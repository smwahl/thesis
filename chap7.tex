\chapter{Tidal Response of Jupiter and Saturn}\label{chap7}

\section{Barotropes}


We assume the liquid planet is in hydrostatic equilibrium,
%
\begin{equation} \nabla P = \rho \nabla U,     
   \label{eq:hydrostatic2} \end{equation}
%
where $P$ is the pressure, $\rho$ is the mass density and $U$ the total effective
potential. Modeling the gravitational field of such a body requires a barotrope
$P(\rho)$ for the body's interior. In this paper, we use the barotrope of
\citet{hubbard2016}, constructed from \textit{ab initio} simulations of
hydrogen-helium mixtures \citep{militzer2013a,militzer2013b}. The $P(\rho)$ relation
is interpolated from a grid of adiabats determined from density functional 
molecular dynamics (DFT-MD) simulations using the Perdew-Burke-Ernzerhof (PBE) functional
\citep{PBE} in combination with a thermodynamic integration technique. The
simulations were performed with cells containing $N_{He}=18$ helium and $N_{H}=220$
hydrogen atoms, corresponding to a solar-like helium mass fraction $Y_0=0.245$. An
adiabat is characterized by an entropy per electron $S/k_B/N_e$
\citep{militzer2013b}, where $k_B$ is Boltzmann's constant and $N_e$ is the number of
electrons. Hereafter we refer to this quantity simply as $S$.

In our treatment, the term ``entropy'' and the symbol $S$ refer to a particular
adiabatic temperature $T(P)$ relationship for a fixed composition H-He mixture
($Y_0=0.245$) as determined from the \textit{ab initio} simulations.  For Jupiter,
the value of $S$ in the outer portion of the planet is determined by matching the
$T(P)$ measurements from the Galileo atmospheric probe (see Figure \ref{fig:eos}).
This adiabatic $T(P)$ is assumed to apply to small perturbations of composition, in
terms of both helium fraction and metallicity. \citet{hubbard2016} demonstrated that
these compositional perturbations have a negligible effect on the temperature
distribution in the interior.

The density perturbations to the equation of state are estimated using the additive
volume law,
%
\begin{equation} 
   V(P, T) = V_H(P, T) + V_{He}(P, T) + V_{Z}(P,T),
\label{eq:volume_law}
\end{equation}
%
where the total volume $V$ is the sum of partial volumes of the main components $V_H$
and $V_{He}$, the heavy element component $V_Z$. \citet{hubbard2016} demonstrated
that this leads to a modified density $\rho$ in terms of the original H-He EOS
density $\rho_0$,
%
\begin{equation} 
   \frac{\rho_0}{\rho} = \frac{1-Y-Z}{1-Y_0} + 
   \frac{ZY_0 + Y - Y_0}{1-Y_0}\frac{\rho_0}{\rho_{He}} + Z\frac{\rho_0}{\rho_Z},
\label{eq:density_ratio}
\end{equation}
%
in which all densities are are evaluated at the same $T(P)$ and $Y_0$ is the helium
fraction used to calculate the H-He equation of state.


\section{Saturn's tidal response} \label{saturn}

\subsection{Saturn interior models}

\citet{lainey2016} present the first determination of the love number $k_2$ for a gas
giant planet using a dataset of astrometric observations of Saturn's coorbital moons.
Their observed value $k_2=0.390 \pm 0.024$ is much larger than the theoretical
prediction of 0.341 by \citet{gavrilov1977}. Here we present calculations suggesting
that the enhancement of Saturn's $k_2$ is the result of the influence of the planet's
rapid rotation, rather than evidence for a non-static tidal response or some other
breakdown of the hydrostatic theory.

For the purposes of this calculation, we use two relatively simple models for
Saturn's interior structure, fitted to physical parameters determined by the
\textit{Voyager} and \textit{Cassini} spacecraft. Table \ref{tab:saturn_params}
summarizes the physical parameters used in our models. We fit our models to minimize
the difference in zonal harmonics from those determined from \textit{Cassini}
\citep{Jacobson2006}.  We consider two different internal rotation rates based on
magnetic field measurements from \textit{Voyager} \citep{desch1981} and
\textit{Cassini} \citep{giampieri2006}, which lead to two different values of $q_{\rm
rot}$. 

In principle, the tidal response of a heterogeneous body will also be different for
satellites with different sizes and orbital parameters. To address this, we also
consider the effect of two major satellites, Tethys and Dione, with different values
for $q_{\rm tid}$ and $R/a$ \citep{archinal2011}. These two satellites, along with
their respective coorbital satellites, were used in the determination of $k_2$ by
\citet{lainey2016}.

For the interior density profile, our first model assumes a constant-density
core surrounded by a polytropic envelope following Eqn. \eqref{eq:poly}. We
constrain the radius of the core to be $a_{\rm core}/a=0.2$, leaving the mass
$m_{\rm core}/M$ as a parameter which is adjusted to match the observed Saturn
$J_2$.  The fitted model using the \textit{Voyager} rotation period matches
both $J_2$ and $J_4$ to within the error bars, but with the \textit{Cassini}
rotation period it matches only $J_2$.  In hydrostatic equilibrium, the two
different rotation rates lead to differences in shape of equipotential surfaces
and, therefore, also to different best fits to $m_{\rm core}/M$. The envelope
polytrope is scaled in order to maintain $M$. Figure
\ref{fig:density_structure} shows the density profile of one such model,
compared to other density models. We consider a model with a total of 128
layers, for which the CMS model has a discretization error \citep{wisdom2016}
smaller than uncertainty in the observations of Saturn's $k_2$.

\begin{figure}[h!]  
  \centering
    \includegraphics[width=22pc]{figs/density_profile.png} \caption{ 
Density structure of simple Saturn models, all fitted to Saturn's observed
$J_2$ \citep{Jacobson2006}.  The blue curve shows an $N=128$ model with a
constant-density core within $r=0.2 a$ and a polytropic outer envelope.  The
red curve shows an $N=4$ model with the same core radius and two additional
spheroids, adjusted to fit both $J_2$ and $J_4$.  For comparison, the dash-dot
curve (teal) shows Saturn model MS24 of \citet{gudkova1999}.  The grey solid
curve shows an unpublished Saturn model based on the density-functional theory,
molecular-dynamics (DFT-MD) equation of state for hydrogen-helium, as used in
the Jupiter model of \citet{hubbard2016}. Figure Credit: William Hubbard.
}
\label{fig:density_structure}
\end{figure}

Our second model has only four spheroids ($N=4$), also depicted in Figure
\ref{fig:density_structure}, with densities and radii adjusted to yield agreement
with both observed $J_2$ and observed $J_4$ as given in Table
\ref{tab:saturn_params}. With a zero-density outermost layer, this leaves two free
paramters, making it the simplest model that can match $J_2$ and $J_4$ exactly.

%\input{tabs/model_param_table}
\begin{table}
    \centering
\caption{Saturn Model Parameters. Identical parameters for Saturn are used with the
    exception of $q_{\rm rot}$, for which the rotation rate from both
    \textit{Cassini} and \textit{Voyager} are considered. A constant core density is
    fitted to match $J_2$, $J_4$, and $J_6$ for a converged figure.
    \label{tab:saturn_params}}
\begin{tabular}{rrrr}
        \hline
{} & Cassini  & Voyager & {} \\
\hline
$GM$ & $3.7931208 \times 10^{7}$ $^{a}$ & - & ${\rm (km^3/s^2) }$
\\
$a$ & $6.0330 \times 10^{4}$       $^{a}$ & - & ${\rm (km) }$  \\ 
$J_2  \times  10^6$  &  $16290.71$ $^{a}$ &  -  &       \\
$J_4  \times  10^6$  &  $-935.83$  $^{a}$ &  -  &       \\
$J_6  \times  10^6$  &  $86.14$    $^{a}$ &  -  &       \\
$q_{\rm rot} $  & $0.1516163$ $^{b}$ & $0.1553029$ $^{c}$ &\\
$r_{\rm core} / a$ & $0.2$ & - &  \\ 
$m_{\rm core} / M$  &  $0.133146$ & $0.140478$ &\\
\hline
\hline
  & Tethys  & Dione & \\
  \hline
$q_{\rm tid} $  & $-2.791103 \times 10^{-8} $ $^{d}$
& $-2.364582 \times 10^{-8}$ $^{d}$ & \\
$R/a $  & $4.8892$ $^{d}$ & $6.2620$ $^{d}$ & \\ 
\hline
\multicolumn{4}{l}{ a. \citet{Jacobson2006}, b. \citet{giampieri2006},} \\
\multicolumn{4}{l}{ c. \citet{desch1981}, d.  \citet{archinal2011}}
\end{tabular}
\end{table}


Finally, we include a third model using a more realistic H-He equation of state
based on DFT-MD simulations, following the preliminary Jupiter model of
\citet{hubbard2016} with a Saturn adiabat.  This model has a density
discontinuity (Fig. \ref{fig:density_structure}) at 0.76 Mbar where the Saturn
adiabat crosses the H-He phase separation curve of \citet{Morales2009}. This
allows the $J_2$ and $J_4$ values to be fitted exactly, by changing the
metallicities above and below the discontinuity. While still schematic, it is
the most realistic of our Saturn models.


The two simple models, while not particularly realistic, capture the major features
of Saturn's internal structure. It is well established that the details of Saturn's
internal structure are largely degenerate, with a wide range of possible core sizes
and densities adequately matching the few observational constraints
\citep{kramm2011,helled2013,Nettelmann2013}.  The qualitative similarities between
our single spheroid and polytrope simulations (Sections \ref{maclaurin} and
\ref{polytrope}) indicate that the rotational enhancement of $k_2$ should be a robust
prediction regardless of the particular details of the interior profile. A comparison
between our polytrope plus core and four layer models provides another test of the
sensitivity of $k_2$ to interior structure.  We do not consider here the influence of
differential rotation \citep{hubbard1982,Kong2013,cao2015,wisdom2016}, which might
have an influence on the gravitational response in comparison to the solid-body
rotation considered here.  

\subsection{Calculated $k_2$ for Saturn} \label{calc_k2}

We take our baseline model to be the $N=128$ CMS core plus polytrope model with
physical parameters fitted to \textit{Cassini} observations. Figure
\ref{fig:saturn_zonal} shows the calculated zonal harmonics $J_n$ up to order
$n=30$.  The even $J_n$ decrease smoothly in magnitude with increasing $n$,
with the slope decreasing at higher $n$.  $J_n$ is negative when $n$ is
divisible by 4, and positive otherwise.  The calculated $J_n$ are essentially
indistinguishable from those calculated for the rotation only case with the
same $q_{\rm rot}$, as is expected given $q_{\rm rot} \gg q_{\rm tid}$. We may
estimate the maximum effect of differential rotation by considering the change
of the calculated $k_2$ as we change the overall planetary rotation period from
the Voyager value ($q_{\rm rot} = 0.155303$) to the Cassini value ($q_{\rm rot}
= 0.151616$), a relative decrease in $q_{\rm rot} $ of about $2 \%$.  From
Table 3, we see that this change increases $k_2$ by about $2 \%$ (holding $J_2$
fixed).  The net effect of a deep-seated smooth variation of rotation rate from
the Voyager value near the equator to the Cassini value near the pole would
presumably be smaller, depending on how much mass is involved in the
differential flow.  \citet{cao2015} have shown that the effect of
realistic deep flow patterns on low order zonal harmonics is small, but a more
quantitative evaluation of their effect on Saturn's $k_2$ remains to be done.

\begin{figure}[h!]  
  \centering
    \includegraphics[width=22pc]{figs/saturn_Jn.pdf}
\caption{ The zonal harmonics $J_n$ for the \textit{Cassini} Saturn model. Positive
values are shown as filled and negative  as empty.}
\label{fig:saturn_zonal}
\end{figure}

Figure \ref{fig:saturn_tesseral} shows the magnitude of $C_{nm}$ for the core plus
polytrope model with \textit{Cassini} rotation. Changing the number of layers,
satellite parameters or the rotation rate to the \textit{Voyager} value leads to a
shift in the values, but the relative magnitudes and signs of $C_{nm}$ remain
approximately the same. In the same figure, we also compare the $C_{nm}$ for a
non-rotating planet having the same density profile $\rho( \lambda_i)$. Here we see
significant shifts in the $C_{nm}$ magnitudes, although the signs remain the same.
For the rotating model, $C_{nm}$ is similar for most points where $n=m$, but with
magnitudes significantly larger when $m<n$. The only exception to this trend is
$C_{31}$ which is lower for the rotating model. These results are all broadly
consistent with the splitting of $k_{nm}$ observed for the polytrope in Section
\ref{polytrope}.

\begin{figure}[h!]  
  \centering
    \includegraphics[width=22pc]{figs/saturn_Cnm_compare.pdf}
\caption{ In red, the tesseral harmonics $C_{nm}$ for the \textit{Cassini} Saturn
model. In black, $C_{nm}$ for the same density profile and same value of $q_{\rm
tid}$, but with $q_{\rm rot}=0$. Positive values are shown as filled and negative as
empty.} \label{fig:saturn_tesseral}
\end{figure}

Table \ref{tab:saturn_results} summarizes our calculated values for $k_2$ for several
different models. The identifying labels ``Cassini'' and ``Voyager'' use the
observed rotation rate from \citet{Jacobson2006}, and \citet{desch1981}
respectively, while ``non-rotating'' is a model with $q_{\rm rot}=0$. The
``non-rotating'' model uses the same ``Cassini'' density profile, meaning that
its density-pressure profile has not been relaxed to be in equilibrium for zero
rotation.  It does, however, allow us to quantify the effect of rotation on the
tidal response by comparison with the ``Cassini'' model. ``Tethys'' and
``Dione'' refer to models with the satellite parameters $q_{\rm tid}$ and $R/a$
corresponding to those satellites, whereas ``no~tide'' is an analogous model
with finite $q_{\rm rot}$ only. ``$N=128$'' uses the polytrope outer envelope
with constant density inner core, whereas ``$N=4$'' is the model which
independently adjusts layer densities to match the observed $J_2$ and $J_4$.
The ``DFT-MD'' models use the H-He equation of state from
\citet{hubbard2016} with $N=511$ layers.

%\input{tabs/saturn_table}
\begin{table}
    \centering
\caption{Calculated Saturn tidal responses \label{tab:saturn_results}}
\begin{tabular}{ccrcr}
    \hline
    {model$^a$} &  & gravitational moment  & & normalized moment \\
    \hline
{Cassini} &  $J_2$     &  $1.62907100025\times10^{-2}   $  &  $J_2/q_{\rm  rot}$  &  $0.10744694879478             $  \\
 {no~tide}   &  $J_4$     &  $-9.2027941201\times10^{-4}   $  &  $J_4/q_{\rm  rot}$  &  $-0.606979160784\times10^{-2} $  \\
 {$N=128$}  &  $J_6$     &  $8.014294995\times10^{-5}     $  &  $J_6/q_{\rm  rot}$  &  $0.5285905549\times10^{-3}    $  \\
 \hline
{non-rotating}  & $C_{22}$  &  $8.5288\times10^{-10}         $  &  $k_2         $      &  $0.36669                      $ \\
{Tethys}        & $J_2$     &  $1.70576\times10^{-9}         $  &  $J_2/q_{\rm  rot}$  &  {-}               \\
{$N=128$}       & $J_4$     &  $-1.351\times10^{-11}         $  &  $J_4/q_{\rm  rot}$  &  {-}               \\
{polytrope}                        & $J_6$     &  $2.2\times10^{-13}            $  &  $J_6/q_{\rm  rot}$  &  {-}               \\
\hline
{Cassini}  & $C_{22}$  &  $9.6070\times10^{-10}         $  &  $k_2         $      &  $0.41304                      $ \\
{Tethys}   & $J_2$     &  $1.629071017501\times10^{-2}  $  &  $J_2/q_{\rm  rot}$  &  $0.1074469499328              $ \\
{$N=128$}  & $J_4$     &  $-9.2027943932\times10^{-4}   $  &  $J_4/q_{\rm  rot}$  &  $-0.60697917880\times10^{-2}  $ \\
{polytrope}                   & $J_6$     &  $8.01429541\times10^{-5}      $  &  $J_6/q_{\rm  rot}$  &  $0.5285905822\times10^{-3}    $ \\
{Voyager}  & $C_{22}$  &  $9.4136\times10^{-10}         $  &  $k_2         $      &  $0.40473                      $ \\
\hline
{Tethys}   & $J_2$     &  $1.629071048760\times10^{-2}  $  &  $J_2/q_{\rm  rot}$  &  $0.1048963407747              $ \\
{$N=128$}  & $J_4$     &  $-9.3570887868\times10^{-4}   $  &  $J_4/q_{\rm  rot}$  &  $-0.60250556585\times10^{-2}  $ \\
{polytrope}                   & $J_6$     &  $8.30176108\times10^{-5}      $  &  $J_6/q_{\rm  rot}$  &  $0.534552720\times10^{-3}     $ \\
\hline
{Cassini}  & $C_{22}$  &  $8.1325\times10^{-10}         $  &  $k_2         $      &  $0.41272                      $ \\
{Dione}    & $J_2$     &  $1.629071019035\times10^{-2}  $  &  $J_2/q_{\rm  rot}$  &  $0.1074469500340              $ \\
{$N=128$}  & $J_4$     &  $-9.2027943688\times10^{-4}   $  &  $J_4/q_{\rm  rot}$  &  $-0.60697917719\times10^{-2}  $ \\
{polytrope}                   & $J_6$     &  $8.01429534\times10^{-5}      $  &  $J_6/q_{\rm  rot}$  &  $0.528590578\times10^{-3}     $ \\
\hline
{Cassini}   &  $C_{22}$  &  $9.6219\times10^{-10}         $  &  $k_2         $      &  $0.41368                      $  \\
{Tethys}    &  $J_2$     &  $1.629071019560\times10^{-2}  $  &  $J_2/q_{\rm  rot}$  &  $0.1074469500686              $  \\
{$N=4$}     &  $J_4$     &  $-9.3583002600\times10^{-4}   $  &  $J_4/q_{\rm  rot}$  &  $-0.61723571821\times10^{-2}  $  \\
                    &  $J_6$     &  $8.61400043\times10^{-5}      $  &  $J_6/q_{\rm  rot}$  &  $0.568144705\times10^{-3}     $  \\
                    \hline
{Cassini}   &  $C_{22}$  &  $9.6235\times10^{-10}         $  &  $k_2         $      &  $0.41375                     $  \\
{Tethys}    &  $J_2$     &  $1.629070920013\times10^{-2}  $  &  $J_2/q_{\rm  rot}$  &  $0.1074469435029              $  \\
{$N=511$}     &  $J_4$     &  $-9.3582993628\times10^{-4}   $  &  $J_4/q_{\rm  rot}$  &  $-0.61723565903\times10^{-2}  $  \\
{DFT-MD}                    &  $J_6$     &  $8.09366588\times10^{-5}      $  &  $J_6/q_{\rm  rot}$  &  $0.533825538\times10^{-3}     $  \\
\hline
{Voyager}   &  $C_{22}$  &  $9.4185\times10^{-10}         $  &  $k_2         $      &  $0.40495                      $  \\
{Tethys}    &  $J_2$     &  $1.629070988378\times10^{-2}  $  &  $J_2/q_{\rm  rot}$  &  $0.104896336887              $  \\
{$N=511$}     &  $J_4$     &  $-9.3583000384\times10^{-4}   $  &  $J_4/q_{\rm  rot}$  & $-0.60258355868\times10^{-2}  $  \\
 {DFT-MD}                   &  $J_6$     &  $8.16661484\times10^{-5}      $  &  $J_6/q_{\rm  rot}$  &  $0.525850615\times10^{-3}     $  \\
\hline
\multicolumn{5}{l}{a. Models are denoted by: rotation rate from \textit{Cassini} or \textit{Voyager}, satellite}\\
\multicolumn{5}{l}{parameters for Tethys or Dione, number of layers $N$, and the equation of state } \\
\multicolumn{5}{l}{used. DFT-MD refers to the H-He equation of state from \citet{hubbard2016}.}
\end{tabular}
\end{table}

All of the rotating models yields a calculated $k_2$ value matching the observation
of \citet{lainey2016} within their error bars. Our baseline model yields
$k_2=0.4130$, while using the \textit{Voyager} observations yields a value
$\sim$0.008 lower. We find that the difference between the $k_2$ values associated
with the satellites Tethys and Dione is $\sim$0.0003, well below the current
sensitivity limit. Using the $\sim$2.5\% higher ``Voyager'' rotation rate leads to a
decrease of $\sim$0.01 in $k_2$. 

In Table \ref{tab:saturn_results}, we also show the calculated $J_2$, $J_4$ and $J_6$
following the convergence of the gravitational field in response to the tidal
perturbation. For the core plus polytrope model, the rotation rate from
\textit{Voyager} is more consistent with the $J_4$ and $J_6$ from
\citet{Jacobson2006}. This doesn't necessarily mean that the \textit{Voyager}
rotation rate is more correct, just that it allows a better fit for our simplified
density model. Nonetheless, our fitted gravitational moments are much closer to each
other than to those from the pre-\textit{Cassini} model of \citet{gavrilov1977}.

In comparison to the other models, the outlier is the non-rotating model, which
underestimates the $k_2$ by $\sim9.4$\% compared to a rotating body with the same
density distribution.  This calculated enhancement accounts for most of the
difference between the observation of $k_2=0.390 \pm .024$ \citep{lainey2016} and the
classical theory result of 0.341 \citep{gavrilov1977}. We attribute our non-rotating
model's larger $k_2$ to our different interior model which matches more recent
constraints on Saturn's zonal gravitational moments $J_2$--$J_6$.

We note that the later theoretical prediction of 0.386 by \citet{vorontsov1984} is
also compatible with the observed $k_2$. Their method considers the effect of
free-oscillations on the tidal response of giant planets. While our rotating models
yield higher values of $k_2$ than \citet{vorontsov1984} our ``non-rotating'' model
produces a $k_2$ smaller than theirs by $\sim$0.02. In principle, it is difficult to
make precise comparisons between models, because of different assumptions about the
interior structure. While consideration of dynamic tidal effects is beyond the scope
of this paper, both effects are likely to influence the tidal response of a real
planet.

%\input{tabs/combined_table}
\begin{table}
    \centering
\caption{Calculated Jovian Tidal Responses\label{tab:combined_table}}
\begin{tabular}{ccc|r}
    \hline
    {planet} & rotation rate & satellite  & $k_2$\\
    \hline
{Jupiter}  &  {Non-rotating}              &  {Io}
&  $0.53725$  \\
\hline
{}         &  {Galileo$^{a}$}  &
{Io$^{a}$}        &  $0.58999$  \\
{}         &  {}                          &
{Europa$^{a}$}    &  $0.58964$  \\
{}         &  {}                          &
{Ganymede$^{a}$}  &  $0.58949$  \\
\hline
{Saturn}   &  {Non-rotating}              &  {Tethys$^{a}$}    &  $0.36669$  \\
\hline
{}         &  {Cassini$^{b}$}  &
{Tethys$^{a}$}    &  $0.41375$  \\
{}         &  {}                          &
{Dione$^{a}$}     &  $0.41272$  \\
\hline
{}         &  {Voyager$^{c}$}  &
{Tethys$^{a}$}    &  $0.40495$  \\
\hline
\multicolumn{4}{l}{ a. \citet{archinal2011}, b. \citet{giampieri2006},} \\
\multicolumn{4}{l}{c. \citet{desch1981}}
\end{tabular}
\end{table}

In addition to the difference in $k_2$, the non-rotating model also predicts slightly
different tidal components of the zonal gravitational moments. Finding the difference
in values between the ``no tide'' model and the analogous tidal model yields
$J_{2,{\rm tid}}=1.7254\times10^{-10}$, $J_{4,{\rm tid}}=-2.732\times10^{-11}$ and
$J_{6,{\rm tid}}=4.14\times10^{-12}$, which are different than calculated zonal
moments for the ``non-rotating'' model.

It may be initially surprising that the four-layer model and the semi-realistic
DFT-MD based models  yield a $k_2$ value only slightly different than the
polytrope model. The three models represent very different density structures
even though they lead to similar low-order zonal harmonics. The fact these
models are indistinguishable by their $k_2$ suggests that the tidal response of
Saturn is only a weak function of the detailed density structure within the
interior of the planet. Indeed, the two models matching $J_4$ are closer to
each other than to the polytrope model that does not match $J_4$. This behavior
can be understood by referring to Eqn.  \eqref{eq:k2_lambda2}, which shows that
to lowest order, $k_2$ and $\Lambda_2$ contain the same information about
interior structure.  This statement is not true when we include a nonlinear
response to rotation and tides.  Thus, future high-precision measurements of
the $k_{nm}$ of Jovian planets, say to better than $0.1\%$, will be useful for
constraining basic parameters such as the interior rotation rate of the planet,
and may help to break the current degeneracy of interior density profiles.  The
theory presented in this paper is intended to match the anticipated precision
of such future measurements.

\section{Jupiter's tidal response} \label{jupiter}

The \textit{Juno} spacecraft began studying Jupiter at close range following its
orbital insertion in early July 2016. The unique low-periapse polar orbit and precise
Doppler measurements of the spacecraft's acceleration will yield parameters of
Jupiter's external gravitational field to unprecedented precision, approaching a
relative precision of $\sim 10^{-9}$ \citep{kaspi2010}. In addition to providing
important information about the planet's interior mass distribution, the
non-spherical components of Jupiter's gravitational field should exhibit a detectable
signal from tides induced by the planet's closer large moons, possibly superimposed
on signals from mass anomalies induced by large-scale dynamic flows in the planet's
interior \citep{cao2015,kaspi2010,kaspi2013}.

As a benchmark for comparison with expected \textit{Juno} data, \citet{hubbard2016}
constructed static interior models of the present state of Jupiter, using a
barotropic pressure-density $P(\rho)$ equation of state for a near-solar mixture of
hydrogen and helium, determined from \textit{ab intio} molecular dynamics simulations
\citep{militzer2013a,militzer2013b}. In this paper, we extend those models to predict
the static tidal response of Jupiter using the three-dimensional concentric Maclaurin
spheroid (CMS) method \citep{wahl2016}.

The \citet{hubbard2016} preliminary Jupiter model is an axisymmetric, rotating model
with a self-consistent gravitational field, shape and interior density profile. It is
constructed to fit pre-\textit{Juno} data for the degree-two zonal gravitational
harmonic $J_2$ \citep{jacobson2003}. While solutions exist matching pre-\textit{Juno}
data for the degree-four harmonic $J_4$, models using the \textit{ab initio} EOS required unphysical
compositions with densities lower than that expected for the pure H-He mixture. As a result, the
preferred model of \citet{hubbard2016} predicts a $J_4$ with an absolute value above
pre-\textit{Juno} error bars. Preliminary Jupiter models consider the effect of a
helium rain layer where hydrogen and helium become immiscible \citep{stevenson1977a}.
The existence of such a layer has important effects for the interior structure of the
planet, since it inhibits convection and mixing between the molecular exterior and
metallic interior portions of the H-He envelope. This circumstance provides a
physical basis for differences in composition and thermal state between the inner and
outer portions of the planet.  Adjustments of the heavy element content and entropy
of the $P(\rho)$ barotrope allow identification of an interior structure consistent
with both pre-\textit{Juno} observational constraints and the \textit{ab initio}
material simulations. The preferred preliminary model predicts a dense inner core
with $\sim$12 Earth masses and an inner hydrogen-helium rich envelope with
$\sim$3$\times$ solar metallicity, using an outer envelope composition matching that
measured by the \textit{Galileo} entry probe.

Although the \textit{Cassini} Saturn orbiter was not designed for direct measurements
of the high degree and order components of Saturn's gravitational field, the first observational
determination of Saturn's second degree Love number $k_2$ was recently reported by
\citet{lainey2016}. This study used an astrometric dataset for Saturn's co-orbital
satellites to fit $k_2$, and identified a value significantly larger than the
theoretical prediction of \citet{gavrilov1977}. The non-perturbative CMS method
obtains values of $k_2$ within the observational error bars for simple models of
Saturn's interior, indicating the high value can be explained completely in terms of
static tidal response \citep{wahl2016}. The perturbative method of
\citet{gavrilov1977} provides an initial estimate of tidally induced terms in the
gravitational potential, but neglects terms on the order of the product of tidal
and rotational perturbations. \citet{wahl2016} demonstrated, that for the
rapidly-rotating Saturn, these terms are significant and sufficient to explain the
observed enhancement of $k_2$.

\section{Equation of state considerations}
The choice of equation of state effects the density structure of the planet, and
consequently, the distribution of heavy elements that is consistent with observational
constraints. For comparison, we also construct models using the \citet{saumon1995}
equation of state (SCvH) for H-He mixtures, which has been used extensively in giant planet
modeling. 

\begin{figure}[h!]  
  \centering
    \includegraphics[width=22pc]{figs/jupiter_eos.pdf}
\caption{ The barotrope used in preferred model Jupiter `DFT-MD\_7.13'. Top:
    temperature-pressure relationship for a hydrogen-Helium mixture with Y=0.245,
    with a entropy $S=7.08$ at pressures below the demixing region, and $S=7.13$ at
    pressures above the demixing region. The helium demixing region is shown by the
    gap and shaded region. The red line shows measurements from the \textit{Galileo}
probe. Bottom: density-pressure relationship for the same barotrope.}
\label{fig:eos}
\end{figure}

\textit{Ab initio} simulations show that, at the temperatures relevant to Jupiter's
interior, there is no distinct, first-order phase transition between molecular
(diatomic, insulating) hydrogen to metallic (monatomic, conducting) hydrogen
\citep{vorberger2007}. In the context of a planet-wide model, however, the transition
takes place over the relatively narrow pressure range between $\sim$1-2 Mbar. Within
a similar pressure range an immiscible region opens in the H-He phase diagram
\cite{morales2013}, which under correct conditions allows for a helium rain layer
\cite{stevenson1977a,stevenson1977b}. By comparing our adiabat calculations to the
\cite{morales2013} phase diagram, we predict such a helium rain layer in present-day
Jupiter \citep{hubbard2016}. The extent of this layer in our models is highlighted in
Figure \ref{fig:eos}. While the detailed physics involved with the formation and
growth of a helium rain layer is poorly understood, the existence of a helium rain
layer has a number of important consequences for the large-scale structure of the
planet. In our models, we assume this process introduces a superadiabatic temperature
gradient and a compositional difference between the outer, molecular layer and inner,
metallic layer.


\begin{table}
\centering
%\resizebox{\linewidth}{!}{%

\caption{Jupiter Model Parameters \label{tab:jupiter_params}}
%\begin{adjustbox}{max width=\textwidth}
\begin{tabular}{l|rrr}
    \hline
    {} & {Jupiter} & {} & {} \\
    \hline
$GM$ & $1.26686535 \times 10^{8}$$^a$  &  ${\rm (km^3/s^2) }$  & \\
$a$ & $7.1492 \times 10^{4}$$^a$         & ${\rm (km) }$   & \\ 
$J_2  \times  10^6$  & $14696.43$$^a$   &        & \\
$J_4  \times  10^6$  & $-587.14$$^a$  &        & \\
$q_{\rm rot} $  & $ .08917920 $$^b$   & & \\
$r_{\rm core} / a$ & $0.15$  & \\ 
\hline \hline
$q_{\rm tid} $  & $-6.872 \times 10^{-7} $ & $-9.169\times 10^{-8 }$  
&  $-6.976\times10^{-8}$  \\
$R/a$  & $5.90$  &  $9.39$  & $14.98$ \\
\hline
\multicolumn{4}{l}{ a. \citet{jacobson2003}, b. \citet{archinal2011}}
\end{tabular}

\end{table}

In summary, the barotrope and resulting suite of axisymmetric Jupiter models that we
use in this investigation are identical to the results presented by
\citet{hubbard2016}. Each model has a central core mass and envelope metallicities
set to fit the observed $J_2$ \citep{jacobson2003}, with densities corrected to be
consistent with non-spherical shape of the rotating planet. Since tidal corrections
to a rotating Jupiter model are of order $10^{-7}$, see
Table~\ref{tab:jupiter_params} and the following section, it is unnecessary to re-fit
the tidally-perturbed models to the barotrope assumed for axisymmetric models. 

\begin{table}
\centering
%\resizebox{\linewidth}{!}{%

\caption{Jupiter model parameters from \cite{hubbard2016}. $S$ is the
specific entropy for the adiabat through the inner or outer H-He envelope. $M$ is the
mass of heavy elements included in each layer. Each model matches
observed  $J_2 = 14696.43 \times 10^{−6}$ \citep{jacobson2003}, JUP230 orbit solution,
to six significant figures. Models denoted as 'DFT-MD' if equation of state based on
\textit{ab initio} simulations or 'SC' for the \citet{saumon1995} equation of
state, with a number denoting the entropy below the helium demixing layer.  The
number of Models denoted with ($J_4$) also match observed $J_4=-596.31\times
10^{-6}$. Model denoted (equal-$Z$) is constrained to have same metallicity in inner
and outer portions of the planet. Preferred interior model shown in bold face.
\label{tab:model_values}}

%\begin{adjustbox}{max width=\textwidth}
\begin{tabular}{l|cc|rrrr}
    \hline
    {} & {$S_{\rm molec.}$} &  {$S_{\rm metal.}$} & {$M_{\rm core}$} & {$M_{\rm Z,molec.}$}
    & {$M_{\rm Z,metal.}$} & {$Z_{\rm global}$} \\
    &  {($S/k_B/N_e$)} & {($S/k_B/N_e$)} & {($M_E$)} &  {($M_E$)} &  {($M_E$)} & \\
    \hline
    DFT-MD   7.24             &  7.08  &  7.24  &  12.5  &  0.9     &  10.3  &  0.07  \\
    DFT-MD  7.24~(equal-$Z$)  &  7.08  &  7.24  &  13.1  &  1.1     &  7.5   &  0.07  \\
    DFT-MD  7.20              &  7.08  &  7.20  &  12.3  &  0.8     &  9.9   &  0.07  \\
    DFT-MD  7.15              &  7.08  &  7.15  &  12.2  &  0.7     &  9.2   &  0.07  \\
    DFT-MD  7.15~($J_4$)      &  7.08  &  7.15  &  9.7   &  $-$0.6  &  14.9  &  0.08  \\
    {\bf DFT-MD 7.13}       & {\bf 7.08}   & {\bf 7.13} & {\bf 12.2}  & {\bf  0.7}   &
    {\bf 8.9}   &  {\bf 0.07}  \\
    DFT-MD  7.13~(low-$Z$)  &  7.08  &  7.15  &  14.0  &  0.2  &  1.1   &  0.05  \\
    DFT-MD  7.08            &  7.08  &  7.08  &  12.0  &  0.6  &  8.3   &  0.07  \\
    SC      7.15            &  7.08  &  7.15  &  4.8   &  3.5  &  28.2  &  0.11  \\
    SC      7.15~($J_4$)    &  7.08  &  7.15  &  4.3   &  3.2  &  29.3  &  0.12  \\
    \hline
\end{tabular}

\end{table}

The physical parameters for each of these models is summarized in Table
\ref{tab:model_values}. The gravitational moments at the planet's surface are
insensitive to the precise distribution of extra heavy-element rich material within the
innermost part of the planet. For instance, 
the gravitational moments do not allow us to discern between a model
with a dense rocky core and a model without a dense rocky core but with same amount
of heavy element distributed in a larger but restricted volume in the deep interior. 
Maintaining a constant core radius is
computationally convenient when finding a converged core mass to $J_{2}$, since it
requires no modification of the radial grid used through the envelope. For this
reason we consider models with a constant core radius of $0.15a$. Decreasing this
radius below $0.15a$ for a given core mass has a negligible effect on the calculated
gravitational moments \citep{hubbard2016}. Figure \ref{fig:density_jupiter} shows
the density profile for two representative models.  In general, models using the
DFT-MD equation of state lead to a larger central core and a lower envelope
metallicity than those using SCvH.  \citet{hubbard2016} also noted that these models
predict a value for $J_4$ outside the reported observational error bars
\citep{jacobson2003}, since they would require unrealistic negative values of $Z$ to
match both $J_2$ and $J_4$.

\begin{figure}[h!]  
  \centering
    \includegraphics[width=22pc]{figs/jupiter_density.pdf}
\caption{   Density structure of Jupiter models (the planetary unit of
density $\rho_{pu}=M/a^3$).  The red curve shows our preferred
    model based on \textit{ab initio} calculations. The blue curve uses the Saumon and
    Chabrier equation of state. The shaded area denotes the helium demixing region.
    Both models have $N=511$ layers and a dense core within $r=0.15a$.  Constant core
    densities are adjusted to match $J_2$ as measured by fits to Jupiter flyby Doppler
    data \citep{jacobson2003}.}
\label{fig:density_jupiter}
\end{figure}

\subsection{State mixing for static Love numbers} \label{state_mixing}

In the CMS method applied to tides, we calculate the tesseral harmonics $C_{nm}$
directly, and the Love numbers $k_{nm}$ are then calculated using Eq.~\ref{eq:kn}.
For the common tidal problem where $q_{\rm tid}$ and $q_{\rm rot}$ are carried to
first order perturbation only, this definition of $k_{nm}$ removes all dependence on
the small parameters $q_{\rm tid}$ and $a/R$, which is convenient for calculating the
expected tidal tesseral terms excited by satellites of arbitrary masses at arbitrary
orbital distances.  However, the high-precision numerical results from our CMS tidal
theory reveal that when $q_{\rm rot} \approx 0.1$, as is the case for Jupiter and
Saturn, the mixed excitation of tidal and rotational harmonic terms in the external
gravity potential has the effect of introducing a small but significant dependence of
$k_{22}$ on $a/R$; see Fig.~\ref{fig:J4_k2}. In the absence of rotation, the CMS
calculations yield results without any state mixing, and the $k_{nm}$ are, as
expected, constant with respect to $a/R$.  It is important to note this effect on the
{\it static} Love numbers because, as we discuss below, dynamical tides can also
introduce a dependence on $a/R$ via differing satellite orbital frequencies.

\begin{figure}[h!]  
  \centering
    \includegraphics[width=22pc]{figs/jupiter_weights.pdf}
\caption{Top: Relative contribution of spheroids to external gravitational zonal
    harmonic coefficients up to order 8. Bottom: Relative contribution of spheroids
    to to external gravitational tesseral coefficients up to order 4. Tesseral
    moments of the same order (i.e. $C_{31}$ and $C_{33}$) have indistinguishable
radial distributions. Values normalized so that each harmonic integrates to unity.
The shaded area denotes the helium demixing region.  }
\label{fig:jupiter_weights}
\end{figure}

\subsection{Calculated static tidal response}

The calculated zonal harmonics $J_n$ and tidal Love numbers $k_{nm}$ for all of the
Jupiter models with Io satellite parameters are shown in
Tab.~\ref{tab:model_harmonics}. Our preferred Jupiter model has a calculated $k_{2}$
of 0.5900. In all cases, these Love numbers are significantly different from those
predicted for a non-rotating planet (see Tab.~\ref{tab:satellite_harmonics}).
Fig.~\ref{fig:tesseral_rotation} shows the different tesseral harmonics $C_{nm}$
calculated with and without rotation. For a non-rotating planet with identical
density distribution to the preferred model we find a much smaller $k_{22}=0.53725$.
\textit{Juno} should, therefore, be able to test for the existence of the rotational
enhancement of the tidal response.


\begin{sidewaystable}
\caption{Gravitational Harmonic Coefficients and Love Numbers\label{tab:model_harmonics}}
%\centering

\begin{adjustbox}{max width=\textheight}
\begin{tabular}{l|rrrr|rrrrrrrrrrrr}
     %\toprule
    \hline
(all $J_n$ $\times$ $10^6$) &  $J_{4}$ &  $J_{6}$ &  $J_{8}$ &  $J_{10}$ &  $k_{22}$ &  $k_{31}$ &  $k_{33}$ &
$k_{42}$ &  $k_{44}$ &  $k_{51}$ &  $k_{53}$ &  $k_{55}$ &  $k_{62}$ &  $k_{64}$ &
$k_{66}$ \\
\hline
pre-\textit{Juno}~observed &  -587.14 &    34.25 &   - & - & - &
- & - & - & - & - & - & - & - &
- & -  \\
(JUP230)$^a$ &  $\pm$1.68 &   $\pm$5.22 &   - & - & - &
- & - & - & - & - & - & - & - &
- & -  \\
\hline
DFT-MD 7.24            &  -597.34 &    35.30 &   -2.561 &     0.212 &   0.59001 &   0.19455 &   0.24424 &   1.79143 &   0.13920 &   0.98041 &   0.84803 &   0.09108 &   6.19365 &   0.52154 &   0.06451 \\
DFT-MD 7.24~(equal-$Z$) &  -599.07 &    35.48 &   -2.579 &     0.214 &   0.59004 &   0.19512 &   0.24498 &   1.79695 &   0.13984 &   0.98531 &   0.85239 &   0.09159 &   6.22719 &   0.52474 &   0.06492 \\
DFT-MD 7.20            &  -596.88 &    35.24 &   -2.556 &     0.211 &   0.59000 &   0.19440 &   0.24404 &   1.78994 &   0.13902 &   0.97903 &   0.84678 &   0.09093 &   6.18392 &   0.52058 &   0.06438 \\
DFT-MD 7.15            &  -596.31 &    35.18 &   -2.549 &     0.211 &   0.58999 &   0.19422 &   0.24381 &   1.78811 &   0.13881 &   0.97733 &   0.84526 &   0.09074 &   6.17202 &   0.51941 &   0.06423 \\
DFT-MD 7.15~($J_4$)     &  -587.14 &    34.18 &   -2.451 &     0.201 &   0.58985 &   0.19118 &   0.23989 &   1.75874 &   0.13537 &   0.95088 &   0.82162 &   0.08794 &   5.98975 &   0.50178 &   0.06195 \\
{\bf DFT-MD 7.13}        & {\bf -596.05} &  {\bf   35.15} &  {\bf  -2.546} &    {\bf
0.210} &  {\bf  0.58999} &   {\bf 0.19413} &   {\bf 0.24370} &   {\bf 1.78728} &
{\bf 0.13871} &   {\bf 0.97655} &   {\bf 0.84456} &   {\bf 0.09066} &   {\bf 6.16658}
&   {\bf 0.51887} &   {\bf 0.06416} \\
DFT-MD 7.13 (low-$Z$)   &  -601.72 &    35.77 &   -2.608 &     0.217 &   0.59009 &   0.19599 &   0.24610 &   1.80546 &   0.14083 &   0.99296 &   0.85924 &   0.09239 &   6.28019 &   0.52985 &   0.06558 \\
DFT-MD 7.08        &  -595.48 &    35.08 &   -2.539 &     0.210 &   0.58998 &   0.19395 &   0.24346 &   1.78542 &   0.13848 &   0.97482 &   0.84301 &   0.09047 &   6.15442 &   0.51767 &   0.06400 \\
SC 7.15           &  -589.10 &    34.86 &   -2.556 &     0.214 &   0.58993 &   0.19112 &   0.24002 &   1.76641 &   0.13699 &   0.96568 &   0.83567 &   0.09024 &   6.12279 &   0.51832 &   0.06449 \\
SC 7.15 ($J_4$)         &  -587.14 &    34.65 &   -2.534 &     0.212 &   0.58991 &   0.19048 &   0.23918 &   1.76013 &   0.13625 &   0.95997 &   0.83054 &   0.08963 &   6.08299 &   0.51443 &   0.06398 \\
\hline
\multicolumn{16}{l}{All Love numbers for a tidal response with $q_{\rm tid}$ and $R/a$
corresponding to Jupiter's Satellite Io. Preferred interior model shown in bold
face.}\\
\multicolumn{16}{l}{a. JUP230 orbit solution \cite{jacobson2003}}
\end{tabular}

\end{adjustbox}
\end{sidewaystable}


The effect of the interior mass distribution for a suite of realistic models has a
minimal effect on the tidal response. Most models using the DFT-MD barotrope are
within a 0.0001 range of values. The one outlier being the model constrained to match
$J_4$ with unphysical envelope composition. The models using the SCvH barotrope
yields slightly lower, but still likely indistinguishable values of $k_{22}$. The
higher order harmonics show larger relative differences between models, but still
below detection levels. Regardless, the zonal harmonic values are more diagnostic for
differences between interior models than the tidal Love numbers. Fig.~\ref{fig:J4_k2}
summarizes these results, and shows that the calculated $k_{22}$ value varies
approximately linearly with $J_4$.  If \textit{Juno} measures higher order tesseral
components of the field, it may be able to verify a splitting of the $k_{nm}$ Love
numbers with different $m$, for instance, a predicted difference between
$k_{31}\sim0.19$ and $k_{33}\sim0.24$.


\begin{figure}[h!]  
  \centering
    \includegraphics[width=22pc]{figs/jupiter_Cnm_rotation.pdf}
\caption{ The tesseral harmonic magnitude $C_{nm}$ for the `DFT\_MD 7.13' Jupiter
model with a tidal perturbation corresponding to Io at its average orbital distance.
Black: the values calculated with Jupiter's rotation rate; red: the values for
a non-rotating body with identical layer densities.  Positive values are shown as
filled and negative as empty.}
\label{fig:tesseral_rotation}
\end{figure}

\begin{figure}[h!]  
  \centering
    \includegraphics[width=22pc]{figs/jupiter_Cnm_satellite.pdf}
\caption{ The tesseral harmonic magnitude $C_{nm}$ for the `DFT\_MD 7.13' Jupiter
model with a tidal perturbation corresponding to different satellites: Io (black),
Europa (red) and Ganymede (blue).}
\label{fig:tesseral_satellites}
\end{figure}

In addition, we find small, but significant, differences between the tidal response
between Jupiter's most influential satellites. Fig. \ref{fig:tesseral_satellites}
shows the calculated $C_{nm}$ for simulations with Io, Europa and Ganymede. We
attribute the dependence on orbital distance to the state mixing described in Section
\ref{state_mixing}. This leads to a difference in $k_{22}$ between the three
satellites (Tab.~\ref{tab:satellite_harmonics}) that may be discernible in
\textit{Juno}'s measurements.


\begin{figure}[h!]  
  \centering
    \includegraphics[width=22pc]{figs/jupiter_J4_k2.pdf}
\caption{ Predicted $k_2$ Love numbers for Jupiter models plotted against $J_4$. The
    favored interior model `DFT-MD\_7.13' with a tidal perturbation from Io is
    denoted by the red star. The other interior models with barotropes based on the
    DFT-MD simulations (blue) have $k_2$ forming a linear trend with $J_4$.  Models
    using the Saumon and Chabrier barotrope (green) plot slightly above this trend.
    The of $k_2$ for a single model `DFT-MD\_7.13' with tidal perturbations from
    Europa and Ganymede (yellow) show larger differences than any resulting from
    interior structure.
    \label{fig:J4_k2}}
\end{figure}

\section{Correction for dynamical tides}

\subsection{Small correction for non-rotating model of Jupiter}

The general problem of the tidal response of a rotationally-distorted liquid
Jovian planet to a time-varying perturbation from an orbiting satellite
has not been solved to a precision equal to that of the static CMS tidal
theory of \citet{wahl2016} and this paper.  However, an elegant approach
based on free-oscillation theory has been applied to the less general problem
of a non-rotating Jovian planet perturbed by a satellite in a circular orbit
\citep{vorontsov1984}.  Let us continue to use the spherical coordinate system $(r,\theta,\phi)$,
where $r$ is radius, $\theta$ is colatitude and $\phi$ is longitude.
Assume that the satellite is in the planet's equatorial
plane ($\theta=\pi/2$) and orbits prograde at angular rate $\Omega_S$.  For a given planet
interior structure, \citet{vorontsov1984} first obtain its eigenfrequencies
$\omega_{\ell m n}$ and orthonormal
eigenfunctions ${\bf u}_{\ell m n}(r,\theta,\phi)$, projected on spherical harmonics
of degree $\ell$ and order $m$ (the index $n=0,1,2,...$ is the number of radial nodes
of the eigenfunction).  Note that in their convention, oscillations moving prograde
(in the direction of increasing $\phi$) have negative $m$, whereas some authors, e.g.
\citet{Marley1993} use the opposite convention.

Treating the tidal response as a forced-oscillation problem, equation (24) of
\citet{vorontsov1984}, the vector tidal displacement ${\bf \xi}$ then reads

\begin{equation}
    {\bf \xi}({\bf r},t) = -\sum_{\ell,m,n} {{({\bf u}_{\ell,m,n},{\bf \nabla}\psi^r_{\ell m})}
\over {\omega^2_{\ell m n} - m^2 \Omega_S^2}} e^{-i m \Omega_S t},
\label{eq:xioft}
\end{equation}
where $({\bf u}_{\ell,m,n},{\bf \nabla} \psi^r_{\ell m})$ is the integrated scalar product of the
vector displacement eigenfunction ${\bf u}_{\ell m n}(r,\theta,\phi)$ and the
gradient of the corresponding term of the satellite's tidal potential
$\psi^r_{\ell m}(r,\theta,\phi,t)$, {\it viz.}
\begin{equation}
    ({\bf u}_{\ell,m,n},{\bf \nabla} \psi^r_{\ell m})=
\int dV \rho_0(r) ({\bf u}_{\ell,m,n} \cdot {\bf \nabla} \psi^r_{\ell m}).
\label{eq:scalprod}
\end{equation}
The integral is taken over the entire spherical volume of the planet,
weighted by the unperturbed spherical mass density distribution $\rho_0(r)$.

\citet{vorontsov1984} then show that, for the nonrotating Jupiter problem, the
degree-two dynamical Love number $k_{2,d}$ is determined to high precision ($\sim$
0.05\%) by off-resonance excitation of the $\ell=2, m=2, n=0$ and $\ell=2, m=-2, n=0$
oscillation modes, such that
%
\begin{equation}
    k_{2,d}={{\omega^2_{220}} \over {\omega^2_{220} - (2 \Omega_S)^2}} k_2,
\label{eq:k2d}
\end{equation}
%
noting that $\omega_{220}$ and $\omega_{2-20}$ are equal for nonrotating Jupiter (all
Love numbers in the present paper written without the subscript {\it d} are
understood to be static).  For a Jupiter model fitted to the observed value of $J_2$,
\citet{vorontsov1984} set $\Omega_S = 0$ to obtain $k_2  = 0.541$, within 0.7\% of
our nonrotating value of 0.53725 (see Table~\ref{tab:satellite_harmonics}).  Setting
$\Omega_S$ to the value for Io, Eq.~\ref{eq:k2d} predicts that $k_{2,d} = 0.547$,
i.e. the dynamical correction increases $k_2$ by 1.2\%.  This effect would be only
marginally detectable by the \textit{Juno} measurements of Jupiter's gravity, given
the expected observational uncertainty.


\begin{table}
\centering
%\resizebox{\linewidth}{!}{%

\caption{Tidal Response for Various Satellites and Non-rotating Model. Tidal response of preferred interior model `DFT\_MD 7.13' with
    $q_{\rm tid}$ and $R/a$ for three large satellites, and for a `non-rotating'
    model with $q_{\rm rot}=0$. In bold face is the same preferred model as in
    \label{tab:satellite_harmonics} }
%\begin{adjustbox}{max width=\textwidth}
\begin{tabular}{l|rrrr}
    \hline
    {} & {\bf Io} & {Io$^a$} &
    {Europa} & {Ganymede} \\
    {}  &  {} &
    {non-} & {} & {} \\
    {}  &  {} & {rotating} & \\
    \hline
    $k_{22}$  &  {\bf  0.58999}  &  0.53725  &  0.58964  &  0.58949  \\
    $k_{31}$  &  {\bf  0.1941}   &  0.2283   &  0.1938   &  0.1937   \\
    $k_{33}$  &  {\bf  0.2437}   &  0.2283   &  0.2435   &  0.2435   \\
    $k_{42}$  &  {\bf  1.787}    &  0.1311   &  4.357    &  12.41    \\
    $k_{44}$  &  {\bf  0.1387}   &  0.1311   &  0.1386   &  0.1386   \\
    $k_{51}$  &  {\bf  0.9766}   &  0.0860   &  2.373    &  6.7486   \\
    $k_{53}$  &  {\bf  0.8446}   &  0.0860   &  2.0289   &  5.740    \\
    $k_{55}$  &  {\bf  0.0907}   &  0.0860   &  0.0906   &  0.0906   \\
    $k_{62}$  &  {\bf  6.167}    &  0.0610   &  37.04    &  302.1    \\
    $k_{64}$  &  {\bf  0.5189}   &  0.0610   &  1.237    &  3.487    \\
    $k_{66}$  &  {\bf  0.0642}   &  0.0610   &  0.0641   &  0.0641   \\
    \hline
    \multicolumn{5}{l}{a.~Non-rotating model has identical density }\\
    \multicolumn{5}{l}{~~ structure to rotating model.}
\end{tabular}

\end{table}

\subsection{ Dynamical effects for rotating model of Jupiter}

For a more realistic model of Jupiter tidal interactions, the dynamical correction to
the tidal response might be larger, and therefore, more detectable.  We have already
shown (Table 4) that inclusion of Jupiter's rotational distortion increases the
static $k_2$ by nearly 10\% above the non-rotating static value for a spherical
planet.  In this section, we note that
Jupiter's rapid rotation may also change Jupiter's dynamic tidal response,
by a factor that remains to be calculated.

In a frame co-rotating with Jupiter at the rate $\Omega_P=2 \pi / 35730$s,
the rate at which the subsatellite point moves is obtained by the scalar difference
$\Delta \Omega = \Omega_S - \Omega_P$, which is negative for all Galilean satellites.  Thus,
in Jupiter's fluid-stationary frame, the subsatellite point moves retrograde
(it is carried to the west by Jupiter's spin).  
For Io, we have $\Delta \Omega = -1.35 \times 10^{-4}$ rad/s.
Jupiter's rotation splits the
$\omega_{2\pm20}$ frequencies \citep{vorontsov1981}, such that
$\omega_{2-20}= 5.24 \times 10^{-4}$ rad/s and
$\omega_{220}= 8.73 \times 10^{-4}$ rad/s.  The oscillation
frequencies of the Jovian modes closest to tidal resonance with Io are
higher than the frequency of the tidal disturbance in
the fluid-stationary frame, but are closer to resonance than
in the case of the non-rotating model considered by
\citet{vorontsov1984}.

An analogous investigation for tides on Saturn raised by
Tethys and Dione yields results similar to the Jupiter values:
tides from Tethys or Dione are closer to resonance with normal modes for $\ell=2$ and
$m=2$ and $m=-2$.  Since our static
value of $k_2$ for Saturn \citep{wahl2016} is robust to various assumptions about interior
structure and agrees well
with the value deduced by \citet{lainey2016}, so far we have no evidence for dynamical
tidal amplification effects in the Saturn system.  

Unlike the investigation of \citet{lainey2016}, which relied on analysis of astrometric data for
Saturn satellite motions, the \textit{Juno} gravity investigation will attempt to directly determine
Jupiter's $k_2$ by analyzing the influence of Jovian tesseral-harmonic terms on the spacecraft orbit.
A discrepancy
between the observed $k_2$ and our predicted static $k_2$
would indicate the need for
a quantitative theory of dynamical tides in rapidly rotating Jovian planets. 


\section{Summary} \label{summary}

The non-perturbative CMS method for calculating a self-consistent shape and
gravitational field of a static liquid planet has been extended to include the
effect of a tidal potential from a satellite. This is expected to represent the
largest contribution to the low-order tesseral harmonics measured by
\textit{Juno} and future spacecraft studies of the gas giants. This approach
has been benchmarked against analytical results for the tidal response of the
constant density Jeans/Roche spheroid, a two constant density layer model and
the polytrope of index unity. 

We highlight for the first time an important effect of rapid rotation on the tidal
response of the gas giants. CMS simulations of the tidal response on bodies with
large rotational flattening show significant deviation in the tesseral harmonics of
the gravitational field as compared to simulations without rotation. This includes
splitting of the love numbers into different $k_{nm}$ for any given order $n>2$.
Meanwhile, it leads to an observable enhancement in $k_2$ compared to a non-rotating
model.

This rotational enhancement of the $k_2$ love number for a simplified interior model
of Saturn agrees with the recent observational result \citep{lainey2016}, which found
$k_2$ to be much higher than previous predictions. Our predicted values of $k_2$ are
robust for reasonable assumptions of interior structure, rotation rate and satellite
parameters.  The \textit{Juno} spacecraft is expected to measure Jupiter's
gravitational field to sufficiently high precision to measure lower order tesseral
components arising from Jupiter's large moons, and we predict an analogous rotational
enhancement of $k_2$ for Jupiter.  Our high-precision tidal theory will be an
important component of the search for non-hydrostatic terms in Jupiter's external
gravity field.

Our study has predicted the static tidal Love numbers $k_{nm}$ for Jupiter and its three
most influential satellites. These results have the following features: (a) They are
consistent with the most recent evaluation of Jupiter's $J_2$ gravitational
coefficient; (b) They are fully consistent with state of the art interior models
\citep{hubbard2016} incorporating DFT-MD equations of state, with a density
enhancement across a region of H-He imiscibility \citep{morales2013}; (c) We use the
non-perturbative CMS method for the first time to calculate high-order tesseral
harmonic coefficients and Love numbers for Jupiter.

The combination of the DFT-MD equation of state and observed $J_{2n}$ strongly limit
the parameter space of pre-\textit{Juno} models. Within this limited parameter space,
the calculated $k_{nm}$ show minimal dependence on details of the interior structure.
Despite this, our CMS calculations predict several interesting features of Jupiter's
tidal response that the \textit{Juno} gravity science system should be able to
detect. In response to the rapid rotation of the planet the $k_2$ tidal Love number
is predicted to be much higher than expected for a non-rotating body. Moreover, the
rotation causes state mixing between different tesseral harmonics, leading to a
dependence of higher order static $k_{nm}$ on both $m$ and the orbital distance of the
satellite. An additional, significant dependence on $a/r$ is expected in the dynamic
tidal response. We present an estimate of the dynamical correction to our
calculations of the static response, but a full analysis of the dynamic theory of
tides has yet to be performed.


