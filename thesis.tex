%% thesis.tex 2014/04/11
%
% Based on sample files of unknown authorship.
%
% The Current Maintainer of this work is Paul Vojta.

\documentclass[phd,12pt]{ucbthesis}
%\usepackage{biblatex}
%\usepackage{longtable}
\usepackage{amsmath}
\usepackage{hyperref}

%\usepackage{lineno,hyperref}
%\modulolinenumbers[5]
%\usepackage{color}
\usepackage[table]{xcolor}
\usepackage{graphicx}
\usepackage{adjustbox}
%\usepackage{graphics}

%\usepackage{deluxetable}
%\usepackage{natbib}
%\bibliographystyle{ieee-alphabetic}

\usepackage[numbers]{natbib}


\usepackage{amssymb}

\usepackage{rotating}

% To compile this file, run "latex thesis", then "biber thesis"
% (or "bibtex thesis", if the output from latex asks for that instead),
% and then "latex thesis" (without the quotes in each case).

% Double spacing, if you want it.  Do not use for the final copy.
% \def\dsp{\def\baselinestretch{2.0}\large\normalsize}
% \dsp

% If the Grad. Division insists that the first paragraph of a section
% be indented (like the others), then include this line:
% \usepackage{indentfirst}




\begin{document}

% Declarations for Front Matter

%\title{Materials in Planet Interiors: From Microscopic to Planetary
%Scales}
\title{Modeling of Planetary Interiors: From Microscopic to Global
Scales}
\author{Sean M Wahl}
\degreesemester{Spring}
\degreeyear{2017}
\degree{Doctor of Philosophy}
\chair{Associate Professor Burkhard Militzer}
\othermembers{ Professor Bruce Buffett, Professor David Romps, Professor Eugene
Chiang}
\numberofmembers{4}
\prevdegrees{B.S. (Massachusetts Institute of Technology) 2011} 
 \field{Earth \& Planetary Science}
\emphasis{Computational Science and Engineering}
\campus{Berkeley}


% For a masters thesis, replace the above \documentclass line with
% \documentclass[masters]{ucbthesis}
% This affects the title and approval pages, which by default calls this
% document a "dissertation", not a "thesis".

\maketitle
% Delete (or comment out) the \approvalpage line for the final version.
\approvalpage
\copyrightpage

% (This file is included by thesis.tex; you do not latex it by itself.)

\begin{abstract}

% The text of the abstract goes here.  If you need to use a \section
% command you will need to use \section*, \subsection*, etc. so that
% you don't get any numbering.  You probably won't be using any of
% these commands in the abstract anyway.

Invasive brag; forbearance.

\end{abstract}


\begin{frontmatter}

\begin{dedication}
\null\vfil
\begin{center}
To my brother, Michael.\\\vspace{12pt}

\begin{quote}
    \textit{Use what talents you possess: the woods would be silent if no
birds sang except those that sang best. - anonymous
}
\end{quote}
\end{center}
\vfil\null
\end{dedication}

% You can delete the \clearpage lines if you don't want these to start on
% separate pages.

\tableofcontents
\clearpage
\listoffigures
\clearpage
\listoftables

\begin{acknowledgements}
First of all, I want to thank my advisor, Prof. Burkhard Militzer, for his
beneficiary advice and continuous support. His expertise in research, programming and
teaching have been invaluable to me. His efforts availability for discussion, attention to
details, and encouragements made it a great experience working with him.
I would next like to thank Prof. Bruce Buffett and Prof. William Hubbard, of the
University of Arizona, who served as secondary advisors, offering a wealth of
knowledge for the various projects I pursued during my time here. 
I would like to thank Jill Banfield for giving me the opportunity and instruction in
teach Mineralogy, the course that was most enlightening to me as an undergrad.

I owe a great deal to my group members and visitors, Shuai Zhang, Hugh Wilson, Kevin
Driver, Fran\c{c}ois Soubiran, Tanis Leonhardi, Josh Tollefson, Felipe Gonzalez and
Shoh Tagawa for the great experience working together, as well as other EPS students,
Ian Rose, Brent Delbridge, Matthew Diamond, David Mangiante, Pamala Kaercher, Qingkai
Kong, Nick Knezek, Marissa Tremblay, Sara Be and postdoc Hiro Matsui, whom I had the
pleasure to work and teach with.
I would also like to recognize all of the past and present EPS graduate student fellows who made
and continue to make the EPS department a wonderful place to work and study.
I would like to give a special thanks to Tyler Arbour, who was a wonderful mentor and
friend within the department, even in the roughest times.

I appreciate the help of the present and past EPS staff, Margie Winn, Crysthel
Catambay, Roxanne Polk, Charley Paffenbarger, Judith Coyote, Nadine Spingola-Hutton,
Allyson Weiyong Tang, Goldie Negelev, Timothy T. Teague, Elana Sabolch, Marion Banks,
Micaelee Ellswythe, Gretchen vonDuering, and Jann Michael Pagdanganan for their help
with all manner of questions regarding working and studying at Berkeley.  I cannot
imagine how my past five years would be without their effort in making EPS run
smoothly.

I am very grateful to the members of of the Juno Science team and Juno Interior
Working group with whom I was able to contribute to a mission of a spacecraft to
another planet, a childhood dream of mine.
I want to thank Prof. Dave Stevenson of Caltech, who provided my first real introduction
to Planetary Science so many years ago, and has continued to be a source of advice
and inspiration.

My musician friends including Laurel North, Ameeta Patel,  John
Slaymaker and the rest of the gang at the Starry Plough, my home away from home, for
always being there to lift my spirits with some tunes.
My best friend, Rena Katz, who made my time living in Berkeley so wonderful.

Last but not the least, I am grateful to my family---my mother, my father, and my
brother. My first teachers and collaborators, to whom I owe my love of science, and
who are a source of unwavering love and support. \\\\

I acknowledge financial support from UC Berkeley Graduate Student Fellowship, NSF,
NASA and CIDER, and computational resources including NAS, NERSC, and UC Berkeley
EPS/Astronomy clusters.

\end{acknowledgements}

\end{frontmatter}

\pagestyle{headings}

% (Optional) \part{First Part}

\chapter{Introduction}\label{chap1}

The materials composing Earth's interior are $\sim$99\% composed of only 8 elements (Fe,
O, Si, Mg, S, Ni, Ca and Al). In Jupiter at least 90\% of the planet is composed
of just hydrogen and helium.  Nonetheless, these materials are the subject of
constant study by experiment and simulation. The extreme high pressures and
temperatures under which these materials exist make them difficult to study.
For the Earth, these conditions extend to $\sim$300 GPa and $\sim$6000 of Kelvin
\citep{alfe2009} (the temperature value remains contentious); inside Jupiter,
the pressures reach in excess of 4 TPa, while the maximum temperatures in the
aftermath of the moon-forming impact would  have been in the tens of thousands
Kelvin. Under such extreme conditions materials can behave in ways that would
not be predicted from their behavior at ambient conditions. Insulating
materials can become metallic at high pressures, and elements that are
nominally excluded can become readily soluble in a material.

An accurate understanding of the properties of planetary materials is essential
in determining their present structure and their evolution over time. Their
properties are inextricably tied to the methods we use to probe the insides of
planets, whether that be through seismology, geodesy, or measurement of their
gravitational or magnetic fields. But understanding their properties alone is
not sufficient. Planetary Science depends on physical models to take these
properties acting on a microscopic material scale and study how they govern
behaviour on planetary length- and time-scales. The results of such models are
often non-unique, due to the relatively small number of direct measurements
that can be made from a planet's surface, and due to the nature of key material
properties such a density  being compatible with a large range of different
compositions. A complete planetary model, therefore, draws on information from a
variety of sources, including cosmochemical inferences and, on Earth, the
record of rock sample exhumed from the planet's interior.

In this thesis, I present a variety of work using computational techniques to
address questions about planetary interiors and evolution, focused on: 1)
determining properties of planetary materials from first-principles physics
simulations, and 2) applying of these derived properties to models of
large-scale planetary structure and processes. In Chapter~\ref{chap2} I discuss
the physics numerical techniques I have used and developed to simulate
planetary materials at high temperatures. Chapters~\ref{chap3} and \ref{chap4}
summarize the results of two studies using these first-principles techniques to
study the dissolution of rock-ice cores in gas giant planets, and the
high-temperature solubility of terrestrial cores and mantles.
Chapter~\ref{chap5} introduces work to build a thermochemical model of iron 
alloys for understanding the generation of magnetic fields in the cores of
Mercury and other small terrestrial bodies. Chapter~\ref{chap6} shows the 
derivation of a non-perturbative method for calculating self-consistent 
gravitational fields of a fluid planet. Chapter~\ref{chap7} uses this method 
to study the tidal response of Jupiter and Saturn. Finally, Chapter~\ref{chap8}
uses the same technique in combination with first-principles equations of state
to interpret the gravitational field measured by the \textit{Juno} spacecraft 
at Jupiter.

\section{Interior structure and formation of the Earth and other planets}

The interior of the Earth, and presumably most other rocky planets and moons,
are differentiated into into a number of layers, as is demonstrated by
seismological and geodynamic studies. It mainly consists of the mantle, which
is likely dominantly of magnesium silicates and oxides, and the core, which is
presumably Fe-Ni alloys with some light elements
\cite{Birch1952,Mcdonough1995}.  The Earth's upper mantle (less than $\sim$7
GPa) and the entire mantles of smaller terrestrial bodies are composed of the
rock peridotite, consisting of the minerals olivine, garnet, and pyroxene. With
increasing pressure these minerals undergo a number of phase transitions
through the transition zone (410-660 km), eventually transforming into the
lower mantle (669-2891 km) composition of bridgemanite (formally Mg-perovskite)
and ferropericlase \cite{Tronnes2010}.  Down to a few hundred kilometers above
the core-mantle boundary ($\sim$136 GPa), there exists the D$^{\prime\prime}$
layer, which is characterized by strong seismic anomalies and lateral
heterogeneities. This layer may involve an additional transition from
bridgemanite to Mg-postperovskite and/or a dense partial melt.  The Earth's
core is divided into a liquid outer core (2891-5150 km, 136-329 GPa),
surrounding a solid inner core (5150-6371 km, 329-364 GPa). Smaller rocky
planets are also expected to have at least a thin liquid outer core, due to the
significant freezing point depression of Fe-S alloys.

The structure of giant-planet interiors is much more limited, but data from
telescope observations and spacecraft missions, as well as cosmochemical
studies based on meteorites and the solar photosphere have aided in
constructing model interiors of Jovian planets \cite{Guillot2004}. Jupiter and
Saturn are composed primarily of hydrogen and helium, but must contain tens of
Earth-masses of heavy elements. The hydrogen-helium mixture transitions from an
insulator to a metal at $\sim$100 GPa. It is not known for sure whether these
planets have compositionally distinct layers, or if the ``rocks'' and ``ices''
exist in solution, and are distributed through the planet (Chapters~\ref{chap3}
and \ref{chap7} deal with this question).  Uranus and Neptune may each have a
small rocky core surrounded by a thick layer of ice-hydrogen-rock mixture, and
another layer near the surface consisting of hydrogen, helium, and ice.  Here
the ``ices'' refer to hydrides of the most abundant light elements (oxygen,
carbon, and nitrogen) that are next to hydrogen, helium and neon, such as
water, methane, and ammonia, which condense from the nebula at relatively low
temperatures. 

Even less is known about the interiors and compositions of exoplanets, the
confirmed number of which has reached 3,586 as of March 2017 \cite{Schneider2011}.
The known exoplanets span a large range of masses from smaller than Earth to
greater than Jupiter.  These detection methods have a selection bias towards
larger planets that are closer in to their stars, thus, many fall into to
category of ``hot Jupiters''. Statistical analyses of the Kepler catalogue,
however, suggest that the most numerous category of planets is that in a
``super-Earth'' or ``sub-Neptune''  mass range. Comparing the planets'
mass-radius data with the equation of state (EOS) of typical planet-forming
materials enables estimating these planets' composition \cite{Seager2007} and gives
some idea of the composition of these planets, although the determination is
non-unique without additional information. The results of the studies in
Chapters~\ref{chap3} and \ref{chap4} have some relevance to exoplanets, as does
Chapter~\ref{chap8} in so far as, many exoplanets are expected to have
analogous interior structures to Jupiter.

The final stages of terrestrial planet formation are expected to include 
a number of energetic collisions between planetesimals and proto-planets. 
These giant impacts have been invoked to explain a number of features of and
differences between planets in the inner solar system. One of the more extreme
examples is the hypothesized moon-forming impact. Recent studies have suggested
even more energetic, high-angular momentum collisions \citep{Cuk2012,Canup2012} than the 
canonical, Mars-sized impactor scenario \citep{canup2004}. A consequence of 
such a giant impact is that at least a portion of the mantle would have been
molten in the aftermath, and may have been heated to temperatures
much higher than those typically considered in studies of Earth materials. 
We pose this question in Chapter~\ref{chap4}.

For the giant planets, there is an ongoing debate regarding the nature of their 
formation: whether a terrestrial-like core must form first for the gas to accrete 
on to, or if they formed directly from the collapse of an instability in the gas. 
The concentration of heavy elements towards the center of the planet is 
indicative of the first scenario, although a comprehensive theory for the 
early evolution of the gas giants is not settled. The work presented in 
Chapters~\ref{chap3} and \ref{chap8} are relevant for this debate.

\section{Methods in high-pressure studies}

Although the studies presented here focus on numerical simulations, the
development of numerical techniques has gone hand in hand with experimental
techniques  for studying materials at high pressure.  There are two main
classes of high-pressure experimental techniques: 1) static compression and 2)
dynamic compression. The first involves applying a constant pressure to a
sample. These techniques include ones based on large-volume press, such as the
multi-anvil apparatus \cite{Ito2007} that can attain pressures of $\sim$30 GPa
(top of the lower mantle) and sometimes $\sim$100 GPa, or diamond anvil cells
\cite{Mao2007}, which reach pressures greater than 360 GPa (center of the
Earth). The second class involves generating and monitoring a shock wave in a
sample. The drivers for dynamic compression experiments include gas
gun\footnote{E.g., Lindhurst Laboratory for Experimental Geophysics at Caltech,
    and more resources listed on
http://mygeologypage.ucdavis.edu/stewart/OLDSITE/ImpactLabs.html.}, intensive
laser\footnote{E.g., National Ignition Facility at Lawrence Livermore National
Laboratory.}, or strong magnetic fields\footnote{E.g., Z Machine at  Sandia
National Laboratories.} \cite{Asimow2015}.

Static compression has the advantage of flexibility in reaching a large range in
desired temperatures and pressures. Large volume presses allow for larger,
centimeter-sized samples, which is helpful for studying equilibrium phase
relations thorough analysis of the recovered samples . Diamond anvil cells are
frequently used for studying isolated materials up to very high pressures,
often with laser-heating to reach the desired temperature. Diamond anvil cell
studies  have benefited greatly from the development of new-generation
synchrotron radiation facilities, which provide high-energy X-ray beams to
determine the structure of $\mu$m-sized samples in situ.

Dynamic compression utilizes shock waves to generate high pressures, which are
necessarily accompanied by simultaneous high temperatures. This technique is
particularly useful for constraining equations of state and detecting phase
transitions under the extreme conditions. During propagation of the shock wave,
the courses of the states of a target mineral are usually along specific paths
that are governed by the EOS. Techniques such as pre-compression and
multi-shock allows reaching states off the principle Hugoniots, and ramp
compression enables high pressure measurements at low temperatures, which can
be useful for studying pressure-driven phase transformations predicted by
ground-state first-principles calculations.

In spite of sweeping advancements in the field, limitations remain for
high-pressure experimental techniques, some of which can be aided by judicious
use of first principles simulations. There are regions of pressure temperature
space that are difficult or impossible to achieve in experiments. Shock
experiments are classically confined to follow the Rankins-Hugoniot curve,
meaning that for a given pressure the temperatures rise much faster than
typical planetary barotropes. Laser heating of diamond anvil cells also has its
limits when it comes to reaching very high temperatures. Thus, there is a range
of conditions in deep planetary interiors, particularly for the giant planets,
that are inaccessible to experiments. There are certain material properties
that are very difficult to measure in situ, and only some questions can be
adequately addressed with recovered samples. In some cases, accuracy of
experimentally determined pressures and densities are insufficient to constrain
equations of state satisfactorily (as is the case for experimental
hydrogen-helium equations of state of interest to the work in
Chapter~\ref{chap8}). In spite of these issues, experiments are ultimately the
way to unambiguously verify theoretical predictions. It is clear that both
fields benefit immensely from contributions from the other.


\chapter{First-Principles simulation Methods}\label{chap2}


In quantum chemistry and condensed matter physics, first-principles methods (often
referred \textit{ab initio} methods) refer to calculations of material properties
using physical models that do not rely on any specific experiment or measured
material property. These calculations, therefore are built up from physical constants
including the speed of light $c$, Planck's constant $h$, as well as the mass  and
charge of constituent electrons ($m$ and $e$) and nuclei ($M_i$ and $Z_i$). Nonetheless,
approximations must be made in order to make solution to the many-body Schr\"odinger
equation tractable.

These methods are of particular use in the study of materials in the deep interiors
of Earth and other planets, because the inherent high pressures and temperatures
pose difficulties for the design and interpretation of experiments. In the case of the
Jovian planets, Jupiter and Saturn, these conditions are so extreme that modern
experimental techniques are unable to recreate the conditions through a large portion
of the planet's interior. In the following chapters I will be addressing questions of
materials in some of these extreme pressure-temperature ranges.

In studies of materials, the most prevalent classes of first-principles methods based
on density functional theory (DFT) or quantum Monte Carlo (QMC). Most of these
techniques are fundamentally zero temperature theories. In this work, as in most
first-principles applications to planetary sciences, I focus on the use of DFT. This
is due largely to the advantage of its better computational efficiency. This
computational efficiency means that DFT can currently be extended to larger atomic
systems and to consider finite temperature more easily than with Monte Carlo based
techniques. The higher precision QMC techniques do have an important place in
determining stability between phases in which energy differences are very small.
Since the focus of the work presented here is on systems involving the exchange of
atoms between phases the energy differences are expected to be large enough for DFT
predictions to be sufficiently accurate.

This chapter introduces the theoretical and computational backgrounds of the density
functional theory molecular dynamics (DFT-MD) technique, which is our workhorse
method for first-principles studies in planetary science. We also describe in depth
the thermodynamic integration method, allows for a calculation of entropy of the
simulated system.


\section{Introduction}
The total energy of quantum system of electrons and nuclei is described by the
Hamiltonian\footnote{Neglecting relativistic, magnetic, and quantum
electrodynamic effects.}
%
 \begin{equation}\label{generalhen}
\mathcal{\hat H}=-\sum_i\frac{\hbar^2}{2m}\nabla_i^2-
\sum_{i,I}\frac{Z_Ie^2}{|\textbf{\textit r}_i-\textbf{\textit R}_I|}+
\sum_{\substack{i,j\\(i\neq j)}}\frac{e^2}{2|\textbf{\textit r}_i-\textbf{\textit r}_j|}-
\sum_I\frac{\hbar^2}{2M_I}\nabla_I^2+
\sum_{\substack{I,J\\(I\neq J)}}\frac{Z_IZ_Je^2}{2|\textbf{\textit R}_I-\textbf{\textit R}_J|}.
\end{equation}
%
Here $\mathbf{r_i}$ describes the configuration of the electrons while $\mathbf{R_I}$ describes the position
of the nuclei.  Because the mass of the nuclei is much greater than that of the
electrons, the kinetic energy of the nuclei can be ignored, and the electrons are
assumed to rearrange instantaneously with any change in the position of the nuclei
(see Chapter 3.1 of \cite{martin-esbook}). This assumption, called the
Born-Oppenheimer or adiabatic approximation, simplifies the general Hamiltonian to
%
\begin{equation}\label{generalhee}
\begin{aligned}
\mathcal{\hat H}&=\hat{T}+V_\text{ext}+V_\text{ee}+V_{II} \\
&=-\sum_i\frac{\hbar^2}{2m}\nabla_i^2-
\sum_{i,I}\frac{Z_Ie^2}{|\textbf{\textit r}_i-\textbf{\textit R}_I|}+
\sum_{\substack{i,j\\(i\neq j)}}\frac{e^2}{2|\textbf{\textit r}_i-\textbf{\textit r}_j|}+
\sum_{\substack{I,J\\(I\neq J)}}\frac{Z_IZ_Je^2}{2|\textbf{\textit R}_I-\textbf{\textit R}_J|},
\end{aligned}
\end{equation}
%
where $\hat{T}, V_\text{ext}, V_\text{ee}$, and $V_{II}$ represent electronic kinetic
energy, electron-nuclei Coulomb absorption\footnote{In general, $V_\text{ext}$ can
also include electric fields and Zeeman terms.}, electron-electron Coulomb repulsion,
and nuclei-nuclei Coulomb repulsion, respectively. $V_{II}$ has an analytical
solution, which allows us to calculate solutions to a many-body equation for the system of electrons 
%
\begin{equation}\label{generaleeq}
\begin{aligned}
\hat{H}\Psi\{\textbf{\textit r}_i\}=\mathcal{E}\Psi\{\textbf{\textit r}_i\}, 
\end{aligned}
\end{equation}
%
for a specified configuration of nuclei $\mathbf{R_I}$ Hamiltonian describing all
electron-electron and electron-nuclei interactions further reduces to
%
\begin{equation}\label{generalhee2}
\begin{aligned}
\hat{H}&=\hat{T}+V_\text{ext}+V_\text{ee} \\
&=-\sum_i\frac{\hbar^2}{2m}\nabla_i^2-
\sum_{i,I}\frac{Z_Ie^2}{|\textbf{\textit r}_i-\textbf{\textit R}_I|}+
\sum_{\substack{i,j\\(i\neq j)}}\frac{e^2}{2|\textbf{\textit r}_i-\textbf{\textit r}_j|}.
\end{aligned}
\end{equation}

\section{Density functional theory}\label{ksdft}

A practical approach to solving the above fully interacting many-body problem
requires some additional assumptions. Of the available approaches to solving
Eq.~\ref{generalhee}, DFT is the most widely used to date. 

The DFT from is based on two theorems proven by Hohenberg and Kohn \cite{Hohenberg1964} in
1964: 1) any property of a system of interacting particles can be determined by the
ground-state density of electrons $n_0(\textbf{\textit r})$; and 2) for any external
potential $V_\text{ext}(\textbf{\textit r})$, a universal energy functional $E[n]$
can be defined whose global minimum value is at $n(\textbf{\textit
r})=n_0(\textbf{\textit r})$. These theorems are formally exact and general, and were
soon made soluble in practice and by the Kohn-Sham approach \cite{Kohn1965} making use
of the following ansatz:
%
\begin{itemize}
\item The exact ground-state density of electrons can be represented by that of an auxiliary system of non-interacting particles.
\item The auxiliary Hamiltonian is chosen to have the usual kinetic operator and an effective local potential $V_\text{eff}^\sigma(\textbf{\textit r})$.
\end{itemize}
This allows solving the many-electron Eq. \ref{generaleeq} by converting it into an independent-particle problem 
\begin{equation}\label{kseq}
\begin{aligned}
\hat{H}_\text{KS}^\sigma\psi_i^\sigma(\textbf{\textit r})=\varepsilon_i^\sigma\psi_i^\sigma(\textbf{\textit r}), 
\end{aligned}
\end{equation}
where $\hat{H}_\text{KS}^\sigma=\hat{T}+V_\text{eff}^\sigma(\textbf r)$ is the spin-dependent single-particle Hamiltonian.

Comparing with the Hohenberg-Kohn energy functional 
\begin{equation}\label{hkeq}
\begin{aligned}
E_\text{HK}
&=T[n]+E_\text{int}[n]+\int{d\textbf{\textit r}}V_\text{ext}(\textbf{\textit r})n(\textbf{\textit r})+E_{II} \\
&=F_\text{HK}[n]+\int{d\textbf{\textit r}}V_\text{ext}(\textbf{\textit r})n(\textbf{\textit r})+E_{II},
\end{aligned}
\end{equation}
in the method of Kohn-Sham the ground-state energy
\begin{equation}\label{kse}
\begin{aligned}
E_\text{KS}&=T_s[n]+E_\text{Hartree}[n]+\int d\textbf{\textit r}V_\text{ext}(\textbf{\textit r})n(\textbf{\textit r})+E_{II}+E_\text{xc}[n],
\end{aligned}
\end{equation}
where 
\begin{equation}\label{numberdensity}
\begin{aligned}
n(\textbf{\textit r})=\sum_\sigma\sum_{i=1}^{N^\sigma}|\psi_i^\sigma(\textbf{\textit r})|^2
\end{aligned}
\end{equation}
is the density of electrons satisfying $\int n(\textbf{\textit r})d(\textbf{\textit r})=N$ (total number of electrons),
\begin{equation}\label{spTs}
\begin{aligned}
T_s=-\frac{\hbar^2}{2m}\sum_\sigma\sum_{i=1}^{N^\sigma}\langle\psi_i^\sigma|\nabla^2|\psi_i^\sigma\rangle=-\frac{\hbar^2}{2m}\sum_\sigma\sum_{i=1}^{N^\sigma}|\nabla\psi_i^\sigma|^2
\end{aligned}
\end{equation}
is the kinetic energy, and 
\begin{equation}\label{spEhatr}
\begin{aligned}
E_\text{Hartree}[n]=\frac{1}{2}\int{d\textbf{\textit r}d\textbf{\textit r}'}\frac{n(\textbf{\textit r})n(\textbf{\textit r}')}{|\textbf{\textit r}-\textbf{\textit r}'|}
\end{aligned}
\end{equation}
is the classical self (Coulomb)-interaction energy, of the independent-particle system. All many-body effects are grouped into the exchange-correlation energy 
\begin{equation}\label{spExc}
\begin{aligned}
E_\text{xc}[n]&=F_\text{HK}[n]-(T_s[n]+E_\text{Hartree}[n]) \\
&=T[n]-T_s[n]+E_\text{int}[n]-E_\text{Hartree}[n].
\end{aligned}
\end{equation}
%
The exact ground-state energy and density of electrons
can be obtained by solving Eqs. \ref{kseq} and \ref{numberdensity} in a
self-consistent iterative way.

The Kohn-Sham approach provides a feasible way of determining the exact ground-state
properties of many-electron systems: given a known $E_\text{xc}[n]$. In principle the
functional form of the exchange-correlation is not known. This leads to the main
assumption of the DFT method. The success of the DFT method relies on the fact that
it is often reasonable to approximate $E_\text{xc}[n]$ as a local or nearly local
functional of the density. 

\subsection{Computational considerations}


\subsubsection{Exchange correlation}

There have been continuing efforts in the aim of exploring practical ways to improve
the approximate the exchange-correlation functional to better match. The simplest
choice is the local density approximation (LSDA, or simply LDA) is derived from the
homogeneous electron gas. It continues to be a popular choice due to it's simplicity,
freedom from required fit parameters, and its relative success in describing many
real materials.

In LDA, the exchange-correlation energy 
\begin{equation}\label{spExclda}
\begin{aligned}
E_\text{xc}[n]
=\int d\textbf{\textit r} n(\textbf{\textit r})\epsilon_\text{xc}([n],\textbf{\textit r})
=\int d\textbf{\textit r} n(\textbf{\textit r})[\epsilon_\text{x}([n],\textbf{\textit r})+\epsilon_\text{c}([n],\textbf{\textit r})].
\end{aligned}
\end{equation}
The exchange energy density (in atomic units) follows the exact expression \cite{martin-esbook}
\begin{equation}\label{ldaxs1}
\begin{aligned}
\epsilon_\text{x}^\sigma=-\frac{3}{4}\left(\frac{6}{\pi}n^\sigma\right)^{1/3}.
\end{aligned}
\end{equation}
In spin-unpolarized systems, $n^\uparrow=n^\downarrow=n/2$, so
\begin{equation}\label{ldaxs0}
\begin{aligned}
\epsilon_\text{x}^\uparrow=\epsilon_\text{x}^\downarrow=\epsilon_\text{x}
=-\frac{3}{4}\left(\frac{3}{\pi}n\right)^{1/3}
=-\frac{3}{4\pi}\left(\frac{9\pi}{4}\right)^{1/3}\frac{1}{r_s},
\end{aligned}
\end{equation}
where $r_s$ characterizes the density of electrons via $1/n=4\pi r_s^3/3$; 
while in partially polarized cases,
\begin{equation}\label{ldaxs2}
\begin{aligned}
\epsilon_\text{x}(n,\zeta)=\epsilon_\text{x}(n,0)+[\epsilon_\text{x}(n,1)-\epsilon_\text{x}(n,0)]f_\text{x}(\zeta),
\end{aligned}
\end{equation}
where $f_\text{x}(\zeta)=[(1+\zeta)^{4/3}+(1-\zeta)^{4/3}-2]/[2(2^{1/3}-1)]$,
$\zeta=(n^\uparrow-n^\downarrow)/n$, and $n=n^\uparrow+n^\downarrow$. For the
correlation energy density, the widely used expression is based on parameterization
\cite{Perdew1981} of accurate quantum Monte Carlo simulations of homogeneous electron gas \cite{Ceperley1980}
\begin{equation}\label{ldaccapz}
\begin{aligned}
\epsilon_\text{c}(r_s)=
\begin{cases}
-0.0480 + 0.031 \ln r_s - 0.0116r_s + 0.0020 r_s\ln r_s & \quad r_s < 1, \\
-0.1423/(1+1.0529\sqrt{r_s}+0.3334r_s) & \quad r_s > 1.
\end{cases}
\end{aligned}
\end{equation}

The success of LDA has also prompted extensive work on designing new functionals. For
example, by considering the non-uniform nature of electron distribution, several
schemes of generalized gradient approximation (GGA), in which the
exchange-correlation functional is a function of both the density of electrons $n$
and its gradient $\nabla n$, have been developed. In recent years one of the most
popular exchange-correlation functionals is a GGA-class functional developed by
Perdew, Burke and Ernzerhof \citep{Perdew1996} (PBE). Most of the DFT simulations
presented in this work were performed using the PBE exchange correlation functional .

There has also been active research (Chapter 5 of \cite{martin-esbook}) on other
rungs of the Jacob's Ladder, such as meta-GGA, hybrid functionals, etc., toward
higher levels of chemical accuracy, at the cost of increased computation time. There
have also been more targeted attempts to solve material specific problems, for
example the introduction of long-range Van der Waals forces using the VdW potentials
\citep{Raymond2015}.


\subsubsection{Self consistent iteration}
Numerically solving the Kohn-Sham equation includes an initial guess of the density,
and an iteration over $n^\text{in}\rightarrow V^\text{in}\rightarrow n^\text{out}$.
In the Kohn-Sham energy functional $E_\text{KS}=T_s[n]+E_\text{pot}[n]$, the kinetic
energy can be expressed as
%
\begin{equation}
\begin{aligned}
T_s[n] & =E_s-\sum_\sigma\int \mathrm{d}\textbf{\textit{r}} V^{\sigma,\text{in}}(\textbf{\textit{r}})n^\text{out}(\textbf{\textit{r}},\sigma) \\
 & =\sum_\sigma\sum_{i=1}^{N^\sigma}\varepsilon_i^\sigma-\sum_\sigma\int \mathrm{d}\textbf{\textit{r}} V^{\sigma,\text{in}}(\textbf{\textit{r}})n^\text{out}(\textbf{\textit{r}},\sigma) \\
& \approx E_s[V_{n^\text{in}}]-\sum_\sigma\int \mathrm{d}\textbf{\textit{r}} V^{\sigma}_{n^\text{in}}(\textbf{\textit{r}})n^\text{in}(\textbf{\textit{r}},\sigma), \label{eqtshwf}
\end{aligned}
\end{equation}
and the potential energy
\begin{equation}
E_\text{pot}[n]=\int \mathrm{d}\textbf{\textit{r}} V_\text{ext}(\textbf{\textit{r}})n(\textbf{\textit{r}})+E_\text{Hartree}[n]+E_{II}+E_{xc}[n]\approx E_\text{pot}[n^\text{in}]. \label{eqpot}
\end{equation}
These allow accurate approximation of the true Kohn-Sham energy with 
\begin{equation}
 E_\text{KS} \approx E_s[V_{n^\text{in}}]-\sum_\sigma\int \mathrm{d}\textbf{\textit{r}} V^{\sigma}_{n^\text{in}}(\textbf{\textit{r}})n^\text{in}(\textbf{\textit{r}},\sigma)+E_\text{pot}[n^\text{in}] \label{eqksapprox},
\end{equation}
for densities near the correct solution. Equation \ref{eqksapprox} is now standard at each step of the self-consistent iteration in solving Kohn-Sham equations (see Chapter 9.2 of \cite{martin-esbook}).

\subsection{Basis sets and pseudopotential}
There are different methods for solving the Kohn-Sham equations. Typically one
chooses a basis set to expand the orbitals, according to the nature of the system.
Plane wave basis, Gaussian basis, and Slater-type orbital basis are often used.  For
electronic structures condensed systems with periodic boundary conditions, such as
those presented here, plane waves is the  natural choice.  They make form a complete,
general basis that allows for easy convergence.

Another noteworthy concept in DFT is the pseudopotential, which is  used in most
materials simulations considering elements with Z$>$2. The idea of pseudopotentials
is to use an effective ionic potential to replace the combined Coulomb potential of
the nucleus and electrons on the valence electrons, whose effect is assumed to be
nearly identical regardless of ionic configuration. The use of pseudopotentials
greatly reduces the size of the basis, which otherwise has to be large to describe
the non-smooth electronic states near the nucleus. Particular attentions has to be
paid to the pseudopotential in studies at extremely high pressures, because a
pseudopotential can be invalidated by overlapping of the core states of nearby atoms.

%The expression of Hamiltonian in Eq. \ref{generalhee} is still valid for valence electrons when using a pseudopential\footnote{Except for ``non-local'' pseudopotentials.} to describe the ionic part.

All DFT simulations presented here were performed using the implementation of the DFT
formalism the Vienna {\it ab initio} simulation package (VASP) \citep{Kresse1996}.
VASP uses projector augmented wave pseudopotentials \citep{Blochl1994} 

\section{Finite Temperature Calculations}

Real materials exist at finite temperatures. This imposes a number of additional
problems for first-principles simulations that are not directly addressed by the 
standard DFT methods. One of the most fundamental consequences is that the time
averaged properties of the material do not correspond to a single, fixed ionic
configuration. Rather, they must be obtained from an ensemble of different
configurations, which must be weighted using principles from statistical mechanics.
In most experimental and ``real world'' applications the ensemble of interest is a
$NPT$ ensemble (one with number of atoms, pressure and temperature). For
first-principles calculations, however, it is typically much easier to consider a
$NVE$ or Microcononical ensemble (with fixed volume and total energy).

Low to intermediate temperatures can be treated as a perturbation to the ground state
(see some discussion in Section 3.1 of \cite{martin-esbook}), leading to
quasi-harmonic approximation (QHA). QHA typically works for solids at temperatures
that are well below their melting temperature, where anharmonic effects are relatively
small. For this reason, QHA can be used for applications in the solid portions of
planets, but is generally insufficient for materials near or above their melting
temperature. For this reason the work presented here focuses on using
first-principles molecular dynamics (FPMD or DFT-MD).

\subsection{Molecular Dynamics}

Molecular dynamics (MD) is a means of sampling different configurations that are
generated through tracking the realistic motions of nuclei, tracking their changing
positions and velocities over time on pico-second timescales. Since these simulations
mimic real processes it allows one to directly observe some properties, such as
diffusion.

At it's heart an MD simulation is just an extension of Newton's law. Considering the
simple case of pair potentials, 
%
\begin{equation}
    V(R) = \sum_{i>j} V({\bf r}_i,{\bf r_j}),
\end{equation}
%
the total force acting on the $i$th atom is
%
\begin{equation}
    {\bf F}_i = m_i{\bf a}_i = - \frac{\partial V}{\partial {\bf r}_i}.
\end{equation}
%
The change of velocity then follows as 
%
\begin{equation}
    \frac{\partial {\bf v}_i}{\partial t} = \frac{ {\bf F}_i}{m_i}
\end{equation}
%
and the change in position as 
%
\begin{equation}
    \frac{\partial {\bf r}_i}{\partial t} = {\bf v}_i.
\end{equation}
%

Molecular dynamics implementations typically use the Verlet algorithm
\citep{Verlet1967} which provides good numerical stability, as well as desirable
physical properties such as time-reversibility. Because the primary goal of MD is in
generating sample configurations higher order integrators are typically not
necessary. The positions new position of an atom  ${\bf r}_{n+1}$ is updated based on it's
previous two positions
%
\begin{equation}
    {\bf r}_{n+1} = 2{\bf r}_{n} - {\bf r}_{n-1} + \left(\frac{ {\bf F}_n }{m}
    \right) \Delta t^2 + O(\Delta t^4)
\end{equation}
%
and the velocity
%
\begin{equation}
    {\bf v}_{n} = \frac{{\bf r}_{n+1} - {\bf r}_{n-1}}{2\Delta t}  + O(\Delta t^2).
\end{equation}
%
The time averaged material properties can then be determined from an
average of the those calculated for each configuration. In the Microcanonical
ensemble the total energy of simulation is constant, but the kinetic and potential
energy, $K$ and $V$ where $E=K+V$ fluctuate over time. The time averaged kinetic
energy is found as 
%
\begin{equation}
    \langle K \rangle = \sum_i \frac{1}{2}m_i\langle  v_i^2 \rangle =
    \frac{1}{2}Nk_BT.
\end{equation}

While $T$ is free to fluctuate between time steps, it is necessary to maintain the
time-averaged temperature over the course of the simulation. This is done through the
use of the Nos\'e-Hoover thermostat \citep{Nose1984}, which maintains $\langle T \rangle$ by
considering heat transfer between the real system and imaginary degrees of freedom.
The important features of this thermostat is that it is proven to obey the
Microcanonical ensemble, thus preserving the ability to use the configurations to
sample the desired properties.

\subsection{Classical Monte Carlo}

Besides MD, the other technique for sampling the Microcanonical ensemble is using
Monte Carlo (MC) techniques. The technique generates a Markov chain of
configurations (${\bf r}_1$,${\bf r}_2$,${\bf r}_3$\ldots) using the Metropolis
algorithm \citep{Metropolis1953}, and then averaging the property of interest over
those configurations. The Metropolis algorithm is as follows:
%
\begin{enumerate}
    \item Start from configuration $R_{\rm old}$.
    \item Propose a random move of an atom within a surrounding box, $R_{\rm old} \to
        R_{\rm new}$.
    \item Compute energies $E_{\rm old}=V(R_{\rm old})$ and $E_{\rm new}=V(R_{\rm
        new})$.
    \item If $E_{\rm new}<E_{\rm old}$ accept the new configuration.
    \item If $E_{\rm new}>E_{\rm old}$ check whether to accept the new configuration
        with probability $A$.
\end{enumerate}
%
where the probably acceptance in the ``up-hill'' case is
%
\begin{equation}
    A(R_{\rm old}\to R_{\rm new}) = \exp\left[-\frac{V(R_{\rm new} - V(R_{\rm old})}
    {k_B T} \right].
\end{equation}
%
In practice, a good rule of thumb is to choose the size of the box for moving the
atoms such that the acceptance rate is near 50\%.  The Boltzmann factor is thus
absorbed into the chain of generated configurations, and properties can be estimated
as their simple average over all the configurations.

There are also a number of different kinds of quantum Monte Carlo (QMC) techniques that
are used extensively in condensed matter physics. These are , however, generally less
computationally efficient than DFT, and are not presented in this work. Classical
Monte Carlo (CMC) using simple pair potentials $V_{ij}=V({\bf r}_i,{\bf r}_j)$ did
play an important role in our implementation of the thermodynamic integration
technique outlined in the subsequent section.

\section{Thermodynamic Integration}

One of the major shortcomings of standard DFT-MD calculations from the viewpoint of
many planetary science problems is the inability to calculate the entropy $S$ of a
simulation. This quantity is of particular interest for a number of planetary
problems because it is needed in order to calculate and compare the Gibb's free
energy $G=E+PV-TS$. While one can identify the pressure of phase transitions at $T=0$
by directly comparing the Helmholtz free energy $H=E+PV$ from DFT, mapping out these
transitions up to thousands of K requires comparing $G$ between different phases.
The technique is essential when considering transitions that involve liquid phases
where use of the QHA is not possible. Calculating the entropy is also important for
predictions of the temperature structure in the deep interiors of planets. In the
simplest case of a vigorously convecting fluid layer, the $T$-$P$ path of constant
entropy is a good approximation

In addition to phase transitions within a
composition, a determination of the entropy also allows us to begin addressing
questions of simple chemical and compositional problems. For instance in Chapter 4 I
present results on the high pressure solubility of iron and MgO, following the
reaction $\rm{MgO}_{\rm sol/liq} + \rm{Fe}_{\rm liq} \Rightarrow \rm{FeMgO}_{\rm
liq}$. This is achieved by running simulations of three separate phases and comparing
the change Gibb's free energy
%
\begin{equation} \label{eqn:gibbs1}
% \frac{ \Delta G }{\rm{ FeMgO \; fun.}} =  \frac{1}{24}G_{\rm{(FeMgO)_{24}}} 
  \Delta G_{\rm{\rm mix}}  =  \frac{1}{24}G_{\rm{(FeMgO)_{24}}} 
- \frac{1}{32} \left[  G_{\rm{(MgO)_{32}}}  + G_{\rm{Fe_{32}}} \right]
\end{equation}
%
associated with the reaction, where $G_{\rm{(MgO)_{32}}}$ and   $G_{\rm{Fe_{32}}}$
are the Gibbs free energies of a pure MgO and iron endmembers with subscripts
referring to the number of atoms in the periodic simulation cell. A representative
snapshot from a MD simulation sell of ${\rm(FeMgO)}_{24}$ is shown in
Fig.~\ref{femgosnap}.

 \begin{figure}[h!] %  figure placement: here, top, bottom, or page
   \centering
   \includegraphics[width=22pc]{figs/mgfeo_vest.png} 
\caption{Snapshot of an MD-DFT simulation of liquid Fe + MgO
    mixture.\label{femgosnap}}
\end{figure}

Thermodynamic integration (TDI) is a technique that considers a fictitious
transformation of one atomic system into another. For instance, one can consider a
system where you slowly change the interaction between atoms, running separate
simulations for each step in the transformation. In our case we calculate an absolute
$S$ by considering the transformation of the DFT system into an
analogous classical system with the same number of atoms, having an analytic
expression for $S$. Fig~\ref{tdi} shows a schematic representation of this
transition. In principle this can also be done between two DFT systems, for
instance transforming some subset of atoms from one element to another 

 \begin{figure}[h!] %  figure placement: here, top, bottom, or page
   \centering
   \includegraphics[width=30pc]{figs/TD_integration.eps} 
\caption{Schematic illustration of a thermodynamic integration between a simulation 
    with DFT interactions and one with classical interactions. \label{tdi}}
\end{figure}

\subsection{Computation of Gibbs Free Energies}

%%%% Discussion from FeMgO paper

The Gibbs free energy of a material includes a contribution from entropy of the system.
Since entropy is not determined in the standard DFT-MD formalism, we adopt a two step
thermodynamic integration method, used in previous studies
\citep{Wilson2010,Wilson2012a,Wahl2013,Gonzalez2014}.  The thermodynamic integration technique
considers the change in Helmholtz free energy for a transformation between two systems
with governing potentials $U_a\left(\mathbf{r_i}\right)$ and
$U_b\left(\mathbf{r_i}\right)$. We define a hybrid potential
$U_{\lambda}=\left(1-\lambda\right)U_a+\lambda U_b$, where $\lambda$ is the fraction of
the potential $U_b\left(\mathbf{r_i}\right)$. The difference in Helmholtz free energy is
then given by
\begin{equation} \label{eqn:td_int}
  \Delta F_{a\to b}\equiv F_b - F_a = \int_{0}^{1}{d\lambda\,\langle U_b\left(\mathbf{r_i}\right) -
  U_a\left(\mathbf{r_i}\right) \rangle_{\lambda}}
\end{equation}
where the bracketed expression represents the ensemble-average over configurations,
$\mathbf{r_i}$, generated in simulations with the hybrid potential at constant volume and
temperature. This technique allows for direct comparisons of the Helmholtz free energy of
DFT phases, $F_{\rm DFT}$, by finding their differences from reference systems with a known
analytic expression, $F_{\rm an}$. 

In practice, it is more computationally efficient to perform the calculation $\Delta
F_{\rm an\to DFT}$ in two steps, each involving an integral of the form of Eqn.
\ref{eqn:td_int}. We introduce an intermediate system governed by classical pair
potentials, $U_{\rm cl}$, found by fitting forces to the DFT trajectories
\citep{Wilson2010,Izvekov2004}. For each pair of elements, find the average force in bins
of radial separation and fit the shape of a potential using a cubic spline function. We
constrain the potential to smoothly approach zero at large radii and use a linear
extrapolation at small radii, where the molecular dynamics simulations provide
insufficient statistics. Examples of these potentials are included in the online
supplementary information. The full energetics of the system is then described as
%$F_{\mathrm{DFT}}=F_{\mathrm{\mathrm{an}}}+\Delta F_{\mathrm{cl}\to \mathrm{DFT}}+\Delta F_{\mathrm{an} \to \mathrm{cl}}$,
\begin{equation} \label{eqn:two_step}
F_{\mathrm{DFT}}=F_{\mathrm{\mathrm{an}}}+\Delta
F_{\mathrm{an} \to \mathrm{cl}}+\Delta F_{\mathrm{cl}\to \mathrm{DFT}}
\end{equation}
where $\Delta F_{\mathrm{cl}\to \mathrm{DFT}}$ requires a small number of DFT-MD
simulations, and $\Delta F_{\mathrm{an} \to \mathrm{cl}}$ numerous, but inexpensive
classical Monte Carlo (CMC) simulations. The method depends on a smooth integration
of $\Delta F_{\mathrm{cl}\to \mathrm{DFT}}$ and avoiding any first order phase
transitions with $\lambda$.  We use five $\lambda$ points, for all $\Delta
F_{\mathrm{cl}\to \mathrm{DFT}}$ integrations. For solids we use a combination of
classical pair and one-body harmonic oscillator potentials for $U_{\rm cl}$
\citep{Wilson2012a,Wahl2013}.  Liquids we use only pair potentials. For solids the
analytical reference system is an Einstein solid with atoms in harmonic potentials
centered on a perfect lattice, while a gas of non-interacting particles is used for
liquids. We found integrating over 5 lambda points to be sufficiently accurate in
most cases, with an increase to 9 lambda points changing our results by $<0.003$ eV
per formula unit in the FeMgO study. In some cases as few as 3 lambda points are
sufficient for the transformation from DFT to pair potentials. 

\section{Equations of State}


\section{Tests of the Thermodynamic Integration Method}

Here we present to tests validating the TDI method for use in multi component systems
using different types of pair potentials. These were performed in conjunction with
the Fe/MgO solubility study presented in Chapter 4.

\subsection{Comparison of thermodynamic integration with different classical potentials}

The classical pair potentials are derived by fitting the forces and
positions along a pre-computed DFT-MD trajectory. The potentials are
constructed to approach zero for large separations. For small separations
where the trajectories provide no information, linear extrapolation is
used, which means our pair potentials are finite at the origin. All of the
results presented in the paper used this fitting procedure.
Figure~\ref{fig:potentials} shows an example for the pair potentials for
liquid MgO and 50~GPa and 6000~K.  While the Mg-Mg and O-O potentials are
purely repulsive, the deep minimum in the Mg-O potential represents the
attractive forces between ions of opposite charge.

\begin{table}[!h]
    \centering
\caption{Comparison of integration paths using different classical potentials.\label{tab:compare_pots}}
\begin{adjustbox}{max width=\textwidth}
\begin{tabular}{clllll}
\hline
Potentials & $F_{\rm an}$ & $F_{\rm an \to cl}$ & $F_{\rm cl}$ & $F_{\rm cl \to DFT}$ & $F_{\rm DFT}$ \\
\hline
Regular, bonding potentials & $-$427.151 & 27.962 & $-$399.189 & $-$233.825 $\pm$ 0.031 & $-$633.014  $\pm$ 0.031 \\
Non-bonding potentials & $-$427.151 & 204.129 & $-$223.022 & $-$409.927 $\pm$ 0.080 & $-$632.950 $\pm$ 0.080 \\
\hline
\end{tabular}
\end{adjustbox}
\end{table}

Table~\ref{tab:compare_pots} provide all terms of the thermodynamic integration
procedure. In order to test how robust our approach is, we constructed a different
set of pair potential where we eliminated all bonding forces. The values of these
non-bonding potentials are constrained to be positive, and asymptote to zero without
a minimum. Obviously they are a poor representation of the DFT forces in the system
and therefore the free energy differences between the DFT and the classical system,
$F_{\rm cl \to DFT}$, given in table \ref{tab:compare_pots}, is much larger than for
our regular potentials.  However, when the values for $F_{an}$, $F_{an \to cl}$, and
$F_{\rm cl \to DFT}$ are added, we recover the results for $F_{DFT}$ within the
1$\sigma$ error bars. This demonstrates that out free energy calculations are not
sensitive to the details how we construct our classical potentials.

\begin{figure}[h!]  
    \centering
    \includegraphics[width=30pc]{figs/potentials_both.pdf}
\caption{Example pair potentials for liquid MgO at 50 GPa and 6000 K. Left: 
  Regular pair potentials fit to DFT-MD simulations, with a linear
  extrapolation at small separation and an asymptote to 0 at large
  separation.  All of the results presented in the paper used this fitting
  procedure. Right: Non-bonding potential fit with the same procedure, but
  constraining values to be positive. Included for comparison with the pair
  potentials in table~\ref{tab:compare_pots}.}
\label{fig:potentials}
\end{figure}


Although the correct final result is found when using an unrealistic
potential, the efficiency for the thermodynamic integration is the highest
when the classical forces best match the DFT forces.
Figure~\ref{fig:integrate_dft} shows the calculated values of $\left<
V_{DFT} - V_{\rm cl} \right>$ as a function of $\lambda$ using the regular
``bonding'' potential, and for the ``non-bonding potential''. Here each
plotted value of lambda represents and independent DFT-MD simulation with
using that fraction of DFT forces, along with the complementary fraction of
classical forces. The integral of this function give the Helmholtz free
energy $F_{\rm cl \to DFT}$. In the first case , the function $\left<
V_{DFT}-V_{\rm cl}\right>$ in figure~\ref{fig:integrate_dft} depends weakly
on $\lambda$. The function is almost linear and the differnce between
values at $\lambda=0$ and 1 is small. When both criteria are satisfied,
very few $\lambda$ points are needed to evaluate the integral. The
simulations with non-bonding potentials experience larger fluctuations  due
to the greater mismatch in forces, leading to a larger statistical
uncertainty.  As a result, these simulations required longer simulation
times to match the results found with bonding potentials.

Figure~\ref{fig:integrate_cmc} shows $\left< V_{\rm cl} \right>$ as a function
of $\lambda$ from classical monte carlo simulations. The integration of
$F_{an \to cl}$function becomes strongly non-linear as $\lambda$
approaches zero, the non-interacting case. Since classical simulations are
approximately 10$^5$ times more efficient, it is possible to obtain very
close sampling of the a cusp in the integrated function. 

\begin{figure}[h!]  
  \centering
    \includegraphics[width=22pc]{figs/dV_dft_both.pdf}
\caption{The integration path to find $F_{cl \to DFT}$ potentials for MgO
at 50 GPa 6000 K, using the regular, bonding pair potentials (upper) and
the non-bonding pair potentials (lower). }
\label{fig:integrate_dft}
\end{figure}

\begin{figure}[h!]  
  \centering
  \includegraphics[width=30pc]{figs/dV_cmc_both.pdf}
\caption{The integration path to find $F_{an \to cl}$ potentials for MgO at 50 GPa 6000
K, using the regular, bonding pair potentials (left) and
the non-bonding pair potentials (right). These are plotted against the
integration parameter, $\lambda$, to the $1/4$ power.}
\label{fig:integrate_cmc}
\end{figure}

Using the definition for the ensemble averaged potential at a given
$\lambda$
\begin{eqnarray}
  \left< V_{\rm cl} \right>_{\lambda} = \frac{ \int \! d{\bf r} \, V_{\rm cl}({\bf r}) 
  e^{ -\beta \lambda V_{\rm cl}({\bf r}) } }
  { Z}, \label{eqn:average}
\end{eqnarray}
where $Z$ is the partition function
\begin{eqnarray}
Z =  \int \! d{\bf r} \, e^{ -\beta \lambda V_{\rm cl}({\bf r}) },
\end{eqnarray}
we find the following expression as $\lambda \to 0$
\begin{eqnarray}
 \left< V_{\rm cl} \right>_{\lambda \to 0} &=& \frac{ \int \! d{\bf r} \, V_{\rm
 cl}}
 {  \int \! d{\bf r} \, 1} \nonumber \\
 &=& \frac{1}{V}  \int \! dr \, r^2 V_{\rm cl}(r) \label{eqn:limit}
\end{eqnarray}
Then from the deriviative of $\left< V_{\rm cl} \right>_{\lambda}$ in
equation \ref{eqn:average} and \ref{eqn:limit}
\begin{eqnarray}
  \frac{d \left< V_{\rm cl} \right>}{d \lambda}
   &=& \frac{1}{Z^2} 
  \left[ Z \int \! d{\bf r} \, \left(-\beta \right)V_{\rm cl}^2({\bf r}) 
    e^{ -\beta \lambda V_{\rm cl}({\bf r}) } 
   - \left(-\beta \right)\left\{  \int \! d{\bf r} \, V_{\rm cl}({\bf r}) e^{ -\beta \lambda
     V_{\rm cl}({\bf r}) } \right\}^2
    \right] \nonumber \\
   &=& (-\beta) \left[ \left< V_{\rm cl}^2 \right> - \left< V_{\rm
   cl}\right>^2 \right] \nonumber \\
   \left. \frac{d \left< V_{\rm cl} \right>}{d \lambda}\right|_{\lambda \to 0} 
   &=& (-\beta) \left[  \frac{1}{V}  \int \! dr \, r^2 V^2_{\rm cl}(r) -
     \left\{ \frac{1}{V}  \int \! dr \, r^2 V_{\rm cl}(r) \right\}^2 \right]
  \end{eqnarray}
This give us the slope and intercept for the integration at $\lambda=0$,
necessary to correctly integrate the cusp.  Becauase of the extreme
difference in computational efficiency, it is always best to adjust the
classical potential to match the DFT forces.


\subsection{Verification of thermodynamic integration in multicomponent systems}

The second test is to verify that, in a multi-component system, the
integration path does not effect the results. An integration path needs to
be constructed that connects a system with Mg-Mg, Mg-O, and O-O pair
potentials with an non-interacting system. In our standard procedure we
turn on all pair potentials simultaneously by changing
$\lambda_1=\lambda_2=\lambda_3$ from 0 to 1.
%
\begin{equation}
      V_{\lambda_1\lambda_2\lambda_3} = \lambda_1 V_{\rm Mg-Mg} + \lambda_2 V_{\rm
      Mg-O} + \lambda_3 V_{\rm O-O} 
\end{equation}
%
 However, as we will now demonstrate, alternative integration paths will give the
 same results. We compare different integration paths to calculate the free energy of
 the classical system, $F_{\rm cl}$, for both the regular and non-bonding potentials.
 This is comparison can not be made directly for $F_{\rm cl \to DFT}$ because
 tracking the interaction of different species separately is not possible in a
 Kohn-Sham formulation.

 \begin{table}[!h]
     \centering
\caption{Comparison of different integration paths using classical potentials. 
  \label{tab:compare_int_path}}

     \begin{tabular}{clllllll}
         \hline
{Potential} & 
            {$F_{\rm an}$} & 
            {$F_{\rm step}$} & 
            {$F_{\rm an} + \sum F_{\rm step}$} \\
         \hline
Non-bonding & $-$427.151 & $F_{000 \to 111}$ = 204.196(27)      & $-$222.955(27)
\\[1mm]\hline\\[-2.5mm]
Non-bonding & $-$427.151 & $F_{000 \to 010}$ = $\,\:$16.328(6)  & $-$222.953(45)\\
           &          & $F_{010 \to 111}$ = 187.870(39) 
\\[1mm]\hline\\[-2.5mm]
Non-bonding & $-$427.151 & $F_{000 \to 101}$ = 166.363(15)      & $-$222.962(21)\\
           &          & $F_{101 \to 111}$ = $\,\:$37.826(18)
\\[1mm]\hline\\[-2.5mm]
Non-bonding & $-$427.151 & $F_{000 \to 100}$ = 104.323(11)      & $-$222.954(52)\\
           &          & $F_{100 \to 110}$ = $\,\:$31.225(10) \\
           &          & $F_{110 \to 111}$ = $\,\:$68.650(31)
\\[1mm]\hline\\[-2.5mm]
Regular     & $-$427.151 & $F_{000 \to 111}$ = $\,\:$28.028(31) & $-$399.123(31) 
\\[1mm]\hline\\[-2.5mm]
Regular     & $-$427.151 & $F_{000 \to 101}$ = ~~182.820(16)      & $-$399.123(37) \\
           &          & $F_{101 \to 111}$ = $-$154.792(21) \\
\hline
%%\tablecomments{Table \ref{data} is published in its entirety in the 
%%electronic edition of the {\it Astrophysical Journal}.  A portion is 
%%shown here for guidance regarding its form and content.}
%\end{deluxetable}
\end{tabular}
\end{table}

In line 2 of table~\ref{tab:compare_int_path}, we turn on the Mg-O
potential in the first integration step ($F_{000 \to 010}$) and then switch
on the Mg-Mg and O-O potentials in the second and final integration step
($F_{101 \to 111}$). The indices refer to the three $\lambda$ values for
Mg-Mg, Mg-O, and O-O potentials, respectively. In line 3 of
table~\ref{tab:compare_int_path}, we interchange both integration steps. In
line 4, we performed three integration steps turning on the one potential
after the other. In the last column we compare the classical free energies
after adding the results from every integration step to $F_{\rm an}$. The
results agree within the statistical uncertainties demonstrating that the
same classical free energies can be obtained for four different integration
paths using non-bonding potentials.

In table~\ref{tab:compare_int_path}, we also show the results for two
integration paths using regular, bonding potentials. We find consistent
results when we either turn on all potentials simultaneously and when we
switch on the Mg-Mg and O-O potentials in the first step and the Mg-O
potential in the second. We were not able, however, to turn on the
attractive Mg-O alone because the system becomes unstable due to the
imbalance between attractive and missing repulsive forces. This is similar
to what happens in the case of a first-order phase transition, which over
which thermodynamic integrations are also invalid. Nevertheless, this test
demonstrates that different integration paths give consistent results also
for the systems with attractive forces when care is taken to taken to avoid
instabilities.

\chapter{Dissolution of Giant Planet Cores}\label{chap5}

\section{Motivation}

Despite recent advances in computational methods improving understanding of
the hydrogen-helium dominated outer layers
\citep{mcmahon12,french12,militzer13,wilson10}, knowledge of the deep interior structure
of giant planets is limited.
Determining the size of a dense central cores in a giant planet is dependent upon
the model and equation of state used.
Current observational evidence yields recent estimates for present day core
mass of
$\sim$0$-$10 \citep{guillot05} and $\sim$14$-$18 \citep{militzer08} Earth masses for Jupiter, and
$\sim$9$-$22 \citep{guillot05} Earth masses for Saturn.
The Juno spacecraft, en route to Jupiter, will improve this constraint with more 
precise measurements of the giant's gravitational field \citep{helled11}. 
Meanwhile, the density profiles of Neptune and Uranus allow non-unique solutions 
for the compositional structure for much of the interior
\citep{guillot99b,guillot05}.

It has long been suggested
\citep{stevenson82a,stevenson82b}, that a portion of this dense material might be
redistributed in solution with hydrogen. As a result, erosion of a dense core
would cause it to shrink over the lifetime of the planet. Possible consequences of this process
are only beginning to be enumerated in evolutionary models
\citep{chabrier07,leconte12,mirouh12}. The establishment of a gradient in
concentration of a heavy dissolved component may change the nature of
convection in a portion of the planets interior. This `double-diffusive'
is hypothesized to reduce the efficiency of heat transfer, thereby altering
the thermal evolution of the planet. Comprehensive understanding of the
process has been limited by the lack of knowledge of the solubility of various
phases in metallic hydrogen, as well as poor understanding of the scaling of
convective efficiency in the presence of competing gradients of composition
and temperature. In this study, we address the first issue for iron metal.

As a result of continuing discoveries by Kepler \citep{borucki10} and other exoplanet surveys, the
number of confirmed planets has climbed to over 800, the majority of which are
giants. This presents a growing sampling of planetary
mass-radii relationships that will be fundamental to understanding the evolution of
giant planet interiors. The range of mass-radius relationships observed for
exoplanets exhibit variation beyond those in the solar system. In some cases,
such as Corot-20b \citep{deleuil11}, the relationships may even defy explanation by
simple structural models. Redistribution of dense core material lowers the
heavy element content required to explain anomalously high observed densities.

The favored model for gas giant formation
\citep{mizuno78,bodenheimer86,pollack96} relies on the early
formation of a large planetary embryo of critical mass to cause runaway
accretion of hydrogen and helium gas. A competing theory involves collapse of
a region of the disk under self-gravity, e.g. \citep{boss97}, but may
have difficulty explaining significant enrichment of refractory elements \citep{hubbard02,guillot05}.
The immediate result of a core-accretion hypothesis is a planet with the
ice-rock-metal embryo residing at the center as a dense core, surrounded by an
extensive layer of metallic hydrogen and helium. The role of core erosion to
the subsequent evolution is a major source of uncertainty, but in principle, can explain shrinking of
cores to masses smaller than those necessary to form the planet under the
core-accretion hypothesis.

Core erosion in giant planets can be addressed by determining the solubility 
of analogous phases. Previous studies have considered an icy
layer of fluid and superionic $\mathrm{H}_2\mathrm{O}$
\citep{wilson12a,wilson13}, and a rocky layer consisting of
MgO \citep{wilson12b} and $\mathrm{SiO}_2$ \citep{gonzalez13}, which have been
shown to separate at relevant conditions \citep{umemoto06}. Assuming the
same gross distribution, elements as terrestrial bodies, the innermost core
would be composed of a dense, metallic alloy composed primarily of iron.

{\it Ab initio} random structure searches
\citep{pickard09} demonstrate that iron remains in a hexagonal close packed (hcp) structure
remains stable up to pressures approaching Jupiter's center, $\sim$2.3 TPa, at
which point it undergoes a phase transition to face centered cubic (fcc)
structure. \citet{stixrude12} demonstrated a gradual decrease in this transition
pressure with temperature. Simulations of liquid hydrogen
\citep{militzer08,militzer13,mcmahon12} undergo a gradual transition from molecular to
metallic, which is complete by $\sim$ 0.4 TPa at low temperatures. {\bf Stable
mixtures of Fe and H have been suggested at lower $P$-$T$ conditions,
applicable to terrestrial cores } \citep{bazhanova12}. 

\section{Material Phases}


All simulations presented here were performed using the Vienna {\it ab initio}
simulation package (VASP) \citep{kresse96}. VASP uses the DFT formalism utilizing
pseudopotentials of the projector augmented wave type \citep{blochl94} and the exchange-correlation
functional of Perdew, Burke and Ernzerhof \citep{perdew96}. The iron
pseudopotential treats a $[\mathrm{Mg}]\mathrm{3pd}^6\mathrm{4s}^2$ electron
configuration as valence states, and a 2$\times$2$\times$2
grid of k-points is used for all simulations. Simulations on hydrogen 
and the solution were performed with a 900 eV cutoff energy for the plane wave expansion, while a 300
eV cutoff was used for iron. A time step of 0.2 fs was used for all liquid simulations, a
0.5$-$1.0 fs time step  was used for high and low temperature iron simulations 
respectively. The $\Delta G_{sol}$ results were confirmed to be well-converged
with respect to the energy cutoff and time step. 
Prohibitively long simulation times required that convergence with respect to
finer k-point meshes be verified over a subset of configurations generated by
a simulation with a 2$\times$2$\times$2 grid.

Iron simulations assume an hcp or fcc structure within their respective
stability regimes \citep{pickard09,stixrude12}. We confirmed Fe to be solid up
to 20000 K at 4 TPa, and to be a liquid at temperatures as low as 15000 K at
1 TPa. We also confirmed that the Gibbs free energy favors hcp stability over fcc
at 1 TPa, though the difference is negligible for our subsequent analysis of
dissolution. We found 32 atom supercells to be sufficient for Fe simulation.
Finite size effects required that we use large 256 atom supercells for
hydrogen, to which one Fe atom was added for the solution. Cubic
supercells are used for fcc and liquid runs. In order to maintain the same
number of atoms for the hcp an orthogonal supercell defined the combination of
hexagonal unit cell vectors $\mathbf{a}$,$\mathbf{a}+\mathbf{b}$,and
$\mathbf{c}$.

Cell volumes at each temperature were determined by fitting a pressure-volume polytrope equation of
state to short DFT-MD simulations. The resulting DFT pressures were all within
0.1\% of the target value. Gibbs free energies were computed for the three 
systems, $\mathrm{Fe}_{32}$, $\mathrm{H}_{256}$
and $\mathrm{H}_{256}\mathrm{Fe}_1$, using the thermodynamic integration method with simulation 
times of 1.0 ps for $\mathrm{H}_{256}$ and $\mathrm{H}_{256}\mathrm{Fe}_1$ 
and 2.5$-$5.0 ps for Fe. $\mathrm{H}_{256}$ and $\mathrm{H}_{256}\mathrm{Fe}_1$
runs with
$\lambda =1$ were extended to 4.0 ns for precise calculations of the internal energy,
which allows for determination of the entropic component of the Helmholtz free
energy. The calculated energies and entropy are presented in Tab. I,
along with the density.
The Gibbs free energy of solvation, calculated using Eq. 1, is presented in
Tab. II for each pressure-temperature condition. A negative $\Delta G_{sol}$
implies that the Gibbs free energy of the solution is lower than that of the separated
phases. Therefore, dissolution is favored at a solute concentration
higher than 1:256. 

\section{Simulation Results}

We find dissolution of iron to be strongly favorable at conditions corresponding to
the interiors of gas giants. Fig. 1 shows the variation of $\Delta G_{sol}$
with temperature and pressure. $\Delta G_{sol}$ exceeds $-$10 eV per iron atom for
plausible temperatures of Jupiter's core. Dissolution remains favorable even
at temperatures far below those predicted by model adiabats
\citep{militzer13,militzer13b}. The energetics
are only weakly dependent on pressure, and  $\Delta G_{sol}$ becomes increasingly
negative with decreasing pressure. The solubility increases with a nearly
linear trend in T, 
yielding slope of $\sim$0.53 meV/K. As a result, solubility is favored
through the entire range of conditions considered, and likely the entire range
for metallic hydrogen regions of giant planets.

Fig. 2 shows a breakdown of the data into contributions by various
thermodynamic parameters. Included in the figure are: $\Delta F_{sol}$, $\Delta U_{sol}$,
$P\Delta V_{sol}$ and $-T\Delta S_{sol}$, respectively, the Helmholtz free energy, internal
energy, volumetric work and entropic contributions contributions to $\Delta
G_{sol}$. Note that $\Delta F_{sol}$, $\Delta U_{sol}$ and $P\Delta V_{sol}$
are calculated independently, while $\Delta G_{sol}=\Delta F_{sol}+P\Delta
V_{sol}$ and $-T\Delta S_{sol}=\Delta F_{sol}-\Delta U_{sol}$ are derived. 
The trend of solubility with temperature is dominated by the entropic term.
The high solubility at low temperatures is
reflected in the negative values of $\Delta U_{sol}$, indicating that the
mixed system is energetically favorable independently of the entropy term. 

Our calculations neglect any interactions between iron atoms in solution, and
thus represent solubility in the low-concentration limit.
$\Delta G_{sol}$ can be related to the volume change associated with
the insertion of an iron of atom into hydrogen, as other contributions are
constant with respect concentration. It can be shown that results for
simulations with a 1:n solute ratio can be generalized to a ratio of 1:m using
  %\begin{mathletters}
  \begin{eqnarray}
  \Delta G_c &\approx& F_0(H_mFe)-F_0(H_m)-F_0(Fe)- \left[
    F_0(H_nFe)-F_0(H_n)-F_0(Fe)\right] \\
    &=& -k_BT\log\left\{ 
    \frac{\left[V(\mathrm{H}_n\mathrm{Fe}) +
    \frac{m-n}{n}V(\mathrm{H}_n)\right]^{m+1}
    \left[V(\mathrm{H}_n)\right]^{n}}
    {\left[V(\mathrm{H}_n)\frac{m}{n}\right]^m
    \left[V(\mathrm{H}_n\mathrm{Fe})\right]^{n + 1}}
   \right\},
\end{eqnarray}
%\end{mathletters}
where $\Delta G_c = \Delta G_{sol}(1:m)- \Delta G_{sol}(1:n)$, and
$V(\mathrm{H}_n)$ and $V(\mathrm{H}_n\mathrm{Fe})$ are the volumes for the
simulations of hydrogen and the solution respectively. 
Fig. 3 shows the shift of
$\Delta G_{sol}$ at 4 TPa at Fe concentrations of 1:100 and 1:1000. $\Delta
G_{sol}$ is decreased for higher concentrations, but not to an extent where
dissolution would become disfavored. At 20000 K
this difference between 1:100 and 1:256 is  $<$2 eV per iron atom, and
at 2000 K is smaller than uncertainty in calculated values of $\Delta
G_{sol}$.


The nature of the Fe-H system poses additional numerical challenges compared to other solutes
considered previously \citep{wilson10,wilson12a,wilson12b,gonzalez13}.
These can largely be attributed to the comparatively large change in volume
and electron density associated with the insertion of an iron atom into
metallic hydrogen. We found it
more efficient to determine cell volumes by fitting an equation of state to a
collection of MD simulations at constant volume, rather than performing
extended constant pressure simulations. 
Finite size effects were also found to be more significant for the Fe-H
system, due to iron's relatively large volume and number of valence electrons. Fig.
4 shows the convergence of a difference in internal energy between H and H-Fe
for MD simulations with 128, 256 and 512 hydrogen atoms. We find 256 hydrogen
atoms to be necessary in contrast to the previous studies that required only 128.
The convergence with k-point grid resolution is also slower
than in previous studies, and presents the greatest uncertainty in this study.
MD calculations with a 3$\times$3$\times$3 k-point grid are prohibitively expensive. An
estimate of this error for the results presented here is obtained by
evaluating the internal energy over configurations sampled from a MD
trajectory with a 2$\times$2$\times$2 k-point grid.\linebreak Fig. 5 shows this
estimated correction of $\Delta G_{sol}$ for the k-point grid used. 
The shifts are on the order $\sim$1 eV per iron atom, but are
consistently negative for both quantities, leading to dissolution being more favorable. 

\section{Discussion}

With the results of previous studies
\citep{wilson12a,wilson12b,gonzalez13}, we can now present a comprehensive
picture for the solubility of typical core materials in liquid
metallic hydrogen.
Dissolution is strongly favored for both iron and water
ice. However, for water the high solubility is attributed entirely to the
entropy, whereas iron has a favorable internal energy component that favors
dissolution at low temperatures.
Both phases are found to be soluble throughout the entire metallic
hydrogen region of both Jupiter and Saturn. The rocky components, MgO and
$\mathrm{SiO}_2$, have more moderate solubilities, with $\mathrm{SiO}_2$ being slightly higher. The saturation
curves are, however, less steep than the adiabats for Jupiter and Saturn. As a
result, solubility is favored for Jupiter's core, but the rocky components of
Saturn's core may be stable given the present uncertainty in the planet's
adiabat.

The rocky components are found to be
stable at lower pressures, approaching the metallic transition. If Mg and Si
were convected upwards with sufficient concentration, they may precipitate, while
Fe and $\mathrm{H}_2\mathrm{O}$ would remain in solution, at least to the molecular-metallic
transition. 
The presence of a significant dissolved component at shallow depths may have
consequences for the density profile and transport properties of
hydrogen, which influence thermal structure and magnetic field
generation.

Core erosion is thermodynamically favorable in gas giant planets,
with the possible exception of smaller, cooler planets, like Saturn. For these planets, the
outer icy layers are soluble, but the rocky layers may not be. The innermost
iron component, though soluble, would be isolated from reaction with hydrogen.
This might allow Saturn to have a larger, less eroded core than
Jupiter, a result consistent with current observational constraints.
Nevertheless, our results imply
that confirmation of a massive core for Jupiter would support the
core-accretion model over gravitational collapse. While erosion of such a core
may be slow due to inefficient double-diffusive convection
\citep{stevenson82a,chabrier07,leconte12,mirouh12}, settling of dispersed
refractory material to form a core is inconsistent with our results. Late formation of a core
would require a large amount material from captured planetesimals
surviving descent to the planet's center.

It may be possible to attribute some
emerging trends in exoplanet mass-radius relationships to the difference in
solubilities between rock and ice, or rock and metal. However, as we have
shown, such thermodynamic differences are likely to only be significant in
smaller, cooler planets, where redistribution of dense material by
double-diffuse convection would be least efficient. The energetics of the
dissolution reaction should be insignificant compared to the role of density
in the redistribution of dense material. The work required to raise an iron of atom to
the molecular-metallic transition is on the order of 1000 eV, whereas the
contribution from the dissolution reaction is $\sim$1$-$10 eV. We conclude
that the process of core erosion is thermodynamically consistent with ab
initio simulations of the relevant materials, and its
significance warrants close consideration in future models of giant planet
evolution.

% Include figure dG-T figure with different lines for all points
 \begin{figure}[H] %  figure placement: here, top, bottom, or page
   \centering
   \includegraphics[width=22pc]{figs/dGall.eps} 
\caption{Gibbs free enerby of solvation for solid Fe in liquid metallic
hydrogen. Negative values favor dissolution for a solute ratio of 1:256.}
\end{figure}

% Figure showing comp
 \begin{figure}[H] %  figure placement: here, top, bottom, or page
   \centering
   \includegraphics[width=22pc]{figs/dGcomp.eps} 
\caption{Breakdown of $\Delta G_{sol}$ into contributions from: internal
energy,
$\Delta U_{sol}$, pressure effects, $P\Delta V_{sol}$, and entropic effects,
$-T\Delta S_{sol}$. Plots show variation with (a) temperature at P=4 TPa, and
(b) pressure at T=2000 K.}
\end{figure}

% Include figure dG-T figure with different lines for all points
 \begin{figure}[H] %  figure placement: here, top, bottom, or page
   \centering
   \includegraphics[width=22pc]{figs/concentration.eps} 
\caption{Shift in $\Delta G_{sol}$ from a system with and Fe:H ratio of 1:256 to 1:100 and
1:1000 in the low-concentration limit.}
\end{figure}

% Include figure dG-T figure with different lines for all points
 \begin{figure}[H] %  figure placement: here, top, bottom, or page
   \centering
   \includegraphics[width=22pc]{figs/finite_size.eps} 
\caption{Energy of insertion for a single Fe atom into supercells containing
128, 256, and 512 atoms. Finite size effects are significant
for $\mathrm{H}_{128}$, but are negligible within error for $\mathrm{H}_{256}$.}
\end{figure}

 \begin{figure}[H] %  figure placement: here, top, bottom, or page
   \centering
   \includegraphics[width=22pc]{figs/kpoints.eps} 
\caption{Estimated corrections to $\Delta G_{sol}$ and $\Delta G_{sol}$
coarseness of k-point grid used in DFT-MD runs. DFT calculations with a $3\times
3\times 3$ k-point grid were performed sampling a trajectory generated by an MD
simulation with a $2\times2\times 2$ k-point mesh. }
\end{figure}

%\begin{table}{rrlcrrrrr}
%\tabletypesize{\scriptsize}
%\tablecolumns{9}
%\tablewidth{0pc}
%\tablecaption{Thermodynamic parameters derived from DFT-MD simulations.\label{data}}
%\tablehead{ \colhead{P} & \colhead{T} & \colhead{System} & \colhead{Phase} &
%\colhead{$\rho$} & \colhead{F} & \colhead{U} & \colhead{G} & \colhead{S}\\
%(GPa) & (K) & ~~~~~- &  - & ($\mathrm{g}/\mathrm{cm}^3$) & (eV)~~~~ & (eV)~~~~
%& (eV)~~~~ & ($\mathrm{k}_b/\mathrm{K}$) }
%\startdata
%400   &  2000   &  $\mathrm{Fe}_{32}$     &  hcp  &  14.408  &  $-$190.8(4)\phantom{0}   &  $-$154.4(0)            &  323.3(4)\phantom{0}     &  211.(4)              \\
%.     &  .      &  $\mathrm{H}_{256}$    &  liq  &  1.2709  &  $-$415.6(5)\phantom{0}   &  $-$228.(0)\phantom{0}  &  419.37(9)               &  1088.(2)             \\
%.     &  .      &  $\mathrm{H}_{256}\mathrm{Fe}$  &  liq  &  1.5192  &  $-$423.8(1)\phantom{0}   &  $-$234.(0)\phantom{0}  &  427.1(8)\phantom{0}     &  1101.(6)             \\
%1000  &  2000   &  $\mathrm{Fe}_{32}$     &  hcp  &  18.279  &  $-$12.4(0)\phantom{0}    &  18.1(2)                &  1000.8(5)\phantom{0}    &  177.(1)              \\
%.     &  .      &  $\mathrm{H}_{256}$    &  liq  &  1.8916  &  23.0(3)\phantom{0}       &  175.9(6)               &  1425.6(6)\phantom{0}    &  887.(3)              \\
%.     &  .      &  $\mathrm{H}_{256}\mathrm{Fe}$  &  liq  &  2.2534  &  20.4(1)\phantom{0}       &  175.(2)\phantom{0}     &  1454.7(1)\phantom{0}    &  898.(3)              \\
%1000  &  2000   &  $\mathrm{Fe}_{32}$     &  fcc  &  18.269  &  $-$10.35(0)              &  19.8(0)                &  1003.4(6)\phantom{0}    &  174.(9)              \\
%.     &  .      &  $\mathrm{H}_{256}$    &  liq  &  1.8916  &  23.0(3)\phantom{0}       &  175.9(6)               &  1425.6(6)\phantom{0}    &  887.(3)              \\
%.     &  .      &  $\mathrm{H}_{256}\mathrm{Fe}$  &  liq  &  2.2534  &  20.4(1)\phantom{0}       &  175.2(3)               &  1454.7(1)\phantom{0}    &  898.(3)              \\
%%1000  &  15000  &  $\mathrm{Fe}_{32}$     &  liq  &  16.970  &  $-$506.2(1)\phantom{0}   &  12(7).\phantom{0(0)}   &  585.(2)\phantom{00}     &  49(0).\phantom{(0)}  \\
%%.     &  .      &  $\mathrm{H}_{256}$    &  liq  &  1.6315  &  $-$2064.1(1)\phantom{0}  &  595.(4)\phantom{0}     &  $-$437.9(2)\phantom{0}  &  2057.(5)             \\
%%.     &  .      &  $\mathrm{H}_{256}\mathrm{Fe}$  &  liq  &  1.9468  &  $-$2091.9(7)\phantom{0}  &  598.(7)\phantom{0}     &  $-$431.7(8)\phantom{0}  &  2081.(6)             \\
%4000  &  2000   &  $\mathrm{Fe}_{32}$     &  fcc  &  28.374  &  754.91(4)                &  777.6(2)\phantom{0}    &  3365.97(1)              &  131.(8)              \\
%.     &  .      &  $\mathrm{H}_{256}$    &  liq  &  3.6375  &  1392.3(8)\phantom{0}     &  1487.6(4)              &  4310.0(0)\phantom{0}    &  552.(7)              \\
%.     &  .      &  $\mathrm{H}_{256}\mathrm{Fe}$  &  liq  &  4.3078  &  1412.3(8)\phantom{0}     &  1508.(4)\phantom{0}    &  4413.4(7)\phantom{0}    &  55(7).\phantom{(0)}  \\
%4000  &  15000  &  $\mathrm{Fe}_{32}$     &  fcc  &  27.826  &  382.1(9)\phantom{0}      &  87(5).\phantom{0(0)}   &  3045.7(0)\phantom{0}    &  38(1).\phantom{(0)}  \\
%.     &  .      &  $\mathrm{H}_{256}$    &  liq  &  3.3618  &  $-$392.9(3)\phantom{0}   &  1917.(3)\phantom{0}    &  2763.9(8)\phantom{0}    &  1787.(3)             \\
%.     &  .      &  $\mathrm{H}_{256}\mathrm{Fe}$  &  liq  &  3.9865  &  $-$392.2(5)\phantom{0}   &  194(3).\phantom{0(0)}  &  2850.7(4)\phantom{0}    &  1807.(2)             \\
%4000  &  20000  &  $\mathrm{Fe}_{32}$     &  fcc  &  27.550  &  176.(9)\phantom{00}      &  92(2).\phantom{0(0)}   &  2866.(8)\phantom{00}    &  43(2).\phantom{(0)}  \\
%.     &  .      &  $\mathrm{H}_{256}$    &  liq  &  3.2731  &  $-$1284.5(8)\phantom{0}  &  208(1).\phantom{0(0)}  &  1957.2(6)\phantom{0}    &  1952.(6)             \\
%.     &  .      &  $\mathrm{H}_{256}\mathrm{Fe}$  &  liq  &  3.8824  &  $-$1294.0(9)\phantom{0}  &  210(4).\phantom{0(0)}  &  2035.5(1)\phantom{0}    &  1972.(0)             \\
%\enddata
%%\tablecomments{Table \ref{data} is published in its entirety in the 
%%electronic edition of the {\it Astrophysical Journal}.  A portion is 
%%shown here for guidance regarding its form and content.}
%\end{deluxetable}


%\begin{deluxetable}{rrcc}
%\tabletypesize{\scriptsize}
%\tablecolumns{4}
%\tablewidth{0pc}
%\tablecaption{Gibbs free energy of solvation for Fe in liquid H\label{solvation}}
%\tablehead{ \colhead{$P$} & \colhead{$T$} &  \colhead{Fe Phase} & \colhead{$\Delta G$} \\
%(GPa) & (K)~ & - & (eV) }
%\startdata
%400 \phantom{0}  & 2000  & hcp  &  $-$2.2 $\pm$ 0.14    \\
%1000\phantom{0}  & 2000  & hcp  &  $-$2.5 $\pm$ 0.12    \\
%1000\phantom{0}  & 2000  & fcc  &  $-$2.3 $\pm$ 0.13    \\
%%1000\phantom{0}  & 15000 & liq  &  $-$12.2 $\pm$ 0.20\phantom{0}   \\
%4000\phantom{0}  & 2000  & fcc  &  $-$1.71 $\pm$ 0.056  \\
%4000\phantom{0}  & 15000 & fcc  &  $-$8.42 $\pm$ 0.066  \\
%4000\phantom{0} & 20000 & fcc  &  $-$11.34 $\pm$ 0.078\phantom{0} \\
%\enddata
%\end{deluxetable}

\chapter{High temperature miscibility of Terrestrial Mantles and Cores}\label{chap4}

\section{Motivation}

Terrestrial planets are, to first order, made up of a metallic iron core and a mantle
composed of silicate and oxide minerals. Chondritic meteorites show that these materials
initially condensed together from the protoplanetary nebula, but became free to separate
and gravitationally stratify as planetesimals grew.  Numerous scenarios have been put
forward to describe how these reservoirs interact depending on the pressure, extent of
melting, and the specific assumptions of rocky phases
\citep{Stevenson1990,Solomatov2007,Rubie2007}. These typically assume the major components
occur in two immiscible phases. Additionally, most studies assume that element
partitioning between the two phases is similar to that observed in experiments performed
at much lower temperatures \citep{Mcdonough1995}. In the case of a hot early history of a
growing planet, neither assumption is necessarily correct. At sufficiently high
temperatures, entropic effects dominate and any mixture of materials will form a single,
homogeneous phase. It is therefore necessary to consider a high temperature mixture of
the `rocky' and metallic terrestrial components. The presence of such a mixed phase will
affect the chemistry of iron-silicate differentiation on the early Earth.

Here we consider a simple representative material for the mixed rock-metal phase as a mixture
of Fe and MgO formed via the reaction
%$\rm{MgO}_{\rm sol/liq} + \rm{Fe}_{\rm liq} \Rightarrow \rm{FeMgO}_{\rm liq}$. 
\begin{equation} 
  \rm{MgO}_{\rm sol/liq} + \rm{Fe}_{\rm liq} \Rightarrow \rm{FeMgO}_{\rm liq}.
\end{equation}
We determine the stability of these phases using first-principles calculations.  At a
given pressure, a system with two separate phases can be described in terms of a
miscibility gap. At low temperatures, homogeneous mixtures with intermediate compositions
are thermodynamically unstable and a heterogeneous mixture of two phases with
compositions near the endmembers is preferred. The exsolution gap is bounded by a solvus
that marks the temperature above which a single mixed phase is stable, and the maximum
temperature along the solvus is referred to as the solvus closure temperature. Here we calculate
the Gibbs free energy of the mixture and the endmembers to determine the solvus closure
temperature for mixture similar to the bulk-composition of a terrestrial planet. These
results can inform future work, by providing the conditions where rock-metal miscibility
plays a role in the differentiation of terrestrial planet interiors.


An order of magnitude calculation shows the gravitational energy released in the
formation of an Earth-mass body, if delivered instantaneously, is sufficient to raise
temperatures inside the body by $\sim$40,000 K.  Redistribution of mass within the body
during core formation can account for another $\sim$4,000 K increase. This energy is
released over the timescale of accretion, $\sim$$10^{8}$ years \citep{Chambers1998}, with
efficient surface heat loss through a liquid-atmosphere interface
\citep{Abe1997,Elkins-Tanton2012}. However, simulations of the final stages of planet
growth \citep{Cameron1991,Chambers1998,Canup2000} suggest that near-instantaneous release
of large quantities of energy through giant impacts is  the rule rather than the
exception.  Simulations of the `canonical' moon-forming impact hypothesis
\citep{canup2004}, in which a Mars-sized body collides with the protoearth, find fractions
of the target's interior shocked well above 10,000 K. More recently, dissipation of
angular momentum from the Earth-Moon system by the evection resonance has loosened
physical constraints on the impact, suggesting that the formation of the moon is better
explained by an even more energetic event than the `canonical' one \citep{Canup2012,Cuk2012}.
It is, therefore, difficult to precisely constrain the temperature of the Earth's
interior in the aftermath of the moon forming impact, much less that of other terrestrial
planets with even more uncertain impact histories. Regardless, there is evidence for
giant impacts throughout the inner solar system, implying temperatures significantly
higher than the present day Earth may have been common. In addition to high temperatures,
giant impacts involve significant physical mixing of iron and rocky materials
\citep{Dahl2010}.  Thus, miscibility may be important even if the impacting bodies have
iron cores that differentiated at lower temperatures and pressures.

Differentiation and core formation is a key event or series of events in terrestrial
planet evolution. The timing and conditions of differentiation have important
consequences for the evolution of the planet, through its effect on the distribution of
elements throughout the planet's interior. The distribution of elements affects the
gravitational stability of solid layers in the mantle, the location of radioactive heat
sources, and the nature of the source of buoyancy driving core convection and magnetic
field generation. Each of these subsequently affect the thermal evolution of the planet.
If this process occurs near the solvus closure temperature, there are likely to be
physical and chemical differences from the processes at conditions where the phases are
completely immiscible. We include a discussion of some of these processes in Section
\ref{sec:discussion}.

\subsection{Simulated system}

Modern high pressure experimental techniques, using static or dynamic compression
techniques, can reach megabar pressures \citep{Boehler2000}. However, experiments at
simultaneous high pressures and temperatures have limitations. Interpretation of mixing
processes during shock wave experiments is difficult, and the samples are not
recoverable. Meanwhile, laser heated diamond anvil cells experience extreme temperature
gradients and require survival of quenched texture to interpret. In both cases, the
methods only cover a small fraction of the $P$-$T$ range expected in the aftermath of a
giant impact.  As a result, simulations based on first-principles theories are an
appropriate means of constraining material properties over a range of such extreme
conditions. 

We performed density functional theory molecular dynamics (DFT-MD) simulations for phases
in a model reaction between liquid iron, and solid (B1) or liquid magnesium oxide. The
change in Gibbs free energy of this system per formula unit FeMgO is described as
\begin{equation} \label{eqn:gibbs}
% \frac{ \Delta G }{\rm{ FeMgO \; fun.}} =  \frac{1}{24}G_{\rm{(FeMgO)_{24}}} 
  \Delta G_{\rm{\rm mix}}  =  \frac{1}{24}G_{\rm{(FeMgO)_{24}}} 
- \frac{1}{32} \left[  G_{\rm{(MgO)_{32}}}  + G_{\rm{Fe_{32}}} \right]
\end{equation}
where $G_{\rm{(MgO)_{32}}}$ and   $G_{\rm{Fe_{32}}}$ are the Gibbs free energies of a
pure MgO and iron endmembers with subscripts referring to the number of atoms in the
periodic simulation cell. $G_{\rm{(FeMgO)_{24}}}$ is the Gibbs free energy of 1:1
stoichiometric liquid solution of the two endmembers. Comparing Gibbs free energies among
a range of compositions, we find the temperature for mixing of the phases in the
1:1 ratio to be a good approximation for the solvus closure temperature.

MgO is the simplest mantle phase to simulate, and a reasonable starting point for a study
of the miscibility of terrestrial materials. Up to $\sim$400 GPa, MgO remains in the
cubic B1 (NaCl) structure \citep{Boates2013}, meaning simulations of only one solid phase
were necessary for the rocky endmember. In order to describe a similar reaction for ${\rm
MgSiO_3}$ perovskite, the Gibbs free energy of MgO and ${\rm SiO_2}$ must also be
calculated to address the possibility of incongruent dissolution of the solid phase
\citep{Boates2013}. More importantly, high pressure experiments observing reactions of
silicates with iron have demonstrated the MgO component has by far the lowest solubility
in iron up to $\sim$3000 K \citep{Knittle1991,Ozawa2008a}. This suggests our results  can
be interpreted as an upper bound for the solvus closure temperature with more realistic
compositions. 

It is worth emphasizing that the mixed FeMgO phase is unlike any commonly studied rocky
phase, in that it does not have a balanced oxide formula. This is by design and is
necessary for the mixing of arbitrary volumes of the metallic and oxide phases.  This is
a separate process from the reaction of the FeO component which transfers O to the
metallic phase at lower temperature, and which is primarily controlled by oxygen fugacity
rather than temperature \citep{Tsuno2013}. We treat the mixed phase as a liquid at all
conditions.  Although we cannot absolutely rule out the possibility of a stable solid
with intermediate composition, such a phase would require a lower Gibbs free energy, and
therefore is consistent with treating our results as an upper bound on the solvus closure
temperature.


All DFT simulations presented here were performed using the Vienna {\it ab initio}
simulation package (VASP) \citep{Kresse1996}. VASP uses projector augmented wave
pseudopotentials \citep{Blochl1994} and the exchange-correlation functional of Perdew,
Burke and Ernzerhof \citep{Perdew1996}. Although the DFT formalism is based on a
zero-temperature theory, DFT-MD simulations at high temperatures have been shown to agree
with theory developed for warm dense matter \citep{Driver2012}. We use an iron
pseudopotential with valence states described by a
$[\mathrm{Mg}]\mathrm{3pd}^6\mathrm{4s}^2$ electron configuration.  For consistency, all
simulations use Balderesci point sampling, a 600 eV cutoff energy for the plane wave
expansion and temperature dependent Fermi-smearing to determine partial orbital
occupations. A time step of between 0.5 and 1.0 fs is used depending on the temperature,
with the smaller time step used for all simulations with temperatures above 6000 K. We
confirmed that the resulting molecular dynamics results are well-converged with respect
to the energy cutoff and time step.  All presented results involve molecular
dynamics simulation 
lengths of at least 2 ps simulated time at each, with longer simulation times having an
insignificant effect on the results of the thermodynamic integration. The largest source
of uncertainty was the finite size effect, which we discuss in detail in
Section \ref{sec:results}.


\section{Results} \label{sec:results}

The online supplementary material includes a table with the results of the calculations
for each composition and $P$-$T$ condition. It includes the density along with the
calculated pressure, internal energy, entropy and Gibbs free energy. The stable phase at
each condition is determined using Eqn. \ref{eqn:gibbs}.  The point at which the trend in
$\Delta G_{\rm mix}$ at constant $P$ changes sign is the inferred solvus temperature at
the 1:1 stoichiometric composition.  Fig.  \ref{fig:components} shows an example of this
trend in $\Delta G$ for the 1:1 mixture at $P=50$ GPa, along with its components $\Delta
U$, $\Delta PV$ and $-\Delta TS$. Using the convention from Eqn. \ref{eqn:gibbs},
positive values favor the separation of the  material into the endmember phases, while
negative values favor the single homogeneous mixed phase. The contributions of the
internal energy and volumetric terms are positive, while the entropy provides the
negative contribution that promotes mixing at sufficiently high temperature. 

\begin{figure}[h!]  
  \centering
    \includegraphics[width=22pc]{figs/components.pdf}
\caption{Gibbs free energy change per formula unit, $G_{\rm{mix}}$ of the reaction
${ \rm MgO_{liq} + Fe_{liq} } \rightarrow {\rm FeMgO_{liq}}$ at $P=50$ GPa (red).  Independently
calculated components of $\Delta G_{\rm{mix}}$: $\Delta U$ (black), $\Delta PV$ (blue),
and $-\Delta TS$ (green). Positive values favor separation into end member phases,
while negative values favor a single mixed phase. $\Delta PV$ values presented here use
the target pressure. Error bars represent the integrated error from the 1 $\sigma$
statistical uncertainty of the molecular dynamics simulations.}
\label{fig:components}
\end{figure}


Using additional calculations at $P=50$  GPa, we estimated uncertainties in our
calculated Gibbs free energies. The most significant contribution to the uncertainty
comes from the finite size of the simulation cells. We estimate the magnitude of this
uncertainty by comparing results from larger simulated cells to those of the original
system. Fig. \ref{fig:finite_size_effect} compares the values of $\Delta G_{\rm mix}$ for
Fe and FeMgO cells with up to twice the number of atoms, and MgO with up to 100 atoms per
cell. From this, we estimate a maximum shift of $<0.1$ eV per formula unit. This
corresponds to an uncertainty in temperature of $\sim$200 K, roughly an order of
magnitude larger than the statistical precision of the calculation. The combined effect
of increasing cell size for all systems leads consistently to lower values of $\Delta
G_{\rm mix}$ at both temperatures, and thus, lower predicted solvus closure temperatures.
For the subsequent analysis, we consider an estimated uncertainty defined as the largest
(positive or negative) shift in the Gibbs free energy for each phase. It should be noted
that these estimated error bars can not be strictly viewed as statistical uncertainties,
since they are unidirectional and based on a small number of independent calculations.
They suggest the likely magnitude by which similar shifts in the calculated Gibb's free
energies will effect the calculated transition temperature at different $P-T$ conditions. 

\begin{figure}[h!]  
  \centering
    \includegraphics[width=20pc]{figs/finite_size_test.pdf}
\caption{Quantifying the finite size effect on $\Delta G_{\rm{mix}}$ for simulations of
the reaction ${ \rm MgO_{liq} + Fe_{liq} } \rightarrow {\rm FeMgO_{liq}}$ at $P=50$ GPa
with different cell sizes. In black are the results for the systems  ${\rm Fe_{32}}$,
${\rm Mg_{32}O_{32}}$ and ${\rm Fe_{24}Mg_{24}O_{24}}$ used at the other P-T conditions.
The other lines show the shift in $\Delta G_{\rm{mix}}$ obtained when the calculation is
repeated for with a larger cell for one (dashed lines) or all (solid lines) of the systems.}
\label{fig:finite_size_effect}
\end{figure}


The effect of pressure and temperature on $\Delta G_{\rm mix}$ is shown in Fig.
\ref{fig:deltaG}. As pressure increases, the slope of $\Delta G_{\rm mix}$ with $T$
remains nearly constant for all simulations with liquid MgO, but the values are shifted to
higher temperatures. This means that the solvus closure temperature has a positive slope
with pressure over the entire range of conditions considered. There is a noticeable
change in the slope of this quantity  when MgO melts.  This corresponds with a weaker
dependence on pressure at high pressures, where the solvus temperature is below the
melting temperature of MgO. 

When determining the energetics of the mixed FeMgO, phase it is important to verify that
the simulation remain in a single mixed phase. At temperatures sufficiently close to the
solvus closure temperature the system should behave as a super-cooled homogeneous
mixture, while at sufficiently low temperatures the simulations could, in principle,
spontaneously  separate into two phases. This would bias the results as interfacial
energies between the separating phases would be included in the calculated Gibbs free
energy.  We were unable to detect phase separation by visual inspection of various of
snapshots as has been seen for hydrogen-helium mixtures in \citep{Soubiran2012}. The pair
correlation function, $g({\bf r})$, can be used as a proxy for separation of phases
\citep{Soubiran2012}.

\begin{figure}[h!]  
  \centering
    \includegraphics[width=22pc]{figs/rdf_Fe_48.pdf}
\caption{Fe-Fe pair correlation functions for mixed Fe + MgO phase. Compares the spatial
  distribution of atoms in simulations at 50 GPa with different temperatures. The 3000 K
  and 5000 K asymptote to values notably less than one, while temperatures near or above
  the solvus closure temperature do not show such a deviation at large $r$.}
\label{fig:rdf}
\end{figure}

Fig. \ref{fig:rdf} shows the Fe-Fe $g(r)$ for simulations of the mixed FeMgO phase at 50
GPa. For temperatures significantly below solvus closure temperature, 3000 and 5000 K,
there are slight negative deviations of the $g(r)_{\rm Fe-Fe}$ at large $r$ from their
expected asymptote to unity. This is consistent with clustering into MgO and Fe-rich
regions, and may indicate spontaneous phase separation  at temperatures well below the
inferred solvus closure temperature. These deviations are minimal or not observed in
simulations near or above the inferred solvus closure temperature. We performed
additional simulations at temperatures close to the solvus closure temperature to verify
that the Gibbs free energy changes linearly as a function of temperature, which is the
expected behaviour without spontaneous phase separation. In spite of deviations in
$g(r)$, we note that including the low temperature simulations in our calculation of the
solvus closure temperature does not significantly change the result at any pressure.

\begin{figure}[h!]  
  \centering
    \includegraphics[width=22pc]{figs/solvus_50GPa.pdf}
\caption{Solvus phase diagram of the Fe-MgO system at $P=50$ GPa. The shape is consistent
  with the composition $X_{\rm MgO}=0.5$ being representative for estimating the solvus
  closure temperature at other pressures. The filled blue region shows an estimate of the
  uncertainty in transition temperature arising from the uncertainties in $G$ in Fig.
  \ref{fig:convex_hull}.}
\label{fig:solvus}
\end{figure}


We also studied the effect of composition on the solvus temperature.  Calculations were
performed on four additional intermediate compositions between the Fe and MgO endembers.
Fig. \ref{fig:convex_hull} shows a convex-hull in $G_{\rm Fe_{1-x}MgO_{x}}$ and $\Delta
G_{\rm mix}$ at 50 GPa and 5000 K. This corresponds to a temperature below the calculated
solvus temperature. The Gibbs free energies of all intermediate components are above the
mixing line between the end members, and form a smooth function with composition. This is
consistent with a binary system with a miscibility gap. Using linear interpolation
between this convex hull and one at 7500 K, we estimate the shape of the miscibility gap
at 50 GPa, as shown in Fig. \ref{fig:solvus}. The miscibility gap is notably asymmetric,
with temperatures decreasing faster towards the Fe-rich endmember than the MgO-rich end.
A similar, more pronounced asymmetry has been experimentally determined for the Fe-FeO
system at lower pressures \citep{Ohtani1984a,Kato1989}. In spite of this, the shape of the
solvus at intermediate compositions, $\sim$0.3-0.9 molar fraction MgO, is relatively
flat. As a result, the temperatures predicted for a 1:1 mixture provide a good estimate
for the solvus closure temperature. We note, however, that the shape of miscibility gap
may be sensitive to uncertainties from the finite size effect. Considering the estimated
errors from the finite size effect test, we can only constrain to be within that
$\sim$0.3-0.9 $X_{\rm MgO}$ range. Regardless of this composition our uncertainty in the
temperature of solvus closure remains $\sim200$ K.


\begin{figure}[h!]  
  \centering
    \includegraphics[width=22pc]{figs/deltaG.pdf}
\caption{Gibbs free energy of mixing for MgO and liquid Fe. Solid lines show conditions
  where MgO was simulated as a liquid, and dashed lines where MgO is in its (B1) solid
  phase.  The filled green region shows an estimate of the uncertainty from finite size
  effects, taken as the maximum shifts in $\Delta G_{\rm{mix}}$ observed our tests of larger
  cells (Fig. \ref{fig:finite_size_effect}).
}
\label{fig:deltaG}
\end{figure}

Fig. \ref{fig:summary} summarizes the results, showing all the conditions at which
simulations were performed.  We find the solvus closure at ambient pressure to be
$4089^{+25}_{-235}$ K.  While there is little experimental work on this exact system, our
results are superficially consistent with extrapolations of the phase diagram for the
Fe-FeO system from low temperatures \citep{Ohtani1984a,McCammon1983}, and with the
`accidental' discovery of the Fe-silicate solvus by \cite{Walker1993}. We find that the
solvus temperature increases with pressure to $6010^{+28}_{-204}$ K at 50 GPa, but its
slope decreases significantly at higher pressures, with a temperatures of
$6767^{+14}_{-135}$ K at 100 GPa and $9365^{+14}_{-130}$ K at 400 GPa.  This transition
also corresponds roughly to the pressure where the trend crosses the MgO melting curve
\citep{Alfe2005, Belonoshko2010,Boates2013}.  Indeed, the simulations used to infer the
closure temperature at these pressures used the solid (B1) structure of MgO.
Unfortunately, it is difficult to check whether the change in slope is a direct result of
this phase transition, as liquid MgO simulations rapidly freeze at temperatures far below
the melting curve.  Conversely, the liquidus of a deep magma ocean might be below the
solvus at these temperatures due to the effect of an $\rm{SiO_2}$ or FeO component in the
silicate/oxide endmember \citep{deKoker2013,Zerr1998}. However, extrapolation of $\Delta
G_{\rm mix}$ from simulations with liquid MgO at higher temperatures (Fig.
\ref{fig:deltaG}) suggest that the change in slope occurs occurs in liquids as well. We
estimate shifts in the inferred solvus closure temperature from finite size effects on
the order of 200 K. The  actual solvus closure temperature may also be shifted by up to a
couple hundred Kelvin, if we also consider the uncertainty in the solvus shape (Fig.
\ref{fig:convex_hull}), since our results refer specifically the solvus temperature for a
1:1 stoichiometric mixture.  The observed change in slope of the solvus temperature and
the relation to the pure MgO melting curve are, however, robust against uncertainties of
this magnitude.

\begin{figure}[h!]  
  \centering
    \includegraphics[width=22pc]{figs/convex_hull_5000.pdf}
\caption{Convex hull of $\Delta G_{\rm{mix}}$ versus formula unit fraction, $X_{\rm
MgO}$, for the Fe-MgO system at $P=50$ GPa and $T=5000$ K (top). Difference between
$\Delta G_{\rm{mix}}$ and a mixing line between the endmembers (bottom). The filled blue
region shows an estimate of the uncertainty from finite size effects, taken as the
maximum shifts in $\Delta G_{\rm{mix}}$ from our tests of larger Fe, MgO and FeMgO
simulations. The estimated error is weighted as a function of composition since the
finite size effects will cancel with that of the end-member as the compositions become
more similar.
}
\label{fig:convex_hull}
\end{figure}

\subsection{Simulation results} \label{sec:simulation}
Table~\ref{tab:results} shows the results of thermodynamic integration
calculations performed for each composition and $P$-$T$ condition. It
includes the density along with the calculated pressure, internal energy
$U$, entropy $S$, and Gibbs free energy, $G$. Calculations are specified by
pressure, $P$, temperature, $T$, and the atomic composition and phase.

To determine the density $\rho$ for a given $P$ and $T$ of interest, we
fitted equations of state to results from DFT-MD simulations. $P$ and $U$
are time-averaged results from DFT-MD simulations with 1$\sigma$
statistical error quoted. $S$ and $G$ are calculated from the two step
thermodynamic integration technique. For quantities calculated using the
thermodynamic integration, the quoted errors were derived by propagating
the errors from each integration point.



\begin{table}[hp]
\centering

\caption{Thermodynamic functions derived from DFT-MD simulations (Part 1/2).
\label{tab:results}}
\begin{adjustbox}{max width=\textwidth}
\begin{tabular}{ccllllll}
    \hline
 {$P$} & {$T$} & {system} &  {$\rho$} & 
 {$P$} & {$U$} & {$S$} &  {$G$}  \\
 {(GPa)} & {(K)} & &  {($\mathrm{g}/\mathrm{cm}^3$)} 
   & {(GPa)} & {(eV)} & {($k_B$)} & {(eV)} \\
   \hline
             0 &   4000 &  $\rm{Mg}_{24}$$\rm{Fe}_{24}$$\rm{O}_{24}$,liq &   2.861 &   $-$0.41(17) &     $-$323(1) &    1064(4) &     $-$689.3(1) \\
             . &      . &                             $\rm{Fe}_{32}$,liq &   6.885 &    0.06(21) &  $-$206.53(9) &   496.7(4) &    $-$377.74(3) \\
             . &      . &                $\rm{Mg}_{32}$$\rm{O}_{32}$,liq &   2.003 &    0.60(14) &   $-$275.0(5) &     776(2) &    $-$542.40(5) \\
             0 &   5000 &  $\rm{Mg}_{24}$$\rm{Fe}_{24}$$\rm{O}_{24}$,liq &   2.556 &    0.15(16) &     $-$289(1) &    1152(3) &     $-$785.4(2) \\
             . &      . &                             $\rm{Fe}_{32}$,liq &   6.340 &   $-$0.05(34) &   $-$189.8(1) &   540.1(4) &    $-$422.52(2) \\
             . &      . &                $\rm{Mg}_{32}$$\rm{O}_{32}$,liq &   1.660 &    0.10(10) &   $-$242.8(8) &     860(2) &    $-$613.17(6) \\
             0 &   6000 &  $\rm{Mg}_{24}$$\rm{Fe}_{24}$$\rm{O}_{24}$,liq &   2.195 &   0.639(88) &   $-$251.9(8) &    1231(2) &     $-$888.3(1) \\
             . &      . &                             $\rm{Fe}_{32}$,liq &   5.740 &   $-$0.02(12) &   $-$170.8(1) &   580.2(3) &    $-$470.80(3) \\
             . &      . &                $\rm{Mg}_{32}$$\rm{O}_{32}$,liq &   1.435 &   0.707(49) &     $-$213(1) &     924(2) &    $-$690.65(9) \\
            50 &   5000 &  $\rm{Mg}_{24}$$\rm{Fe}_{24}$$\rm{O}_{24}$,liq &   5.237 &   49.28(40) &   $-$315.5(7) &     963(2) &     $-$502.0(2) \\
             . &      . &                             $\rm{Fe}_{32}$,liq &   8.805 &   51.36(34) &   $-$203.3(3) &   475.0(6) &    $-$302.73(2) \\
             . &      . &                $\rm{Mg}_{32}$$\rm{O}_{32}$,liq &   3.582 &   51.41(26) &   $-$261.4(5) &     703(1) &    $-$377.59(9) \\
             . &      . &    $\rm{Mg}_{6}$$\rm{Fe}_{54}$$\rm{O}_{6}$,liq &   7.797 &   50.61(43) &   $-$370.3(5) &     988(1) &    $-$579.56(9) \\
             . &      . &  $\rm{Mg}_{16}$$\rm{Fe}_{38}$$\rm{O}_{16}$,liq &   6.347 &   49.70(24) &   $-$338.0(8) &     995(2) &    $-$540.61(8) \\
             . &      . &  $\rm{Mg}_{30}$$\rm{Fe}_{13}$$\rm{O}_{30}$,liq &   4.461 &   49.73(22) &   $-$302.3(7) &     911(2) &    $-$470.07(8) \\
             . &      . &   $\rm{Mg}_{36}$$\rm{Fe}_{4}$$\rm{O}_{36}$,liq &   3.838 &    47.6(12) &     $-$317(3) &     855(8) &     $-$459.4(1) \\
             . &      . &  $\rm{Mg}_{30}$$\rm{Fe}_{30}$$\rm{O}_{30}$,liq &   5.237 &   47.94(21) &     $-$400(1) &    1190(3) &    $-$627.26(8) \\
             . &      . &  $\rm{Mg}_{48}$$\rm{Fe}_{48}$$\rm{O}_{48}$,liq &   5.237 &   48.49(21) &     $-$640(1) &    1908(3) &    $-$1004.9(2) \\
             . &      . &                             $\rm{Fe}_{64}$,liq &   8.805 &   50.57(18) &   $-$407.6(3) &   947.6(6) &    $-$605.53(2) \\
             . &      . &                $\rm{Mg}_{40}$$\rm{O}_{40}$,liq &   3.582 &    40.2(11) &     $-$365(3) &     788(8) &     $-$470.9(2) \\
             . &      . &                $\rm{Mg}_{50}$$\rm{O}_{50}$,liq &   3.582 &   44.43(37) &     $-$438(2) &    1023(4) &     $-$587.2(2) \\
            50 &   6000 &  $\rm{Mg}_{24}$$\rm{Fe}_{24}$$\rm{O}_{24}$,liq &   5.083 &   49.36(25) &   $-$291.5(7) &    1027(2) &     $-$587.3(1) \\
             . &      . &                             $\rm{Fe}_{32}$,liq &   8.537 &   50.95(21) &   $-$191.6(1) &   506.5(3) &    $-$345.03(2) \\
             . &      . &                $\rm{Mg}_{32}$$\rm{O}_{32}$,liq &   3.446 &   49.50(24) &   $-$243.1(5) &     754(1) &    $-$439.09(4) \\
            50 &   7500 &  $\rm{Mg}_{24}$$\rm{Fe}_{24}$$\rm{O}_{24}$,liq &   4.868 &   50.09(41) &     $-$254(2) &    1110(3) &     $-$725.7(1) \\
             . &      . &                             $\rm{Fe}_{32}$,liq &   8.164 &   50.59(52) &   $-$173.9(3) &   545.9(5) &    $-$413.29(2) \\
             . &      . &                $\rm{Mg}_{32}$$\rm{O}_{32}$,liq &   3.301 &   51.09(60) &     $-$212(1) &     823(2) &    $-$540.85(6) \\
             . &      . &    $\rm{Mg}_{6}$$\rm{Fe}_{54}$$\rm{O}_{6}$,liq &   7.221 &   49.04(17) &   $-$317.2(3) &  1119.5(5) &    $-$806.98(5) \\
             . &      . &  $\rm{Mg}_{16}$$\rm{Fe}_{38}$$\rm{O}_{16}$,liq &   5.867 &   49.54(29) &   $-$278.9(8) &    1142(1) &    $-$772.40(6) \\
             . &      . &  $\rm{Mg}_{30}$$\rm{Fe}_{13}$$\rm{O}_{30}$,liq &   4.115 &   49.90(27) &   $-$243.4(7) &    1058(1) &    $-$683.67(8) \\
             . &      . &   $\rm{Mg}_{36}$$\rm{Fe}_{4}$$\rm{O}_{36}$,liq &   3.538 &   50.10(17) &   $-$250.7(7) &    1020(1) &    $-$664.90(8) \\
             . &      . &  $\rm{Mg}_{30}$$\rm{Fe}_{30}$$\rm{O}_{30}$,liq &   4.868 &   49.80(16) &     $-$321(1) &    1382(2) &    $-$907.77(9) \\
              . &      . &  $\rm{Mg}_{48}$$\rm{Fe}_{48}$$\rm{O}_{48}$,liq &   4.868 &   50.13(15) &     $-$507(1) &    2222(2) &    $-$1451.7(2) \\
              . &      . &                             $\rm{Fe}_{64}$,liq &   8.164 &   50.62(24) &   $-$347.7(3) &  1090.7(5) &    $-$825.75(2) \\
              . &      . &                $\rm{Mg}_{40}$$\rm{O}_{40}$,liq &   3.301 &   50.75(23) &   $-$264.9(7) &    1026(1) &    $-$674.98(4) \\
              . &      . &                $\rm{Mg}_{50}$$\rm{O}_{50}$,liq &   3.301 &   51.40(27) &     $-$330(1) &    1284(2) &    $-$843.19(7) \\
%% Note: The commented out data in the table is repeated in a continuation of the table,
%% so that it nicely continues onto a second page.
%%             50 &  10000 &  $\rm{Mg}_{24}$$\rm{Fe}_{24}$$\rm{O}_{24}$,liq &   4.546 &   51.31(32) &     -197(1) &    1210(2) &     -976.4(1) \\
%%              . &      . &                             $\rm{Fe}_{32}$,liq &   7.536 &   49.92(47) &   -142.0(2) &   601.3(3) &    -537.28(3) \\
%%              . &      . &                $\rm{Mg}_{32}$$\rm{O}_{32}$,liq &   3.054 &   50.64(51) &     -163(1) &     909(2) &     -727.8(1) \\
%%            100 &   6000 &  $\rm{Mg}_{24}$$\rm{Fe}_{24}$$\rm{O}_{24}$,liq &   6.144 &  100.54(60) &     -264(1) &     969(3) &     -375.5(1) \\
%%              . &      . &                             $\rm{Fe}_{32}$,liq &   9.771 &   97.45(69) &   -188.0(5) &     476(1) &    -244.43(4) \\
%%              . &      . &                $\rm{Mg}_{32}$$\rm{O}_{32}$,sol &   4.432 &  100.04(18) &   -261.7(3) &   598.8(7) &    -269.70(8) \\
%%            100 &   6700 &  $\rm{Mg}_{24}$$\rm{Fe}_{24}$$\rm{O}_{24}$,liq &   6.042 &   98.34(38) &   -254.2(7) &     998(1) &     -434.5(1) \\
%%              . &      . &                             $\rm{Fe}_{32}$,liq &   9.660 &   99.69(82) &   -179.6(5) &     495(1) &    -273.67(4) \\
%%              . &      . &                $\rm{Mg}_{32}$$\rm{O}_{32}$,sol &   4.381 &  100.46(35) &   -250.5(6) &     627(1) &    -307.19(8) \\
%%            100 &   7000 &  $\rm{Mg}_{24}$$\rm{Fe}_{24}$$\rm{O}_{24}$,liq &   6.027 &  101.60(35) &   -238.4(8) &    1027(1) &     -461.2(1) \\
%%              . &      . &                             $\rm{Fe}_{32}$,liq &   9.611 &  100.24(30) &   -176.2(2) &   502.5(3) &    -286.64(2) \\
%%              . &      . &                $\rm{Mg}_{32}$$\rm{O}_{32}$,sol &   4.360 &  100.53(42) &   -245.4(3) &   637.3(6) &    -323.25(7) \\
%%            100 &   8000 &  $\rm{Mg}_{24}$$\rm{Fe}_{24}$$\rm{O}_{24}$,liq &   5.932 &  103.41(68) &     -218(1) &    1068(2) &     -551.3(2) \\
%%              . &      . &                             $\rm{Fe}_{32}$,liq &   9.388 &   99.43(71) &   -165.0(5) &   527.1(8) &    -331.12(4) \\
%%              . &      . &                $\rm{Mg}_{32}$$\rm{O}_{32}$,liq &   4.034 &  100.57(67) &   -183.8(7) &     785(1) &    -393.43(7) \\
%%            100 &  10000 &  $\rm{Mg}_{24}$$\rm{Fe}_{24}$$\rm{O}_{24}$,liq &   5.664 &  100.86(98) &     -176(2) &    1147(2) &     -742.5(2) \\
%%              . &      . &                             $\rm{Fe}_{32}$,liq &   8.998 &   98.79(49) &   -142.5(4) &   567.2(4) &    -425.41(2) \\
%%              . &      . &                $\rm{Mg}_{32}$$\rm{O}_{32}$,liq &   3.842 &   99.23(52) &     -145(2) &     856(2) &    -534.78(8) \\
%%            100 &  15000 &  $\rm{Mg}_{24}$$\rm{Fe}_{24}$$\rm{O}_{24}$,liq &   5.157 &  101.17(74) &      -75(2) &    1281(2) &  -1267.13(10) \\
%%              . &      . &                             $\rm{Fe}_{32}$,liq &   8.145 &  102.48(80) &    -81.2(5) &   645.5(4) &    -688.15(2) \\
%%            400 &   8000 &  $\rm{Mg}_{24}$$\rm{Fe}_{24}$$\rm{O}_{24}$,liq &   9.117 &  398.69(56) &      -24(1) &     921(2) &      390.9(2) \\
%%              . &      . &                             $\rm{Fe}_{32}$,liq &  13.469 &   397.7(10) &    -93.1(8) &     453(2) &      145.0(2) \\
%%              . &      . &                $\rm{Mg}_{32}$$\rm{O}_{32}$,sol &   6.487 &  399.95(25) &    -69.8(3) &   580.4(5) &      354.3(1) \\
%%            400 &   9300 &  $\rm{Mg}_{24}$$\rm{Fe}_{24}$$\rm{O}_{24}$,liq &   9.010 &  400.51(49) &        5(2) &     974(2) &      285.7(2) \\
%%              . &      . &                             $\rm{Fe}_{32}$,liq &  13.583 &  398.04(53) &    -95.6(5) &   445.8(7) &      92.59(8) \\
%%              . &      . &                $\rm{Mg}_{32}$$\rm{O}_{32}$,sol &   6.434 &  400.24(54) &    -51.3(5) &   615.1(8) &      286.8(1) \\
%%            400 &  10000 &  $\rm{Mg}_{24}$$\rm{Fe}_{24}$$\rm{O}_{24}$,liq &   8.954 &  399.46(67) &       17(2) &     998(2) &      225.5(2) \\
%%              . &      . &                             $\rm{Fe}_{32}$,liq &  13.263 &   397.6(25) &      -75(2) &     485(2) &       65.3(1) \\
%%              . &      . &                $\rm{Mg}_{32}$$\rm{O}_{32}$,sol &   6.393 &  397.31(43) &    -41.8(3) &   632.4(5) &      249.7(1) \\
%%            400 &  12500 &  $\rm{Mg}_{24}$$\rm{Fe}_{24}$$\rm{O}_{24}$,liq &   8.753 &  399.86(65) &       63(1) &    1071(1) &        2.5(2) \\
%%              . &      . &                             $\rm{Fe}_{32}$,liq &  12.938 &   400.3(12) &    -46.0(9) &   530.1(9) &     -44.36(5) \\
%%              . &      . &                $\rm{Mg}_{32}$$\rm{O}_{32}$,liq &   6.100 &  396.44(75) &       61(1) &     794(1) &       82.3(1) \\
%%            400 &  15000 &  $\rm{Mg}_{24}$$\rm{Fe}_{24}$$\rm{O}_{24}$,liq &   8.553 &   399.7(11) &      117(2) &    1138(2) &     -235.9(1) \\
%%              . &      . &                             $\rm{Fe}_{32}$,liq &  12.688 &  403.62(99) &    -19.3(7) &   562.1(6) &    -161.93(3) \\
%%              . &      . &                $\rm{Mg}_{32}$$\rm{O}_{32}$,liq &   5.943 &   396.6(11) &      114(2) &     858(2) &      -95.8(1) \\
\hline
\end{tabular}
\end{adjustbox}
\end{table}

\begin{table}[hp]
\centering

\caption{Thermodynamic functions derived from DFT-MD simulations (Part 2/2).}
\begin{adjustbox}{max width=\textwidth}
\begin{tabular}{ccllllll}
    \hline
 {$P$} & {$T$} & {system} &  {$\rho$} & 
 {$P$} & {$U$} & {$S$} &  {$G$}  \\
 {(GPa)} & {(K)} & &  {($\mathrm{g}/\mathrm{cm}^3$)} 
   & {(GPa)} & {(eV)} & {($k_B$)} & {(eV)} \\
   \hline
            50 &  10000 &  $\rm{Mg}_{24}$$\rm{Fe}_{24}$$\rm{O}_{24}$,liq &   4.546 &   51.31(32) &     $-$197(1) &    1210(2) &     $-$976.4(1) \\
             . &      . &                             $\rm{Fe}_{32}$,liq &   7.536 &   49.92(47) &   $-$142.0(2) &   601.3(3) &    $-$537.28(3) \\
             . &      . &                $\rm{Mg}_{32}$$\rm{O}_{32}$,liq &   3.054 &   50.64(51) &     $-$163(1) &     909(2) &     $-$727.8(1) \\
           100 &   6000 &  $\rm{Mg}_{24}$$\rm{Fe}_{24}$$\rm{O}_{24}$,liq &   6.144 &  100.54(60) &     $-$264(1) &     969(3) &     $-$375.5(1) \\
             . &      . &                             $\rm{Fe}_{32}$,liq &   9.771 &   97.45(69) &   $-$188.0(5) &     476(1) &    $-$244.43(4) \\
             . &      . &                $\rm{Mg}_{32}$$\rm{O}_{32}$,sol &   4.432 &  100.04(18) &   $-$261.7(3) &   598.8(7) &    $-$269.70(8) \\
           100 &   6700 &  $\rm{Mg}_{24}$$\rm{Fe}_{24}$$\rm{O}_{24}$,liq &   6.042 &   98.34(38) &   $-$254.2(7) &     998(1) &     $-$434.5(1) \\
             . &      . &                             $\rm{Fe}_{32}$,liq &   9.660 &   99.69(82) &   $-$179.6(5) &     495(1) &    $-$273.67(4) \\
             . &      . &                $\rm{Mg}_{32}$$\rm{O}_{32}$,sol &   4.381 &  100.46(35) &   $-$250.5(6) &     627(1) &    $-$307.19(8) \\
           100 &   7000 &  $\rm{Mg}_{24}$$\rm{Fe}_{24}$$\rm{O}_{24}$,liq &   6.027 &  101.60(35) &   $-$238.4(8) &    1027(1) &     $-$461.2(1) \\
             . &      . &                             $\rm{Fe}_{32}$,liq &   9.611 &  100.24(30) &   $-$176.2(2) &   502.5(3) &    $-$286.64(2) \\
             . &      . &                $\rm{Mg}_{32}$$\rm{O}_{32}$,sol &   4.360 &  100.53(42) &   $-$245.4(3) &   637.3(6) &    $-$323.25(7) \\
           100 &   8000 &  $\rm{Mg}_{24}$$\rm{Fe}_{24}$$\rm{O}_{24}$,liq &   5.932 &  103.41(68) &     $-$218(1) &    1068(2) &     $-$551.3(2) \\
             . &      . &                             $\rm{Fe}_{32}$,liq &   9.388 &   99.43(71) &   $-$165.0(5) &   527.1(8) &    $-$331.12(4) \\
             . &      . &                $\rm{Mg}_{32}$$\rm{O}_{32}$,liq &   4.034 &  100.57(67) &   $-$183.8(7) &     785(1) &    $-$393.43(7) \\
           100 &  10000 &  $\rm{Mg}_{24}$$\rm{Fe}_{24}$$\rm{O}_{24}$,liq &   5.664 &  100.86(98) &     $-$176(2) &    1147(2) &     $-$742.5(2) \\
             . &      . &                             $\rm{Fe}_{32}$,liq &   8.998 &   98.79(49) &   $-$142.5(4) &   567.2(4) &    $-$425.41(2) \\
             . &      . &                $\rm{Mg}_{32}$$\rm{O}_{32}$,liq &   3.842 &   99.23(52) &     $-$145(2) &     856(2) &    $-$534.78(8) \\
           100 &  15000 &  $\rm{Mg}_{24}$$\rm{Fe}_{24}$$\rm{O}_{24}$,liq &   5.157 &  101.17(74) &      $-$75(2) &    1281(2) &  $-$1267.13(10) \\
             . &      . &                             $\rm{Fe}_{32}$,liq &   8.145 &  102.48(80) &    $-$81.2(5) &   645.5(4) &    $-$688.15(2) \\
           400 &   8000 &  $\rm{Mg}_{24}$$\rm{Fe}_{24}$$\rm{O}_{24}$,liq &   9.117 &  398.69(56) &      $-$24(1) &     921(2) &      390.9(2) \\
             . &      . &                             $\rm{Fe}_{32}$,liq &  13.469 &   397.7(10) &    $-$93.1(8) &     453(2) &      145.0(2) \\
             . &      . &                $\rm{Mg}_{32}$$\rm{O}_{32}$,sol &   6.487 &  399.95(25) &    $-$69.8(3) &   580.4(5) &      354.3(1) \\
           400 &   9300 &  $\rm{Mg}_{24}$$\rm{Fe}_{24}$$\rm{O}_{24}$,liq &   9.010 &  400.51(49) &        5(2) &     974(2) &      285.7(2) \\
             . &      . &                             $\rm{Fe}_{32}$,liq &  13.583 &  398.04(53) &    $-$95.6(5) &   445.8(7) &      92.59(8) \\
             . &      . &                $\rm{Mg}_{32}$$\rm{O}_{32}$,sol &   6.434 &  400.24(54) &    $-$51.3(5) &   615.1(8) &      286.8(1) \\
           400 &  10000 &  $\rm{Mg}_{24}$$\rm{Fe}_{24}$$\rm{O}_{24}$,liq &   8.954 &  399.46(67) &       17(2) &     998(2) &      225.5(2) \\
             . &      . &                             $\rm{Fe}_{32}$,liq &  13.263 &   397.6(25) &      $-$75(2) &     485(2) &       65.3(1) \\
             . &      . &                $\rm{Mg}_{32}$$\rm{O}_{32}$,sol &   6.393 &  397.31(43) &    $-$41.8(3) &   632.4(5) &      249.7(1) \\
           400 &  12500 &  $\rm{Mg}_{24}$$\rm{Fe}_{24}$$\rm{O}_{24}$,liq &   8.753 &  399.86(65) &       63(1) &    1071(1) &        2.5(2) \\
             . &      . &                             $\rm{Fe}_{32}$,liq &  12.938 &   400.3(12) &    $-$46.0(9) &   530.1(9) &     $-$44.36(5) \\
             . &      . &                $\rm{Mg}_{32}$$\rm{O}_{32}$,liq &   6.100 &  396.44(75) &       61(1) &     794(1) &       82.3(1) \\
           400 &  15000 &  $\rm{Mg}_{24}$$\rm{Fe}_{24}$$\rm{O}_{24}$,liq &   8.553 &   399.7(11) &      117(2) &    1138(2) &     $-$235.9(1) \\
             . &      . &                             $\rm{Fe}_{32}$,liq &  12.688 &  403.62(99) &    $-$19.3(7) &   562.1(6) &    $-$161.93(3) \\
             . &      . &                $\rm{Mg}_{32}$$\rm{O}_{32}$,liq &   5.943 &   396.6(11) &      114(2) &     858(2) &      $-$95.8(1) \\
\hline
\end{tabular}
\end{adjustbox}
\end{table}

Simulations of $\rm{Fe}_{32}$, $\rm{Mg}_{32}$$\rm{O}_{32}$ and
$\rm{Mg}_{24}$$\rm{Fe}_{24}$$\rm{O}_{24}$ are included for every $P$-$T$
condition. Additional compositions
($\rm{Mg}_{36}$$\rm{Fe}_{4}$$\rm{O}_{36}$,
$\rm{Mg}_{30}$$\rm{Fe}_{13}$$\rm{O}_{30}$,
$\rm{Mg}_{16}$$\rm{Fe}_{38}$$\rm{O}_{16}$ and
$\rm{Mg}_{6}\rm{Fe}_{54}\rm{O}_{6}$) were performed for 50 GPa at 5000 and
7500 K, to test the compositional dependence of the Fe-MgO solvus.
Finally, simulations with 1:1 stoichiometries but larger cells
$\rm{Fe}_{64}$, $\rm{Mg}_{30}\rm{Fe}_{30}\rm{O}_{30}$,
$\rm{Mg}_{45}\rm{Fe}_{45}\rm{O}_{45}$ $\rm{Mg}_{40}$$\rm{O}_{40}$, and
$\rm{Mg}_{50}$$\rm{O}_{50}$.




\subsection{Saturation limits} \label{sec:saturation}

$\Delta G_{mix}$ can be related to the volume change associated with the
insertion of an iron of atom into hydrogen, as other contributions are
constant with respect concentration. It can be shown that results for
simulations with a 1:$n$ solute ratio can be generalized to a ratio of
1:$m$ using
\begin{eqnarray}
  \Delta G_c &\approx& F_0(Fe_mMgO)-F_0(Fe_m)-F_0(MgO) \nonumber \\ && -
  \left[F_0(Fe_nMgO)-F_0(Fe_n)-F_0(MgO)\right] \nonumber \\ &=&
  -k_BT\log\left\{ \frac{\left[V(\mathrm{Fe}_n\mathrm{MgO}) +
    \frac{m-n}{n}V(\mathrm{Fe}_n)\right]^{m+1}
    \left[V(\mathrm{Fe}_n)\right]^{n}}
    {\left[V(\mathrm{Fe}_n)\frac{m}{n}\right]^m
    \left[V(\mathrm{Fe}_n\mathrm{MgO})\right]^{n + 1}} \right\},
\end{eqnarray}
where $\Delta G_c = \Delta G_{mix}(1:m)- \Delta G_{mix}(1:n)$, and
$V(\mathrm{Fe}_n)$ and $V(\mathrm{Fe}_n\mathrm{MgO})$ are the volumes for
the simulations of hydrogen and the solution respectively. This allows us
to approximate the saturation limit for MgO in Fe based only on $\Delta
G_{mix}$ and $V$ of our lowest concentration simulation, and $V$ of both of
the endmember compositions.  We note that this low-concentration limit
assumes that the self interaction of the MgO `solute' is negligible in our
lowest concentration, $\rm{Mg}_{6}\rm{Fe}_{54}\rm{O}_{6}$. While this is
not exact, we present it as a estimate for extrapolating these results to
low MgO concentrations. In doing so, we demonstrate that these calculations
are consistent with Mg concentrations the below detection limit of
laser-heated diamond anvil cell experiments performed at lower
temperatures.

\section{Discussion} \label{sec:discussion}

In the extreme case where a significant fraction of the planet is in a mixed iron-rock
phase, the early evolution will be quite different than prevailing theories.
Differentiation of material accreted onto the planet is delayed until the planet cools to
below the solvus closure temperature, allowing iron to exsolve and sink to the core.
This study provides an estimate of the temperatures required to mix the Mg-rich rocky
mantle with the core of a terrestrial planet. At the surface, the complete mixing of Fe
and MgO is achieved at $\sim$4000 K (Fig. \ref{fig:summary}), which is well above the
melting point of silicates. At core-mantle boundary pressures, the critical temperature
would be $\sim$7000 K.  This is below higher estimates for the melting temperature of
pure MgO \citep{Alfe2005,Belonoshko2010,Boates2013} and $\rm{MgSiO_3}$ perovskite
\citep{Zerr1993}. There are significant disparities among calculations and experiments
on the melting \citep{Belonoshko1994,Belonoshko1997,Alfe2005,Zerr1998}
temperatures in the lower mantle, disagreeing even on which phases represent the solidus
and liquidus. The melting behavior in our MgO simulations are consistent with the
high-temperature melting curve of recent first-principles simulations
\citep{Alfe2005,Belonoshko2010,Boates2013}.  Regardless, the solvus remains well above
the solidus for more realistic compositions of the silicate mantle
\citep{deKoker2013,Zerr1998,Holland1997}.

\begin{figure}[h!]  
  \centering
    \includegraphics[width=22pc]{figs/solvus.pdf}
\caption{Pressure dependence of the solvus closure. The $P$-$T$ condition of all
  thermodynamic integration calculations are included. Blue markers denote conditions
  where MgO was treated as a liquid. Green markers denote conditions where MgO was
  treated as a solid (B1).  Red circles show the solvus closure temperature inferred from
  simulations at the same pressure. The estimated uncertainty in the solvus closure
  temperature from finite size effects is shown by the filled red region.  The dashed,
  black line shows the MgO melting temperature from molecular dynamics from DFT-md with
  PBE exchange correlation function\citep{Boates2013}, which is consistent with other
  first-principles calculations \citep{Alfe2005,Belonoshko2010}}
\label{fig:summary}
\end{figure}


\subsection{Evolution of a fully mixed planet}

For a sufficiently energetic impact, or series of impacts, a planet might be heated to
such high temperatures, that the entire planet maybe be an approximately homogeneous
mixture of the iron and rock components. Such an extreme scenario is unlikely for an
Earth-sized planet, and likely violates geochemical observations that preclude complete
mixing of the Earth's primitive mantle \citep{Mukhopadhyay2012}. Nonetheless, considering the
evolution of a planet from a fully-mixed state is useful for demonstrating the effects
our phase diagram on the mixing behavior in a planet. A fully mixed state is also not so
far-fetched for super-Earths since heating from release of gravitational energy scales as
roughly $M^{2/3}$. %$(M/M_{\Earth})^{2/3}$. 
Fig.~\ref{fig:mantlecoremixing} shows a schematic diagram detailing some of the
processes involved with the formation of this fully mixed state.

\begin{figure}[h!]  
  \centering
    \includegraphics[width=22pc]{figs/mantlecoremixing.eps}  
\caption{ Schematic diagram showing the sequence of physical processes proposed
for core mantle mixing in the aftermath of a giant impact.}
\label{fig:mantlecoremixing}
\end{figure}

The depth at which the phases separate from the fully mixed state is determined by the
pressure dependence of solvus closure. Following such a large impact, the planet will
quickly evolve to a magma ocean state, and a higher-temperature adiabat will be rapidly
re-established. Fig. \ref{fig:isentropes} compares the solvus closure temperature to the
calculated isentropes of the mixed FeMgO phase. For a homogeneous, vigorously convecting
liquid layer of the planet, these approximate adiabatic temperature profiles of the
interior of the planet at different points in its evolution.  The comparison is
qualitatively the same if the isentropes for either endmember is used instead of the
mixed phase. At pressures above 50 GPa, the isentropes have a notably steeper slope than
the solvus closure temperature. At lower pressures, $<$50 GPa the slopes are identical within the
estimated uncertainty. As a result, separation begins in the exterior of the planet and
proceeds inwards as the planet cools. Since iron separating in the outer portion of the
planet is denser than the rocky phase, it would sink until it reached a depth where it
dissolves into the mixed phase again. This may promote compositional stratification, and
possibly multi-layer convection between an upper iron-poor and deeper iron-rich layer.
The extent to which this process can stratify the planet depends on the competition
between growth of liquid Fe droplets and their entrainment in convective flows
\citep{Solomatov2007}.

Based on the Fe-MgO solvus closure temperature presented here, transition of a planet
from a fully mixed state to separated rocky and metallic phases would occur while the entire
planet is at least partially molten. Consequently, a fully mixed state in an Earth-mass
planet would be short lived, since cooling timescales for a deep magma ocean are short in
comparison to the timescale of accretion \citep{Abe1997,Elkins-Tanton2012}. This also
means little record of such an event is likely to survive to the present day Earth. Indeed,
there is little unambiguous evidence for a magma ocean, despite it being a seemingly
unavoidable consequence of the moon-forming impact hypothesis. The high surface
temperatures of some rocky exoplanets \citep{Pepe2013} might allow for prolonged cooling
times from a such a mixed state.

\begin{figure}[h!]  
  \centering
    \includegraphics[width=22pc]{figs/isentropes.pdf}  
\caption{Calculated isentropes for the mixed FeMgO liquid phase compared to the solvus
  closure temperature. These results favor the mixed phase to remain stable at depth. The
  dashed, black line shows the MgO melting temperature from \citep{Boates2013}.The filled
  regions represent the propagation of estimated errors from finite size effects.}
\label{fig:isentropes}
\end{figure}

At the relatively low pressures of growing planetesimals
\citep{Kleine2002,Sramek2012,Weiss2013}, these results predict that core formation begins
at temperatures well below the solvus. As a result, complete mixing of a planet must
overcome the gravitationally stable differentiated structure. This will impede upward
mixing of a dense core even at temperatures above solvus closure, leading to an inefficient
double-diffusive convection state like that proposed for the giant planets
\citep{Chabrier2007}. Material accreted while the planet was above the solvus would,
however, remain in a fully mixed outer layer, and evolve according to the picture
presented in Fig. \ref{fig:isentropes}. Substantial mechanical mixing during giant impact
events \citep{canup2004,Canup2012,Cuk2012} would also enhance mixing prior to the setup of
a double-diffusive state.

\subsection{Consequences of localized heating}

Despite the implausibility of a fully-mixed earth, related processes may become important
as material is added by impacts with smaller differentiated bodies at temperatures near
or above solvus closure. Since peak shock temperatures are related to the velocity of the
impact rather than the size of the impactor, smaller-scale events can create localized
regions where the temperature exceeds the solvus closure temperature.  Assuming iron from
the shocked region can be rapidly delivered to the core without significant cooling
\citep{Monteux2009}, material equilibrated near or above the solvus can be delivered to
the core, through a mantle of lower average temperature. In the case that heat transfer
from the sinking iron diapir is negligible, the comparison between the solvus closure
temperature and adiabats is valid for the fraction of the planet in contact with the
sinking iron. In other words, the temperature in the sinking iron will follow a nearly
adiabatic path, with Fe and MgO becoming more soluble as the pressure rises. This means
that a fraction of the iron delivered to the core could have equilibrated with the rocky
mantle at much higher temperatures than on average. The differentiation of a fraction of
the planet in the presence of a mixed phase would likely effect partitioning of both major
and minor elements between the core and mantle. \cite{Walker1993}
suggested that deviations in siderophile element partitioning behavior occur near the
solvus closure temperature for iron-silicate mixtures. However, this interpretation has
been questioned in light of the confounding effect of drastic changes in oxygen partitioning
with pressure \citep{Frost2010}. Better characterization of element-partitioning at such
high temperatures could constrain what fraction of the mantle could have been
equilibrated in this fashion.


\begin{figure}[h!]  
  \centering
    \includegraphics[width=30pc]{figs/exsolution_figure_2.eps}  
\caption{ A thermodynamic model depicting a hypothetical concentration curve of Mg in
    iron as a function of temperature: Points above the curve are super-saturated.
    Rapid mixing of a ‘cold’ reservoir ‘A’ with a ‘hot’ reservoir ‘B’ results in an
    intermediate, super-saturated state ‘C’. The extent of exsolution predicted for
    state C, depends on the super-saturation, and thus the shape of the concentration
    curve. 
}
\label{fig:exsolution}
\end{figure}


One important consequence of high-temperature equilibration is the delivery of excess,
nominally insoluble, light components to the core. This will occur if iron is
equilibrated with rocky materials at near-solvus temperatures, and rapidly delivered to
the core before it can cool and re-equilibrate with the mantle at lower temperatures.
This would be followed by exsolution of a Mg-rich material at the top of the cooling
core. This process has been suggested as a possible solution to the problem of the
Earth's core having  insufficient energy to generate a magnetic field before nucleation
of the inner core \citep{Stevenson2012}. Fig.~\ref{fig:exsolution} details the
energetics of such a process. If the interpretation of Fig.
\ref{fig:isentropes} can be extended to more iron rich compositions, then exsolution will
occur at the top of the core, depositing sediments of Mg-rich material at the core-mantle
boundary \citep{buffett2000}. As a result, the effect of this sedimentation on core
convection is analogous to the exclusion of light elements from the growing inner core.
Fig. \ref{fig:concentrations} shows an extrapolation of our results to predict the
saturation of MgO in Fe at 50 GPa as a function of temperature.  This is done using a
function for $G$ in terms of the cell volumes derived in the low-concentration limit
\citep{Wilson2012a,Wahl2013}, and details are presented in the supplemental material. From
this we predict a $>$1\% MgO saturation limit down to 4200 K, with concentrations steeply
decreasing to be below detection limits in high-pressure experiments by $\sim$3000 K
\citep{Knittle1991,Ozawa2008a}. In principle, high-temperature equilibration could also
explain a bulk-mantle iron concentration in disequilibrium with the present-day core
\citep{Ozawa2008a,Tsuno2013}. However, the shape of the MgO-rich side of the calculated
exsolution gap (Fig. \ref{fig:solvus}) contradicts this, since the Fe content of the MgO
endmember shows a significant deviation for only a small range of temperatures.

\begin{figure}[h!]  
  \centering
    \includegraphics[width=22pc]{figs/concentrations.pdf}  
\caption{Extrapolated saturation limits of MgO in Fe at 50 GPa at various temperatures.
  Extrapolation is under the assumption that the solution behaves in the
  low-concentration limit. The dashed vertical line is the most Fe-rich composition from
  Fig. \ref{fig:convex_hull}, from which the extrapolation is made. The filled regions
  represent the propagation of estimated errors from finite size effects.}
\label{fig:concentrations}
\end{figure}

The solvus closure temperatures for material with the bulk terrestrial planet composition
marks the transition to a regime where where miscibility is a dominant effect in the
evolution of the planet. These results present an estimate of those temperatures based on
the simplified Fe-MgO system. Where possible, our simulated system was chosen to provide
an upper limit for these temperatures, so we expect miscibility for realistic terrestrial
compositions at possibly lower temperatures than those found here. The solvus closure
temperature found here for the Fe-MgO system is at temperatures low enough, that it was
likely overcome for some fraction of the planet during accretion. Energetic impact events
are now thought to have been commonplace during the formation of the terrestrial planets,
and the role of miscibility between the most abundant rocky and metallic materials should
be considered to adequately assess their early evolution.




\chapter{Thermodynamics of Convection with a Phase Transition}\label{chap5}


\section{Motivation}

One of the more surprising findings of the MESSENGER spacecraft to Mercury was
the confirmation that the smallest terrestrial planet has an internally
generated, dipolar magnetic field, which is likely driven by a combination of
thermal and compositional buoyancy sources. This observation places constraints
on the thermal and energetic state of Mercury's large iron core and on mantle
dynamics because dynamo operation is strongly dependent on the amount of heat
extracted from the core by the mantle. However, other observations point to
several factors that should inhibit a present-day dynamo. These include
physical constraints on a thin, possibly non-convecting mantle, as well as
properties of liquid iron alloys that promote compositional stratification in
the core.

The lack of a simple relationship between the size of the planetary body and the presence of a
magnetic field in terrestrial planets of moons is striking. The thermodynamics of
dynamo generation exhibit a competition between heat loss by convection and heat loss
by conduction.  Dipolar magnetic fields arise from helical flows that develop within
a rotating conducting liquid undergoing turbulent convection. The properties of iron
alloys, and particularly their melting temperature, is strongly influenced by the
presence of light elements \citep{sanloup2000}. As a result, the existence of a
magnetic field depends on planet formation and evolution in a complicated fashion. I
propose to study the influence of non-ideal mixing behavior in liquid alloys of iron
and sulfur on thermal and compositional convection. Mercury's magnetic field strength
has posed problems for standard dynamo models \citep{Christensen2006,Stanley2005},
and partial crystallization resulting from non-ideal mixing provides a possible
mechanism to explain this.

Spacecraft observations have confirmed the presence of internally generated
magnetic fields for Mercury \citep{Anderson2011} and Ganymede \cite{Kivelson1996}.
The internal structure of both bodies is constrained by measurements of
gravitational moments \citep{Smith2012,Hauck2006}. However, these measurements are
not sufficiently precise to determine what portion of the cores are liquid, nor
how much light component is contained in the cores. Thermal evolution
calculations of both planets \citep{Hauck2004,Hauck2006,Breuer2007,Bland2008} suggest
that several wt.\% S is necessary to preserve a substantial unfrozen layer in
the core, despite inefficient stagnant-lid convection
\citep{Solomatov2000,Hauck2004,Breuer2007} and tidal heating from a hypothesized
resonance in Ganymede's orbital history \citep{Showman1997,Bland2008}. Some models
for Mercury's formation suggest minimal accretion of volatile elements like
sulfur, but surface observations from MESSENGER \citep{Nittler2011,Mccubbin2012} are
inconsistent with extensive devolatilization.

The primary goals of this work is to develop an automated system to generate an
interpolated thermodynamic model using experimentally determined, eutectic
phase diagrams, to calculate adiabatic profiles for the thermodynamic model
using a parcel method, and to evaluate the effect calculated adiabatic profiles
have on global budgets of energy and entropy. This work was never published and
is presented here in an incomplete state.

A part of this work was integrated with and added upon as part of a Cooperative
Institute for Dynamic Earth Research (CIDER) summer program project. This includes
work by the present author in collaboration with Brent Delbridge, and Ian Rose, of UC
Berkeley, and Grace Cox of the University of Leeds, under the advisorship of Jessica
Irving (Princeton), Bill McDonough and Laurent Montesi (Maryland), and  the last two
sections of this chapter present results from this collaboration.

\section{Iron alloy properties}

It has long been recognized that alloying components are abundant in the cores
of terrestrial planets and that they must play an important role in thermal
evolution and dynamo generation. Indeed, it is now largely accepted that the
exclusion of a light, alloying component is the most important contributor to
convection in the Earth's core \citep{Lister1995}. Therefore,
consolidating the present understanding about chemical state of Mercury's
interior is essential for determining the state of
Mercury's dynamo.

For sufficiently high concentrations of S and Si, the alloy encounters a
liquid-liquid immiscibility gap leading to the partitioning of these elements between
different phases. Within the pressure range and for reasonable compositions, an “iron
snow” state \citep{Chen2008,Williams2009}, where iron crystallization initiates in
the outer portions of the core, must also be considered for Mercury. We, therefore,
require a coupled model of chemistry and thermal evolution for Mercury's core to
determine the constraints on composition of the core based on the planets
gravitational moments \citep{Smith2012}, and combine this with constraints from
entropy budget calculations.  Since these constraints are limited, it is necessary to
develop a means of testing a large number of possible interior structures and
compositions.

 \begin{figure}[h] %  figure placement: here, top, bottom, or page
   \centering
   \includegraphics[width=26pc]{figs/liquidi.png} 
   \caption{Liquidus relationships for Fe-S alloys generated from the interpolated thermodynamic 
   model for a range of light-element composition in wt.\% S. The sharp peak and trough lead to 
   a region of partial crystallization for a range of thermal states of the core.}
   \label{fig:liquidi}
\end{figure}

The pressures present in the cores of Mercury ($\sim$8$-$40 GPa) and Ganymede
($\sim$8$-$12 GPa) are significantly lower than those for the Earth's core and
accessible to a wider variety of experimental techniques. At these pressures, 
FeS has been discovered to undergo multiple first-order phase transitions
\citep{Fei1997,Fei2000}, stabilizing new phases $Fe_3S_2$ and $Fe_3S$ at 14 and 21 GPa 
respectively. Fe-S melts undergo analogous changes in compacity
\citep{Morard2007} and associated deviations from ideal mixing behavior
\citep{Chen2008}. The Fe-FeS system shows eutectic melting behavior with eutectic
sulfur composition decreasing from $\sim$30 wt.\% S at ambient pressure to $\sim$12 
wt.\% S at 40 GPa \citep{Chudinovskikh2007}. The eutectic temperature shows
anomalous behavior over a pressure range $\sim$5$-$20 GPa \citep{Fei1997,Chen2008}.
\citet{Chen2008} also found liquidus temperatures at intermediate
compositions on the Fe side of the eutectic to deviate from those predicted by
an ideal mixing model. These anomalous features in the Fe-S phase diagram lead
to the prediction of `iron snow', partial crystallization near the top of the
core with the lower portion remaining completely molten \citep{Hauck2006,Chen2008}.
However, this process has yet to be analyzed in a thermodynamically consistent
manner.

\section{Thermodynamic model from material data}

I have created a working `pipeline' in Matlab for generating a thermodynamic model
from experimental data \citep{Brett1969,Fei1997,Chen2008,Stewart2007}. Data for
$X$-$T$ phase diagrams at constant pressure are fit using a smoothing-spline. To best
account for the changing shape and eutectic composition, these spline fits are then
interpolated with $P$ as a linear combination of fits at the two nearest values of
$P$
%
\begin{equation}
  X(P,T) = X_{eut}(P)\sum_{i=1,2}\xi_i(P)\bar{X}_i(\bar{T}),
\end{equation}
%
where $\bar{X}$, $\bar{T}$ are fractional coordinates with respect to the
values at the eutectic and pure Fe endmembers, and $\xi_i$
is linear mixing parameter. For parcel
calculations, derivative relationships between $P$, $X$ and $T$ can be related
using the lever rule. Additional parameters such as density, heat capacities
and latent heat of fusion are included in a fashion allowing them to be
specified as functions of $P$, $X$ and $T$. With this pipeline, it should be
straightforward to repeat the analysis with other eutectic systems, such as
silicate liquids, and test the affect of varying parameters.
The salient feature of the system is the variation of the
liquidus with light-element fraction, shown in Fig.~\ref{fig:liquidi}, as this
determines where crystallization occurs.

 \begin{figure}[h] %  figure placement: here, top, bottom, or page
   \centering
   \includegraphics[width=26pc]{figs/adiabats.png} 
   \caption{$P-T$ profiles for an adiabatic parcel calculations with 6 wt.\% S using the interpolated Fe-S model. Starting 
   temperatures are spaced by 50 K, and integrated from low to high pressure. The solid black line represents 
   the melting temperature for pure Fe, and the dashed line the liquidus at 6 wt.\% S.}
   \label{fig:adiabats}
\end{figure}


\section{Parcel calculations}

Parcel calculations using the interpolated thermodynamic model are performed
by numerical integration of an expression derived from manipulation of the
first law of thermodynamics
%
\begin{equation}
  \label{eqn:firstlaw}
  d T = -d P \frac{\left[-T \sum_i \alpha_i\nu_ix_i +
    L\left(\frac{\partial{X}}{\partial{P}}\right)_T \right] } 
  {\left[ \sum_i C_{P,i}x_i + L \left(
    \frac{\partial{X}}{\partial{T}}\right)_P \right] },
\end{equation}
%
where $\alpha_i$, $c_{P,i}$, $\nu_i$, and $L$ are the coefficient of thermal
expansion, specific heat capacity, specific volume, and specific latent heat,
respectively, for a phase with mass fraction $x_i$. Eqn. \ref{eqn:firstlaw}
enforces the constraint of zero heat transfer as the pressure on the parcel is
changed. Calculations are carried out by specifying a parcel composition,
temperature and starting pressure, and integrating over a range of pressures.
An example calculation for 6 wt.\% S, for a range of starting temperatures is
presented in Fig.~\ref{fig:adiabats}. I find that for reasonable choices of
parameters, the calculated adiabat within the `iron snow' region is maintained
within $\sim$10 degrees of the liquidus over a range of $\sim$50$-$100 degrees
in starting temperatures. Meanwhile, the temperatures outside the adiabat show
negligible perturbation from simple single-phase adiabats.

 \begin{figure}[h] %  figure placement: here, top, bottom, or page
   \centering
\begin{tabular}{cc}
 \includegraphics[width=20pc]{figs/profiles.png} &
 \includegraphics[width=15pc]{figs/Liquidus_model.png} \\
\end{tabular}
   \caption{ Left: pressure, gravity, density, and temperature profiles of an
interior model for Mercury, with 6 wt.\% S.  Right: Liquidus curves for different
pressures in the Fe-FeS system, from compiled and interpolated experimental data
(Wicks and Knezek, pers. comm.), credit: Nick Knezek.}
  \label{fig:interior_model}
\end{figure}


The onset of crystallization in the outer portion of a terrestrial core may
affect convection and dynamo generation in multiple ways. 
\citet{Hauck2006} suggested that settling of crystals might contribute to
convection by releasing gravitational energy. It is unclear, however, that
crystal settling can drive large scale convective motion. The crystal fractions
for the simulation represented in Fig. \ref{fig:adiabats} also remains
relatively small for conditions during which a separate `iron snow' region
exists. One can consider the influence of this process through use of a
perturbed adiabat with standard models for convection and dynamo generation.

\section{Core energy and entropy budgets} \label{budget}

To determine the energy and entropy budgets, I make use of standard iterative
methods for determining a mean core state, constrained by a planet mass, core
mass and a requirement that $P=0$ at the surface (e.g. \citet{Lister1995}). From
this, I obtain radial profiles of material properties along a calculated
adiabat.  I use descriptions of the core energy and entropy budget
\citep{Gubbins1979,Lister1995,Lister2003} to compare the evolution of mean core state
with and without consideration of partial crystallization. Evolution
calculations are simplified by assuming fractional crystallization in the
core has a negligible affect on the thermal state of the mantle, allowing the
results of existing thermal evolution calculations \citep{Hauck2004,Breuer2007} to
be used as boundary conditions. Since the conductive heat flux is proportional
to $\nabla T$, the steepened adiabat causes a decrease in the convective
heat flux. In the description of the entropy flux, this manifests itself as a
term proportional to $k\left(\nabla T / T \right)^2$ arising from the
divergence of the conductive heat flow \citep{Lister2003}. I can compare the
magnitude of this contribution to standard estimates for the contribution from
thermal and compositional convection.


 \begin{figure}[h] %  figure placement: here, top, bottom, or page
   \centering
\begin{tabular}{cc}
 \includegraphics[width=18pc]{figs/clapeyron_1.png} &
 \includegraphics[width=18pc]{figs/core_energetics.png} \\
\end{tabular}
\caption{Left: Clapeyron slopes for different interpolations of the iron
melting curve, compared to adiabats with various parameterizations.  Right:  Models
for latent heat and gravitational energy release from a solidifying core, with
corresponding thermal energy change.}
\label{fig:core_energy}
\end{figure}

The semi-analytic methods presented above allow for analysis of the
affect of non-ideal melting on the energetics over long time-scales. However,
the details of the dynamics of convection are also of importance to magnetic field
generation. Numerical calculations of convection would, therefore, be a useful
supplement to the results presented here. They would help evaluate the validity of
key assumptions, such as the persistence of an adiabatic state through the
`iron snow' region. This could be achieved through modification of the CALYPSO,
a geodynamo code which passes standard benchmarks \citep{Christensen2001}. This
requires implementation of phase tracking, and a contribution to the energy
equation from latent heat release. 


\section{Mercury interior structure model}

As a culmination of a 2014 CIDER Summer Program, we have a working prototype for a
code designed to calculate self-consistent internal structures. The code uses an
efficient iterative procedure to calculate density, gravity, pressure and temperature
profiles. It can also find an inner core radius that is consistent with a given core
temperature profile. In order to do this we have made use of the code base provided
by the BurnMan project \citep{Cottaar2014}, a continuation of a prior CIDER
collaboration. This framework allows for integration of thermodynamic properties of
minerals in a straightforward and consistent manner. A major benefit of designing the
structural model code around BurnMan is the systematic inclusion of uncertainty
in the experimentally determined thermodynamic properties, which will be an important
step in establishing the ability of any given set of constraints to determine the
state of Mercury's core. 

We have developed a 3-layer interior structure model with a growing inner core. This
involves a more complicated version of the calculations presented in
Section~\ref{superearth}, involving more specific descriptions of the in multiple
layers with attention paid to the conditions at the interfaces.  Material properties
are calculated using a Mie-Gr\"{u}neisen-Debye EOS, using the BurnMan code. It finds
adiabatic temperature profiles consistent with the pressure of the inner core
boundary and the composition of the liquid. The interior planet model is given a list
of different layers of a given mass and homogeneous composition. 

The BurnMan code is then called upon to calculate a $P-T$ barotrope for a given
mantle mineralogy and composition. We adapted the {\verb burnman.Mineral }
class to describe an equation of state for solid and liquid iron alloys.
For the solids we used experimental data for pure gamma iron and a 17 wt. \% Si solid
solution \citep{Tsujino2013,Gleason2013,Lin2003}. For the liquid we
used the results of experiments of pure iron \citep{Anderson1994,Desai1986}, as well
as solutions with sulfur and silicon \citep{Balog2003,Kaiura1979}. For the liquids
the Gr\"{u}neisen is not well defined, although we choose to use this
approximation in order to utilize existing BurnMan code for the core materials.

Included in the interior model is a simple treatment of the partitioning of light
elements between the solid inner and liquid outer cores. As the inner core grows,
light elements become concentrated in the outer core according to their initial
abundance, partitioning coefficient, and mass fraction in the solid and liquid
reservoirs. In these models we assume that sulfur partitions entirely into the liquid
(distribution coefficient, $D=0$), whereas silicon partitions nearly equally between
the solid and liquid.

 \begin{figure}[h] %  figure placement: here, top, bottom, or page
   \centering
\begin{tabular}{cc}
 \includegraphics[width=18pc]{figs/thermal_evolution.png}
 \includegraphics[width=18pc]{figs/inner_core_growth.png}
\end{tabular}
\caption{Left: Thermal evolution of the Mercurian mantle and core. This
  thermal evolution model couples the core thermodynamics in the previous section
  with the parameterized convection model of \citep{Stevenson1983}. The colored
  regions show the solution for models with $\pm \mathrm{100}$ degrees C. Note the
  break in slope of the $\mathrm{T_{cmb}}$ temperature with the onset of inner core
  growth at $\sim 2.5\times10^2~\mathrm{Ma}$. Right: Growth of the inner core versus
  time. This model run yields an inner core of $ \sim 1400~\mathrm{km}$, slightly
  exceeding the upper bound of inner core size as constrained by \citep{Dumberry2015}.
  Figure credit: Ion Rose.}
  \label{fig:thermal}
\end{figure}

Also shown in Fig.~\ref{fig:interior_model} is a model liquidus, fit to experimental
phase diagrams. This particular liquidus model was modified from one compiled by June
Wicks and Nick Knezek (pers. comm., CIDER 2014). This model is an interpolation, very
similar to the one presented earlier in the chapter, but with additional experimental
sources included in the model. The melting curve of the Fe-S system has enigmatic
features, which may not be adequately captured by a  linear interpolation. The onset
of ``snow'' regions in the core is extremely sensitive to this interpolation. 

Comparing the slope of the interpolated melting temperature to the slope of the
adiabatic profile determines the crystallization behavior of the core, with a
steeper melting temperature corresponding to  ``Earth-like'' conditions. For higher
values of the thermal expansivity, $\alpha$, the core will be ``snowing'' at all
times. For lower $\alpha$, the onset of snowing occurs with increasing S-content.

Once the layer masses and compositions are defined for the interior structure model. We
use an iterative method to compute consistent boundary radii for each of the layers,
along with consistent profiles of gravity and pressure as a function of radius from
the center of the planet. The pressures are then fed back to the description of the
adiabatic barotropes to update the pressures and densities as a function of radius of
the planet. This description also naturally leads to the moments of inertia for each
of these layers, for which there are indirect measurements from Mercury's orbital librations
\citep{Margot2012}. It also allows for a calculation of the radial contraction of the
planet, which may be recorded in geological features such as scarps on the planet's
surface.


\section{Coupling with mantle convection}


Between any two interior structure models with an incremental change in the
solidification of the core, there is a change in the energy of the system. The change
in density profile, from contraction and light element partitioning in the core can
be translated into a change in the gravitational energy of the system. Likewise, the
change in temperature profile can be used to approximate a change in thermal energy
in combinations with a heat capacity, $C_P$. The solidification of the inner core
also has an accompanying latent heat release, although this is a smaller effect.

\begin{figure}[h] %  figurVVe placement: here, top, bottom, or page
   \centering
\begin{tabular}{c}
 \includegraphics[width=30pc]{figs/CMB_flux.pdf} 
\end{tabular}
\caption{ Heat flux variations due to insolation for a conducting mantle with
    negligible internal heating. The total CMB heat flux is $\sim 0.6~\mathrm{TW}$,
and peak-to-peak variations are about 20\%. Figure Credit: Ian Rose.}
\label{fig:flux}
\end{figure}

In order to relate these global changes in energy to a predicted timescale, one must
consider the rate of heat loss through the mantle and lithosphere of the planet.  The
interior structure model is coupled to a parameterized convection model
\citep{Stevenson1983} for the thermal evolution of the planet. Shown on the right in
Fig.~\ref{fig:core_energy} are the changes in latent heat, gravitational and thermal
energy in the core per change in core mantle boundary temperature for bulk
composition of 6 wt.\% S. Over a large number of steps, we can thus track the energy
and entropy fluxes through different portions of the planet.

Mercury's unusual 3:2 spin-orbit resonance causes persistent temperature variations
at the surface.  This boundary condition may create significant heat-flux variations
at the CMB, especially if the mantle is not convecting.  Here we solve a simple
conduction equation in the Mercurian mantle to calculate an estimate of heat flux at
the CMB.  This heat-flux variation can then used to inform the boundary conditions of
a dynamo simulation using the \texttt{Calypso} code, and entropy-budget models for
geodynamos such as that introduced in Section~\ref{budget}. Results from an example
calculation of such a geodynamo entropy budget calculation are shown in
Figure~\ref{entropy}.



 \begin{figure}[h] %  figurVVe placement: here, top, bottom, or page
   \centering
\begin{tabular}{c}
 \includegraphics[width=22pc]{figs/entropy_fig.png} 
\end{tabular}
\caption{Entropy budget and inner core radius for a core with 6 wt.\% S and
  current CMB heat flux of 0.5~TW. In this model, there is insufficient entropy to
  drive a dynamo before inner core solidification (negative Ohmic dissipation), but
  compositional sources that arise from inner core growth increase the available
entropy such that a present-day dynamo can be sustained (positive Ohmic dissipation).
Figure credit: Grace Cox. }
\label{entropy}
\end{figure}

\chapter{Calculating Gravitational Moments for an Interior Structure }\label{chap4}

%volume integral in spherical coordinates
\newcommand{\sphint}{\int_{-1}^1 d\mu' \int_{0}^{2\pi} d\phi' \int_{r' > r} dr'}
\newcommand{\sphshort}{\int_{\tau} d\tau}
\newcommand{\muint}{\int_{-1}^1 d\mu'}
\newcommand{\phiint}{\int_{0}^{2\pi} d\phi'}
\newcommand{\xiint}{\int_{b/a}^{\xi} d\xi'}

\section{Motivation}

The gas giants Jupiter and Saturn rotate so rapidly that adequate treatment of the
non-spherical part of their gravitational potential requires either a very high-order
perturbative, or better, an entirely non-perturbative approach \citep{hubbard2012,
hubbard2013, hubbard2014, wisdom1996,wisdom2016}. Here we present an extension of the
Concentric Maclaurin Spheroid (CMS) method of \citet{hubbard2012, hubbard2013} to three
dimensions to include the tidal perturbation from a satellite. This allows for
high-precision simulations of static tidal response, consistent with the
planet's shape and interior mass distribution. The presence of a large rotational
bulge produces an observable effect on the tidal response of giant planets. This
effect, which has not been previously revealed by linear tidal-response theories
applied to spherical-equivalent interior models, has implications for the observed
tidal responses of Jupiter and Saturn.

The \textit{Juno} spacecraft is expected to measure the strength of Jupiter's
gravitational field to an unprecedented precision ($\sim$ one part in $10^9$)
\citep{kaspi2010}, potentially revealing a weak signal from the planet's interior
dynamics. Also present in Jupiter's gravitational field will be tesseral-harmonic
terms produced by tides raised by the planet's large satellites.  In fact, close to
the planet, the gravitational signal from Jupiter's tides has a similar magnitude to
the predicted signal from models of deep internal dynamics
\citep{cao2015,kaspi2010,kaspi2013}. An accurate prediction of the planet's
hydrostatic tidal response will, therefore, be essential for interpreting the
high-precision measurements provided by the \textit{Juno} gravity science experiment.

Although the \textit{Cassini} Saturn orbiter was not designed for direct measurement
of high-order components of Saturn's gravitational field, it has already provided
gravitational information relevant to the planet's interior structure.
\citet{lainey2016} used an astrometry dataset of the orbits of Saturn's co-orbital
satellites to make the first determination of the planet's $k_2$ love number. Their
observed $k_2$ was significantly larger than the theoretical prediction of
\citet{gavrilov1977}. A mismatch between an observed $k_2$ and the value predicted
for a Saturn model fitted to the planet's low-degree zonal harmonics $J_2$ and $J_4$
would raise questions about the adequacy of the hydrostatic (non-dynamic) theory of
tides.  

In this paper we present theoretical results for simplified Saturn interior models
matching the planet's observed low-degree zonal harmonics.  When these models are
analyzed with the full 3-d CMS theory including rotation and tides, we predict a
gravitational response in line with the observed $k_2$ value of \citet{lainey2016},
suggesting that the observation can be completely understood in terms of a static
tidal response.  A similar test will be possible for Jupiter once its $k_2$ has been
measured by the \textit{Juno} spacecraft.

There is extensive literature on the problem of the shape and gravitational potential
of a liquid planet in hydrostatic equilibrium, responding to its own rotation and to
an external gravitational potential from a satellite; see, e.g., a century-old
discussion in \citet{jeans1919}.  Many classical geophysical investigations use a
perturbation approach, obtaining the planet's linear and higher-order response to
small deviations of the potential from spherical symmetry. A good discussion of the
application of perturbation theory to rotational response, the so-called theory of
figures, is found in \citet{zharkov1978}, while a pioneering calculation of the tidal
response of giant planets is presented by \citet{gavrilov1977}.

\citet{hubbard2012} introduced an iterative numerical method, based on the theory of
figures, for calculating the self-consistent shape and gravitational field of a
constant density, rotating fluid body to high precision. In the CMS method, integrals
over the mass distribution are solved using Gaussian quadrature to obtain the
gravitational multipole moments. This method was extended to non-constant density
profiles by \citet{hubbard2013}, by approximating the barotropic pressure-density
relationship with multiple concentric constant-density (Maclaurin) spheroids.  Here a
spheroid is defined as a smooth shape obtained from deforming a sphere in three
dimensions and is more general than an ellipsoid, whose shape is uniquely defined
by 3 parameters.  This approach mitigates problems with cancellation of
terms that arise in a purely numerical solution to the general equation of
hydrostatic equilibrium, and has a typical relative precision of $\sim 10^{-12}$. The
CMS method has been benchmarked against analytical results for simple models
\citep{hubbard2014} and against an independent, non-perturbative numerical method
\citep{wisdom1996,wisdom2016}. 

The theory of \citet{gavrilov1977} begins with an interior model of Saturn fitted to
the values of $J_2$ and $J_4$ observed at that time.  This interior model tabulates
the mass density $\rho$ as a function of $s$, where $s$ is the mean radius of the
constant-density surface.  Tidal perturbation theory is then applied to this
spherical-equivalent Saturn.  The \citet{gavrilov1977} approach is sufficient for an
initial estimate of the tidally-induced terms in the external potential, but it
neglects terms which are of the order of the product of the tidal perturbation and
the rotational perturbation.  Here we demonstrate that, for a rapidly-rotating giant
planet, the latter terms make a significant contribution to the love numbers
$k_{nm}$, as well as (unobservably small) tidal contributions to the gravitational
moments $J_n$.

\citet{vorontsov1984} introduced a novel approach to calculation of the tidal
response of giant planets.  Rather than treating the problem as a purely static
one, as we do here, they considered the case of a non-rotating giant planet
orbited by a single satellite with an inertial orbital frequency $\Omega_s$.
They then calculated the response of the planet's normal oscillation modes to
the perturbation, noting that the mode frequencies (whose oscillation periods
are measured in hours) are much higher than satellite orbital frequencies
(satellite periods are measured in days). For such off-resonance excitation, it
is unnecessary to consider damping (as parameterized by the tidal quality
factor $Q$) in calculating the  tidal response.  Taking the limit
$\Omega_s \rightarrow 0$, \citet{vorontsov1984} obtained the static tidal
response of the non-rotating planet and thus its love number $k_2$.  We compare
the Vorontsov et al. Saturn $k_2$ with our value in Section \ref{calc_k2}, below.

An analogous problem has been studied for the tidal response of Galilean
satellite Io by \citet{zharkov2004} and \citet{zharkov2010}, and for close-in
exoplanets by \citet{correia2013}. These works consider the second order
approximations through a higher order perturbative theory. Our problem is
different, however, in that the tidal and rotational perturbations for Io are
of comparable magnitude, while the large influence of rotation on a much weaker
tidal response found here for Saturn is unlike Io.  Similarly, close-in,
tidally locked exoplanets have comparable tidal and rotational perturbations.

\citet{folonier2015} presented a method for approximating the love numbers of a
non-homogeneous body using Clairaut theory for the equilibrium ellipsoidal
figures.  This results in an expression for the love number $k_2$ for a body
composed of concentric ellipsoids, parameterized by their flattening
parameters. In the case of the constant density spheroid, there is a well-known
result that the equipotential surface is an ellipsoid. However, in bodies with
more complicated density distributions, the equipotential surfaces will have a
more general spheroidal shape.  Because of the small magnitude of tidal
perturbations, the method of \citet{folonier2015} works in the limit of slow
rotation despite this limitation.  However, the method does not account for the
coupled effect of tides and rotation, and does not predict love numbers of
order higher than $k_2$.  Within these constraints, we show below that our
extended CMS method yields results that are in excellent agreement with results
from \citet{folonier2015}.

Although our theory is quite general and can be used to calculate a rotating
planet's static tidal response to multiple satellites located at arbitrary
latitudes, longitudes, and radial distances, for application to Jupiter and
Saturn it suffices to consider the effect of a single perturbing satellite
sitting on an orbital plane at zero inclination to the planet's equator.  Since
tidal distortions are always very small compared with rotational distortion,
and Jupiter's Galilean satellites, as well many of Saturn's larger satellites,
are on orbits with low inclination, the tidal response to multiple satellites
can be obtained by a linear superposition of the perturbation from each body.
Extension of our theory to a system with a large satellite on an inclined
orbit, such as Neptune-Triton, would be straightforward, but is not considered
here.

\section{Concentric Maclaurin Spheroid Method }

In the co-rotating frame of the planet in hydrostatic equilibrium,  the pressure $P$,
the mass density $\rho$ and the total effective potential $U$ are related by
%
\begin{equation} \begin{aligned} \nabla P = \rho \nabla U. \end{aligned}
    \label{eq:hydrostatic} \end{equation}
%
The total effective potential can be separated into three components,
%
\begin{equation} \begin{aligned} U = V + Q + W, \end{aligned}
    \label{eq:potential_components} \end{equation}
%
where $V$ is the gravitational potential arising from the mass distribution
within the planet, $Q$ is the centrifugal potential corresponding to a rotation
frequency $\omega$, and $W$ is the tidal potential arising from a satellite
with mass $m_{\rm s}$ at planet-centered coordinates $(R,\mu_s,\phi_s)$, where
$R$ is the satellite's orbital distance from the origin, $\mu_s=\cos \theta$,
where $\theta$ is the satellite's planet-centered colatitude and $\phi_s$ is
the planet-centered longitude.  In this investigation, we treat only the static
tides in the corotating frame of the planet, and thus  we always place the
satellite at angular coordinates $\mu_s=0$ and $\phi_s=0$.  The relative
magnitudes of $V$, $Q$, and $W$ can be described in terms of two
non-dimensional numbers:
%
\begin{equation} q_{\rm rot} = \frac{\omega^2 a^3}{GM} \label{eq:qrot} \end{equation}
%
for the rotational perturbation and
%
\begin{equation} q_{\rm tid} = -\frac{3m_{\rm s}a^3}{MR^3} \label{eq:qtid}
\end{equation}
%
for the tidal perturbation, where $G$ is the universal gravitational constant, and
$M$ and $a$ are the mass and maximum equatorial radius of the planet. The planet-satellite
system is described by these two small parameters along with a third parameter, the
ratio $a/R$. 

Since CMS theory is nonperturbative, in principle our results are valid to all powers
of these small parameters and their products (until we reach the computer's numerical
precision limit). For the giant-planet tidal problems that we consider here, terms of
second and higher order in $q_{\rm tid}$ are always negligible, but terms linear in
$q_{\rm tid}$ and multiplied by various powers of $q_{\rm rot}$ and $a/R$ contribute
above the numerical noise level.  It is, in fact, terms of order $q_{\rm tid} \cdot
q_{\rm rot}$ that contribute most importantly to the new results of this paper.

We introduce dimensionless planetary units of pressure $P_{\rm pu}$, density
$\rho_{\rm pu}$, and total potential $U_{\rm pu}$, such that
%
\begin{equation} \begin{aligned} P \equiv& \frac{GM^2}{a^4} P_{\rm pu} \\ \rho
        \equiv& \frac{M}{a^3} \rho_{\rm pu} \\ U \equiv& \frac{GM}{a} U_{\rm pu}. \\
    \end{aligned} \label{eq:planetary_units} \end{equation}
%
The CMS method considers a model planet composed of $N$ nested spheroids of constant
density as depicted in Figure \ref{fig:spheroid}. We label these spheroids with index
$i=0,1,2,\dots,N-1$, with $i = 0$ corresponding to the outermost spheroid and $i=N-1$
corresponding to the innermost spheroid. Each spheroid is constrained to have a point
at radial distance $a_i$ from the planet's center of mass, such that each of these
fixed points has the same angular coordinates as the sub-satellite point
$(\mu=0,\phi=0)$.  Accordingly, the $a_0$ of the outermost spheroid corresponds to
its the largest principal axis, if the perturbing satellite is in the equatorial
plane.

When $q_{\rm tid}=0$, the potential is axially symmetric and the problem can be
solved in two spatial dimensions. However, when both $q_{\rm tid}$ and $q_{\rm rot}$
are nonzero, the symmetry is broken, meaning that each spheroid has a fully triaxial
figure with the surface described by
%
\begin{equation} \zeta_i \equiv r_i(\mu,\phi) / a_i, \label{eq:shape} \end{equation}
%
such that $\zeta_0$ represents the shape of the outer surface.

Taking advantage of the principle of superposition for a linear relationship between
the potential $V$ and the mass density $\rho$, the total $V$ is given by the sum of
the potential arising from each individual spheroid \citep{hubbard2013}. This allows us to
approximate any monotonically increasing density profile, with the density of the
$i$th spheroid represented by the density jump
%
\begin{equation} \delta \rho_i = \begin{cases} \rho_i - \rho_{i-1}, & i>0 \\ \rho_0,
        & i=0.  \end{cases} \label{eq:density_increment} \end{equation}
%
This parameterization of density has the added benefit of naturally handling
discontinuities in $\rho$, as would be expected for a giant planet with a dense
central core.

\section{Extension to Three Dimensions} \label{method_derivation}

\subsection{Calculation of gravitational potential} \label{method_derivation}

The general expansion of $V$ in spherical coordinates ${\bf r} = (r,\mu =
\cos{\theta},\phi)$ is
%
\begin{equation}
    \begin{aligned} V&(r,\mu,\phi) = \\
        &\frac{G}{r} \left[ \sum_{n=0}^{\infty} P_n(\mu) \sphshort
            \rho(r')P_n(\mu')\left(\frac{r'}{r}\right)^k \right. \\ & +
            \sum_{n=0}^{\infty} \sum_{m=1}^{n} P^m_n(\mu)\cos(m\phi) \sphshort
            \frac{2(n-m)!}{(n+m)!} \rho(r')P_n^m(\mu')\cos(m\phi')
            \left(\frac{r'}{r}\right)^k \\ & \left.  + \sum_{n=0}^{\infty}
            \sum_{m=1}^{n} P^m_n(\mu)\sin(m\phi) \sphshort \frac{2(n-m)!}{(n+m)!}
            \rho(r')P_n^m(\mu')\sin(m\phi') \left(\frac{r'}{r}\right)^k \right]
        \end{aligned} \label{eq:general_expansion}
    \end{equation}

\citep{zharkov1978}, where $P_n$ and $P_n^m$ are the Legendre and associated Legendre
polynomials, 
%
\begin{equation*} d\tau = r'^2 dr' \sin(\theta')d\theta' d\phi' = r'^2 dr' d\mu' d\phi',
\end{equation*}
%
and the origin, ${\bf r}=(0,0,0)$, is the center of mass of the planet.  The
potential at a general point within the planet has a contribution from mass both
interior and exterior to that point, for which the exponent $k$ in Eqn.
\eqref{eq:general_expansion} is different:
%
\begin{equation*} k = \begin{cases} n, & r' < r \\ -(n+1), & r' > r .  \end{cases}
\end{equation*}

The centrifugal potential $Q$ depends only on $r$ and $\mu$
%
\begin{equation} \begin{aligned} Q(r,\mu) = \frac{1}{3}r^2\omega^2\left[1-P_2(\mu)
        \right].  \end{aligned} \label{eq:centrifugal} \end{equation}
%
The tidal potential $W$ for a satellite at position ${\bf R}=(R,\mu_s,\phi_s)$ is
%
\begin{equation} \begin{aligned} W({\bf r}) = \frac{Gm_{\rm s}}{\left| {\bf R} - {\bf
        r} \right|}.  \end{aligned} \label{eq:tidal} \end{equation}
%
The general expansion of $W$ around the center of mass of the planet is obtained by
using the summation theorem for spherical harmonics \citep{gavrilov1977}  
%
\begin{equation} \begin{aligned} W(r,\mu,\phi) =& \frac{Gm_{\rm s}}{R}
        \sum_{n=2}^\infty \left( \frac{r}{R} \right)^n \left[ P_n(\mu) P_n(\mu_s) \phantom{\frac{1}{1}}\right. \\
        & \left. + 2 \sum_{m=1}^{n} \frac{(n-m)!}{(n+m)!} \cos(m\phi - m\phi_s)
    P_n^m(\mu) P_n^m(\mu_s) \right].  \end{aligned} \label{eq:tidal_general}
    \end{equation}

Following \citet{hubbard2013}, we derive non-dimensional quantities in terms of the
planet mass $M$ and maximum radius $a=a_0$. For each spheroid, we define a
dimensionless radius of each spheroid
%
\begin{equation} \begin{aligned} \lambda_i \equiv& a_i/a \\
    \end{aligned} \label{eq:lambda} \end{equation}
%
and dimensionless density increment, based on the mean density of the planet
%
\begin{equation} \begin{aligned} \bar{\rho} =& \frac{3M}{a^3}\frac{1}{\muint \phiint
        \zeta_0^3} \\ \delta_i \equiv& \frac{\delta\rho_i}{\bar{\rho}}.
        \label{eq:delta} \end{aligned} \end{equation}
%
A non-dimensional mass of the planet is then given by the integral expression
%
\begin{equation} M^* = \frac{1}{3} \sum_{j=0}^{N-1} \delta_j \lambda_j^3 \muint \phiint
    \zeta_j^3,  \label{eq:mass} \end{equation}
%
which is equal to unity when $\delta_j$ is properly normalized for $\zeta_j$.  The
contribution to the potential is expanded in terms of interior and external zonal
harmonics $J_{i,n}$ and $J_{i,n}'$. For the tidal problem, we must also consider the
analogous $C_{i,nm}$, $C'_{i,nm}$, $S_{i,nm}$ and $S'_{i,nm}$ \citep{zharkov1978}.
These contribute linearly to the total moment evaluated exterior to the planet's
surface
%
\begin{equation} J_n = \sum_{i=0,N-1} J_{i,n}.  \label{eq:total_J2} \end{equation}
%
% Alias for tildaed variables
\newcommand{\til}{\widetilde}
%
The layer-specific harmonics are then normalized by radius as 
%
\begin{equation} \begin{aligned}[c] \til{J}_{i,n} \equiv&
        \frac{J_{i,n}}{\lambda_i^n}, & \til{J}'_{i,n} \equiv&
        J'_{i,n}\lambda_i^{(n+1)} \\ \til{S}_{i,nm} \equiv&
        \frac{S_{i,nm}}{\lambda_i^n}, & \til{S}'_{i,nm} \equiv&
        S'_{i,nm}\lambda_i^{(n+1)} \\ \til{C}_{i,nm} \equiv&
        \frac{C_{i,nm}}{\lambda_i^n}, & \til{C}'_{i,nm} \equiv&
        C'_{i,nm}\lambda_i^{(n+1)}.  \end{aligned} \label{eq:normalization}
\end{equation}
% 
Following the derivation in \citet{hubbard2013} and generalizing the expressions for
full three dimensional volume integrals, we find the normalized interior harmonics
%
\begin{equation} \begin{aligned} \til{J}_{i,n} &=& -\frac{3}{n+3} \frac{ \delta_i
            \lambda_i^3 \muint P_n(\mu') \phiint \zeta_i^{(n+3)} } { \sum_{j=0}^{N-1}
        \delta_j \lambda_j^3 \muint \phiint \zeta_j^3} \\ \til{C}_{nm} &=&
        \frac{6(n-m)!}{(n+3)(n+m)!} \frac{ \delta_i \lambda_i^3 \muint P_n^m(\mu')
            \phiint \zeta_i^{(n+3)} \cos(m\phi') } { \sum_{j=0}^{N-1} \delta_j
        \lambda_j^3 \muint \phiint \zeta_j^3} \\ \til{S}_{nm} &=&
        \frac{6(n-m)!}{(n+3)(n+m)!} \frac{ \delta_i \lambda_i^3 \muint P_n^m(\mu')
            \phiint \zeta_i^{(n+3)} \sin(m\phi') } { \sum_{j=0}^{N-1} \delta_j
        \lambda_j^3 \muint \phiint \zeta_j^3}, \\ \end{aligned} \label{eq:harmonics}
\end{equation}
%
and the exterior harmonics
%
\begin{equation} \begin{aligned} \til{J}'_{i,n} &=& -\frac{3}{2-n} \frac{ \delta_i
            \lambda_i^3 \muint P_n(\mu') \phiint \zeta_i^{(-n+2)} } {
                \sum_{j=0}^{N-1} \delta_j \lambda_j^3 \muint \phiint \zeta_j^3} \\
                \til{C}'_{nm} &=&  \frac{6(n-m)!}{(2-n)(n+m)!} \frac{ \delta_i
                    \lambda_i^3 \muint P_n^m(\mu') \phiint \zeta_i^{(-n+2)}
                \cos(m\phi') } { \sum_{j=0}^{N-1} \delta_j \lambda_j^3 \muint \phiint
            \zeta_j^3} \\ \til{S}'_{nm} &=& \frac{6(n-m)!}{(2-n)(n+m)!} \frac{
                \delta_i \lambda_i^3 \muint P_n^m(\mu') \phiint  \zeta_i^{(-n+2)}
            \sin(m\phi') } { \sum_{j=0}^{N-1} \delta_j \lambda_j^3 \muint \phiint
        \zeta_j^3} \\ \end{aligned} \label{eq:harmonics_prime} \end{equation}
%
with a special case for $n=2$
%
\begin{equation} \begin{aligned} \til{J}'_{i,n} &=& -3 \frac{ \delta_i \lambda_i^3
        \muint P_n(\mu') \phiint \log(\zeta_i) } { \sum_{j=0}^{N-1} \delta_j
    \lambda_j^3 \muint \phiint \zeta_j^3} \\ \til{C}'_{nm} &=&
    \frac{6(n-m)!}{(n+m)!} \frac{ \delta_i \lambda_i^3 \muint P_n^m(\mu') \phiint
    \log(\zeta_i) \cos(m\phi') } { \sum_{j=0}^{N-1} \delta_j \lambda_j^3 \muint
\phiint \zeta_j^3} \\ \til{S}'_{nm} &=& \frac{6(n-m)!}{(n+m)!} \frac{ \delta_i
\lambda_i^3 \muint P_n^m(\mu') \phiint  \log(\zeta_i) \sin(m\phi') } {
    \sum_{j=0}^{N-1} \delta_j \lambda_j^3 \muint \phiint \zeta_j^3} \\ \end{aligned}
\label{eq:harmonics_prime_n=2} \end{equation}
%
and
%
\begin{equation} \begin{aligned} J''_{i,0} =& \frac{2\pi\delta_i a_0^3}{3 M}.
    \end{aligned} \end{equation}

The shape of the surface of the planet is defined by the equipotential relationship
%
\begin{equation} U(\zeta,\mu,\phi,\mu_s,\phi_s) -  U(1,0,0,\mu_s,\phi_s) = 0,
    \label{eq:equipotential_surface} \end{equation}
%
where the potential in planetary units at an arbitrary point on the planet's surface
%
\begin{equation} \begin{aligned} U(\zeta,\mu,\phi,\mu_s,\phi_s) =& \frac{1}{\zeta_0}
        \left[ 1 - \sum^{N-1}_{i=0} \sum^\infty_{n=1} \lambda_i^n \zeta_0^{-n}
            \left\{ P_{n}(\mu) \til{J}_{i,n} \phantom{\sum_{1}^{n}}\right. \right. \\
            & \left. - \sum_{m=1}^{n} P_n^m(\mu) \left( \til{C}_{i,nm}\cos(m\phi) +
            \til{S}_{i,nm}\sin(m\phi) \right) \right\}  \\ & + \frac{1}{3} q_{\rm
            rot} \zeta_0^3 [1 - P_2(\mu)] \\ & -\frac{1}{3} \zeta_0^{3} q_{\rm tid}
            \sum_{n=2}^\infty \left(\frac{a}{R}\right)^{(n-2)} \zeta_0^{(n-2)}
            \left\{ P_n(\mu) P_n(\mu_s) \phantom{\sum_1^n}\right. \\ & \left. \left.
        + 2 \sum_{m=1}^{n} \frac{(n-m)!}{(n+m)!} \cos(m\phi - m\phi_s) P_n^m(\mu)
    P_n^m(\mu_s) \right\} \right]\\ \end{aligned} \label{eq:surface_potential}
    \end{equation}
%
matches the reference potential at the sub-satellite point
%
\begin{equation} \begin{aligned} U(1,0,0,\mu_s,\phi_s) =& 1 - \sum^{N-1}_{i=0}
        \sum^\infty_{n=1} \lambda_i^n   \left\{ P_{n}(0) \til{J}_{i,n} -
        \sum_{m=1}^{n} P_n^m(0) \til{C}_{i,nm} \right\}  \\ &+ \frac{1}{2} q_{\rm
        rot}  -\frac{1}{3} q_{\rm tid} \sum_{n=2}^\infty
        \left(\frac{a}{R}\right)^{(n-2)} \left\{ P_n(0) P_n(\mu_s) \phantom{\sum_a^b}
        \right. \\ & \left. + 2 \sum_{m=1}^{n} \frac{(n-m)!}{(n+m)!} \cos(- m\phi_s)
    P_n^m(0) P_n^m(\mu_s) \right\}. \\ \end{aligned} \label{eq:surface_subsatellite}
    \end{equation}
%
Similarly, the shapes of the interior spheroids are found by solving
%
\begin{equation} U_j(\zeta,\mu,\phi,\mu_s,\phi_s) -  U_j(1,0,0,\mu_s,\phi_s) = 0,
    \label{eq:equipotential_interior} \end{equation}
%
where
%
\begin{equation} \label{eq:Uj} \begin{aligned} U_j(\zeta_j,\mu,\phi,\mu_s,\phi_s) =&
        -\frac{1}{\zeta_j \lambda_j} \left[ \sum^{N-1}_{i=j} \sum^\infty_{n=0} \left(
            \frac{\lambda_i}{\lambda_j} \right)^n \zeta_j^{-n} \left\{ P_{n}(\mu)
            \til{J}_{i,n} \phantom{\sum_a^b} \right. \right. \\ & \left. -
            \sum_{m=1}^{n} P_n^m(\mu) \left( \til{C}_{i,nm}\cos(m\phi) +
            \til{S}_{i,nm}\sin(m\phi) \right) \right\}  \\ &+ \sum^{j-1}_{i=0}
            \sum^\infty_{n=0} \left( \frac{\lambda_j}{\lambda_i} \right)^{n+1}
            \zeta_j^{n+1} \left\{ \til{J}'_{i,n} P_n(\mu) \phantom{ \sum_1^m} \right.
            \\ & \left. - \sum_{m=1}^{n} P_n^m(\mu) \left( \til{C}'_{i,nm}\cos(m\phi)
        + \til{S}'_{i,nm}\sin(m\phi) \right) \right\} \\ &+ \left. \sum^{j-1}_{i=0}
    J''_{i,0} \lambda_j^3 \zeta_j^3 \right] + \frac{1}{3} q_{\rm rot} \lambda_j^2
    \zeta_j^2 [1 - P_2(\mu)] \\ &-\frac{1}{3} \lambda_j^2 \zeta_j^{2} q_{\rm tid}
    \sum_{n=2}^\infty \left(\frac{a\lambda_j}{R}\right)^{(n-2)} \zeta_j^{(n-2)}
    \left\{ P_n(\mu) P_n(\mu_s) \phantom{\sum_a^b} \right. \\ & \left. + 2
    \sum_{m=1}^{n} \frac{(n-m)!}{(n+m)!} \cos(m\phi - m\phi_s) P_n^m(\mu)
P_n^m(\mu_s) \right\} \end{aligned} \end{equation}
%
and
%
\begin{equation} \begin{aligned} U_{j}(1,0,0,\mu_s,\phi_s) =& -
        \frac{1}{\lambda_j}\left[ \sum^{N-1}_{i=j} \sum^\infty_{n=0} \left(
            \frac{\lambda_i}{\lambda_j} \right)^n   \left\{ P_{n}(0) \til{J}_{i,n} -
            \sum_{m=1}^{n} P_n^m(0) \til{C}_{i,nm}\right\} \right.\\ & +
        \sum^{j-1}_{i=0} \sum^\infty_{n=0} \left( \frac{\lambda_j}{\lambda_i}
    \right)^{n+1} \left\{ \til{J}'_{i,n} P_n(0) - \sum_{m=1}^{n} P_n^m(0)
\til{C}'_{i,nm} \right\} \\ & \left. + \sum^{j-1}_{i=0} J''_{i,0} \lambda_j^3 \right]
+\frac{1}{2} q_{\rm rot} \lambda_j^2 \\ & - \frac{1}{3} \lambda_j^2 q_{\rm tid}
\sum_{n=2}^\infty \left(\frac{a\lambda_j}{R}\right)^{(n-2)} \left\{  P_n(0)
    P_n(\mu_s) \phantom{\sum_1^1} \right. \\ & \left. + 2 \sum_{m=1}^{n}
    \frac{(n-m)!}{(n+m)!} \cos(- m\phi_s) P_n^m(0) P_n^m(\mu_s) \right\}. \\
\end{aligned} \label{eq:interior_subsatellite} \end{equation}
%
From Eqn. \eqref{eq:interior_subsatellite}, we also find the potential at the center
of the planet
%
\begin{equation} \begin{aligned} U_{\rm center} = -\sum_{i=0}^{N-1} \sum_{n=0}^\infty
        \lambda_i \left\{ \til{J}'_{i,n} - \sum_{m=1}^{n} \til{C}'_{i,nm} \right\}.
    \end{aligned} \label{eq:central_potential} \end{equation}

%
Taking the limit of Eqn. \eqref{eq:central_potential} as the radius goes to zero
yields
%
\begin{equation} \begin{aligned} U_{\rm center} &=\lim_{\zeta_j\to 0} U_j(\zeta_j)\\
        &= - \sum_{i=0}^{N-1} \frac{J'_{i,n=0}}{ \lambda_i}, \end{aligned}
\end{equation} correcting a typographical error in Eqn. 49 of \citet{hubbard2013}.
%
In solving equations \eqref{eq:equipotential_surface} and
\eqref{eq:equipotential_interior}, we also require their analytical derivatives
%
\begin{equation} \begin{aligned} \frac{d \left[U(\zeta,\mu,\phi,\mu_s,\phi_s) -
        U(1,0,0,\mu_s,\phi_s) \right] }{d\zeta} =& \frac{dU(\zeta,\mu,\phi)}{d\zeta}
        \\ \frac{d \left[ U_j(\zeta_j,\mu,\phi,\mu_s,\phi_s) -
        U_{j}(1,0,0,\mu_s,\phi_s) \right]}{d\zeta_j} =& \frac{d
        U_j(\zeta_j,\mu,\phi)}{d\zeta_j}.  \label{eq:equipotential_derivative}
    \end{aligned} \end{equation}

\subsection{Gaussian quadrature} \label{quadrature}

The preceding expressions give the gravitational potential and equipotential shapes,
as a function of $q_{\rm rot}$ and $q_{\rm tid}$, within a layered planet with $N$
concentric spheroids. In the limit of $N\to \infty$, the solution would apply to an
arbitrary monotonically increasing barotropic relation, $\rho(P)$. 

For practical applications, we need to find the potential as a multipole expansion up
to a maximum degree $n_{\max}$. For the results presented here, we use $n_{\max}=30$.
The angular integrals in equations
\eqref{eq:harmonics}~--~\eqref{eq:harmonics_prime_n=2} can be evaluated using
Gaussian quadratures on a two dimensional grid. Here we use Legendre-Gauss
integration to integrate polar angles over $L_{1}=48$ quadrature points
$\mu_{\alpha}=\cos(\theta_{\alpha})$, $\alpha = 1,2,\dots L_{1}$, with the
corresponding weights $\omega_{\alpha}$ over the interval
$0<\mu<1$. At any point in the calculation, we must keep track of radius values for
each layer on a 2D grid of quadrature points $\zeta_{i\alpha\beta}$. For efficiency,
we pre-calculate the values of all of the Legendre and associated Legendre polynomials
at each polar quadrature point, $P_n(\mu_\alpha)$ and $P_n^m(\mu_\alpha)$.

For the azimuthal angle, we encounter integrals of the form
%
\begin{equation} \begin{aligned} I_{c,m} \equiv \int^{2\pi}_0 f(\phi)
        \cos(m\phi)d\phi \\ I_{s,m} \equiv \int^{2\pi}_0 f(\phi) \sin(m\phi)d\phi
    \end{aligned} \label{eq:cheb_integral} \end{equation}
%
when calculating the tesseral harmonics. For these, we use Chebyshev-Gauss
integration with $L_{2}=96$ quadrature points $\eta_{\beta}=\cos(\phi_{\beta})$,
$\beta = 1,2,\dots L_{2}$, with the corresponding weights $\omega_{\beta}$, $\beta =
1,2,\dots L_{2}$ over the interval $0<\phi<2\pi$  
%
\begin{equation} \begin{aligned} d\eta = -\sin(\phi)d\phi \\ d\phi = -
        \frac{d\eta}{\sqrt{1-\eta^2}}.  \end{aligned} \end{equation}
%
Using the identity $(\sin\theta)^{m-k}=(1-\mu^2)^{\frac{m-k}{2}}$, the sinusoidal
functions can be expanded as
%
\begin{equation} \begin{aligned} \cos{m\phi} = \sum^m_{k=0} { m \choose k } \eta^k
        (1-\eta^2)^{\frac{m-k}{2}} \cos\left\{ \frac{\pi}{2}(m-k) \right\} \\
        \sin{m\phi} = \sum^m_{k=0} { m \choose k } \eta^k (1-\eta^2)^{\frac{m-k}{2}}
        \sin\left\{ \frac{\pi}{2}(m-k) \right\}.  \end{aligned} \end{equation}
%
Substituting these into Eqn. \eqref{eq:cheb_integral} and splitting the integral into
two intervals $0<\phi<\pi$ and $\pi<\phi<2\pi$ yields
%
\begin{equation} \begin{aligned} I_{c,m} =& \sum^m_{k=0}{ m \choose k } \cos \left[
        \frac{\pi}{2}(m-k) \right] \left\{ \int^{1}_{-1} \eta^k f(\cos^{-1}(-\eta))
        \left[ 1 - \eta^2 \right]^{\frac{m-k}{2}} d\eta \right. \\ &- \left.
        \int^{1}_{-1} \eta^k f(\cos^{-1}\eta) \left[ 1 - \eta^2
        \right]^{\frac{m-k}{2}} d\eta \right\} \\ =& \sum^m_{k=0}{ m \choose k } \cos
        \left[ \frac{\pi}{2}(m-k) \right] \\ &* \left\{ \pm \sum^{L_2}_{\beta =1}
        \omega_\beta \eta_{\beta}^k f(\pi - \cos^{-1}(\eta_{\beta})) \left[ 1 -
        \eta_\beta^2 \right]^{\frac{m-k}{2}} \right. \\ &- \left. \sum^{L_2}_{\beta
        =1} \omega_\beta \eta_\beta^k f(\cos^{-1}\eta_\beta) \left[ 1 - \eta_\beta^2
        \right]^{\frac{m-k}{2}} \right\}, \end{aligned} \end{equation}
%
where the sign of the second sum depends on the parity of $m$.  When calculating the
zonal harmonics, the integral $I_{c,m}(f(\mu_\alpha,\phi_\beta))$ reduces to the
axisymmetric solution with $m=0$. The zonal harmonics Eqn.  \eqref{eq:harmonics} can,
therefore, be calculated via the summation
%
\begin{equation} \begin{aligned} \til{J}_{i,n} \approx - \left( \frac{3}{n+3} \right)
        \left( \frac{ \delta_i \lambda_i^3 \sum^{L_1}_{\alpha =1} \omega_{\alpha}
        P_n(\mu_{\alpha}) I_{c,0}( \zeta_{i\alpha\beta}^{(n+3)})} { \sum_{j=0}^{N-1}
        \delta_j \lambda_j^3  \sum^{L_1}_{\alpha =1} \omega_{\alpha} I_{c,0}(
        \zeta_{j\alpha\beta}^{3} )} \right) \end{aligned} \label{eq:zonal_quad}
\end{equation}
%
and the tesseral harmonics likewise via
%
\begin{equation} \begin{aligned} \til{C}_{nm} &\approx&  \frac{6(n-m)!}{(n+3)(n+m)!}
        \left( \frac{ \delta_i \lambda_i^3 \sum^{L_1}_{\alpha =1} \omega_{\alpha}
        P^m_n(\mu_{\alpha})  I_{c,m}(\zeta_{i\alpha\beta}^{(n+3)}) } {
            \sum_{j=0}^{N-1} \delta_j \lambda_j^3 \sum^{L_1}_{\alpha =1}
            \omega_{\alpha} I_{c,0}(\zeta_{j\alpha\beta}^3)} \right).  \end{aligned}
        \label{eq:tesseral_quad} \end{equation}
%
There are analogous expressions for $I_{s,m}$ and $S_{nm}$, but these evaluate to
zero in all calculations presented here due to the symmetry of the model.

\subsection{Iterative procedure} \label{iterative}

We begin with initial estimates for the shape of each surface
$\zeta_{i\alpha\beta,0}$ and for the moments $\til{J}_{i,n}$, $\til{J}'_{i,n}$,
$\til{J}''_i$, $\til{C}_{i,nm}$, $\til{C}'_{i,nm}$, $\til{S}_{i,nm}$, and
$\til{S}'_{i,nm}$. For each iteration $t$ the level surfaces are then updated using a
single Newton-Raphson integration step. 
%
\begin{equation} \begin{aligned} \zeta_{i\alpha\beta,t+1} = \zeta_{i\alpha\beta,t} -
        \frac{ f( \zeta_{i\alpha\beta,t})}{ f'( \zeta_{i\alpha\beta,t}) }
    \end{aligned} \label{eq:newton} \end{equation}
%
where $f$ is the equipotential relation, Equations
\eqref{eq:equipotential_surface}~--~\eqref{eq:surface_subsatellite} for the outermost
surface and Equations
\eqref{eq:equipotential_interior}~--~\eqref{eq:interior_subsatellite} for interior
layers, and $f'$ is the first derivative of that function with respect to $\zeta$, Eqn.
\eqref{eq:equipotential_derivative}. The multipole moments are then calculated for
the updated $\zeta_{i\alpha\beta}$ via Equations
\eqref{eq:harmonics}~--~\eqref{eq:harmonics_prime_n=2}. These two steps are repeated
until all of the exterior moments, $J_{n}$, $C_{nm}$ and $S_{nm}$,
have converged such that the difference between successive iterations falls below a
specified tolerance. Starting with a naive guess for the initial state, a typical
calculation achieves a precision much higher than would be required for comparison
with \textit{Juno} measurements after about 40 iterations.

In simulations with a finite $q_{\rm rot}$ and $q_{\rm tid}$, we typically find an
initial converged equilibrium shape with a non-zero, first-order harmonic coefficient
$C_{11}$ of the order of $q_{\rm rot} \cdot q_{\rm tid}$ or smaller. This indicates
that the center of mass of the system is shifted slightly along the planet-satellite
axis from the origin of the initial coordinate system. To remove this term, we apply
a translation to the shape function of $\Delta x=-a\cdot C_{11}$ in the direction of
the satellite. This correction requires approximating the coordinates $(\mu',\phi')$
in the uncorrected frame that correspond to the quadrature points $\mu_\alpha$ and
$\phi_\beta$ in the corrected frame, so that the correct shape $\zeta$ is integrated
to find the moments in the corrected frame. For a value of $q_{\rm tid}$ similar to
the gas giants, this correction yields a body with $C_{11}$ on the order of the
specified tolerance. For systems with a much larger $q_{\rm tid}$ (of which there are
none in our planetary system), this second-order effect might affect the precision of
the calculation.  The residual effect is below the numerical noise level for the
Saturn models presented in this paper.

\subsection{Calculation of the barotrope} \label{barotrope}

We first calculate the density of each uniform layer; for the $j$th layer we have 
%
\begin{equation}
\begin{aligned}
    \rho_{j,{\rm pu}} = \frac{\sum_{i=0}^j\delta_i}{\sum_{k=0}^{N-1}
        \delta_k \lambda_k^3 \muint \phiint \zeta_k^3}.
\end{aligned}
\label{eq:barotrope}
\end{equation}
%
Using this expression, we calculate the total potential $U_{\rm pu}$ on the surface
of each layer and at the center using Equations \eqref{eq:surface_subsatellite} and
\eqref{eq:interior_subsatellite}~--~\eqref{eq:central_potential}. Since the density
is constant between interfaces, the hydrostatic equilibrium relation,
Eqn. \eqref{eq:hydrostatic} is trivially integrated to obtain the pressure at the
bottom of the $j$th layer.
%
\begin{equation}
\begin{aligned}
    P_{j,{\rm pu}} = P_{j-1,{\rm pu}} + \rho_{j-1,{\rm pu}}
    (U_{j,{\rm pu}}-U_{j-1,{\rm pu}})
\end{aligned}
\end{equation}

After obtaining a converged hydrostatic-equilibrium model for N spheroids with the
above array using the initial density profile $\delta_j$, one calculates the arrays
$U_{j,{\rm pu}}$ and $P_{j,{\rm pu}}$. Next, one calculates an array of desired
densities 
%
\begin{equation}
\begin{aligned}
    \rho_{j,{\rm pu},{\rm desired}} = \rho\left(\frac{1}{2}(P_{j+1}+P_j)\right),
\end{aligned}
\label{eq:rhodesired}
\end{equation}
%
where $\rho(P)$ is the inverse of the adopted barotrope $P(\rho)$.  Finding the
difference between the desired densities of subsequent layers then gives a new array
of $\delta_j$ for use in the next iteration. In our implementation, it is also
necessary to scale these densities by a constant factor to obtain the correct total
mass of the CMS model. 

Self-gravity from the model's rotational and tidal deformation will cause a small
change in the density profile from that expected for a spherical body.  In practice,
only relatively large changes in the shape of the body will cause a significant
deviation in the density profile. Since $q_{\rm rot} \gg q_{\rm tid}$, the influence
of rotation dominates the shape of the body. For this reason, we can use an
axisymmetric, rotation-only model as described in \citet{hubbard2013} to find a
converged density structure for a given barotrope and specified $q_{\rm rot}$, and
then perform a single further iteration with tides added to find the hydrostatic
solution for that density profile.  Because the tide-induced density changes are very
small, it is unnecessary to iterate with Eqn.  \eqref{eq:rhodesired} to relax the
configuration further for the triaxial figure. Converging the density-pressure
profile to a prescribed barotrope and a fully triaxial figure with relatively large
$q_{\rm tid}$ is significantly more computationally expensive, and is irrelevant to
any giant planet in our planetary system.

\section{Comparison with test cases} \label{tests}

\subsection{Spheroid of constant density} \label{maclaurin}

The well-known special case of a single constant-density spheroid is an
important test, because it has a closed form, analytical solution to the theory
of figures \citep{tassoul2015}. In the case of non-zero $q_{\rm rot}$ it is
conventionally referred to as the Maclaurin spheroid, as the Jeans spheroid for
finite $q_{\rm tid}$, and the Roche spheroid in the general case. In
equilibrium, the spheroid will have an ellipsoidal shape. In the limit of a
low-amplitude tidal perturbation and zero rotation, the love number for all
permitted $n$ is
%
\begin{equation}
\begin{aligned}
    k_n = \frac{3}{2(n-1)}
\end{aligned}
\label{eq:mac_kn}
\end{equation}
%
\citep{munk2009}. 

From our simulation results, we calculate the love numbers as
%
\begin{equation}
\begin{aligned}
    k_{nm} = -\frac{2}{3}\frac{(n+m)!}{(n-m)!}\frac{C_{nm}}{P_n^m(0)q_{\rm tid}}
    \left( \frac{a}{R} \right)^{2-n}.
\end{aligned}
\label{eq:kn}
\end{equation}
%
For simulations with finite $q_{\rm tid}$ and $q_{\rm rot}=0$, we find our calculated
$k_{nm}$ to be degenerate with $m$ in accordance with the analytical result. For a
given value of $n$, 
%
\begin{equation}
    k_{nm} = \begin{cases}
        0, & n \mathrm{~and~} m \mathrm{~opposite~parity} \\
        \mathrm{const}, & n \mathrm{~and~} m \mathrm{~same~parity}. \\
    \end{cases}
    \label{eq:degenerate}
\end{equation}
%
Figure \ref{fig:maclaurin_tidal_only} shows the calculated $k_n$ for the non-rotating
Jeans spheroid as a function of $q_{\rm tid}$ up to order $n=6$, with $R/a$ taken
to be that for Tethys and Saturn. For a small tidal perturbation, we find that $k_n$
approaches the analytical result of Eqn. \eqref{eq:mac_kn}. Conversely, as $q_{\rm
rot}$ approaches unity from below, the love numbers diverge, with $k_n$ decreasing
for $n\leq3$ and increasing for $n>3$. The departure from the analytical solution
becomes significant ($\left|\Delta k_n\right| > 0.1$) for $-q_{\rm tid}>10^{-3}$,
whereas for values representative of the largest Saturnian satellites, $k_2$ matches
the analytic value to within our numerical precision.

In general, the tidal response of a gas giant planet will not be a perturbation to a
perfect sphere, but to a spheroidal shape dominated by rotational flattening.
Therefore, simulation of the tidal response in the absence of rotation is not
generally applicable to real gas giants. When we simulate a Roche spheroid with
both finite $q_{\rm rot}$ and $q_{\rm tid}$, we find a different behavior for
$k_{nm}$ as defined by Eqn. \eqref{eq:kn}. Figure \ref{fig:maclaurin_rotation_effect}
shows the calculated $k_{nm}$ for a spheroid with a constant $q_{\rm tid}$
and a variable $q_{\rm rot}$. When the magnitude of $q_{\rm rot}$ is comparable to
$q_{\rm tid}$, the tidal response matches the expected analytical result. However,
for $q_{\rm rot}>10^{-3}$, we can see that the degeneracy of $k_{nm}$ with $m$ is
broken, and all permitted $k_{nm}$ deviate from the expected values. In other words,
Eqn.  \eqref{eq:degenerate} becomes
%
\begin{equation}
    \begin{cases}
        k_{nm}=0, & n \mathrm{~and~} m \mathrm{~opposite~parity} \\
        k_{nm}\neq\mathrm{const}, & n \mathrm{~and~} m \mathrm{~same~parity}, \\
    \end{cases}
    \label{eq:nondegenerate}
\end{equation}
%
and all permitted $k_{nm}$ deviate from the expected values. We also note that these
deviations become pronounced earlier for the higher order $n$.

\subsection{Two-layered spheroid} \label{two_layer}

Proceeding to more complicated interior structures has proved challenging for
analytical or semi-analytical methods. Even the next simplest model with two
constant-density layers does not have a closed form solution for arbitrary order $n$.
\citet{folonier2015} present an extension of Clairaut theory for a multi-layer planet
under the approximation that the level surfaces are perfect ellipsoids. Under this
approximation, they derive an analytic solution for the distortion in response to a
tidal perturbation only. This yields an expression for $k_2$ as a function of two
ratios of properties of the two layers, $a_1/a$ and $\rho_0/\rho_1$. Table
\ref{tab:folonier_table} shows a comparison of our calculated $k_2$ with the analytic
result from \citet{folonier2015} for a selection of parameters spanning a range of
$a_1/a$ and $\rho_0/\rho_1$. All of our results using the CMS method differ from
those using Clairaut theory by less than $10^{-5}$. This provides an important test
of the correctness of the interior potentials used in our approach. It also indicates
that ellipsoids, while not exact, are a very good approximation for the degree 2
tidal response shape in the limit of very small $q_{\rm tid}$, and $q_{\rm rot}=0$. 

\subsection{Polytrope of index unity} \label{polytrope}

The polytrope of index unity defines a more realistic barotrope that also lends
itself to semi-analytic analyses. It corresponds to the relation
%
\begin{equation}
    P = K\rho^2
    \label{eq:poly}
\end{equation}
%
where the polytropic constant $K$ can be chosen to match the planet's physical
parameters. For a non-rotating $n=1$ polytrope, the density distribution is given by
%
\begin{equation}
    \rho = \rho_c \frac{\sin \pi \lambda}{\pi \lambda}
    \label{eq:poly_rho}
\end{equation}
%
where $\rho_c$ is the density at the center of the planet.  To obtain the first
approximation of $\delta_j$, we differentiate Eqn. \eqref{eq:poly_rho} by $\lambda$:
%
\begin{equation}
    \frac{d(\rho/\rho_c)}{d\lambda} = \frac{\cos \pi \lambda}{\lambda} 
    - \frac{\sin \pi \lambda}{\pi \lambda^2}.
\end{equation}
%
We then correct this profile to be consistent with the given $q_{\rm rot}$ via the
method introduced in Section \ref{barotrope}. Scaling the densities to maintain the
total mass of the planet has a straightforward interpretation for a polytropic
barotrope, as it is equivalent to changing $K$.

For the constant-density Roche spheroid the lowest degree love number was 
%
\begin{equation}
    k_2 = \frac{3}{2}.
\end{equation}
% 
Considering only the linear response to a purely rotational perturbation, we define a
general degree 2 linear response parameter $\Lambda_2$ as 
%
\begin{equation}
    J_2 = \Lambda_2 q_{\rm rot}.
    \label{eq:linear_response}
\end{equation}
%
Whereas $\Lambda_2=1/2$ for the Roche spheroid, for the polytrope of index unity
the analytic result is \citep{hubbard1975} 
%
\begin{equation}
\begin{aligned}
    \Lambda_2 = \left( \frac{5}{\pi^2} - \frac{1}{3} \right).
\end{aligned}
\end{equation}
%
Considering linear response only, one finds in general
%
\begin{equation}
\begin{aligned}
    k_2 = 3 \Lambda_2 ,
\end{aligned}
 \label{eq:k2_lambda2}
\end{equation}
%
valid in the limit $q_{\rm rot} \ll 1$ and $q_{\rm tid} \ll 1$, for any barotrope in
hydrostatic equilibrium.  Thus, for the polytrope of index unity in this limit,
%
\begin{equation}
\begin{aligned}
    k_2 = \frac{15}{\pi^2} - 1 = 0.519817755.
    \label{eq:poly_estimate}
\end{aligned}
\end{equation}
%
We compare this to a CMS simulation of the $n=1$ polytrope model with 128 layers,
$q_{\rm rot}=0$, $q_{\rm tid}=10^{-6}$, and Tethys' $R/a$. The simulation results
agree with the expected relation $J_2=2C_{22}$ to numerical precision, and yield
$k_2=0.519775$. This provides a test of the multi-layer CMS approach subject to a
tidal-only perturbation.  The CMS result matches our Eqn. \eqref{eq:poly_estimate}
benchmark to better than the precision with which we could measure this parameter
using the \textit{Juno} spacecraft.  The small difference can be attributed to
approximation of a continuous polytrope by 128 layers in the CMS simulation.
\citet{wisdom2016} (Eqn.  15) show the relative discretization error of a CMS
polytrope model to be $\sim 10^{-3}$ for $N=128$, roughly consistent with our
calculated difference. 

Similar to the calculations on the constant density spheroid in Section \ref{maclaurin}, we
performed additional $N=128$ polytrope simulations with finite $q_{\rm tid}$ and
$q_{\rm rot}=0$. Once again, we find our calculated $k_{nm}$ to be degenerate with
$m$ for the tidal-only simulations, in agreement with Eqn. \eqref{eq:degenerate}.
Figure \ref{fig:polytrope_tidal_only} shows the behavior of $k_n$ for $n\leq6$ for
these tidal-only polytrope simulations.  We only present these results up to $q_{\rm
tid}\sim10^{-4}$, because above that value effects of the triaxial shape on the
pressure-density profile would require iterated relaxation to the polytropic
relation, as discussed in Section \ref{barotrope}. We observe that realistic values
for $q_{\rm tid}$ have negligible effect on the tidal response. Even for the
Io-Jupiter system, the effect of finite $q_{\rm tid}$ on $k_{nm}$ is near the
numerical noise level. The general behavior is quite similar to the case of the
single Jeans spheroid.  For small tidal perturbations, the polytrope $k_n$
approach values smaller than the single spheroid case, with $k_2$ asymptoting to
the analytic limit in Eqn.  \eqref{eq:poly_estimate}.  Similar to the single  
spheroid, the behavior as $q_{\rm rot}$ increases from zero sees $k_n$ decrease for
$n\leq3$ and increase for $n>3$.  The deviation from the low $q_{\rm tid}$ value is
also less pronounced for the more realistic polytrope density distribution than for
the single spheroid. This is to be expected since there is less mass concentrated
in the outer portion of the polytrope model. 

Figure \ref{fig:polytrope_rotation_effect} shows the effect of variable $q_{\rm rot}$
on polytrope models with constant $q_{\rm tid}$. Once again, we find that $k_{nm}$
degeneracy with respect to $m$ breaks, in agreement with Eqn.
\eqref{eq:nondegenerate}, as $q_{\rm rot}$ increases. Although the splitting of
$k_{nm}$ is somewhat diminished from the single Roche spheroid results, the
deviations are still significant at large values of $q_{\rm rot}\sim 10^{-2}$
consistent with the rapidly-rotating gas giants. The shift in $k_{nm}$ shows a nearly
linear increase in magnitude with increasing $q_{\rm rot}$, with potentially
observable increases in $k_2$ for both the ice giant and gas giant planets. The
general behavior of $k_{nm}$ is very similar between these tests with two very
different density profiles. The relative magnitudes and directions of all $k_{nm}$ up
to $n=6$ are similar between the two cases. This indicates that the effect should be
ubiquitous in all fast-spinning liquid bodies, and relatively insensitive to the
density profile of the planet.

\begin{figure}[h!]  
  \centering
    \includegraphics[width=26pc]{figs/spheroid.pdf}
\caption{ Conceptual diagram of a Concentric Maclaurin Spheroid (CMS) model with a tidal
perturbation from a satellite.}
\label{fig:spheroid}
\end{figure}

\begin{figure}[h!]  
  \centering
    \includegraphics[width=22pc]{figs/maclaurin_tidal_only.pdf}
\caption{ The effect of tidal perturbation strength on the tidal love numbers of a
non-rotating constant density (Jeans) spheroid up to order 6. The love numbers $k_n$ are degenerate
with respect to $m$. The orbital radius is taken to be that of Tethys.}
\label{fig:maclaurin_tidal_only}
\end{figure}

\begin{figure}[h!]  
  \centering
    \includegraphics[width=22pc]{figs/maclaurin_rotation_effect.pdf}
\caption{ The effect of rotation rate on the tidal love numbers of a constant density
    (Roche) spheroid up to order 6. The $k_{nm}$ for a given $n$ are found to split
    at high rotation rates. $q_{\rm tid}$ is kept constant at $1.0\times10^{-6}$, and
    the orbital radius is taken to be that of Tethys.}
\label{fig:maclaurin_rotation_effect}
\end{figure}

\begin{figure}[h!]  
  \centering
    \includegraphics[width=22pc]{figs/polytrope_tidal_linear.pdf}
\caption{ The effect of tidal perturbation strength on the tidal love numbers of a
    non-rotating planet with an $n=1$ polytrope equation of state, up to order 6.
    $\Delta k_n$ is the shift in love number $k_n$ from the limit of low $q{\rm
    tid}$. The love numbers $k_n$ are degenerate with respect to $m$. The orbital
    radius is taken to be that of Tethys. The vertical, dashed gray lines show $q_{\rm tid}$
    for Tethys-Saturn and Io-Jupiter.}
\label{fig:polytrope_tidal_only}
\end{figure}

\begin{figure}[h!]  
  \centering
    \includegraphics[width=22pc]{figs/polytrope_rotation_both.pdf}
\caption{ Top: The effect of rotation rate on the tidal love numbers of a planet with an
    $n=1$ polytrope equation of state, up to order 6. The $k_{nm}$ for a given $n$
    are found to split at high rotation rates. $q_{\rm tid}$ is kept constant at
    $1.0\times10^{-6}$, and the orbital radius is taken to be that of Tethys. The
    vertical, dashed gray lines show $q_{\rm rot}$ for Neptune, Uranus, Jupiter and Saturn.
    Bottom: Shift in $k_{nm}$ as a function of $q_{\rm rot}$ on a linear scale.}
\label{fig:polytrope_rotation_effect}
\end{figure}

%\input{tabs/folonier_table}

\chapter{Tidal Response of Jupiter and Saturn}\label{chap7}

\section{Barotropes}


We assume the liquid planet is in hydrostatic equilibrium,
%
\begin{equation} \nabla P = \rho \nabla U,     
   \label{eq:hydrostatic2} \end{equation}
%
where $P$ is the pressure, $\rho$ is the mass density and $U$ the total effective
potential. Modeling the gravitational field of such a body requires a barotrope
$P(\rho)$ for the body's interior. In this paper, we use the barotrope of
\citet{hubbard2016}, constructed from \textit{ab initio} simulations of
hydrogen-helium mixtures \citep{militzer2013a,militzer2013b}. The $P(\rho)$ relation
is interpolated from a grid of adiabats determined from density functional 
molecular dynamics (DFT-MD) simulations using the Perdew-Burke-Ernzerhof (PBE) functional
\citep{PBE} in combination with a thermodynamic integration technique. The
simulations were performed with cells containing $N_{He}=18$ helium and $N_{H}=220$
hydrogen atoms, corresponding to a solar-like helium mass fraction $Y_0=0.245$. An
adiabat is characterized by an entropy per electron $S/k_B/N_e$
\citep{militzer2013b}, where $k_B$ is Boltzmann's constant and $N_e$ is the number of
electrons. Hereafter we refer to this quantity simply as $S$.

In our treatment, the term ``entropy'' and the symbol $S$ refer to a particular
adiabatic temperature $T(P)$ relationship for a fixed composition H-He mixture
($Y_0=0.245$) as determined from the \textit{ab initio} simulations.  For Jupiter,
the value of $S$ in the outer portion of the planet is determined by matching the
$T(P)$ measurements from the Galileo atmospheric probe (see Figure \ref{fig:eos}).
This adiabatic $T(P)$ is assumed to apply to small perturbations of composition, in
terms of both helium fraction and metallicity. \citet{hubbard2016} demonstrated that
these compositional perturbations have a negligible effect on the temperature
distribution in the interior.

The density perturbations to the equation of state are estimated using the additive
volume law,
%
\begin{equation} 
   V(P, T) = V_H(P, T) + V_{He}(P, T) + V_{Z}(P,T),
\label{eq:volume_law}
\end{equation}
%
where the total volume $V$ is the sum of partial volumes of the main components $V_H$
and $V_{He}$, the heavy element component $V_Z$. \citet{hubbard2016} demonstrated
that this leads to a modified density $\rho$ in terms of the original H-He EOS
density $\rho_0$,
%
\begin{equation} 
   \frac{\rho_0}{\rho} = \frac{1-Y-Z}{1-Y_0} + 
   \frac{ZY_0 + Y - Y_0}{1-Y_0}\frac{\rho_0}{\rho_{He}} + Z\frac{\rho_0}{\rho_Z},
\label{eq:density_ratio}
\end{equation}
%
in which all densities are are evaluated at the same $T(P)$ and $Y_0$ is the helium
fraction used to calculate the H-He equation of state.


\section{Saturn's tidal response} \label{saturn}

\subsection{Saturn interior models}

\citet{lainey2016} present the first determination of the love number $k_2$ for a gas
giant planet using a dataset of astrometric observations of Saturn's coorbital moons.
Their observed value $k_2=0.390 \pm 0.024$ is much larger than the theoretical
prediction of 0.341 by \citet{gavrilov1977}. Here we present calculations suggesting
that the enhancement of Saturn's $k_2$ is the result of the influence of the planet's
rapid rotation, rather than evidence for a non-static tidal response or some other
breakdown of the hydrostatic theory.

For the purposes of this calculation, we use two relatively simple models for
Saturn's interior structure, fitted to physical parameters determined by the
\textit{Voyager} and \textit{Cassini} spacecraft. Table \ref{tab:saturn_params}
summarizes the physical parameters used in our models. We fit our models to minimize
the difference in zonal harmonics from those determined from \textit{Cassini}
\citep{Jacobson2006}.  We consider two different internal rotation rates based on
magnetic field measurements from \textit{Voyager} \citep{desch1981} and
\textit{Cassini} \citep{giampieri2006}, which lead to two different values of $q_{\rm
rot}$. 

In principle, the tidal response of a heterogeneous body will also be different for
satellites with different sizes and orbital parameters. To address this, we also
consider the effect of two major satellites, Tethys and Dione, with different values
for $q_{\rm tid}$ and $R/a$ \citep{archinal2011}. These two satellites, along with
their respective coorbital satellites, were used in the determination of $k_2$ by
\citet{lainey2016}.

For the interior density profile, our first model assumes a constant-density
core surrounded by a polytropic envelope following Eqn. \eqref{eq:poly}. We
constrain the radius of the core to be $a_{\rm core}/a=0.2$, leaving the mass
$m_{\rm core}/M$ as a parameter which is adjusted to match the observed Saturn
$J_2$.  The fitted model using the \textit{Voyager} rotation period matches
both $J_2$ and $J_4$ to within the error bars, but with the \textit{Cassini}
rotation period it matches only $J_2$.  In hydrostatic equilibrium, the two
different rotation rates lead to differences in shape of equipotential surfaces
and, therefore, also to different best fits to $m_{\rm core}/M$. The envelope
polytrope is scaled in order to maintain $M$. Figure
\ref{fig:density_structure} shows the density profile of one such model,
compared to other density models. We consider a model with a total of 128
layers, for which the CMS model has a discretization error \citep{wisdom2016}
smaller than uncertainty in the observations of Saturn's $k_2$.

\begin{figure}[h!]  
  \centering
    \includegraphics[width=22pc]{figs/density_profile.png} \caption{ 
Density structure of simple Saturn models, all fitted to Saturn's observed
$J_2$ \citep{Jacobson2006}.  The blue curve shows an $N=128$ model with a
constant-density core within $r=0.2 a$ and a polytropic outer envelope.  The
red curve shows an $N=4$ model with the same core radius and two additional
spheroids, adjusted to fit both $J_2$ and $J_4$.  For comparison, the dash-dot
curve (teal) shows Saturn model MS24 of \citet{gudkova1999}.  The grey solid
curve shows an unpublished Saturn model based on the density-functional theory,
molecular-dynamics (DFT-MD) equation of state for hydrogen-helium, as used in
the Jupiter model of \citet{hubbard2016}. Figure Credit: William Hubbard.
}
\label{fig:density_structure}
\end{figure}

Our second model has only four spheroids ($N=4$), also depicted in Figure
\ref{fig:density_structure}, with densities and radii adjusted to yield agreement
with both observed $J_2$ and observed $J_4$ as given in Table
\ref{tab:saturn_params}. With a zero-density outermost layer, this leaves two free
paramters, making it the simplest model that can match $J_2$ and $J_4$ exactly.

%\input{tabs/model_param_table}
\begin{table}
    \centering
\caption{Saturn Model Parameters. Identical parameters for Saturn are used with the
    exception of $q_{\rm rot}$, for which the rotation rate from both
    \textit{Cassini} and \textit{Voyager} are considered. A constant core density is
    fitted to match $J_2$, $J_4$, and $J_6$ for a converged figure.
    \label{tab:saturn_params}}
\begin{tabular}{rrrr}
        \hline
{} & Cassini  & Voyager & {} \\
\hline
$GM$ & $3.7931208 \times 10^{7}$ $^{a}$ & - & ${\rm (km^3/s^2) }$
\\
$a$ & $6.0330 \times 10^{4}$       $^{a}$ & - & ${\rm (km) }$  \\ 
$J_2  \times  10^6$  &  $16290.71$ $^{a}$ &  -  &       \\
$J_4  \times  10^6$  &  $-935.83$  $^{a}$ &  -  &       \\
$J_6  \times  10^6$  &  $86.14$    $^{a}$ &  -  &       \\
$q_{\rm rot} $  & $0.1516163$ $^{b}$ & $0.1553029$ $^{c}$ &\\
$r_{\rm core} / a$ & $0.2$ & - &  \\ 
$m_{\rm core} / M$  &  $0.133146$ & $0.140478$ &\\
\hline
\hline
  & Tethys  & Dione & \\
  \hline
$q_{\rm tid} $  & $-2.791103 \times 10^{-8} $ $^{d}$
& $-2.364582 \times 10^{-8}$ $^{d}$ & \\
$R/a $  & $4.8892$ $^{d}$ & $6.2620$ $^{d}$ & \\ 
\hline
\multicolumn{4}{l}{ a. \citet{Jacobson2006}, b. \citet{giampieri2006},} \\
\multicolumn{4}{l}{ c. \citet{desch1981}, d.  \citet{archinal2011}}
\end{tabular}
\end{table}


Finally, we include a third model using a more realistic H-He equation of state
based on DFT-MD simulations, following the preliminary Jupiter model of
\citet{hubbard2016} with a Saturn adiabat.  This model has a density
discontinuity (Fig. \ref{fig:density_structure}) at 0.76 Mbar where the Saturn
adiabat crosses the H-He phase separation curve of \citet{Morales2009}. This
allows the $J_2$ and $J_4$ values to be fitted exactly, by changing the
metallicities above and below the discontinuity. While still schematic, it is
the most realistic of our Saturn models.


The two simple models, while not particularly realistic, capture the major features
of Saturn's internal structure. It is well established that the details of Saturn's
internal structure are largely degenerate, with a wide range of possible core sizes
and densities adequately matching the few observational constraints
\citep{kramm2011,helled2013,Nettelmann2013}.  The qualitative similarities between
our single spheroid and polytrope simulations (Sections \ref{maclaurin} and
\ref{polytrope}) indicate that the rotational enhancement of $k_2$ should be a robust
prediction regardless of the particular details of the interior profile. A comparison
between our polytrope plus core and four layer models provides another test of the
sensitivity of $k_2$ to interior structure.  We do not consider here the influence of
differential rotation \citep{hubbard1982,Kong2013,cao2015,wisdom2016}, which might
have an influence on the gravitational response in comparison to the solid-body
rotation considered here.  

\subsection{Calculated $k_2$ for Saturn} \label{calc_k2}

We take our baseline model to be the $N=128$ CMS core plus polytrope model with
physical parameters fitted to \textit{Cassini} observations. Figure
\ref{fig:saturn_zonal} shows the calculated zonal harmonics $J_n$ up to order
$n=30$.  The even $J_n$ decrease smoothly in magnitude with increasing $n$,
with the slope decreasing at higher $n$.  $J_n$ is negative when $n$ is
divisible by 4, and positive otherwise.  The calculated $J_n$ are essentially
indistinguishable from those calculated for the rotation only case with the
same $q_{\rm rot}$, as is expected given $q_{\rm rot} \gg q_{\rm tid}$. We may
estimate the maximum effect of differential rotation by considering the change
of the calculated $k_2$ as we change the overall planetary rotation period from
the Voyager value ($q_{\rm rot} = 0.155303$) to the Cassini value ($q_{\rm rot}
= 0.151616$), a relative decrease in $q_{\rm rot} $ of about $2 \%$.  From
Table 3, we see that this change increases $k_2$ by about $2 \%$ (holding $J_2$
fixed).  The net effect of a deep-seated smooth variation of rotation rate from
the Voyager value near the equator to the Cassini value near the pole would
presumably be smaller, depending on how much mass is involved in the
differential flow.  \citet{cao2015} have shown that the effect of
realistic deep flow patterns on low order zonal harmonics is small, but a more
quantitative evaluation of their effect on Saturn's $k_2$ remains to be done.

\begin{figure}[h!]  
  \centering
    \includegraphics[width=22pc]{figs/saturn_Jn.pdf}
\caption{ The zonal harmonics $J_n$ for the \textit{Cassini} Saturn model. Positive
values are shown as filled and negative  as empty.}
\label{fig:saturn_zonal}
\end{figure}

Figure \ref{fig:saturn_tesseral} shows the magnitude of $C_{nm}$ for the core plus
polytrope model with \textit{Cassini} rotation. Changing the number of layers,
satellite parameters or the rotation rate to the \textit{Voyager} value leads to a
shift in the values, but the relative magnitudes and signs of $C_{nm}$ remain
approximately the same. In the same figure, we also compare the $C_{nm}$ for a
non-rotating planet having the same density profile $\rho( \lambda_i)$. Here we see
significant shifts in the $C_{nm}$ magnitudes, although the signs remain the same.
For the rotating model, $C_{nm}$ is similar for most points where $n=m$, but with
magnitudes significantly larger when $m<n$. The only exception to this trend is
$C_{31}$ which is lower for the rotating model. These results are all broadly
consistent with the splitting of $k_{nm}$ observed for the polytrope in Section
\ref{polytrope}.

\begin{figure}[h!]  
  \centering
    \includegraphics[width=22pc]{figs/saturn_Cnm_compare.pdf}
\caption{ In red, the tesseral harmonics $C_{nm}$ for the \textit{Cassini} Saturn
model. In black, $C_{nm}$ for the same density profile and same value of $q_{\rm
tid}$, but with $q_{\rm rot}=0$. Positive values are shown as filled and negative as
empty.} \label{fig:saturn_tesseral}
\end{figure}

Table \ref{tab:saturn_results} summarizes our calculated values for $k_2$ for several
different models. The identifying labels ``Cassini'' and ``Voyager'' use the
observed rotation rate from \citet{Jacobson2006}, and \citet{desch1981}
respectively, while ``non-rotating'' is a model with $q_{\rm rot}=0$. The
``non-rotating'' model uses the same ``Cassini'' density profile, meaning that
its density-pressure profile has not been relaxed to be in equilibrium for zero
rotation.  It does, however, allow us to quantify the effect of rotation on the
tidal response by comparison with the ``Cassini'' model. ``Tethys'' and
``Dione'' refer to models with the satellite parameters $q_{\rm tid}$ and $R/a$
corresponding to those satellites, whereas ``no~tide'' is an analogous model
with finite $q_{\rm rot}$ only. ``$N=128$'' uses the polytrope outer envelope
with constant density inner core, whereas ``$N=4$'' is the model which
independently adjusts layer densities to match the observed $J_2$ and $J_4$.
The ``DFT-MD'' models use the H-He equation of state from
\citet{hubbard2016} with $N=511$ layers.

%\input{tabs/saturn_table}
\begin{table}
    \centering
\caption{Calculated Saturn tidal responses \label{tab:saturn_results}}
\begin{tabular}{ccrcr}
    \hline
    {model$^a$} &  & gravitational moment  & & normalized moment \\
    \hline
{Cassini} &  $J_2$     &  $1.62907100025\times10^{-2}   $  &  $J_2/q_{\rm  rot}$  &  $0.10744694879478             $  \\
 {no~tide}   &  $J_4$     &  $-9.2027941201\times10^{-4}   $  &  $J_4/q_{\rm  rot}$  &  $-0.606979160784\times10^{-2} $  \\
 {$N=128$}  &  $J_6$     &  $8.014294995\times10^{-5}     $  &  $J_6/q_{\rm  rot}$  &  $0.5285905549\times10^{-3}    $  \\
 \hline
{non-rotating}  & $C_{22}$  &  $8.5288\times10^{-10}         $  &  $k_2         $      &  $0.36669                      $ \\
{Tethys}        & $J_2$     &  $1.70576\times10^{-9}         $  &  $J_2/q_{\rm  rot}$  &  {-}               \\
{$N=128$}       & $J_4$     &  $-1.351\times10^{-11}         $  &  $J_4/q_{\rm  rot}$  &  {-}               \\
{polytrope}                        & $J_6$     &  $2.2\times10^{-13}            $  &  $J_6/q_{\rm  rot}$  &  {-}               \\
\hline
{Cassini}  & $C_{22}$  &  $9.6070\times10^{-10}         $  &  $k_2         $      &  $0.41304                      $ \\
{Tethys}   & $J_2$     &  $1.629071017501\times10^{-2}  $  &  $J_2/q_{\rm  rot}$  &  $0.1074469499328              $ \\
{$N=128$}  & $J_4$     &  $-9.2027943932\times10^{-4}   $  &  $J_4/q_{\rm  rot}$  &  $-0.60697917880\times10^{-2}  $ \\
{polytrope}                   & $J_6$     &  $8.01429541\times10^{-5}      $  &  $J_6/q_{\rm  rot}$  &  $0.5285905822\times10^{-3}    $ \\
{Voyager}  & $C_{22}$  &  $9.4136\times10^{-10}         $  &  $k_2         $      &  $0.40473                      $ \\
\hline
{Tethys}   & $J_2$     &  $1.629071048760\times10^{-2}  $  &  $J_2/q_{\rm  rot}$  &  $0.1048963407747              $ \\
{$N=128$}  & $J_4$     &  $-9.3570887868\times10^{-4}   $  &  $J_4/q_{\rm  rot}$  &  $-0.60250556585\times10^{-2}  $ \\
{polytrope}                   & $J_6$     &  $8.30176108\times10^{-5}      $  &  $J_6/q_{\rm  rot}$  &  $0.534552720\times10^{-3}     $ \\
\hline
{Cassini}  & $C_{22}$  &  $8.1325\times10^{-10}         $  &  $k_2         $      &  $0.41272                      $ \\
{Dione}    & $J_2$     &  $1.629071019035\times10^{-2}  $  &  $J_2/q_{\rm  rot}$  &  $0.1074469500340              $ \\
{$N=128$}  & $J_4$     &  $-9.2027943688\times10^{-4}   $  &  $J_4/q_{\rm  rot}$  &  $-0.60697917719\times10^{-2}  $ \\
{polytrope}                   & $J_6$     &  $8.01429534\times10^{-5}      $  &  $J_6/q_{\rm  rot}$  &  $0.528590578\times10^{-3}     $ \\
\hline
{Cassini}   &  $C_{22}$  &  $9.6219\times10^{-10}         $  &  $k_2         $      &  $0.41368                      $  \\
{Tethys}    &  $J_2$     &  $1.629071019560\times10^{-2}  $  &  $J_2/q_{\rm  rot}$  &  $0.1074469500686              $  \\
{$N=4$}     &  $J_4$     &  $-9.3583002600\times10^{-4}   $  &  $J_4/q_{\rm  rot}$  &  $-0.61723571821\times10^{-2}  $  \\
                    &  $J_6$     &  $8.61400043\times10^{-5}      $  &  $J_6/q_{\rm  rot}$  &  $0.568144705\times10^{-3}     $  \\
                    \hline
{Cassini}   &  $C_{22}$  &  $9.6235\times10^{-10}         $  &  $k_2         $      &  $0.41375                     $  \\
{Tethys}    &  $J_2$     &  $1.629070920013\times10^{-2}  $  &  $J_2/q_{\rm  rot}$  &  $0.1074469435029              $  \\
{$N=511$}     &  $J_4$     &  $-9.3582993628\times10^{-4}   $  &  $J_4/q_{\rm  rot}$  &  $-0.61723565903\times10^{-2}  $  \\
{DFT-MD}                    &  $J_6$     &  $8.09366588\times10^{-5}      $  &  $J_6/q_{\rm  rot}$  &  $0.533825538\times10^{-3}     $  \\
\hline
{Voyager}   &  $C_{22}$  &  $9.4185\times10^{-10}         $  &  $k_2         $      &  $0.40495                      $  \\
{Tethys}    &  $J_2$     &  $1.629070988378\times10^{-2}  $  &  $J_2/q_{\rm  rot}$  &  $0.104896336887              $  \\
{$N=511$}     &  $J_4$     &  $-9.3583000384\times10^{-4}   $  &  $J_4/q_{\rm  rot}$  & $-0.60258355868\times10^{-2}  $  \\
 {DFT-MD}                   &  $J_6$     &  $8.16661484\times10^{-5}      $  &  $J_6/q_{\rm  rot}$  &  $0.525850615\times10^{-3}     $  \\
\hline
\multicolumn{5}{l}{a. Models are denoted by: rotation rate from \textit{Cassini} or \textit{Voyager}, satellite}\\
\multicolumn{5}{l}{parameters for Tethys or Dione, number of layers $N$, and the equation of state } \\
\multicolumn{5}{l}{used. DFT-MD refers to the H-He equation of state from \citet{hubbard2016}.}
\end{tabular}
\end{table}

All of the rotating and non-rotating models yields a calculated $k_2$ value matching
the observation of \citet{lainey2016} within their error bars, with the non-rotating
models are on the low end of that range and the rotating models on the high end of
that range. Our baseline model yields $k_2=0.4130$, while using the \textit{Voyager}
observations yields a value $\sim$0.008 lower. We find that the difference between
the $k_2$ values associated with the satellites Tethys and Dione is $\sim$0.0003,
well below the current sensitivity limit. Using the $\sim$2.5\% higher ``Voyager''
rotation rate leads to a decrease of $\sim$0.01 in $k_2$. 

In Table \ref{tab:saturn_results}, we also show the calculated $J_2$, $J_4$ and $J_6$
following the convergence of the gravitational field in response to the tidal
perturbation. For the core plus polytrope model, the rotation rate from
\textit{Voyager} is more consistent with the $J_4$ and $J_6$ from
\citet{Jacobson2006}. This doesn't necessarily mean that the \textit{Voyager}
rotation rate is more correct, just that it allows a better fit for our simplified
density model. Nonetheless, our fitted gravitational moments are much closer to each
other than to those from the pre-\textit{Cassini} model of \citet{gavrilov1977}.

In comparison to the other models, the outlier is the non-rotating model, which
underestimates the $k_2$ by $\sim9.4$\% compared to a rotating body with the same
density distribution.  This calculated enhancement accounts for most of the
difference between the observation of $k_2=0.390 \pm .024$ \citep{lainey2016} and the
classical theory result of 0.341 \citep{gavrilov1977}. We attribute our non-rotating
model's larger $k_2$ to our different interior model which matches more recent
constraints on Saturn's zonal gravitational moments $J_2$--$J_6$, although the
\citet{lainey2016} error bars are still large enough to permit our non-rotating
model.

We note that the later theoretical prediction of 0.386 by \citet{vorontsov1984} is
also compatible with the observed $k_2$. Their method considers the effect of
free-oscillations on the tidal response of giant planets. While our rotating models
yield higher values of $k_2$ than \citet{vorontsov1984} our ``non-rotating'' model
produces a $k_2$ smaller than theirs by $\sim$0.02. In principle, it is difficult to
make precise comparisons between models, because of different assumptions about the
interior structure. While consideration of dynamic tidal effects is beyond the scope
of this paper, both effects are likely to influence the tidal response of a real
planet.

%\input{tabs/combined_table}
\begin{table}
    \centering
\caption{Calculated Jovian Tidal Responses\label{tab:combined_table}}
\begin{tabular}{ccc|r}
    \hline
    {planet} & rotation rate & satellite  & $k_2$\\
    \hline
{Jupiter}  &  {Non-rotating}              &  {Io}
&  $0.53725$  \\
\hline
{}         &  {Galileo$^{a}$}  &
{Io$^{a}$}        &  $0.58999$  \\
{}         &  {}                          &
{Europa$^{a}$}    &  $0.58964$  \\
{}         &  {}                          &
{Ganymede$^{a}$}  &  $0.58949$  \\
\hline
{Saturn}   &  {Non-rotating}              &  {Tethys$^{a}$}    &  $0.36669$  \\
\hline
{}         &  {Cassini$^{b}$}  &
{Tethys$^{a}$}    &  $0.41375$  \\
{}         &  {}                          &
{Dione$^{a}$}     &  $0.41272$  \\
\hline
{}         &  {Voyager$^{c}$}  &
{Tethys$^{a}$}    &  $0.40495$  \\
\hline
\multicolumn{4}{l}{ a. \citet{archinal2011}, b. \citet{giampieri2006},} \\
\multicolumn{4}{l}{c. \citet{desch1981}}
\end{tabular}
\end{table}

In addition to the difference in $k_2$, the non-rotating model also predicts slightly
different tidal components of the zonal gravitational moments. Finding the difference
in values between the ``no tide'' model and the analogous tidal model yields
$J_{2,{\rm tid}}=1.7254\times10^{-10}$, $J_{4,{\rm tid}}=-2.732\times10^{-11}$ and
$J_{6,{\rm tid}}=4.14\times10^{-12}$, which are different than calculated zonal
moments for the ``non-rotating'' model.

It may be initially surprising that the four-layer model and the semi-realistic
DFT-MD based models  yield a $k_2$ value only slightly different than the
polytrope model. The three models represent very different density structures
even though they lead to similar low-order zonal harmonics. The fact these
models are indistinguishable by their $k_2$ suggests that the tidal response of
Saturn is only a weak function of the detailed density structure within the
interior of the planet. Indeed, the two models matching $J_4$ are closer to
each other than to the polytrope model that does not match $J_4$. This behavior
can be understood by referring to Eqn.  \eqref{eq:k2_lambda2}, which shows that
to lowest order, $k_2$ and $\Lambda_2$ contain the same information about
interior structure.  This statement is not true when we include a nonlinear
response to rotation and tides.  Thus, future high-precision measurements of
the $k_{nm}$ of Jovian planets, say to better than $0.1\%$, will be useful for
constraining basic parameters such as the interior rotation rate of the planet,
and may help to break the current degeneracy of interior density profiles.  The
theory presented in this paper is intended to match the anticipated precision
of such future measurements.

\section{Jupiter's tidal response} \label{jupiter}

The \textit{Juno} spacecraft began studying Jupiter at close range following its
orbital insertion in early July 2016. The unique low-periapse polar orbit and precise
Doppler measurements of the spacecraft's acceleration will yield parameters of
Jupiter's external gravitational field to unprecedented precision, approaching a
relative precision of $\sim 10^{-9}$ \citep{kaspi2010}. In addition to providing
important information about the planet's interior mass distribution, the
non-spherical components of Jupiter's gravitational field should exhibit a detectable
signal from tides induced by the planet's closer large moons, possibly superimposed
on signals from mass anomalies induced by large-scale dynamic flows in the planet's
interior \citep{cao2015,kaspi2010,kaspi2013}.

As a benchmark for comparison with expected \textit{Juno} data, \citet{hubbard2016}
constructed static interior models of the present state of Jupiter, using a
barotropic pressure-density $P(\rho)$ equation of state for a near-solar mixture of
hydrogen and helium, determined from \textit{ab intio} molecular dynamics simulations
\citep{militzer2013a,militzer2013b}. In this paper, we extend those models to predict
the static tidal response of Jupiter using the three-dimensional concentric Maclaurin
spheroid (CMS) method \citep{wahl2016}.

The \citet{hubbard2016} preliminary Jupiter model is an axisymmetric, rotating model
with a self-consistent gravitational field, shape and interior density profile. It is
constructed to fit pre-\textit{Juno} data for the degree-two zonal gravitational
harmonic $J_2$ \citep{jacobson2003}. While solutions exist matching pre-\textit{Juno}
data for the degree-four harmonic $J_4$, models using the \textit{ab initio} EOS required unphysical
compositions with densities lower than that expected for the pure H-He mixture. As a result, the
preferred model of \citet{hubbard2016} predicts a $J_4$ with an absolute value above
pre-\textit{Juno} error bars. Preliminary Jupiter models consider the effect of a
helium rain layer where hydrogen and helium become immiscible \citep{stevenson1977a}.
The existence of such a layer has important effects for the interior structure of the
planet, since it inhibits convection and mixing between the molecular exterior and
metallic interior portions of the H-He envelope. This circumstance provides a
physical basis for differences in composition and thermal state between the inner and
outer portions of the planet.  Adjustments of the heavy element content and entropy
of the $P(\rho)$ barotrope allow identification of an interior structure consistent
with both pre-\textit{Juno} observational constraints and the \textit{ab initio}
material simulations. The preferred preliminary model predicts a dense inner core
with $\sim$12 Earth masses and an inner hydrogen-helium rich envelope with
$\sim$3$\times$ solar metallicity, using an outer envelope composition matching that
measured by the \textit{Galileo} entry probe.

Although the \textit{Cassini} Saturn orbiter was not designed for direct measurements
of the high degree and order components of Saturn's gravitational field, the first observational
determination of Saturn's second degree Love number $k_2$ was recently reported by
\citet{lainey2016}. This study used an astrometric dataset for Saturn's co-orbital
satellites to fit $k_2$, and identified a value significantly larger than the
theoretical prediction of \citet{gavrilov1977}. The non-perturbative CMS method
obtains values of $k_2$ within the observational error bars for simple models of
Saturn's interior, indicating the high value can be explained completely in terms of
static tidal response \citep{wahl2016}. The perturbative method of
\citet{gavrilov1977} provides an initial estimate of tidally induced terms in the
gravitational potential, but neglects terms on the order of the product of tidal
and rotational perturbations. \citet{wahl2016} demonstrated, that for the
rapidly-rotating Saturn, these terms are significant and sufficient to explain the
observed enhancement of $k_2$.

\section{Equation of state considerations}
The choice of equation of state effects the density structure of the planet, and
consequently, the distribution of heavy elements that is consistent with observational
constraints. For comparison, we also construct models using the \citet{saumon1995}
equation of state (SCvH) for H-He mixtures, which has been used extensively in giant planet
modeling. 

\begin{figure}[h!]  
  \centering
    \includegraphics[width=22pc]{figs/jupiter_eos.pdf}
\caption{ The barotrope used in preferred model Jupiter `DFT-MD\_7.13'. Top:
    temperature-pressure relationship for a hydrogen-Helium mixture with Y=0.245,
    with a entropy $S=7.08$ at pressures below the demixing region, and $S=7.13$ at
    pressures above the demixing region. The helium demixing region is shown by the
    gap and shaded region. The red line shows measurements from the \textit{Galileo}
probe. Bottom: density-pressure relationship for the same barotrope.}
\label{fig:eos}
\end{figure}

\textit{Ab initio} simulations show that, at the temperatures relevant to Jupiter's
interior, there is no distinct, first-order phase transition between molecular
(diatomic, insulating) hydrogen to metallic (monatomic, conducting) hydrogen
\citep{vorberger2007}. In the context of a planet-wide model, however, the transition
takes place over the relatively narrow pressure range between $\sim$1-2 Mbar. Within
a similar pressure range an immiscible region opens in the H-He phase diagram
\cite{morales2013}, which under correct conditions allows for a helium rain layer
\cite{stevenson1977a,stevenson1977b}. By comparing our adiabat calculations to the
\cite{morales2013} phase diagram, we predict such a helium rain layer in present-day
Jupiter \citep{hubbard2016}. The extent of this layer in our models is highlighted in
Figure \ref{fig:eos}. While the detailed physics involved with the formation and
growth of a helium rain layer is poorly understood, the existence of a helium rain
layer has a number of important consequences for the large-scale structure of the
planet. In our models, we assume this process introduces a superadiabatic temperature
gradient and a compositional difference between the outer, molecular layer and inner,
metallic layer.


\begin{table}
\centering
%\resizebox{\linewidth}{!}{%

\caption{Jupiter Model Parameters \label{tab:jupiter_params}}
%\begin{adjustbox}{max width=\textwidth}
\begin{tabular}{l|rrr}
    \hline
    {} & {Jupiter} & {} & {} \\
    \hline
$GM$ & $1.26686535 \times 10^{8}$$^a$  &  ${\rm (km^3/s^2) }$  & \\
$a$ & $7.1492 \times 10^{4}$$^a$         & ${\rm (km) }$   & \\ 
$J_2  \times  10^6$  & $14696.43$$^a$   &        & \\
$J_4  \times  10^6$  & $-587.14$$^a$  &        & \\
$q_{\rm rot} $  & $ .08917920 $$^b$   & & \\
$r_{\rm core} / a$ & $0.15$  & \\ 
\hline \hline
$q_{\rm tid} $  & $-6.872 \times 10^{-7} $ & $-9.169\times 10^{-8 }$  
&  $-6.976\times10^{-8}$  \\
$R/a$  & $5.90$  &  $9.39$  & $14.98$ \\
\hline
\multicolumn{4}{l}{ a. \citet{jacobson2003}, b. \citet{archinal2011}}
\end{tabular}

\end{table}

In summary, the barotrope and resulting suite of axisymmetric Jupiter models that we
use in this investigation are identical to the results presented by
\citet{hubbard2016}. Each model has a central core mass and envelope metallicities
set to fit the observed $J_2$ \citep{jacobson2003}, with densities corrected to be
consistent with non-spherical shape of the rotating planet. Since tidal corrections
to a rotating Jupiter model are of order $10^{-7}$, see
Table~\ref{tab:jupiter_params} and the following section, it is unnecessary to re-fit
the tidally-perturbed models to the barotrope assumed for axisymmetric models. 

\begin{table}
\centering
%\resizebox{\linewidth}{!}{%

\caption{Jupiter model parameters from \cite{hubbard2016}. $S$ is the
specific entropy for the adiabat through the inner or outer H-He envelope. $M$ is the
mass of heavy elements included in each layer. Each model matches
observed  $J_2 = 14696.43 \times 10^{−6}$ \citep{jacobson2003}, JUP230 orbit solution,
to six significant figures. Models denoted as 'DFT-MD' if equation of state based on
\textit{ab initio} simulations or 'SC' for the \citet{saumon1995} equation of
state, with a number denoting the entropy below the helium demixing layer.  The
number of Models denoted with ($J_4$) also match observed $J_4=-596.31\times
10^{-6}$. Model denoted (equal-$Z$) is constrained to have same metallicity in inner
and outer portions of the planet. Preferred interior model shown in bold face.
\label{tab:model_values}}

%\begin{adjustbox}{max width=\textwidth}
\begin{tabular}{l|cc|rrrr}
    \hline
    {} & {$S_{\rm molec.}$} &  {$S_{\rm metal.}$} & {$M_{\rm core}$} & {$M_{\rm Z,molec.}$}
    & {$M_{\rm Z,metal.}$} & {$Z_{\rm global}$} \\
    &  {($S/k_B/N_e$)} & {($S/k_B/N_e$)} & {($M_E$)} &  {($M_E$)} &  {($M_E$)} & \\
    \hline
    DFT-MD   7.24             &  7.08  &  7.24  &  12.5  &  0.9     &  10.3  &  0.07  \\
    DFT-MD  7.24~(equal-$Z$)  &  7.08  &  7.24  &  13.1  &  1.1     &  7.5   &  0.07  \\
    DFT-MD  7.20              &  7.08  &  7.20  &  12.3  &  0.8     &  9.9   &  0.07  \\
    DFT-MD  7.15              &  7.08  &  7.15  &  12.2  &  0.7     &  9.2   &  0.07  \\
    DFT-MD  7.15~($J_4$)      &  7.08  &  7.15  &  9.7   &  $-$0.6  &  14.9  &  0.08  \\
    {\bf DFT-MD 7.13}       & {\bf 7.08}   & {\bf 7.13} & {\bf 12.2}  & {\bf  0.7}   &
    {\bf 8.9}   &  {\bf 0.07}  \\
    DFT-MD  7.13~(low-$Z$)  &  7.08  &  7.15  &  14.0  &  0.2  &  1.1   &  0.05  \\
    DFT-MD  7.08            &  7.08  &  7.08  &  12.0  &  0.6  &  8.3   &  0.07  \\
    SC      7.15            &  7.08  &  7.15  &  4.8   &  3.5  &  28.2  &  0.11  \\
    SC      7.15~($J_4$)    &  7.08  &  7.15  &  4.3   &  3.2  &  29.3  &  0.12  \\
    \hline
\end{tabular}

\end{table}

The physical parameters for each of these models is summarized in Table
\ref{tab:model_values}. The gravitational moments at the planet's surface are
insensitive to the precise distribution of extra heavy-element rich material within the
innermost part of the planet. For instance, 
the gravitational moments do not allow us to discern between a model
with a dense rocky core and a model without a dense rocky core but with same amount
of heavy element distributed in a larger but restricted volume in the deep interior. 
Maintaining a constant core radius is
computationally convenient when finding a converged core mass to $J_{2}$, since it
requires no modification of the radial grid used through the envelope. For this
reason we consider models with a constant core radius of $0.15a$. Decreasing this
radius below $0.15a$ for a given core mass has a negligible effect on the calculated
gravitational moments \citep{hubbard2016}. Figure \ref{fig:density_jupiter} shows
the density profile for two representative models.  In general, models using the
DFT-MD equation of state lead to a larger central core and a lower envelope
metallicity than those using SCvH.  \citet{hubbard2016} also noted that these models
predict a value for $J_4$ outside the reported observational error bars
\citep{jacobson2003}, since they would require unrealistic negative values of $Z$ to
match both $J_2$ and $J_4$.

\begin{figure}[h!]  
  \centering
    \includegraphics[width=22pc]{figs/jupiter_density.pdf}
\caption{   Density structure of Jupiter models (the planetary unit of
density $\rho_{pu}=M/a^3$).  The red curve shows our preferred
    model based on \textit{ab initio} calculations. The blue curve uses the Saumon and
    Chabrier equation of state. The shaded area denotes the helium demixing region.
    Both models have $N=511$ layers and a dense core within $r=0.15a$.  Constant core
    densities are adjusted to match $J_2$ as measured by fits to Jupiter flyby Doppler
    data \citep{jacobson2003}.}
\label{fig:density_jupiter}
\end{figure}

\subsection{State mixing for static Love numbers} \label{state_mixing}

In the CMS method applied to tides, we calculate the tesseral harmonics $C_{nm}$
directly, and the Love numbers $k_{nm}$ are then calculated using Eq.~\ref{eq:kn}.
For the common tidal problem where $q_{\rm tid}$ and $q_{\rm rot}$ are carried to
first order perturbation only, this definition of $k_{nm}$ removes all dependence on
the small parameters $q_{\rm tid}$ and $a/R$, which is convenient for calculating the
expected tidal tesseral terms excited by satellites of arbitrary masses at arbitrary
orbital distances.  However, the high-precision numerical results from our CMS tidal
theory reveal that when $q_{\rm rot} \approx 0.1$, as is the case for Jupiter and
Saturn, the mixed excitation of tidal and rotational harmonic terms in the external
gravity potential has the effect of introducing a small but significant dependence of
$k_{22}$ on $a/R$; see Fig.~\ref{fig:J4_k2}. In the absence of rotation, the CMS
calculations yield results without any state mixing, and the $k_{nm}$ are, as
expected, constant with respect to $a/R$.  It is important to note this effect on the
{\it static} Love numbers because, as we discuss below, dynamical tides can also
introduce a dependence on $a/R$ via differing satellite orbital frequencies.

\begin{figure}[h!]  
  \centering
    \includegraphics[width=22pc]{figs/jupiter_weights.pdf}
\caption{Top: Relative contribution of spheroids to external gravitational zonal
    harmonic coefficients up to order 8. Bottom: Relative contribution of spheroids
    to to external gravitational tesseral coefficients up to order 4. Tesseral
    moments of the same order (i.e. $C_{31}$ and $C_{33}$) have indistinguishable
radial distributions. Values normalized so that each harmonic integrates to unity.
The shaded area denotes the helium demixing region.  }
\label{fig:jupiter_weights}
\end{figure}

\subsection{Calculated static tidal response}

The calculated zonal harmonics $J_n$ and tidal Love numbers $k_{nm}$ for all of the
Jupiter models with Io satellite parameters are shown in
Tab.~\ref{tab:model_harmonics}. Our preferred Jupiter model has a calculated $k_{2}$
of 0.5900. In all cases, these Love numbers are significantly different from those
predicted for a non-rotating planet (see Tab.~\ref{tab:satellite_harmonics}).
Fig.~\ref{fig:tesseral_rotation} shows the different tesseral harmonics $C_{nm}$
calculated with and without rotation. For a non-rotating planet with identical
density distribution to the preferred model we find a much smaller $k_{22}=0.53725$.
\textit{Juno} should, therefore, be able to test for the existence of the rotational
enhancement of the tidal response.


\begin{sidewaystable}
\caption{Gravitational Harmonic Coefficients and Love Numbers\label{tab:model_harmonics}}
%\centering

\begin{adjustbox}{max width=\textheight}
\begin{tabular}{l|rrrr|rrrrrrrrrrrr}
     %\toprule
    \hline
(all $J_n$ $\times$ $10^6$) &  $J_{4}$ &  $J_{6}$ &  $J_{8}$ &  $J_{10}$ &  $k_{22}$ &  $k_{31}$ &  $k_{33}$ &
$k_{42}$ &  $k_{44}$ &  $k_{51}$ &  $k_{53}$ &  $k_{55}$ &  $k_{62}$ &  $k_{64}$ &
$k_{66}$ \\
\hline
pre-\textit{Juno}~observed &  -587.14 &    34.25 &   - & - & - &
- & - & - & - & - & - & - & - &
- & -  \\
(JUP230)$^a$ &  $\pm$1.68 &   $\pm$5.22 &   - & - & - &
- & - & - & - & - & - & - & - &
- & -  \\
\hline
DFT-MD 7.24            &  -597.34 &    35.30 &   -2.561 &     0.212 &   0.59001 &   0.19455 &   0.24424 &   1.79143 &   0.13920 &   0.98041 &   0.84803 &   0.09108 &   6.19365 &   0.52154 &   0.06451 \\
DFT-MD 7.24~(equal-$Z$) &  -599.07 &    35.48 &   -2.579 &     0.214 &   0.59004 &   0.19512 &   0.24498 &   1.79695 &   0.13984 &   0.98531 &   0.85239 &   0.09159 &   6.22719 &   0.52474 &   0.06492 \\
DFT-MD 7.20            &  -596.88 &    35.24 &   -2.556 &     0.211 &   0.59000 &   0.19440 &   0.24404 &   1.78994 &   0.13902 &   0.97903 &   0.84678 &   0.09093 &   6.18392 &   0.52058 &   0.06438 \\
DFT-MD 7.15            &  -596.31 &    35.18 &   -2.549 &     0.211 &   0.58999 &   0.19422 &   0.24381 &   1.78811 &   0.13881 &   0.97733 &   0.84526 &   0.09074 &   6.17202 &   0.51941 &   0.06423 \\
DFT-MD 7.15~($J_4$)     &  -587.14 &    34.18 &   -2.451 &     0.201 &   0.58985 &   0.19118 &   0.23989 &   1.75874 &   0.13537 &   0.95088 &   0.82162 &   0.08794 &   5.98975 &   0.50178 &   0.06195 \\
{\bf DFT-MD 7.13}        & {\bf -596.05} &  {\bf   35.15} &  {\bf  -2.546} &    {\bf
0.210} &  {\bf  0.58999} &   {\bf 0.19413} &   {\bf 0.24370} &   {\bf 1.78728} &
{\bf 0.13871} &   {\bf 0.97655} &   {\bf 0.84456} &   {\bf 0.09066} &   {\bf 6.16658}
&   {\bf 0.51887} &   {\bf 0.06416} \\
DFT-MD 7.13 (low-$Z$)   &  -601.72 &    35.77 &   -2.608 &     0.217 &   0.59009 &   0.19599 &   0.24610 &   1.80546 &   0.14083 &   0.99296 &   0.85924 &   0.09239 &   6.28019 &   0.52985 &   0.06558 \\
DFT-MD 7.08        &  -595.48 &    35.08 &   -2.539 &     0.210 &   0.58998 &   0.19395 &   0.24346 &   1.78542 &   0.13848 &   0.97482 &   0.84301 &   0.09047 &   6.15442 &   0.51767 &   0.06400 \\
SC 7.15           &  -589.10 &    34.86 &   -2.556 &     0.214 &   0.58993 &   0.19112 &   0.24002 &   1.76641 &   0.13699 &   0.96568 &   0.83567 &   0.09024 &   6.12279 &   0.51832 &   0.06449 \\
SC 7.15 ($J_4$)         &  -587.14 &    34.65 &   -2.534 &     0.212 &   0.58991 &   0.19048 &   0.23918 &   1.76013 &   0.13625 &   0.95997 &   0.83054 &   0.08963 &   6.08299 &   0.51443 &   0.06398 \\
\hline
\multicolumn{16}{l}{All Love numbers for a tidal response with $q_{\rm tid}$ and $R/a$
corresponding to Jupiter's Satellite Io. Preferred interior model shown in bold
face.}\\
\multicolumn{16}{l}{a. JUP230 orbit solution \cite{jacobson2003}}
\end{tabular}

\end{adjustbox}
\end{sidewaystable}


The effect of the interior mass distribution for a suite of realistic models has a
minimal effect on the tidal response. Most models using the DFT-MD barotrope are
within a 0.0001 range of values. The one outlier being the model constrained to match
$J_4$ with unphysical envelope composition. The models using the SCvH barotrope
yields slightly lower, but still likely indistinguishable values of $k_{22}$. The
higher order harmonics show larger relative differences between models, but still
below detection levels. Regardless, the zonal harmonic values are more diagnostic for
differences between interior models than the tidal Love numbers. Fig.~\ref{fig:J4_k2}
summarizes these results, and shows that the calculated $k_{22}$ value varies
approximately linearly with $J_4$.  If \textit{Juno} measures higher order tesseral
components of the field, it may be able to verify a splitting of the $k_{nm}$ Love
numbers with different $m$, for instance, a predicted difference between
$k_{31}\sim0.19$ and $k_{33}\sim0.24$.


\begin{figure}[h!]  
  \centering
    \includegraphics[width=22pc]{figs/jupiter_Cnm_rotation.pdf}
\caption{ The tesseral harmonic magnitude $C_{nm}$ for the `DFT\_MD 7.13' Jupiter
model with a tidal perturbation corresponding to Io at its average orbital distance.
Black: the values calculated with Jupiter's rotation rate; red: the values for
a non-rotating body with identical layer densities.  Positive values are shown as
filled and negative as empty.}
\label{fig:tesseral_rotation}
\end{figure}

\begin{figure}[h!]  
  \centering
    \includegraphics[width=22pc]{figs/jupiter_Cnm_satellite.pdf}
\caption{ The tesseral harmonic magnitude $C_{nm}$ for the `DFT\_MD 7.13' Jupiter
model with a tidal perturbation corresponding to different satellites: Io (black),
Europa (red) and Ganymede (blue).}
\label{fig:tesseral_satellites}
\end{figure}

In addition, we find small, but significant, differences between the tidal response
between Jupiter's most influential satellites. Fig. \ref{fig:tesseral_satellites}
shows the calculated $C_{nm}$ for simulations with Io, Europa and Ganymede. We
attribute the dependence on orbital distance to the state mixing described in Section
\ref{state_mixing}. This leads to a difference in $k_{22}$ between the three
satellites (Tab.~\ref{tab:satellite_harmonics}) that may be discernible in
\textit{Juno}'s measurements.


\begin{figure}[h!]  
  \centering
    \includegraphics[width=22pc]{figs/jupiter_J4_k2.pdf}
\caption{ Predicted $k_2$ Love numbers for Jupiter models plotted against $J_4$. The
    favored interior model `DFT-MD\_7.13' with a tidal perturbation from Io is
    denoted by the red star. The other interior models with barotropes based on the
    DFT-MD simulations (blue) have $k_2$ forming a linear trend with $J_4$.  Models
    using the Saumon and Chabrier barotrope (green) plot slightly above this trend.
    The of $k_2$ for a single model `DFT-MD\_7.13' with tidal perturbations from
    Europa and Ganymede (yellow) show larger differences than any resulting from
    interior structure.
    \label{fig:J4_k2}}
\end{figure}

\section{Correction for dynamical tides}

\subsection{Small correction for non-rotating model of Jupiter}

The general problem of the tidal response of a rotationally-distorted liquid
Jovian planet to a time-varying perturbation from an orbiting satellite
has not been solved to a precision equal to that of the static CMS tidal
theory of \citet{wahl2016} and this paper.  However, an elegant approach
based on free-oscillation theory has been applied to the less general problem
of a non-rotating Jovian planet perturbed by a satellite in a circular orbit
\citep{vorontsov1984}.  Let us continue to use the spherical coordinate system $(r,\theta,\phi)$,
where $r$ is radius, $\theta$ is colatitude and $\phi$ is longitude.
Assume that the satellite is in the planet's equatorial
plane ($\theta=\pi/2$) and orbits prograde at angular rate $\Omega_S$.  For a given planet
interior structure, \citet{vorontsov1984} first obtain its eigenfrequencies
$\omega_{\ell m n}$ and orthonormal
eigenfunctions ${\bf u}_{\ell m n}(r,\theta,\phi)$, projected on spherical harmonics
of degree $\ell$ and order $m$ (the index $n=0,1,2,...$ is the number of radial nodes
of the eigenfunction).  Note that in their convention, oscillations moving prograde
(in the direction of increasing $\phi$) have negative $m$, whereas some authors, e.g.
\citet{Marley1993} use the opposite convention.

Treating the tidal response as a forced-oscillation problem, equation (24) of
\citet{vorontsov1984}, the vector tidal displacement ${\bf \xi}$ then reads

\begin{equation}
    {\bf \xi}({\bf r},t) = -\sum_{\ell,m,n} {{({\bf u}_{\ell,m,n},{\bf \nabla}\psi^r_{\ell m})}
\over {\omega^2_{\ell m n} - m^2 \Omega_S^2}} e^{-i m \Omega_S t},
\label{eq:xioft}
\end{equation}
where $({\bf u}_{\ell,m,n},{\bf \nabla} \psi^r_{\ell m})$ is the integrated scalar product of the
vector displacement eigenfunction ${\bf u}_{\ell m n}(r,\theta,\phi)$ and the
gradient of the corresponding term of the satellite's tidal potential
$\psi^r_{\ell m}(r,\theta,\phi,t)$, {\it viz.}
\begin{equation}
    ({\bf u}_{\ell,m,n},{\bf \nabla} \psi^r_{\ell m})=
\int dV \rho_0(r) ({\bf u}_{\ell,m,n} \cdot {\bf \nabla} \psi^r_{\ell m}).
\label{eq:scalprod}
\end{equation}
The integral is taken over the entire spherical volume of the planet,
weighted by the unperturbed spherical mass density distribution $\rho_0(r)$.

\citet{vorontsov1984} then show that, for the nonrotating Jupiter problem, the
degree-two dynamical Love number $k_{2,d}$ is determined to high precision ($\sim$
0.05\%) by off-resonance excitation of the $\ell=2, m=2, n=0$ and $\ell=2, m=-2, n=0$
oscillation modes, such that
%
\begin{equation}
    k_{2,d}={{\omega^2_{220}} \over {\omega^2_{220} - (2 \Omega_S)^2}} k_2,
\label{eq:k2d}
\end{equation}
%
noting that $\omega_{220}$ and $\omega_{2-20}$ are equal for nonrotating Jupiter (all
Love numbers in the present paper written without the subscript {\it d} are
understood to be static).  For a Jupiter model fitted to the observed value of $J_2$,
\citet{vorontsov1984} set $\Omega_S = 0$ to obtain $k_2  = 0.541$, within 0.7\% of
our nonrotating value of 0.53725 (see Table~\ref{tab:satellite_harmonics}).  Setting
$\Omega_S$ to the value for Io, Eq.~\ref{eq:k2d} predicts that $k_{2,d} = 0.547$,
i.e. the dynamical correction increases $k_2$ by 1.2\%.  This effect would be only
marginally detectable by the \textit{Juno} measurements of Jupiter's gravity, given
the expected observational uncertainty.


\begin{table}
\centering
%\resizebox{\linewidth}{!}{%

\caption{Tidal Response for Various Satellites and Non-rotating Model. Tidal response of preferred interior model `DFT\_MD 7.13' with
    $q_{\rm tid}$ and $R/a$ for three large satellites, and for a `non-rotating'
    model with $q_{\rm rot}=0$. In bold face is the same preferred model as in
    \label{tab:satellite_harmonics} }
%\begin{adjustbox}{max width=\textwidth}
\begin{tabular}{l|rrrr}
    \hline
    {} & {\bf Io} & {Io$^a$} &
    {Europa} & {Ganymede} \\
    {}  &  {} &
    {non-} & {} & {} \\
    {}  &  {} & {rotating} & \\
    \hline
    $k_{22}$  &  {\bf  0.58999}  &  0.53725  &  0.58964  &  0.58949  \\
    $k_{31}$  &  {\bf  0.1941}   &  0.2283   &  0.1938   &  0.1937   \\
    $k_{33}$  &  {\bf  0.2437}   &  0.2283   &  0.2435   &  0.2435   \\
    $k_{42}$  &  {\bf  1.787}    &  0.1311   &  4.357    &  12.41    \\
    $k_{44}$  &  {\bf  0.1387}   &  0.1311   &  0.1386   &  0.1386   \\
    $k_{51}$  &  {\bf  0.9766}   &  0.0860   &  2.373    &  6.7486   \\
    $k_{53}$  &  {\bf  0.8446}   &  0.0860   &  2.0289   &  5.740    \\
    $k_{55}$  &  {\bf  0.0907}   &  0.0860   &  0.0906   &  0.0906   \\
    $k_{62}$  &  {\bf  6.167}    &  0.0610   &  37.04    &  302.1    \\
    $k_{64}$  &  {\bf  0.5189}   &  0.0610   &  1.237    &  3.487    \\
    $k_{66}$  &  {\bf  0.0642}   &  0.0610   &  0.0641   &  0.0641   \\
    \hline
    \multicolumn{5}{l}{a.~Non-rotating model has identical density }\\
    \multicolumn{5}{l}{~~ structure to rotating model.}
\end{tabular}

\end{table}

\subsection{ Dynamical effects for rotating model of Jupiter}

For a more realistic model of Jupiter tidal interactions, the dynamical correction to
the tidal response might be larger, and therefore, more detectable.  We have already
shown (Table 4) that inclusion of Jupiter's rotational distortion increases the
static $k_2$ by nearly 10\% above the non-rotating static value for a spherical
planet.  In this section, we note that
Jupiter's rapid rotation may also change Jupiter's dynamic tidal response,
by a factor that remains to be calculated.

In a frame co-rotating with Jupiter at the rate $\Omega_P=2 \pi / 35730$s,
the rate at which the subsatellite point moves is obtained by the scalar difference
$\Delta \Omega = \Omega_S - \Omega_P$, which is negative for all Galilean satellites.  Thus,
in Jupiter's fluid-stationary frame, the subsatellite point moves retrograde
(it is carried to the west by Jupiter's spin).  
For Io, we have $\Delta \Omega = -1.35 \times 10^{-4}$ rad/s.
Jupiter's rotation splits the
$\omega_{2\pm20}$ frequencies \citep{vorontsov1981}, such that
$\omega_{2-20}= 5.24 \times 10^{-4}$ rad/s and
$\omega_{220}= 8.73 \times 10^{-4}$ rad/s.  The oscillation
frequencies of the Jovian modes closest to tidal resonance with Io are
higher than the frequency of the tidal disturbance in
the fluid-stationary frame, but are closer to resonance than
in the case of the non-rotating model considered by
\citet{vorontsov1984}.

An analogous investigation for tides on Saturn raised by
Tethys and Dione yields results similar to the Jupiter values:
tides from Tethys or Dione are closer to resonance with normal modes for $\ell=2$ and
$m=2$ and $m=-2$.  Since our static
value of $k_2$ for Saturn \citep{wahl2016} is robust to various assumptions about interior
structure and agrees well
with the value deduced by \citet{lainey2016}, so far we have no evidence for dynamical
tidal amplification effects in the Saturn system.  

Unlike the investigation of \citet{lainey2016}, which relied on analysis of astrometric data for
Saturn satellite motions, the \textit{Juno} gravity investigation will attempt to directly determine
Jupiter's $k_2$ by analyzing the influence of Jovian tesseral-harmonic terms on the spacecraft orbit.
A discrepancy
between the observed $k_2$ and our predicted static $k_2$
would indicate the need for
a quantitative theory of dynamical tides in rapidly rotating Jovian planets. 


\section{Summary} \label{summary}

The non-perturbative CMS method for calculating a self-consistent shape and
gravitational field of a static liquid planet has been extended to include the
effect of a tidal potential from a satellite. This is expected to represent the
largest contribution to the low-order tesseral harmonics measured by
\textit{Juno} and future spacecraft studies of the gas giants. This approach
has been benchmarked against analytical results for the tidal response of the
constant density Jeans/Roche spheroid, a two constant density layer model and
the polytrope of index unity. 

We highlight for the first time an important effect of rapid rotation on the tidal
response of the gas giants. CMS simulations of the tidal response on bodies with
large rotational flattening show significant deviation in the tesseral harmonics of
the gravitational field as compared to simulations without rotation. This includes
splitting of the love numbers into different $k_{nm}$ for any given order $n>2$.
Meanwhile, it leads to an observable enhancement in $k_2$ compared to a non-rotating
model.

This rotational enhancement of the $k_2$ love number for a simplified interior model
of Saturn agrees with the recent observational result \citep{lainey2016}, which found
$k_2$ to be much higher than previous predictions. Our predicted values of $k_2$ are
robust for reasonable assumptions of interior structure, rotation rate and satellite
parameters.  The \textit{Juno} spacecraft is expected to measure Jupiter's
gravitational field to sufficiently high precision to measure lower order tesseral
components arising from Jupiter's large moons, and we predict an analogous rotational
enhancement of $k_2$ for Jupiter.  Our high-precision tidal theory will be an
important component of the search for non-hydrostatic terms in Jupiter's external
gravity field.

Our study has predicted the static tidal Love numbers $k_{nm}$ for Jupiter and its three
most influential satellites. These results have the following features: (a) They are
consistent with the most recent evaluation of Jupiter's $J_2$ gravitational
coefficient; (b) They are fully consistent with state of the art interior models
\citep{hubbard2016} incorporating DFT-MD equations of state, with a density
enhancement across a region of H-He imiscibility \citep{morales2013}; (c) We use the
non-perturbative CMS method for the first time to calculate high-order tesseral
harmonic coefficients and Love numbers for Jupiter.

The combination of the DFT-MD equation of state and observed $J_{2n}$ strongly limit
the parameter space of pre-\textit{Juno} models. Within this limited parameter space,
the calculated $k_{nm}$ show minimal dependence on details of the interior structure.
Despite this, our CMS calculations predict several interesting features of Jupiter's
tidal response that the \textit{Juno} gravity science system should be able to
detect. In response to the rapid rotation of the planet the $k_2$ tidal Love number
is predicted to be much higher than expected for a non-rotating body. Moreover, the
rotation causes state mixing between different tesseral harmonics, leading to a
dependence of higher order static $k_{nm}$ on both $m$ and the orbital distance of the
satellite. An additional, significant dependence on $a/r$ is expected in the dynamic
tidal response. We present an estimate of the dynamical correction to our
calculations of the static response, but a full analysis of the dynamic theory of
tides has yet to be performed.



\chapter{Interpreting Jupiter's Gravitational Field from \textit{Juno}}\label{chap8}

\section{Introduction} \label{sec:intro}

The \textit{Juno} spacecraft entered an orbit around Jupiter in July of 2016, and
since then has measured Jupiter's gravitational field to high precision
\citep{bolton2017}.  Here we present a preliminary suite of interior structure models
for comparison with the low order gravitational moments ($J_2$, $J_4$, $J_6$ and
$J_8$) measured by \textit{Juno} during its first two perijoves \citep{Folkner2017}. 

A well constrained interior structure is  a primary means of testing models for
the formation of the giant planets. The abundance and distribution of elements
heavier than helium  (subsequently referred to as ``heavy elements'') in the
planet is key in relating gravity measurements to formation processes. In the
canonical model for the formation of Jupiter, a dense core composed
$\sim$10~$M_\oplus$ (Earth masses) of rocky and icy material forms first,
followed by a period of rapid runaway accretion of nebular gas
\citep{Mizuno1978,Bodenheimer1986,Pollack1996}. Recent formation models suggest
that even in the core accretion scenario, the core can be small ($\sim$ 2
$M_\oplus$) or be diffused with the envelope
\citep{venturini2016,lozovsky2017}. If Jupiter formed by gravitational
instability, i.e., the collapse of a region of the disk under self-gravity
\citep{Boss1997}, there is no requirement for a core, although a core could
still form at a later stage \citep{helled2014}.  Even if the planet initially
formed with a distinct rock-ice core, at high pressures and temperatures these
core materials become soluble in liquid metallic hydrogen
\citep{Stevenson1985,wilson2012a,Wilson2012b,Wahl2013,Gonzalez2013}. As a
result, the core will erode and the heavy material will be redistributed
outward to some extent. In this study we consider the effect of such a dilute
core, in which the heavy elements have expanded to a significant fraction of
Jupiter's radius.

Significant progress has been made in understanding hydrogen-helium mixtures at
planetary conditions
\citep{saumon1995,Saumon2004,Vorberger2007,Militzer2008,Fortney2010,Nettelmann2012,militzer2013a,becker2013,militzer2016},
but interior model predictions are still sensitive to the hydrogen-helium
equation of state used \citep{hubbard2016,miguel2016}. 
In Section \ref{sec:barotropes} we describe the derivation of barotropes from a
hydrogen-helium equation of state based on \textit{ab-initio} materials
simulations \citep{militzer2013a,hubbard2016}, make comparisons to other
equations of states, and consider simple perturbations to better understand
their effect on the models. In Section \ref{sec:model} we describe details of
these models including a predicted layer of ongoing helium rain-out
\citep{stevenson1977a,stevenson1977b,Morales2009,Lorenzen2009,Wilson2010,morales2013},
with consideration of a dilute core in Section \ref{sec:dilute}. We then
describe the results of these models in terms of their calculated $J_n$
(Section \ref{sec:trends}) and heavy element mass and distribution (Section
\ref{sec:core_mass}). Finally, in Section \ref{sec:conclusion} we discuss these
results in relation to the present state of  measurements of, as well as theory
for the formation and evolution of Jupiter.

\section{Materials and Methods} \label{sec:methods}

\subsection{Equations of State}

The \textit{ab initio} simulations for MH13 were performed at a single, solar-like
helium mass fraction, $Y_0=0.245$. The precise abundance and distribution for both
helium and heavy element fractions are, \textit{a priori} unknown.  These are
quantified in terms of their local mass fractions, $Y$ and  $Z$. Our models consider
different proportions of both components by perturbing the densities using a relation
derived from the additive value law \citep{hubbard2016}. For the helium density we
use the pure helium end-member of SCvH. We assume a density ratio of heavy element to
hydrogen helium mixture,  $\rho_0 / \rho_Z$, of  $0.38$ for pressures below $100$
GPa, corresponding to heavy element composition measured by the \textit{Galileo}
entry probe \citep{Wong2004}, and $0.42$ for a solar fraction at higher pressures;
see discussion in \citet{hubbard2016}.  
The MH13 equation of state  uses density functional theory molecular dynamics
(DFT-MD) simulations in combination with a thermodynamic integration to find the
entropy of the simulated material. This allows one to directly characterize an
adiabat for the \textit{ab initio} equation of state as the $T(P)$ path in which the
simulated entropy per electron $S/k_B/N_e$ remains constant.  Here $k_B$ is
Boltzmann's constant and $N_e$ is the number of electrons. In the following
discussion, the term ``entropy'' and the symbol $S$ are used interchangeably to refer
to the particular adiabatic temperature profile through regions of the planet
presumed to be undergoing efficient convection. In this work, we assume that the
compositional perturbations have a negligible effect on the isentropic temperature
profile \citep{Soubiran2016}.


Models calculated with REOS3 followed the approach described by
\citet{miguel2016}: We fitted separately the core mass and composition
in heavy elements. The helium content of the molecular region was
fixed to the Galileo value while the increase in helium abundance in
the metallic region was calculated to reproduce the protosolar
value. The abundance of heavy elements was allowed to be different in
the molecular and metallic regions.   


\subsection{Barotropes} \label{sec:barotropes}
In this paper we consider interior density profiles in hydrostatic equilibrium,
%
\begin{equation} \nabla P = \rho \nabla U,     
    \label{eq:hydrostatic} \end{equation}
%
where $P$ is the pressure and $\rho$ is the mass density. In order to find a consistent
density profile, we use a barotrope $P(\rho)$ corresponding to isentropic profiles
constructed from various equations of state. 

Most of the results presented are based on  density functional theory molecular
dynamics (DFT-MD) simulations of hydrogen-helium mixtures from
\citet{militzer2013a} and \citet{militzer2013b} (MH13).  For densities below
those determined by the \textit{ab initio} simulations ($P<5$~GPa), we use the
\citet{saumon1995} equation of state (SCvH), which has been used extensively in
giant planet modeling. The benefits of this simulation technique lie in its
ability to determine the behavior of mixture through the metallization
transition, and to directly calculate entropy for the estimation of adiabtic
profiles. The barotropes are parameterized in
terms of helium and heavy element mass fraction $Y$ and $Z$, and specific
entropy $S$ as a proxy for the adiabatic temperature profile; for additional
details see Supplementary Section S1.

For comparison, we consider  models using the \textit{ab initio} equations of
state of hydrogen and helium calculated by \citet{becker2013}(REOS3) with the
procedure for estimating the entropy described by \citet{miguel2016}. Finally,
we also consider models using the SCvH EOS through the entire pressure range of
the planet.  Although the SCvH EOS does not fit the most recent data from
high-pressure shockwave experiments \citep{hubbard2016,miguel2016}, it is 
useful for comparison since it has been used to constrain Jupiter models
in the past \citep[e.g.][]{Saumon2004}.  

Different equations of state affect model outcomes in part by placing
constraints on the allowable abundance and distribution of heavy elements. The
DFT-MD isentrope consistent with the \textit{Galileo} probe measurements has
higher densities, and a less steep isentropic temperature profile than SCvH in
the vicinity of the metallization transition
\citep{militzer2013a,militzer2016}. The H-Reos equation of state has a similar
shape to the $T(P)$ profile, but has an offset in temperature of several
hundred K through much of the molecular envelope
\citep{Nettelmann2012,hubbard2016,miguel2016}. 

DFT-MD simulation is the best technique at present for determining
densities of hydrogen-helium mixtures over most of conditions in a giant planet
($P>5$ GPa).  There is, however, a poorly characterized uncertainty in density
for DFT-MD calculations. Shock-wave experiments are consistent with DFT, but
can only test their accuracy to, at best $\sim$6 \%
\citep{Knudson2004,Brygoo2015}.  Moreover, there is a necessary extrapolation
between $\sim$5 GPa, where the simulations become too computationally expensive
\citep{militzer2013a,militzer2013b}, and $\sim$10 bar where the deepest
temperature measurements from the \textit{Galileo} probe were obtained
\citep{Seiff1997}.  We consider perturbations to the MH13 equation of state in
the form of an entropy jump, $\Delta S$, at a prescribed pressure in the outer,
molecular envelope; increases of $S$ from 7.07 up to 7.30 (with S in units of
Boltzmann constant per electron) are considered. These perturbations test the
effect of a density decrease through the entire envelope ($P=$0.01 GPa), at the
switch from SCvH to DFT (5.0 GPa), and near the onset of the metalization
transition (50.0 GPa). 

Gravitational moments for the models are calculated using the non-perturbative
concentric Maclaurin spheroid (CMS) method
\citep{hubbard2012,hubbard2013,hubbard2016,wahl2016}; see Supplementary Section
S2 for additional details. 

\subsection{Model assumptions}\label{sec:model}
%\subsection{Helium Rain Layer} \label{sec:rain}

One of the most significant structural features of Jupiter's interior
arises from a pressure-induced immiscibility of hydrogen and helium, which allows for
rain-out of helium from the planet's exterior to interior
\citep{stevenson1977a,stevenson1977b}. \textit{Ab initio} simulations
\citep{Morales2009,Lorenzen2009,Wilson2010,morales2013} predict that the onset of this
immiscibility occurs around $\sim$100 GPa, over a similar pressure range as the
molecular to metallic transition in hydrogen. At higher pressures, the miscibility
gap closure temperature remains nearly constant with pressure, such that in the deep
interior temperatures are sufficient for helium to become miscible again.

The MH13 adiabats cross the \citet{morales2013} phase diagram such 
that helium rain-out occurs between $\sim$100-300 GPa \citep{militzer2016}.
This is consistent with the sub-solar $Y$ measurement made by the
\textit{Galileo} entry probe \citep{Zahn1998}. The REOS3 adiabats are
significantly warmer and require adjustments to the phase diagram in order to
explain the observations \citep{nettelmann2015}.  Although the detailed physics
involved with the formation and growth of a helium rain layer is poorly
understood \citep{Fortney2010}, the existence of a helium rain layer has a
number of important consequences for the thermal and compositional structure of
the planet.

We calculate the abundance of helium in both the upper helium-poor
(molecular hydrogen) region and lower helium-rich (metallic hydrogen) region by
enforcing a helium to hydrogen ratio that is globally protosolar. We also allow
for a compositional gradient of heavy elements across the layer with a mass
mixing ratio that changes from $Z_1$ in the lower layer to $Z_2$ in the upper
layer. 


\subsection{Dilute Core} \label{sec:dilute}

The thermodynamic stability of various material phases in giant planet
interiors has been assessed using DFT-MD calculations
\citep{wilson2012a,Wilson2012b,Wahl2013,Gonzalez2013}.  These calculations
suggest that at the conditions at the center of Jupiter, all likely abundant
dense materials will dissolve into the metallic hydrogen-helium envelope. Thus,
a dense central core of Jupiter is expected to be presently eroded or eroding.
However, the redistribution of heavy elements amounts to a large gravitational
energy cost and the efficiency of that erosion is difficult to assess
\citep[see][]{Guillot2004}.  It was recently shown by \citet{vazan2016},  that
redistribution of heavy elements by convection is possible, unless the initial
composition gradient is very steep.  Some formation models suggest that a
gradual distribution of heavy elements is an expected outcome, following the
deposition of planetesimals in the gaseous envelope \citep{lozovsky2017}. The
formation of a compositional gradient could lead to double-diffusive convection
\citep{Chabrier2007,Leconte2013} in Jupiter's deep interior, which could lead
to a slow redistribution of heavy elements, even on planetary evolution
timescales.

In a selection of the models presented here, we consider Jupiter's ``core'' to
be a region of the planet in which $Z$ is enriched by a constant factor
compared to the envelope region exterior to it. This means that the model core
is a diffuse region composed largely of the hydrogen-helium mixture. In fact,
this configuration is not very different from the internal structure derived by
\citet{lozovsky2017} for proto-Jupiter.  Given the current uncertainty in the
evolution of a dilute core, we consider models with core in various degrees of
expansion, $0.15<r/r_J<0.6$.  In a few models, we also test the importance of
the particular shape of the dilute core profile by considering a core with a
Gaussian $Z$ profile instead.  Fig.~\ref{fig:density} demonstrates the density
profiles resulting from these different assumptions about the distribution of
core heavy elements. 

\section{Results} \label{sec:results}

\subsection{Reference Interior Model}

The reference model (model A) fixes parameters in the outer (molecular) envelope to those
measured by the \textit{Galileo} entry probe: $S=7.074$, $Y=0.2333$ and $Z=0.0169$.
It should be noted that the $Z$ from \textit{Galileo} is based on a measurement
showing sub-solar ratio of ${\rm H}_2{\rm O}$ to other ices (i.e. ${\rm CH}_4$ and
${\rm NH}_3$) \citep{Wong2004}. It has been hypothesized that the entry probe may
have descended through an anomalously dry region of Jupiter's atmosphere, in which
case this value of $Z$ may be an underestimate. The helium ratio of the deep
(metallic) envelope is chosen assuming that the \textit{Galileo} $Y$ was depleted
from a solar composition by helium rain , and the deep entropy is chosen as a
moderate enhancement across the helium rain layer, $S=7.13$.  An upper and lower
pressure of the helium rain layer are determined by finding where the two adiabatic
profiles for the inner and outer envelope intersect the \citep{morales2013} phase
diagram. This step is done self-consistently for all values of $S$, except in a few
extreme cases where the corresponding adiabat does not intersect the phase diagram. 

The interior structures of the REOS3 models presented here differ in the treatment of
the helium rain, assuming a 3-layer boundary with a sharp transition between the
molecular and metallic envelopes. The difference $J_6$ between the REOS3 model with
the compact core (model X) and the perturbed EOS (model F) can be attributed to this
structural difference. 
%A version of the perturbed MH13 model using the same
%assumption for a sharp transition matches model X.

The MH13 models assume that the helium-rain layer is
superadiabatic, a natural consequence of inefficient convection
\citep{militzer2016}. In the case of the REOS3 models, because the adiabat is
significantly warmer, the presence of such a superadiabatic region has minor
quantitative consequences on the solutions and was not considered. In that
case, we used the approach described in \citet{miguel2016}.  

\subsection{Comparison to \textit{Juno}} \label{sec:comparison}

% Check exactly what results are being presented (There were 3 different versions of
% thePJ1 + PJ2 Results.

The even zonal moments observed by \textit{Juno} after the first two perijoves
\citep{Folkner2017} are broadly consistent with the less precise predictions of
\citet{Campbell1985} and \citet{Jacobson2003}, but inconsistent with the more
recent JUP310 solution \citep{Jacobson2013}.  Table~\ref{tab:models} compares
these observations with a few representative models. 
%While the formal
%uncertainties on these quantities are already quite small \citep{Folkner2017}, the dynamical
%contributions to them is still unknown. 
Although the solid-body (static) contribution dominates this low-order, even
part of the gravity spectrum \citep{Hubbard1999}, a small dynamical
contribution above \textit{Juno}'s expected sensitivity must be considered
\citep{Kaspi2010}.  For sufficiently deep flows, these contributions could be
many times larger than \textit{Juno}'s formal uncertainties for $J_n$
\citep{Kaspi2017}, and thus represent the conservative estimate of uncertainty
for the purpose of constraining the interior structure. Thus, ongoing gravity
measurements by \textit{Juno}, particularly of odd and high order, even $J_n$,
will continue to improve our understanding of Jupiter's deep interior
\citep{Kaspi2013}.  Marked in yellow in Fig~\ref{fig:j4j6}, is the possible
uncertainty considering a wide range of possible flows, and finding a
corresponding density distribution assuming the large scale flows are to
leading order geostrophic \citep{Kaspi2009}.  The relatively small range in our
model $J_6$ and $J_8$ compared to these uncertainties suggests flow in Jupiter
are shallower than the most extreme cases considered by \citet{Kaspi2017}. 
%This density
%distribution is then integrated to calculate the dynamical contribution to the
%gravity spectrum.


%\subsection{Reference Model} \label{sec:reference}
\subsection{Model Trends} \label{sec:trends}

It is evident that the $J_n$ observed by \textit{Juno} are not consistent with
the ``preferred'' model put forward by \citet{hubbard2016}, even considering
differential rotation. Nonetheless, we begin with a similar model (Model A in
Tab.  \ref{tab:models}) since it is illustrative of the features of the model
using the MH13 equation of state with reasonable pre-\textit{Juno} estimates
for model parameters. A detailed description of the reference model is included
Supplementary Section S3.

In order to increase $J_4$ for a given planetary radius and $J_2$,  
one must either increase the density below the 100 GPa pressure level or
conversely decrease the density above that level \citep[][their
Fig.~5]{Guillot1999}. We explore two possibilities: either we raise the density
in the metallic region by expanding the central core, or we consider the
possibility of an increased entropy in the molecular region.

Fig.~\ref{fig:j4j6} shows the effect of increasing the radius of the dilute
core on $J_4$ and $J_6$. Starting with the MH13 reference model with
$r/r_J=0.15$ (Model A), the core radius is increased incrementally to
$r/r_J\sim0.4$, above which the model becomes unable to fit $J_2$.  Therefore,
considering an extended core shifts the higher order moments towards the
\textit{Juno} values, but is unable to reproduce $J_4$, even considering a
large dynamical contribution to $J_n$.  Supplementary Fig.~S1 shows a similar
trend for $J_8$, although the relative change in $J_8$ with model parameters
compared to the observed value is less significant than for $J_4$ and $J_6$. 

Precisely matching \textit{Juno}'s value for $J_4$ with the MH13 based models
presented here, requires lower densities than the reference model through at
least a portion of the outer, molecular envelope. In the absence of additional
constraints, this can be accomplished by lowering $Y$ or $Z$, or by increasing
$S$ (and consequently the temperature). In Fig.~\ref{fig:j4j6} this manifests
itself as a nearly linear trend in $J_4$ and $J_6$ (black `+' symbols), below
which there are no calculated points.  This trend also improves the agreement
of $J_4$ and $J_6$ with \textit{Juno} measurements, but with a steeper slope in
$J_6/J_4$ than that from the dilute core.  For $\Delta S\sim0.14$ applied at
$P=$0.01 GPa, a model with this perturbed equation of state can match the
observed $J_4$, with a mismatch in  $J_6$ of $\sim0.1\times10^{-6}$ below the
observed value (Model F).  When the $\Delta S$ perturbation is applied at
higher pressures ($P=5.0$ and $50.0$ GPa), a larger $\Delta S$ is needed to
produce the same change in $J_4$.

We also consider a number of models with both a decrease in the density of the outer,
molecular layer and a dilute core. Here we present MH13 models where the core
radius is increased for models with outer envelope $Z=0.010$, $0.007$ or $0.0$.
Above $Z\sim0.010$ the models are unable to simultaneously match $J_2$ and $J_4$. The
models with $Z=0.010$ and $Z=0.007$ can both fit $J_4$, but with a $J_6$ 
$\sim0.1\times 10^{-6}$ above the observed value (Models C \& D). These models also
require extremely dilute cores with $r/r_J\sim0.5$ in order to match $J_4$. A more
extreme model with no heavy elements ($Z=0$) included in the outer, molecular
envelope (Model B) can simultaneously match $J_4$ and $J_6$ within the current
uncertainty, with a less expansive core with $r/r_J\sim0.27$. The dilute
core using the Gaussian profile and an outer envelope $Z=0.007$ (Model E), has a very similar
trend in $J_4$--$J_6$, although it is shifted to slightly lower values of $J_6$.

There are a number of other model parameters which lead to similar, but less
pronounced, trends than the dilute core. Starting with Model C, we test
shifting the onset pressure for helium rain, between 50 to 200 GPa, and the
entropy in the deep interior, $S=7.07$ to $7.30$ (lower frame in Fig.
\ref{fig:j4j6}).  Both modifications exhibit a similar slope in $J_4$--$J_6$ to
the models with different core radii, but spanning a smaller range in $J_4$
than for the dilute core trend. 

The models using REOS3 have a significantly hotter adiabatic $T$ profile than
MH13.  Models R and S (\ref{tab:models}) are two example solutions obtained
with the REOS3 adiabat, for a 3-layer model with a compact core, and when
adding a dilute core, respectively.  Because of the flexibility due to the
larger $Z$ values that are required to fit Jupiter's mean density, there is a
wide range of solutions \citep{Nettelmann2012, miguel2016} with $J_4$ values
that can extend all the way from $-599\times 10^{-6}$ to $-586\times 10^{-6}$, spanning the
range of values of the MH13 solutions. Model T corresponds to a model calculated 
with the same $\Delta Z$ discontinuity at the molecular-metallic transition as Model S 
but with a compact instead of dilute core. This shows that, as in the case of the MH13 EOS, 
with all other parameters fixed, a dilute core yields larger $J_4$ values.  

For both DFT-based equations of state, we find that heavy element abundances
must increase in the planet's deep interior. The required $\Delta Z$ across the
helium rain layer is increased with the REOS3 equation of state, and
decreased by considering a dilute core.  Regardless of the EOS used, including
a diffuse core has a similar effect on $J_6$, increasing the value by a
similar amount for similar degree of expansion, when compared to an analogous
model with a compact core. Thus $J_6$ may prove to be a useful constraint in
assessing the degree of expansion of Jupiter's core. 

\subsection{Predicted Core Mass} \label{sec:core_mass}

Fig. \ref{fig:coremass} displays the total mass of heavy elements, along with
the proportion of that mass in the dilute core. Models using MH13 with dilute
cores, have core masses between 10 and 24 ~$M_\oplus$ (Earth masses), with
gradual increase from 24 to 27~$M_\oplus$ for the total heavy elements in the
planet. Of the models able to fit the observed $J_4$, those with heavy element
contents closer to the \textit{Galileo} value have more extended cores
containing a greater mass of heavy elements. 

The perturbation of the equation of state with an entropy jump, has an opposite
effect on the predicted core mass with respect to the dilute core, despite the similar
effect on the calculated $J_n$. For increasingly large $\Delta S$ perturbations, core
mass decreases, to $\sim$8~$M_\oplus$, while total heavy element mass increases.  As
this perturbation is shifted to higher pressures the change in core mass becomes less
pronounced, for a given value of $\Delta Z$. In all the cases considered here, the
MH13 equation of state predicts significantly larger core masses and lower
total heavy element mass than the SCvH equation of state.

All of the models depicted in Fig. \ref{fig:coremass} represent fairly conservative
estimates of the heavy element mass. For any such model, there is a trade-off in
densities that can be introduced where the deep interior is considered to be hotter
(higher $S$), and that density deficit is balanced by a higher value of $Z$. 
It is also possible, that a dilute core would
introduce a superadiabatic temperature  profile, which would allow for a similar
trade-off in densities and additional mass in the dilute core.  Constraining this
requires an evolutionary model to constrain the density and temperature gradients
through the dilute core \citep{Leconte2012,Leconte2013}, and has not been
considered here.  Shifting the onset pressure of helium rain can shift the core mass
by $\sim$2~$M_\oplus$ in either direction. If the majority
of the heavy core material is denser rocky phase \citep{Soubiran2016}, the
corresponding smaller value of $\rho_0/\rho_Z$  results in a simultaneous decrease in
core mass and total $Z$ of $\sim$2--4~$M_\oplus$.

Using the REOS3, both models with a small, compact core of $\sim$6~$M_\oplus$ or a
diluted core of $\sim$19~$M_\oplus$ are possible, along with a continuum of
intermediate solutions.  These models have a much larger total mass of heavy
elements, $46$ and $34\,\rm M_\oplus$, a direct consequence of the higher
temperatures of that EOS \citep[see][]{miguel2016}.  The enrichment in heavy elements
over the solar value in the molecular envelope correspond to about $1$ for model R
and $1.4$ for model S, pointing to a water abundance close to the solar value in the
atmosphere of the planet.  In spite of the difference in total mass of heavy
elements, the relationship between core mass and radius is similar for MH13 and
REOS3.

In lieu of additional constraints we can likely bracket the core mass between
6--25~$M_\oplus$, with larger masses corresponding to a more dilute profile of
the core. These  masses for the dilute core are broadly consistent those
required by the core-collapse formation model \cite{Pollack1996}, as well as
models  that account for the dissolution of planetesimals \citep{lozovsky2017}.
The mass of heavy elements in the envelope, and thus the total heavy element
mass is strongly affected by the equation of state, with MH13 predicting
5--6$\times$ solar fraction of total heavy elements in Jupiter and REOS3 around 
$7-10$$\times$ solar fraction.


\section{Conclusion} \label{sec:conclusion}

After only two perijoves the \textit{Juno} gravity science experiment has
significantly improved the measurements of the low order, even gravitational
moments $J_2$--$J_8$ \citep{Folkner2017}. The formal uncertainty on these
measured $J_n$ is already sufficiently small that they would be able to
distinguish small differences between interior structure models, assuming that
the contribution to these low order moments arises primarily from the static
interior density profile. Considering a wide range of possible dynamical
contributions increases the effective uncertainty of the static $J_2$--$J_8$ by
orders of magnitude \citep{Kaspi2017}. It is expected that the dynamical
contribution to $J_n$ will be better constrained following future perijove
encounters by the \textit{Juno} spacecraft with measurements of odd and higher
order even $J_n$ \citep{Kaspi2013}. 

Even with this greater effective uncertainty, it is possible to rule out
a portion of the models presented in this study, primarily on the basis on the
observed $J_4$. The reference model, using a DFT-MD equation of state with
direct calculation of entropy in tandem with a consistent hydrogen-helium
phase-diagram is incompatible with a simple interior structure  model
constrained by composition and temperature from the \textit{Galileo} entry
probe. 

Our models suggest that a dilute core, expanded through a region 0.3--0.5
times the planet's radius is helpful for fitting the observed $J_n$.  Moreover,
for a given $J_4$ the degree to which the core is expanded affects $J_6$ and
$J_8$ in a predictable, model independent manner, such that further
constraining $J_6$ and $J_8$ may allow one to determine whether Jupiter's
gravity requires such a dilute core. Such a core might arise through erosion
of an initially compact rock-ice core, or through a differential rate of
planetesimal accretion during growth, although both present theoretical
challenges.

Using the REOS3 approach leads to a wider range of possibilities which include
solutions with the standard 3-layer model approach or assuming the presence of
a dilute core.  In any case, as for the MH13 solutions, the REOS3 solutions
require the abundance of heavy elements to increase in the deep envelope. This
indicates that Jupiter's envelope has not been completely mixed. 

These results present a challenge for evolutionary modelling of Jupiter's deep
interior  \citep[e.g.][]{vazan2016,mankovich2016}.  The physical processes
involved with the formation and stability of a dilute core are not understood.
It strongly depends on the formation process of the planet and the mixing at
the early stages after formation, and also enters a hydrodynamical regime of
double diffusive convection where competing thermal and compositional gradients
can result in inefficient mixing of material \citep{Leconte2012,Mirouh2012}.
The timescale for the formation and evolution of such features, especially on
planetary length scales is still poorly understood.  In particular, it is not
known whether there would be enough convective energy to expand 10~$M_\oplus$
or more of material to 0.3 to 0.5$\times$ Jupiter's radii. It is also presently
unknown whether it is plausible to expand the core to this degree without fully
mixing the entire planet, and without resorting to extremely fortuitous choices
in parameters.  Since Jovian planets are expected to go through periods of
rapid cooling shortly after accretion \citep{Fortney2010}, if they are mostly
convective, it is likely that much of the evolution of a dilute core would
have to occur early on in the planet's history when the convective energy is
greatest. This presents a challenge for explaining interior models requiring a
large $\Delta Z$ across the helium rain layer, as such a layer would form after the
period of most intense mixing.

In our preliminary models, those able to fit $J_4$ have lower densities in
portions of the outer molecular envelope than MH13. This is achieved though
modifying abundances of helium and heavy elements to be lower than those measured by
the \textit{Galileo} entry probe, or invoking a hotter non-adiabatic
temperature profile. Some formation scenarios \citep[e.g.][]{Mousis2012} can
account for relatively low envelope ${\rm H}_2{\rm O}$ content ($\sim2\times$
solar), but our models would require even more extreme depletions for this
to be explained by composition alone.  Alternatively there might be an
overestimate of the density inherent to the DFT simulations of MH13 of the
order of $\sim$3\% for $P<100$

Interior models could, therefore, be improved through further theoretical and
experimental studies of hydrogen-helium mixtures, particularly in constraining
density in the pressure range below $\sim$100 GPa, where the models are most
sensitive to changes in the equation of state. More complicated equation of state
perturbations, including the onset and width of the metallization transition
\citep{Knudson2017} may be worth considering in future modelling efforts.
Similarly, the interior modeling effort will be aided by an independent
measurement of atmospheric ${\rm H}_2{\rm O}$ from \textit{Juno}'s microwave
radiometer (MWR) instrument \citep{helled2014b}.

%Text here ===>>>

%%

%  Numbered lines in equations:
%  To add line numbers to lines in equations,
%  \begin{linenomath*}
%  \begin{equation}
%  \end{equation}
%  \end{linenomath*}



%% Enter Figures and Tables near as possible to where they are first mentioned:
%
% DO NOT USE \psfrag or \subfigure commands.
%
% Figure captions go below the figure.
% Table titles go above tables;  other caption information
%  should be placed in last line of the table, using
% \multicolumn2l{$^a$ This is a table note.}
%
%----------------
% EXAMPLE FIGURE
%
% \begin{figure}[h]
% \centering
% when using pdflatex, use pdf file:
% \includegraphics[width=20pc]{figsamp.pdf}
%
% when using dvips, use .eps file:
% \includegraphics[width=20pc]{figsamp.eps}
%
% \caption{Short caption}
% \label{figone}
%  \end{figure}
%

\begin{figure}[h]
  \begin{center}
    \noindent\includegraphics[width=22pc]{figs/jplot.pdf}
  \end{center}
\caption{Improvement in measurements of Jupiter's first even zonal harmonics, as a function of year
(abscissa).  All $J_n$ values are normalized to $a$ = 71492 km, and referenced to theoretical
values from a recent Jupiter model \citep{HM16}, horizontal red line. }
\label{fig:jplot}
\end{figure}



\begin{figure}[h]
  \begin{center}
    \noindent\includegraphics[width=22pc]{figs/Jupiter_adiabat68.eps}
  \end{center}
  \caption{Different theoretical predictions for Jupiter's interior adiabat. The shaded areas corresponds to the layers
    in figure~\ref{fig:our_Jupiter_model}}. 
  \label{fig:jup}
\end{figure}

\begin{figure}[h]
  \begin{center}
    \noindent\includegraphics[width=22pc]{figs/Jupiter_adiabat67_immiscibility.pdf}
  \end{center}
  \caption{Hydrogen-helium miscibility diagram. The solid lines show
    DFT-MD adiabats from \citet{MH13} labeled with their entropy in
    units of $k_b$ per electron. The shaded area is the immiscibility
    region calculated by \citet{Morales2013} that we extrapolated
    towards higher pressures. }
  \label{fig:imm}
\end{figure}

\begin{figure}[h]
  \begin{center}
    \noindent\includegraphics[width=22pc]{figs/galileo.eps}
  \end{center}
  \caption{Different theoretical predictions for Jupiter's interior adiabat. The shaded areas corresponds to the layers
    in figure~\ref{fig:galileo_match}}. 
  \label{fig:jup}
\end{figure}


\begin{figure}[h]
\centering

\includegraphics[width=22pc]{figs/density_profile_inset.eps}

\caption{Density profiles of representative models. Solid lines denote models
    using MH13, while dashed use REOS3. In black is a model with $S$, $Y$ and $Z$
    matching that measured by the \textit{Galileo} entry probe, and a core with constant
    enrichment of heavy elements inside $r/r_J$=$0.15$.  In red (Model
    D) $Z$=$0.007$ in the molecular envelope and constant $Z$-enriched, dilute
    core expanded to $r/r_J\sim0.50$ to fit the $J_4$ observed by
    \textit{Juno}. In blue (Model E) with $Z$=$0.007$ also fitting $J_4$ with
    Gaussian $Z$ profile. In orange (Model R) and green (Model S) are profiles
    for the REOS3 models fitting $J_4$ with a compact and dilute core,
    respectively. (Inset) Schematic diagram showing the approximate location of
the helium rain layer, and dilute core.}
\label{fig:density}
\end{figure}

\begin{figure}[h]
  \begin{center}
      \noindent\includegraphics[width=22pc]{figs/MR4.eps}
  \end{center}
  \caption{The two diagrams show the
    fractional radius and pressure as a function of fractional mass
    For a representative Jupiter interior model.}
\label{fig:our_Jupiter_model}
\end{figure}

\begin{figure}[h]o
\centering

\includegraphics[width=22pc]{figs/jgr_J4J6_edit.eps}

\caption{Zonal gravitational moments $J_4$ and $J_6$ for interior models
    matching the measured $J_2$.
    %
    (Upper) The blue rectangle shows the uncertainty of the \textit{Juno}
    measurements as of perijove 2 \citep{Folkner2017}. The yellow region shows the
    effective uncertainty in the static contribution due possible deep differential
    rotation \citep{Kaspi2017}. The blue star is the reference (Model A) with
    $Z=Z_{\rm Gal}$ matching that measured by the \textit{Galileo} entry probe, and
    an core of $r/r_J$=$0.15$. The blue squares show how these results change as a
    dilute core with a constant $Z$ enrichment with increasing $r$ is considered. The
    green and red circles denote similar expanding core trends with lowered outer
    envelope heavy element fraction to $Z$=$0.007$ and $Z$=$0.01$, respectively.  The
    `+'s denote models which take perturb the MH13 EOS by introducing a jump in $S$
    at $P$=$0.01$ (black), $P$=$5.0$ (blue) and $P$=$50.0$ GPa (red). Black diamonds
    show models using the SCvH EOS at all conditions.     
    %
    (Lower) The stars denote models B,C,D,E, \& F in
    Table~\ref{tab:models}. Violet diamonds show models using the REOS3 EOS
    (Models R, S \& T).  Black and green `x's show models starting with the
    green star (dilute core, $Z$=$0.007$) and changing the $S$ of the deep
    interior or the onset pressure of helium rain. Red, green and cyan stars
show models fitting the measured $J_4$ with the radius of the dilute core.
Black Star shows model fitting $J_4$ with with the entropy jump magnitude
$\Delta S$. 
%Orange 'x's show models with a sharp molecular-metallic transition.
}
\label{fig:j4j6}
\end{figure}


\begin{figure}[h]
\centering

%\setfigurenum{S1} %%Change number for each figure
\includegraphics[width=22pc]{figs/jgr_J4J8.eps}

\caption{ Zonal gravitational moments $J_4$ and $J_8$ for interior models
    matching the measured $J_2$. The rectangles show the uncertainty of the
    \textit{Juno} measurements as of perijove 2 \citep{Folkner2017}.The yellow region
    shows the effective uncertainty in the static contribution due possible deep
    differential rotation \citep{Kaspi2017}. Symbols refer to identical models as in
Fig. 2~in the main text. }
\label{fig:j4j8}
\end{figure}


\begin{figure}[h]
\centering

\includegraphics[width=22pc]{figs/jgr_coremass_edit.eps}

\caption{Mass of heavy elements in the core of the model versus the total heavy
    element mass in Jupiter predicted by the model. Symbols refer to identical models
    as in Fig~\ref{fig:j4j6}. The stars denote models included in
    Table~\ref{tab:models}.  Horizontal lines display the values of $M_{\rm
    Z,total}$, corresponding to 5, 6, 7 and 8$\times$ solar abundance of heavy elements.
}
\label{fig:coremass}
\end{figure}

% ---------------
% EXAMPLE TABLE
%
% \begin{table}
% \caption{Time of the Transition Between Phase 1 and Phase 2$^{a}$}
% \centering
% \begin{tabular}{l c}
% \hline
%  Run  & Time (min)  \\
% \hline
%   $l1$  & 260   \\
%   $l2$  & 300   \\
%   $l3$  & 340   \\
%   $h1$  & 270   \\
%   $h2$  & 250   \\
%   $h3$  & 380   \\
%   $r1$  & 370   \\
%   $r2$  & 390   \\
% \hline
% \multicolumn{2}{l}{$^{a}$Footnote text here.}
% \end{tabular}
% \end{table}


\begin{sidewaystable}
\caption{Comparison of selected models to observed gravitational moments}
\label{tab:models}
%\centering
\begin{tabular}{llrrrrrrrrrrrrr}
     %\toprule
    \hline
    
    & Model~Description$^{a}$ &    $Z_1$$^{c}$ &        $Z_2$ &            $J_2$ &          $J_4$ &         $J_6$        &        $J_8$ &        $J_{10}$ &     $C/Ma^2$ &  $r_{\rm core}/r_J$ &     $M_{\rm core}$ &        $M_{Z,{\rm env}}$ &  $M_{Z,{\rm total}}$ &  $Z_{\rm global}$ \\
\hline
& \it{Juno}~observed$^{b}$                        & & & $14696.514$   & $-586.623$   &  $34.244$   & $-2.502$ &   &  &         &  &         &  &       \\
 &                                         & & & $\pm 0.272$   & $\pm 0.363$   &  $\pm 0.236$  & $\pm 0.311$ &  &  &         &  &         &  &       \\
\hline
A & MH13,~$Z_{\rm~Gal}$,~compact~core                &  0.0169  &  0.0298  &  14696.641  &  -594.511  &  34.998  &  -2.533  &  0.209  &  0.26391  &  0.150  &  13.2  &  10.5  &  23.6  &  0.0744  \\
\rowcolor{blue!15}
B & MH13,~dilute~core                   &  0.0000  &  0.0451  &  14696.641  &  -586.577  &  34.196  &  -2.457  &  0.202  &  0.26400  &  0.270  &  10.4  &  13.9  &  24.2  &  0.0762  \\
\rowcolor{blue!15}
C & MH13,~dilute~core                  &  0.0100  &  0.0114  &  14696.467  &  -586.613  &  34.360  &  -2.481  &  0.205  &  0.26396  &  0.498  &  18.5  &  7.3   &  25.8  &  0.0812  \\
\rowcolor{blue!15}
D & MH13,~dilute~core                 &  0.0071  &  0.0199  &  14696.641  &  -586.585  &  34.392  &  -2.486  &  0.205  &  0.26396  &  0.530  &  21.3  &  5.1   &  26.4  &  0.0831  \\
\rowcolor{blue!15}
E & MH13,~Gaussian~core                &  0.0071  &  0.0087  &  14696.467  &  -586.588  &  34.336  &  -2.479  &  0.204  &  0.26397  &  --     &  23.5  &  3.3   &  26.8  &  0.0843  \\
\rowcolor{blue!15}
F & Perturbed~MH13,~compact~core  &  0.0169  &  0.0526  &  14696.466  &  -586.588  &  34.117  &  -2.444  &  0.200  &  0.26400  &  0.150  &  9.3   &  15.9  &  25.1  &  0.0791  \\
G & SCvH,~compact~core                 &  0.0820  &  0.0916  & 14696.641 & -587.437 &  34.699  &  -2.541  &  0.212  &  0.26393  &  0.150  &  1.5   &  32.7 &  34.2  &  0.1076  \\
\rowcolor{blue!15}
R & REOS3,~compact~core & 0.0131 & 0.1516 &  14696.594 & -586.631 & 34.186 &  -2.457 & 0.202 & 0.26443 & 0.110 & 6.21 & 40.0 & 46.2 & 0.1454 \\
\rowcolor{blue!15}
S & REOS3,~dilute~core & 0.0209 & 0.0909 & 14696.755 & -586.658 & 34.346 & -2.480 & 0.204 & 0.26442 & 0.533 & 19.2 & 14.5 & 33.7 & 0.1061 \\
T & REOS3,~compact~core, low $J_4$ & 0.0293 & 0.0993 & 14696.381 & -593.646 & 34.933 & -2.529 & 0.209 & 0.26432 & 0.122 & 8.9 & 27.0 & 35.9 & 0.1129 \\
\hline
\multicolumn{15}{l}{$J_n$ in parts per million. Shaded rows are models match the \textit{Juno} observed $J_2$--$J_8$ within the current uncertainty.} \\
\multicolumn{15}{l}{$^{a}$Equation of state used, dilute or compact core, $Z_{\rm~Gal}$ denotes model with $Z_1$ matching \textit{Galileo} probe measurement. $^{b}$\citet{Folkner2017}.} \\
\multicolumn{15}{l}{$^{b}$\citet{Folkner2017}.} \\
\multicolumn{15}{l}{$^{c}$$Z_1$ denotes the heavy element fraction in molecular envelope, $Z_2$ denotes heavy element fraction in the metallic envelope, but exterior to the core. } 
\end{tabular}
\end{sidewaystable}



\chapter{Conclusions}\label{chap9}



% \appendix
% \chapter{More Monticello Candidates}

%\printbibliography

%\nocite{*}
%\bibliographystyle{plain}
%\bibliographystyle{ieeetr}
%\bibliography{thesis}

%\bibliography{references}
%\bibliography{references.bib}

\bibliographystyle{plainnat}
\bibliography{references}

\end{document}
