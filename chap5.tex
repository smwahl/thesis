\chapter{Thermodynamics of Convection with a Phase Transition}\label{chap3}


\section{Motivation}

One of the more surprising findings of the MESSENGER spacecraft to Mercury was
the confirmation that the smallest terrestrial planet has an internally
generated, dipolar magnetic field, which is likely driven by a combination of
thermal and compositional buoyancy sources. This observation places constraints
on the thermal and energetic state of Mercury’s large iron core and on mantle
dynamics because dynamo operation is strongly dependent on the amount of heat
extracted from the core by the mantle. However, other observations point to
several factors that should inhibit a present-day dynamo. These include
physical constraints on a thin, possibly non-convecting mantle, as well as
properties of liquid iron alloys that promote compositional stratification in
the core.

The lack of a simple relationship between the size of the body and the presence
of a magnetic field in terrestrial planets of moons is striking. The
thermodynamics of dynamo generation exhibit a competition between heat loss by
convection and heat loss by conduction.  Dipolar magnetic fields arise from
helical flows that develop within a rotating conducting liquid undergoing
turbulent convection. The properties of iron alloys, and particularly their
melting temperature, is strongly influenced by the presence of light elements
\citep{sanloup2000}. As a result, the existence of a magnetic field depends, in a
complicated fashion, on planet formation and evolution. I propose to study the
influence of non-ideal mixing behavior in liquid alloys of iron and sulfur on
thermal and compositional convection. Mercury's magnetic field strength has
posed problems for standard dynamo models \citep{Christensen2006,Stanley2005}, and
partial crystallization resulting from non-ideal mixing provides a possible
mechanism to explain this.

Spacecraft observations have confirmed the presence of internally generated
magnetic fields for Mercury \citep{Anderson2011} and Ganymede \cite{Kivelson1996}.
The internal structure of both bodies is constrained by measurements of
gravitational moments \citep{Smith2012,Hauck2006}. However, these measurements are
not sufficiently precise to determine what portion of the cores are liquid, nor
how much light component is contained in the cores. Thermal evolution
calculations of both planets \citep{Hauck2004,Hauck2006,Breuer2007,Bland2008} suggest
that several wt.\% S is necessary to preserve a substantial unfrozen layer in
the core, despite inefficient stagnant-lid convection
\citep{Solomatov2000,Hauck2004,Breuer2007} and tidal heating from a hypothesized
resonance in Ganymede's orbital history \citep{Showman1997,Bland2008}. Some models
for Mercury's formation suggest minimal accretion of volatile elements like
sulfur, but surface observations from MESSENGER \citep{Nittler2011,Mccubbin2012} are
inconsistent with extensive devolatization.

The primary goals of this work is to develop an automated system to generate an
interpolated thermodynamic model using experimentally determined, eutectic
phase diagrams, to calculate adiabatic profiles for the thermodynamic model
using a parcel method, and to evaluate the effect calculated adiabatic profiles
have on global budgets of energy and entropy. This work was never published and
is presented here in an incomplete state.

A part of this work was integrated and added upon as part of a CIDER project with the
author in collaboration with Brent Delbridge, and Ian Rose, of UC Berkeley, and Grace
Cox of the University of Leeds, under the advisorship of Jessica Irving (Princeton),
Bill McDonough and Laurent Montesi (Maryland), and  The last two Sections of this
chapter present results from this collaboration.

\section{Iron alloy properties}

It has long been recognized that alloying components are abundant in the cores
of terrestrial planets and that they must play an important role in thermal
evolution and dynamo generation. Indeed, it is now largely accepted that the
exclusion of a light, alloying component is the most important contributor to
convection in the Earth’s core \citep{Lister1995}. Therefore
consolidating the present understanding about chemical state of Mercury’s
interior is essential to knowing what can and cannot be demonstrated about
Mercury’s dynamo.

For sufficiently high concentrations of S and Si, the two would partition in a
liquid-liquid immiscibility gap in the planets outer core. Within the pressure
range and for reasonable compositions an “iron snow” state
\citep{Chen2008,Williams2009}, where iron crystallization initiates in the outer portions of
the core, must also be considered for Mercury. We therefore require a
coupled model of chemistry and for Mercury’s core to determine the constraints
on composition of the core based on the planets gravitational moments
\citep{Smith2012}, as well as constraints from our own entropy budget calculations.
Since these constraints are very limited, it is necessary to develop a means of
testing a very large number of possible interior structures and compositions.

 \begin{figure}[h] %  figure placement: here, top, bottom, or page
   \centering
   \includegraphics[width=26pc]{figs/liquidi.png} 
   \caption{Liquidus relationships for Fe-S alloys generated from the interpolated thermodynamic 
   model for a range of light-element composition in wt.\% S. The sharp peak and trough lead to 
   a region of partial crystallization for a range of thermal states of the core.}
   \label{fig:liquidi}
\end{figure}

The pressures present in the cores of Mercury ($\sim$8$-$40 GPa) and Ganymede
($\sim$8$-$12 GPa) are significantly lower than those for the Earth's core and
accessible to a wider variety of experimental techniques. At these pressures, 
FeS has been discovered to undergo multiple first-order phase transitions
\citep{Fei1997,Fei2000}, stabilizing new phases $Fe_3S_2$ and $Fe_3S$ at 14 and 21 GPa 
respectively. Fe-S melts undergo analogous changes in compacity
\citep{Morard2007} and associated deviations from ideal mixing behavior
\citep{Chen2008}. The Fe-FeS system shows eutectic melting behavior with eutectic
sulfur composition decreasing from $\sim$30 wt.\% S at ambient pressure to $\sim$12 
wt.\% S at 40 GPa \citep{Chudinovskikh2007}. The eutectic temperature shows
anomalous behavior over a pressure range $\sim$5$-$20 GPa \citep{Fei1997,Chen2008}.
\citet{Chen2008} also found liquidus temperatures at intermediate
compositions on the Fe side of the eutectic to deviate from those predicted by
an ideal mixing model. These anomalous features in the Fe-S phase diagram lead
to the prediction of `iron snow', partial crystallization near the top of the
core with the lower portion remaining completely molten \citep{Hauck2006,Chen2008}.
However, this process has yet to be analyzed in a thermodynamically consistent
manner.

\section{Thermodynamic model from material data}

I have created a working `pipeline' in Matlab for generating a thermodynamic model from
experimental data \citep{Brett1969,Fei1997,Chen2008,Stewart2007}. Data for 
$X$-$T$ phase diagrams at constant
pressure are fit with a smoothing-spline.To best account for the changing
shape and eutectic composition, these spline fits are then
interpolated with $P$ as a linear combination of fits at the two nearest
values of $P$
\begin{equation}
  X(P,T) = X_{eut}(P)\sum_{i=1,2}\xi_i(P)\bar{X}_i(\bar{T}),
\end{equation}
where $\bar{X}$, $\bar{T}$ are fractional coordinates with respect to the
values at the values at the eutectic and pure Fe endmembers, and $\xi_i$
is linear mixing parameter. For parcel
calculations, derivative relationships between $P$, $X$ and $T$ can be related
using the lever rule. Additional parameters such as density, heat capacities
and latent heat of fusion are included in a fashion allowing them to be
specified as functions of $P$, $X$ and $T$. With this pipeline, it should be
straightforward to repeat the analysis with other eutectic systems, such as
silicate liquids, and test the affect of varying parameters.
The salient feature of the system is the variation of the
liquidus with light-element fraction, shown in Fig.~\ref{fig:liquidi}, as this
determines where crystallization occurs.

 \begin{figure}[h] %  figure placement: here, top, bottom, or page
   \centering
   \includegraphics[width=26pc]{figs/adiabats.png} 
   \caption{$P-T$ profiles for an adiabatic parcel calculations with 6 wt.\% S using the interpolated Fe-S model. Starting 
   temperatures are spaced by 50 K, and integrated from low to high pressure. The solid black line represents 
   the melting temperature for pure Fe, and the dashed line the liquidus at 6 wt.\% S.}
   \label{fig:adiabats}
\end{figure}


\section{Parcel calculations}

Parcel calculations using the interpolated thermodynamic model are performed
by numerical integration of an expression derived from manipulation of the
first law of thermodynamics
\begin{equation}
  \label{eqn:firstlaw}
  d T = -d P \frac{\left[-T \sum_i \alpha_i\nu_ix_i +
    L\left(\frac{\partial{X}}{\partial{P}}\right)_T \right] } 
  {\left[ \sum_i C_{P,i}x_i + L \left(
    \frac{\partial{X}}{\partial{T}}\right)_P \right] },
\end{equation}
where $\alpha_i$, $c_{P,i}$, $\nu_i$, and $L$ are the coefficient of thermal
expansion, specific heat capacity, specific volume, and specific latent heat,
respectively, for a phase with mass fraction $x_i$. Eqn. \ref{eqn:firstlaw}
enforces the constraint of zero heat transfer as the pressure on the parcel is
changed. Calculations are carried out by specifying a parcel composition,
temperature and starting pressure, and integrating over a range of pressures.
An example calculation for 6 wt.\% S, for a range of starting temperatures is
presented in Fig.~\ref{fig:adiabats}. I find that for reasonable choices of
parameters, the calculated adiabat within the `iron snow' region is maintained
within $\sim$10 degrees of the liquidus over a range of $\sim$50$-$100 degrees
in starting temperatures. Meanwhile, the temperatures outside the adiabat show
negligible perturbation from simple single-phase adiabats.

 \begin{figure}[h] %  figure placement: here, top, bottom, or page
   \centering
\begin{tabular}{cc}
 \includegraphics[width=20pc]{figs/profiles.png} &
 \includegraphics[width=15pc]{figs/Liquidus_model.png} \\
\end{tabular}
   \caption{ Left: pressure, gravity, density, and temperature profiles of an
interior model for Mercury, with 6 wt.\% S.  Right: Liquidus curves for different
pressures in the Fe-FeS system, from compiled and interpolated experimental data
(Wicks and Knezek, pers. comm.), credit: Nick Knezek.}
  \label{fig:interior_model}
\end{figure}


The onset of crystallization in the outer portion of a terrestrial core may
affect convection and dynamo generation in multiple ways. 
\citet{Hauck2006} suggested that settling of crystals might contribute to
convection by releasing gravitational energy. It is unclear, however, that
crystal settling can drive large scale convective motion. The crystal fractions
for the simulation represented in Fig. \ref{fig:adiabats} also remains
relatively small for conditions during which a separate `iron snow' region
exists. One can consider the influence of this process through use of a
perturbed adiabat with standard models for convection and dynamo generation.

\section{Core energy and entropy budgets}

To determine the energy and entropy budgets, I make use of standard iterative
methods for determining a mean core state, constrained by a planet mass, core
mass and a requirement that $P=0$ at the surface (e.g. \citet{Lister1995}). From
this, I will obtain radial profiles of material properties along a calculated
adiabat.  I will use descriptions of the core energy and entropy budget
\citep{Gubbins1979,Lister1995,Lister2003} to compare the evolution of mean core state
with and without consideration of partial crystallization. Evolution
calculations will be simplified by assuming fractional crystallization in the
core has a negligible affect on the thermal state of the mantle, allowing the
results of existing thermal evolution calculations \citep{Hauck2004,Breuer2007} to
be used as boundary conditions Since the conductive heat flux is proportional
to $\nabla T$, the steepened adiabat will cause a decrease in the convective
heat flux. In the description of the entropy flux, this manifests itself as a
term proportional to $k\left(\nabla T / T \right)^2$ arising from the
divergence of the conductive heat flow \citep{Lister2003}. I will compare the
magnitude of this contribution to standard estimates for the contribution from
thermal and compositional convection.


 \begin{figure}[h] %  figure placement: here, top, bottom, or page
   \centering
\begin{tabular}{cc}
 \includegraphics[width=18pc]{figs/clapeyron_1.png} &
 \includegraphics[width=18pc]{figs/core_energetics.png} \\
\end{tabular}
\caption{Left: Clapeyron slopes for different interpolations of the iron
melting curve, compared to adiabats with various parameterizations.  Right:  Models
for latent heat and gravitational energy release from a solidifying core, with
corresponding thermal energy change.}
\label{fig:core_energy}
\end{figure}

The semi-analytic methods presented above will allow for analysis of the
affect of non-ideal melting on the energetics over long time-scales. However,
the details of the dynamics of convection are also of importance to magnetic field
generation. Numerical calculations of convection will, therefore, be a useful
supplement to the results presented here. They will help evaluate the validity of
key assumptions, such as the persistence of an adiabatic state through the
`iron snow' region. This will be achieved through modification of the CALYPSO,
a geodynamo code which passes standard benchmarks \citep{Christensen2001}. This
requires implementation of phase tracking, and a contribution to the energy
equation from latent heat release. 



\section{Coupling with mantle convection}

We have a working prototype for a code designed to calculate self-consistent
internal structures. The code uses an efficient iterative procedure to
calculate density, gravity, pressure and temperature profiles. It can also find
an inner core radius that is In order to do this we have made use of the
code base provided by the BURNMAN project \citep{Cottaar2014}, a continuation
of a prior CIDER collaboration. This framework allows for integration of
thermodynamic properties of minerals in a straightforward and consistent
manner. A major benefit of designing the structural model code around BURNMAN
will be the systematic inclusion of uncertainty in the experimentally
determined thermodynamic properties, which will be an important step in
establishing the ability of any given set of constraints to determine the state
of Mercury’s core. 

 \begin{figure}[h] %  figure placement: here, top, bottom, or page
   \centering
\begin{tabular}{cc}
 \includegraphics[width=18pc]{figs/thermal_evolution.png}
 \includegraphics[width=18pc]{figs/inner_core_growth.png}
\end{tabular}
\caption{Left: Thermal evolution of the Mercurian mantle and core. This
  thermal evolution model couples the core thermodynamics in the previous section
  with the parameterized convection model of \citep{Stevenson1983}. The colored
  regions show the solution for models with $\pm \mathrm{100}$ degrees C. Note the
  break in slope of the $\mathrm{T_{cmb}}$ temperature with the onset of inner core
  growth at $\sim 2.5\times10^2~\mathrm{Ma}$. Right: Growth of the Inner Core versus
  time. This model run yields an inner core of $ \sim 1400~\mathrm{km}$, slightly
  exceeding the upper bound of inner core size as constrained by \citep{Dumberry2015}.}
  \label{fig:thermal}
\end{figure}

We have developed a 3-layer interior structure model with a growing inner core.
Material properties are calculated using a Mie-Gr\"{u}neisen-Debye EOS, using the BurnMan
code. It finds adiabatic temperature profiles consistent with the pressure of the 
inner core boundary and the composition of the liquid. 

Also shown in Fig.~\ref{fig:interior_model} is a model liquidus, fit to experimental
phase diagrams. The melting curve of the Fe-S system has enigmatic features, which
may not be captured captured by a  linear interpolation. The onset of ``snow''
regions in the core is extremely sensitive to this interpolation.


 \begin{figure}[h] %  figurVVe placement: here, top, bottom, or page
   \centering
\begin{tabular}{c}
 \includegraphics[width=30pc]{figs/CMB_flux.pdf} 
\end{tabular}
\caption{ Heat flux variations due to insolation for a conducting mantle with
    negligible internal heating. The total CMB heat flux is $\sim 0.6~\mathrm{TW}$,
and peak-to-peak variations are about 20\%. Figure Credit: Ian Rose.}
\label{fig:flux}
\end{figure}


Comparing the slope of the interpolated melting temperature to the slope of the
adiabatic profile determines the crystallization behavior of the core, with a
steeper melting temperature corresponding to  ``Earth-like'' conditions. For higher
values of the thermal expansivity, $\alpha$, the core will be ``snowing'' at all
times. For lower $\alpha$, the onset of snowing occurs with increasing S-content.



The interior structure model is coupled to a parameterized convection model for
the thermal evolution of the planet. Shown on the right in Fig.~\ref{fig:core_energy}
are the changes in latent heat, gravitational and thermal energy in the core per
change in core mantle boundary temperature for bulk composition of 6 wt.\% S. 



Mercury's unusual 3:2 spin-orbit resonance causes persistent temperature variations
at the surface.  This boundary condition may create significant heat-flux variations
at the CMB, especially if the mantle is not convecting.  Here we solve a simple
conduction equation in the Mercurian mantle to calculate an estimate of heat flux at
the CMB.  This heat-flux variation is then used to inform the boundary conditions of
a dynamo simulation using the \texttt{Calypso} code.

 \begin{figure}[h] %  figurVVe placement: here, top, bottom, or page
   \centering
\begin{tabular}{c}
 \includegraphics[width=22pc]{figs/entropy_fig.png} 
\end{tabular}
\caption{Entropy budget and inner core radius for a core with 6 wt.\% S and
  current CMB heat flux of 0.5~TW. In this model, there is insufficient entropy to
  drive a dynamo before inner core solidification (negative Ohmic dissipation), but
  compositional sources that arise from inner core growth increase the available
entropy such that a present-day dynamo can be sustained (positive Ohmic dissipation).
Figure credit: Grace Cox. }
\label{entropy}
\end{figure}
