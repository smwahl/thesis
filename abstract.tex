% (This file is included by thesis.tex; you do not latex it by itself.)

\begin{abstract}

% The text of the abstract goes here.  If you need to use a \section
% command you will need to use \section*, \subsection*, etc. so that
% you don't get any numbering.  You probably won't be using any of
% these commands in the abstract anyway.

The inherent difficulty of making and interpreting measurements of the deep interior
of the Earth and other planets necessitates constructing models of their structure
and evolution.  Properties of materials at actual planetary conditions are a key
input to these models. For the Earth these conditions extend to the hundreds of GPa
and thousands of Kelvin; for the energetic impact events and within the gas giants
the range extends to several TPa and perhaps tens of thousands Kelvin.  Despite
tremendous advances in experimental techniques, much of this range of conditions
remains out of reach, and thus, computer simulations of materials play an important
role in characterizing materials within planetary interiors.

This thesis presents a variety of work using computational techniques to: 1)
determine properties of planetary materials from first-principles simulations, and 2)
apply these derived properties to models of large-scale planetary structure and
processes. First-principles calculations are unique in their ability to simulate a
nearly unbounded range of pressure-temperature conditions, including those beyond the
capacity of any experimental techniques. 

In this thesis, I discuss the physics and numerical techniques I have used and
developed to simulate planetary materials at high pressures and temperatures, and to
interpret and condense the results of these calculations. I also present results of
studies applying the first-principles techniques to specific problems in planetary
science. I test the stability of compact rocky cores in the metallic hydrogen-helium
envelopes of gas giants, finding that such cores are likely to undergo dissolution
and erosion.  I then explore the miscibility of terrestrial cores and mantles at
extreme temperatures. I predict that this mixed rock-metal state is of
importance in catastrophic giant impacts that are now thought to be commonplace in
the early history of the terrestrial planets, or deep inside ``super-Earth''
exoplanets. 

I continue by detailing studies applying material equations of state from simulation
and experiment. I describe work towards developing a more comprehensive
thermo-chemical model of liquid iron alloys integrated with models of Mercury's
thermal history and magnetic field energetics. I then describe the derivation and
implementation of a new numerical, non-perturbative method for precise calculations
of gravitational field strength for a rotating, liquid planet with tides. I then look
at the consequences of this new method for the tidal responses of Jupiter and Saturn,
finding a significant, previously uncharacterized contribution arising from the
influence of rotation.  Finally, I detail an ongoing effort using interior structure
models of Jupiter to interpret the drastically improved measurements of Jupiter's
gravity by the \textit{Juno} spacecraft mission. I find evidence for the
existence of a dilute core in spite of difficulties reconciling first-principles
equations of state with observations of the planet's atmosphere.

\end{abstract}
