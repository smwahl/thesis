\chapter{Dissolution of Giant Planet Cores}\label{chap3}

\section{Motivation}

Despite recent advances in computational methods improving understanding of the
hydrogen-helium dominated outer layers
\citep{Mcmahon2012,French2012,militzer2013a,Wilson2010}, knowledge of the deep
interior structure of giant planets is limited.  Determining the size of a
dense central cores in a giant planet is dependent upon the model and equation
of state used.  Current observational evidence yields recent estimates for
present day core mass of $\sim$0$-$10 \citep{Guillot2005},  $\sim$14$-$18
\citep{Militzer2008}, and $\sim$10-25 Earth masses \citep{Wahl2017} for
Jupiter, and $\sim$9$-$22 \citep{Guillot2005} Earth masses for Saturn.  The
Juno spacecraft, en route to Jupiter, will improve this constraint with more
precise measurements of the giant's gravitational field \citep{Helled2011}.
Meanwhile, the density profiles of Neptune and Uranus allow non-unique
solutions for the compositional structure for much of the interior
\citep{Guillot1999b,Guillot2005}.

It has long been suggested \citep{Stevenson1982a,Stevenson1982b}, that a
portion of this dense material might be redistributed in solution with
hydrogen. As a result, erosion of a dense core would cause it to shrink over
the lifetime of the planet. Possible consequences of this process are only
beginning to be enumerated in evolutionary models
\citep{Chabrier2007,Leconte2012,Mirouh2012}. The establishment of a gradient in
concentration of a heavy dissolved component may change the nature of
convection in a portion of the planets interior. This hypothesized
`double-diffusive' region reduces the efficiency of heat transfer, thereby
altering the thermal evolution of the planet. A schematic diagram of the
resulting dissolved core is presented in Fig.~\ref{fig:dissolve_core}.
Comprehensive understanding of the process has been limited by the lack of
knowledge of the solubility of various phases in metallic hydrogen, as well as
poor understanding of the scaling of convective efficiency in the presence of
competing gradients of composition and temperature. In this study, we address
the first issue for iron metal.

 \begin{figure}[h] %  figure placement: here, top, bottom, or page
   \centering
   \includegraphics[width=22pc]{figs/dissolving_core.eps} 
   \caption{Schematic diagram demonstrating the difference between a compact and dissolving core for Jupiter.\label{fig:dissolve_core}}
\end{figure}



As a result of continuing discoveries by Kepler \citep{Borucki2010} and other exoplanet surveys, the
number of confirmed planets has climbed to over 800, a large proportion of which are
giants. This presents a growing sampling of planetary
mass-radii relationships that will be fundamental to understanding the evolution of
giant planet interiors. The range of mass-radius relationships observed for
exoplanets exhibit variation beyond those in the solar system. In some cases,
such as Corot-20b \citep{deleuil2011}, the relationships may even defy explanation by
simple structural models. Redistribution of dense core material lowers the
heavy element content required to explain anomalously high observed densities.

The favored model for gas giant formation
\citep{Mizuno1978,Bodenheimer1986,Pollack1996} relies on the early
formation of a large planetary embryo of critical mass to cause runaway
accretion of hydrogen and helium gas. A competing theory involves collapse of
a region of the disk under self-gravity, e.g. \citep{Boss1997}, but may
have difficulty explaining significant enrichment of refractory elements \citep{Hubbard2002,Guillot2005}.
The immediate result of a core-accretion hypothesis is a planet with the
ice-rock-metal embryo residing at the center as a dense core, surrounded by an
extensive layer of metallic hydrogen and helium. The role of core erosion to
the subsequent evolution is a major source of uncertainty, but in principle, can explain shrinking of
cores to masses smaller than those necessary to form the planet under the
core-accretion hypothesis.

Core erosion in giant planets can be addressed by determining the solubility 
of analogous phases. Previous studies have considered an icy
layer of fluid and superionic $\mathrm{H}_2\mathrm{O}$
\citep{Wilson2012a,Wilson2013}, and a rocky layer consisting of
MgO \citep{wilson12b} and $\mathrm{SiO}_2$ \citep{Gonzalez2013}, which have been
shown to separate at relevant conditions \citep{umemoto2006}. Assuming the
same gross distribution, elements as terrestrial bodies, the innermost core
would be composed of a dense, metallic alloy composed primarily of iron.

{\it Ab initio} random structure searches
\citep{Pickard2009} demonstrate that iron remains in a hexagonal close packed (hcp) structure
remains stable up to pressures approaching those at Jupiter's center, $\sim$2.3 TPa, at
which point it undergoes a phase transition to face centered cubic (fcc)
structure. \citet{stixrude2012} demonstrated a gradual decrease in this transition
pressure with temperature. Simulations of liquid hydrogen
\citep{Militzer2008,militzer2013a,Mcmahon2012} undergo a gradual transition from molecular to
metallic, which is complete by $\sim$ 0.4 TPa at low temperatures.  Stable
mixtures of Fe and H have been suggested at lower $P$-$T$ conditions,
applicable to terrestrial cores \citep{Bazhanova2012}. 

\section{Material phases}

We performed density functional theory molecular dynamics (DFT-MD) simulations
to determine the energetics of a dissolution reaction, in which solid iron
dissolves in pure liquid hydrogen. We calculate a Gibbs free energy of
solvation:
\begin{equation} \label{feh_gibbs}
  \Delta G_{sol}\left(\mathrm{Fe}:256\mathrm{H}\right) = G\left(\mathrm{H}_{256}\mathrm{Fe}\right) -
  \left[ G\left(\mathrm{H}_{256}\right) +
    \frac{1}{32}G\left(\mathrm{Fe}_{32}\right) \right]\mathrm{,}
\end{equation}
where $G\left(\mathrm{H}_{256}\right)$ and $G\left(\mathrm{Fe}_{32}\right)$
are the Gibbs free
energies of a pure hydrogen liquid and solid or liquid iron.
$G\left(\mathrm{H}_{256}\mathrm{Fe}\right)$ is the Gibbs free energy of 1:256 liquid solution of iron
in hydrogen. We assume that analysis of a single low-concentration solution is
sufficient to determine the onset of core erosion, since the reservoir of
metallic hydrogen would be much larger than the core. This does not rule out
non-ideal effects of higher concentrations that might exist in a narrow,
poorly convecting layer at the top of a core. 

\begin{table}[h]
%\tabletypesize{\scriptsize}
%\tablecolumns{9}
%\tablewidth{0pc}
\centering
%\resizebox{\linewidth}{!}{%

\caption{Thermodynamic parameters derived from DFT-MD simulations.\label{data}}
\begin{adjustbox}{max width=\textwidth}
\begin{tabular}{rrlcrrrrr}
    \hline
 {$P$} & {$T$} & {System} & {Phase} &
{$\rho$} & {$F$} & {$U$} & {$G$} & {$S$}\\
(GPa) & (K) & ~~~~~- &  - & ($\mathrm{g}/\mathrm{cm}^3$) & (eV)~~~~ & (eV)~~~~
& (eV)~~~~ & ($\mathrm{k}_b/\mathrm{K}$) \\
\hline
400   &  2000   &  $\mathrm{Fe}_{32}$     &  hcp  &  14.408  &  $-$190.8(4)\phantom{0}   &  $-$154.4(0)            &  323.3(4)\phantom{0}     &  211.(4)              \\
.     &  .      &  $\mathrm{H}_{256}$    &  liq  &  1.2709  &  $-$415.6(5)\phantom{0}   &  $-$228.(0)\phantom{0}  &  419.37(9)               &  1088.(2)             \\
.     &  .      &  $\mathrm{H}_{256}\mathrm{Fe}$  &  liq  &  1.5192  &  $-$423.8(1)\phantom{0}   &  $-$234.(0)\phantom{0}  &  427.1(8)\phantom{0}     &  1101.(6)             \\
1000  &  2000   &  $\mathrm{Fe}_{32}$     &  hcp  &  18.279  &  $-$12.4(0)\phantom{0}    &  18.1(2)                &  1000.8(5)\phantom{0}    &  177.(1)              \\
.     &  .      &  $\mathrm{H}_{256}$    &  liq  &  1.8916  &  23.0(3)\phantom{0}       &  175.9(6)               &  1425.6(6)\phantom{0}    &  887.(3)              \\
.     &  .      &  $\mathrm{H}_{256}\mathrm{Fe}$  &  liq  &  2.2534  &  20.4(1)\phantom{0}       &  175.(2)\phantom{0}     &  1454.7(1)\phantom{0}    &  898.(3)              \\
1000  &  2000   &  $\mathrm{Fe}_{32}$     &  fcc  &  18.269  &  $-$10.35(0)              &  19.8(0)                &  1003.4(6)\phantom{0}    &  174.(9)              \\
.     &  .      &  $\mathrm{H}_{256}$    &  liq  &  1.8916  &  23.0(3)\phantom{0}       &  175.9(6)               &  1425.6(6)\phantom{0}    &  887.(3)              \\
.     &  .      &  $\mathrm{H}_{256}\mathrm{Fe}$  &  liq  &  2.2534  &  20.4(1)\phantom{0}       &  175.2(3)               &  1454.7(1)\phantom{0}    &  898.(3)              \\
1000  &  15000  &  $\mathrm{Fe}_{32}$     &  liq  &  16.970  &  $-$506.2(1)\phantom{0}   &  12(7).\phantom{0(0)}   &  585.(2)\phantom{00}     &  49(0).\phantom{(0)}  \\
.     &  .      &  $\mathrm{H}_{256}$    &  liq  &  1.6315  &  $-$2064.1(1)\phantom{0}  &  595.(4)\phantom{0}     &  $-$437.9(2)\phantom{0}  &  2057.(5)             \\
.     &  .      &  $\mathrm{H}_{256}\mathrm{Fe}$  &  liq  &  1.9468  &  $-$2091.9(7)\phantom{0}  &  598.(7)\phantom{0}     &  $-$431.7(8)\phantom{0}  &  2081.(6)             \\
4000  &  2000   &  $\mathrm{Fe}_{32}$     &  fcc  &  28.374  &  754.91(4)                &  777.6(2)\phantom{0}    &  3365.97(1)              &  131.(8)              \\
.     &  .      &  $\mathrm{H}_{256}$    &  liq  &  3.6375  &  1392.3(8)\phantom{0}     &  1487.6(4)              &  4310.0(0)\phantom{0}    &  552.(7)              \\
.     &  .      &  $\mathrm{H}_{256}\mathrm{Fe}$  &  liq  &  4.3078  &  1412.3(8)\phantom{0}     &  1508.(4)\phantom{0}    &  4413.4(7)\phantom{0}    &  55(7).\phantom{(0)}  \\
4000  &  15000  &  $\mathrm{Fe}_{32}$     &  fcc  &  27.826  &  382.1(9)\phantom{0}      &  87(5).\phantom{0(0)}   &  3045.7(0)\phantom{0}    &  38(1).\phantom{(0)}  \\
.     &  .      &  $\mathrm{H}_{256}$    &  liq  &  3.3618  &  $-$392.9(3)\phantom{0}   &  1917.(3)\phantom{0}    &  2763.9(8)\phantom{0}    &  1787.(3)             \\
.     &  .      &  $\mathrm{H}_{256}\mathrm{Fe}$  &  liq  &  3.9865  &  $-$392.2(5)\phantom{0}   &  194(3).\phantom{0(0)}  &  2850.7(4)\phantom{0}    &  1807.(2)             \\
4000  &  20000  &  $\mathrm{Fe}_{32}$     &  fcc  &  27.550  &  176.(9)\phantom{00}      &  92(2).\phantom{0(0)}   &  2866.(8)\phantom{00}    &  43(2).\phantom{(0)}  \\
.     &  .      &  $\mathrm{H}_{256}$    &  liq  &  3.2731  &  $-$1284.5(8)\phantom{0}  &  208(1).\phantom{0(0)}  &  1957.2(6)\phantom{0}    &  1952.(6)             \\
.     &  .      &  $\mathrm{H}_{256}\mathrm{Fe}$  &  liq  &  3.8824  &  $-$1294.0(9)\phantom{0}  &  210(4).\phantom{0(0)}  &  2035.5(1)\phantom{0}    &  1972.(0)             \\
\hline
\end{tabular}%
%}
\end{adjustbox}
\end{table}

All simulations presented here were performed using the Vienna {\it ab initio}
simulation package (VASP) \citep{Kresse1996}. VASP uses the DFT formalism utilizing
pseudopotentials of the projector augmented wave type \citep{Blochl1994} and the exchange-correlation
functional of Perdew, Burke and Ernzerhof \citep{Perdew1996}. The iron
pseudopotential treats a $[\mathrm{Mg}]\mathrm{3pd}^6\mathrm{4s}^2$ electron
configuration as valence states, and a 2$\times$2$\times$2
grid of k-points is used for all simulations. Simulations on hydrogen 
and the solution were performed with a 900 eV cutoff energy for the plane wave expansion, while a 300
eV cutoff was used for iron. A time step of 0.2 fs was used for all liquid simulations, a
0.5$-$1.0 fs time step  was used for high and low temperature iron simulations 
respectively. The $\Delta G_{sol}$ results were confirmed to be well-converged
with respect to the energy cutoff and time step. 
Prohibitively long simulation times required that convergence with respect to
finer k-point meshes be verified over a subset of configurations generated by
a simulation with a 2$\times$2$\times$2 grid. A snapshot from a representative DFT-MD simulation of a iron atom dissolved in hydrogen is shown in Fig.~\ref{fig:HFe_snap}

\begin{figure}[h] %  figure placement: here, top, bottom, or page
   \centering
   \includegraphics[width=30pc]{figs/chg1_smoothed.png} 
\caption{ Simulation snapshot showing the electron density of an iron atom dissolved in liquid
    metallic hydrogen.\label{fig:HFe_snap}}
\end{figure}

Iron simulations assume an hcp or fcc structure within their respective
stability regimes \citep{Pickard2009,stixrude2012}. We confirmed Fe to be solid up
to 20000 K at 4 TPa, and to be a liquid at temperatures as low as 15000 K at
1 TPa. We also confirmed that the Gibbs free energy favors hcp stability over fcc
at 1 TPa, though the difference is negligible for our subsequent analysis of
dissolution. We found 32 atom supercells to be sufficient for Fe simulation.
Finite size effects required that we use large 256 atom supercells for
hydrogen, to which one Fe atom was added for the solution. Cubic
supercells are used for fcc and liquid runs. In order to maintain the same
number of atoms for the hcp an orthogonal supercell with edges defined the combination of
hexagonal unit cell vectors $\mathbf{a}$,$\mathbf{a}+\mathbf{b}$, and
$\mathbf{c}$.

\begin{table}[h]
    \centering
%\tabletypesize{\scriptsize}
%\tablecolumns{9}
%\tablewidth{0pc}
\caption{Gibbs free energy of solvation for Fe in liquid H\label{solvation}}
\begin{tabular}{rrcc}
    \hline
{$P$} & {$T$} &  {Fe Phase} & {$\Delta G$} \\
(GPa) & (K)~ & - & (eV) \\
\hline
400 \phantom{0}  & 2000  & hcp  &  $-$2.2 $\pm$ 0.14    \\
1000\phantom{0}  & 2000  & hcp  &  $-$2.5 $\pm$ 0.12    \\
1000\phantom{0}  & 2000  & fcc  &  $-$2.3 $\pm$ 0.13    \\
1000\phantom{0}  & 15000 & liq  &  $-$12.2 $\pm$ 0.20\phantom{0}   \\
4000\phantom{0}  & 2000  & fcc  &  $-$1.71 $\pm$ 0.056  \\
4000\phantom{0}  & 15000 & fcc  &  $-$8.42 $\pm$ 0.066  \\
4000\phantom{0} & 20000 & fcc  &  $-$11.34 $\pm$ 0.078\phantom{0} \\
\hline
\end{tabular}
\end{table}


Cell volumes at each temperature were determined by fitting a pressure-volume
polytrope equation of state to short DFT-MD simulations. The resulting DFT
pressures were all within 0.1\% of the target value. Gibbs free energies were
computed for the three systems, $\mathrm{Fe}_{32}$, $\mathrm{H}_{256}$ and
$\mathrm{H}_{256}\mathrm{Fe}_1$, using the thermodynamic integration method
with simulation times of 1.0 ps for $\mathrm{H}_{256}$ and
$\mathrm{H}_{256}\mathrm{Fe}_1$ and 2.5$-$5.0 ps for Fe. $\mathrm{H}_{256}$ and
$\mathrm{H}_{256}\mathrm{Fe}_1$ runs with $\lambda =1$ were extended to 4.0 ps
for precise calculations of the internal energy, which allows for determination
of the entropic component of the Helmholtz free energy. The calculated energies
and entropy are presented in Tab.~\ref{data}, along with the density.  The Gibbs free
energy of solvation, calculated using Eq.~\ref{feh_gibbs} , is presented in Tab.~\ref{solvation} for each
pressure-temperature condition. A negative $\Delta G_{sol}$ implies that the
Gibbs free energy of the solution is lower than that of the separated phases.
Therefore, dissolution is favored at a solute concentration higher than 1:256. 


\section{Simulation results}

We find dissolution of iron to be strongly favorable at conditions corresponding to
the interiors of gas giants. Fig.~\ref{fig:dGall}  shows the variation of $\Delta G_{sol}$
with temperature and pressure. $\Delta G_{sol}$ exceeds $-$10 eV per iron atom for
plausible temperatures of Jupiter's core. Dissolution remains favorable even
at temperatures far below those predicted by model adiabats
\citep{militzer2013a,militzer2013b}. The energetics
are only weakly dependent on pressure, and  $\Delta G_{sol}$ becomes increasingly
negative with decreasing pressure. The solubility increases with a nearly
linear trend in T, 
yielding slope of $\sim$0.53 meV/K. As a result, solubility is favored
through the entire range of conditions considered, and likely the entire range
for metallic hydrogen regions of giant planets.

% Include figure dG-T figure with different lines for all points
 \begin{figure}[h] %  figure placement: here, top, bottom, or page
   \centering
   \includegraphics[width=22pc]{figs/dGall.eps} 
\caption{Gibbs free energy of solvation for solid Fe in liquid metallic
    hydrogen. Negative values favor dissolution for a solute ratio of 1:256. \label{fig:dGall}}
\end{figure}

Fig.~\ref{fig:dGcomp} shows a breakdown of the data into contributions by various
thermodynamic parameters. Included in the figure are: $\Delta F_{sol}$, $\Delta U_{sol}$,
$P\Delta V_{sol}$ and $-T\Delta S_{sol}$, respectively, the Helmholtz free energy, internal
energy, volumetric work and entropic contributions contributions to $\Delta
G_{sol}$. Note that $\Delta F_{sol}$, $\Delta U_{sol}$ and $P\Delta V_{sol}$
are calculated independently, while $\Delta G_{sol}=\Delta F_{sol}+P\Delta
V_{sol}$ and $-T\Delta S_{sol}=\Delta F_{sol}-\Delta U_{sol}$ are derived. 
The trend of solubility with temperature is dominated by the entropic term.
The high solubility at low temperatures is
reflected in the negative values of $\Delta U_{sol}$, indicating that the
mixed system is energetically favorable independently of the entropy term. 

% Figure showing comp
 \begin{figure}[h] %  figure placement: here, top, bottom, or page
   \centering
   \includegraphics[width=22pc]{figs/dGcomp.eps} 
\caption{Breakdown of $\Delta G_{sol}$ into contributions from: internal
energy,
$\Delta U_{sol}$, pressure effects, $P\Delta V_{sol}$, and entropic effects,
$-T\Delta S_{sol}$. Plots show variation with (a) temperature at P=4 TPa, and
(b) pressure at T=2000 K. \label{fig:dGcomp}}
\end{figure}

Our calculations neglect any interactions between iron atoms in solution, and
thus represent solubility in the low-concentration limit.
$\Delta G_{sol}$ can be related to the volume change associated with
the insertion of an iron of atom into hydrogen, as other contributions are
constant with respect concentration. It can be shown that results for
simulations with a 1:n solute ratio can be generalized to a ratio of 1:m using
  %\begin{mathletters}
  \begin{eqnarray}
  \Delta G_c &\approx& F_0(H_mFe)-F_0(H_m)-F_0(Fe)- \left[
    F_0(H_nFe)-F_0(H_n)-F_0(Fe)\right] \\
    &=& -k_BT\log\left\{ 
    \frac{\left[V(\mathrm{H}_n\mathrm{Fe}) +
    \frac{m-n}{n}V(\mathrm{H}_n)\right]^{m+1}
    \left[V(\mathrm{H}_n)\right]^{n}}
    {\left[V(\mathrm{H}_n)\frac{m}{n}\right]^m
    \left[V(\mathrm{H}_n\mathrm{Fe})\right]^{n + 1}}
   \right\},
\end{eqnarray}
%\end{mathletters}
where $\Delta G_c = \Delta G_{sol}(1:m)- \Delta G_{sol}(1:n)$, and
$V(\mathrm{H}_n)$ and $V(\mathrm{H}_n\mathrm{Fe})$ are the volumes for the
simulations of hydrogen and the solution respectively. 
Fig.~\ref{fig:concentrations} shows the shift of
$\Delta G_{sol}$ at 4 TPa at Fe concentrations of 1:100 and 1:1000. $\Delta
G_{sol}$ is decreased for higher concentrations, but not to an extent where
dissolution would become disfavored. At 20000 K
this difference between 1:100 and 1:256 is  $<$2 eV per iron atom, and
at 2000 K is smaller than uncertainty in calculated values of $\Delta
G_{sol}$.

% Include figure dG-T figure with different lines for all points
 \begin{figure}[h] %  figure placement: here, top, bottom, or page
   \centering
   \includegraphics[width=22pc]{figs/concentration.eps} 
\caption{Shift in $\Delta G_{sol}$ from a system with and Fe:H ratio of 1:256 to 1:100 and
1:1000 in the low-concentration limit.}
\end{figure}


The nature of the Fe-H system poses additional numerical challenges compared to other solutes
considered previously \citep{Wilson2010,Wilson2012a,wilson12b,Gonzalez2013}.
These can be attributed to the comparatively large change in volume
and electron density associated with the insertion of an iron atom into
metallic hydrogen. We found it
more efficient to determine cell volumes by fitting an equation of state to a
collection of MD simulations at constant volume, rather than performing
extended constant pressure simulations. 
Finite size effects were also found to be more significant for the Fe-H
system, due to iron's relatively large volume and number of valence electrons. Fig.~\ref{fig:finite_size}
4 shows the convergence of a difference in internal energy between H and H-Fe
for MD simulations with 128, 256 and 512 hydrogen atoms. We find 256 hydrogen
atoms to be necessary in contrast to the previous studies that required only 128. 


% Include figure dG-T figure with different lines for all points
 \begin{figure}[h] %  figure placement: here, top, bottom, or page
   \centering
   \includegraphics[width=22pc]{figs/finite_size.eps} 
\caption{Energy of insertion for a single Fe atom into supercells of liquid metallic hydrogen containing
128, 256, and 512 atoms. Finite size effects are significant
for $\mathrm{H}_{128}$, but are negligible within error for $\mathrm{H}_{256}$. \label{fig:finite_size}}
\end{figure}

The convergence with k-point grid resolution is also slower
than in previous studies, and presents the greatest uncertainty in this study.
MD calculations with a 3$\times$3$\times$3 k-point grid are prohibitively expensive. An
estimate of this error for the results presented here is obtained by
evaluating the internal energy over configurations sampled from a MD
trajectory with a 2$\times$2$\times$2 k-point grid. Fig.~\ref{fig:kpoints} shows this
estimated correction of $\Delta G_{sol}$ for the k-point grid used. 
The shifts are on the order $\sim$1 eV per iron atom, but are
consistently negative for both quantities, leading to dissolution being more favorable. 


 \begin{figure}[h] %  figure placement: here, top, bottom, or page
   \centering
   \includegraphics[width=22pc]{figs/kpoints.eps} 
\caption{Estimated corrections to $\Delta G_{sol}$ and $\Delta G_{sol}$
coarseness of k-point grid used in DFT-MD runs. DFT calculations with a $3\times
3\times 3$ k-point grid were performed sampling a trajectory generated by an MD
simulation with a $2\times2\times 2$ k-point mesh. \label{fig:kpoints}}
\end{figure}

\section{Discussion}

With the results of previous studies
\citep{Wilson2012a,wilson12b,Gonzalez2013}, we can now present a comprehensive
picture for the solubility of typical core materials in liquid
metallic hydrogen.
Dissolution is strongly favored for both iron and water
ice. However, for water the high solubility is attributed entirely to the
entropy, whereas iron has a favorable internal energy component that favors
dissolution at low temperatures.
Both phases are found to be soluble throughout the entire metallic
hydrogen region of both Jupiter and Saturn. The rocky components, MgO and
$\mathrm{SiO}_2$, have more moderate solubilities, with $\mathrm{SiO}_2$ being slightly higher. The saturation
curves are, however, less steep than the adiabats for Jupiter and Saturn. As a
result, solubility is favored for Jupiter's core, but the rocky components of
Saturn's core may be stable given the present uncertainty in the planet's
adiabat.

The rocky components are found to be stable at lower pressures, approaching the
metallic transition. If Mg and Si are advected upwards in sufficient
concentration, they may precipitate, while Fe and $\mathrm{H}_2\mathrm{O}$
would remain in solution, at least to the molecular-metallic transition.  The
presence of a significant dissolved component at shallow depths may have
consequences for the density profile and transport properties of hydrogen,
which influence thermal structure and magnetic field generation.
Fig.~\ref{fig:mgoerosion} shows the relative solubility fields of these
candidate core materials.

 \begin{figure}[h] %  figure placement: here, top, bottom, or page
   \centering
   \includegraphics[width=22pc]{figs/MgO_core-erosion.png} 
\caption{ Solubility fields for MgO and H$_2$0 ice compared to P-T state for Jupiter and Saturn
    core-envelope boundary. Figure from \citet{wilson12b}, credit: Hugh Wilson.\label{fig:mgoerosion}}
\end{figure}

Core erosion is thermodynamically favorable in gas giant planets,
with the possible exception of smaller, cooler planets, like Saturn. For these planets, the
outer icy layers are soluble, but the rocky layers may not be. The innermost
iron component, though soluble, might remain isolated from reaction with hydrogen.
This might allow Saturn to have a larger, less eroded core than
Jupiter, a result consistent with current observational constraints.
Nevertheless, our results imply
that confirmation of a massive core for Jupiter would support the
core-accretion model over gravitational collapse. While erosion of such a core
may be slow due to inefficient double-diffusive convection
\citep{Stevenson1982a,Chabrier2007,Leconte2012,Mirouh2012}, settling of dispersed
refractory material to form a core is inconsistent with our results. Late formation of a core
would require a large amount material from captured planetesimals
surviving descent to the planet's center.

It may be possible to attribute some
emerging trends in exoplanet mass-radius relationships to the difference in
solubilities between rock and ice, or rock and metal. However, as we have
shown, such thermodynamic differences are likely to only be significant in
smaller, cooler planets, where redistribution of dense material by
double-diffuse convection would be least efficient. The energetics of the
dissolution reaction should be insignificant compared to the role of density
in the redistribution of dense material. The work required to raise an iron of atom to
the molecular-metallic transition is on the order of 1000 eV, whereas the
contribution from the dissolution reaction is $\sim$1$-$10 eV. We conclude
that the process of core erosion is thermodynamically consistent with ab
initio simulations of the relevant materials, and its
significance warrants close consideration in future models of giant planet
evolution.




