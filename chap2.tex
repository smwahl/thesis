\chapter{First-Principles simulation Methods}\label{chap2}

\section{Density Functional Theory}

\section{Thermodynamic Integration}


%%%% From F-H Manuscript
Compression of materials with modern experimental techniques can reach megabar
pressures, however the pressure-temperature conditions near Jupiter's core ($>$4 TPa) remain
inaccessible. Temperatures in shockwave experiments climb rapidly at high
pressures in comparison with planetary adiabats, while diamond anvil cell
experiments
are best suited for low temperatures. As a result, simulations based on ab 
initio theories are best suited for directly probing the conditions of gas 
giant interiors.


We performed density functional theory molecular dynamics (DFT-MD) simulations
to determine the energetics of a dissolution reaction, in which solid iron
dissolves in pure liquid hydrogen. We calculate a Gibbs free energy of
solvation:
\begin{equation}
  \Delta G_{sol}\left(\mathrm{Fe}:256\mathrm{H}\right) = G\left(\mathrm{H}_{256}\mathrm{Fe}\right) -
  \left[ G\left(\mathrm{H}_{256}\right) +
    \frac{1}{32}G\left(\mathrm{Fe}_{32}\right) \right]\mathrm{,}
\end{equation}
where $G\left(\mathrm{H}_{256}\right)$ and $G\left(\mathrm{Fe}_{32}\right)$
are the Gibbs free
energies of a pure hydrogen liquid and solid or liquid iron.
$G\left(\mathrm{H}_{256}\mathrm{Fe}\right)$ is the Gibbs free energy of 1:256 liquid solution of iron
in hydrogen. We assume that analysis of a single low-concentration solution is
sufficient to determine the onset of core erosion, since the reservoir of
metallic hydrogen would be much larger than the core. This does not rule out
non-ideal effects of higher concentrations that might exist in a narrow,
poorly convecting layer at the top of a core. 

\subsection{Computation of Gibbs Free Energies}

Free energy calculations require the determination of a contribution from
entropy, which is not determined from the standard DFT-MD formalism. To achieve
this, we adopt a two step thermodynamic integration method, as 
used in previous studies \citep{morales09,wilson10,wilson12a,wilson12b}. The method
requires integration of the change in Helmholtz free energy over an
unphysical, yet thermodynamically permissible, transformation between two
systems governed by potentials $U_a\left(\mathbf{r_i}\right)$ and 
$U_b\left(\mathbf{r_i}\right)$. We define a hybrid potential 
$U_{\lambda}=\left(1-\lambda\right)U_a+\lambda U_b$, where $\lambda$ is the
fraction of the potential $U_b\left(\mathbf{r_i}\right)$. The difference is
Helmholtz free energy is then given by
\begin{equation}
  \Delta F_{b\to a}\equiv F_b - F_a = \int_{0}^{1}{d\lambda\,\langle U_b\left(\mathbf{r_i}\right) -
  U_a\left(\mathbf{r_i}\right) \rangle_{\lambda}}
\end{equation}
where the averaging is over configurations generated
during simulations of the system governed by the hybrid potential. 

To increase the efficiency, we calculated the Helmholtz free energy in two steps 
each involving an integral of the form of Eq. 2. We first find $\Delta
F_{\mathrm{DFT}\to \mathrm{cl}}$,
between the systems governed by DFT and classical pair potentials, which are
fit via a force-matching method \citep{izvekov04}. In the second step $\Delta
F_{\mathrm{cl}\to \mathrm{an}}$, between the pair potentials and a reference system with an
analytic solution $F_{\mathrm{an}}$. The Helmholtz energy for the DFT systems,
$F_{\mathrm{DFT}}=F_{\mathrm{\mathrm{an}}}+\Delta F_{\mathrm{DFT}\to \mathrm{cl}}+\Delta F_{\mathrm{cl} \to \mathrm{an}}$, may then be
compared directly. The first step requires DFT-MD simulations for a small
number of $\lambda$ values, while the second uses applies a much faster classical Monte Carlo
approach at numerous values of $\lambda$ to ensure a smooth integration.

Finding a suitable reference system is essential to the method, as it allows
$\Delta F_{\mathrm{cl}\to \mathrm{DFT}}$ to be found with a small number of 
number of integration steps, and prevents solids from melting or transforming to a new
structure. For liquid systems, we integrate to a non-interacting
ideal gas, using classical two-body pair potentials as the intermediate
step. For solid iron, we integrate to an Einstein solid, a system of
non-interacting, 3-d harmonic oscillators. For the intermediate classical
solid system we use a 50-50
'mixture' of two-body and one-body harmonic oscillator potentials. Spring
constants for the Einstein terms are found from the mean-squared displacement 
of atoms from their ideal lattice sites during a DFT-MD simulation. 


%%%% Similar discussion from FeMgO paper

The Gibbs free energy of a material includes a contribution from entropy of the system.
Since entropy is not determined in the standard DFT-MD formalism, we adopt a two step
thermodynamic integration method, used in previous studies
\citep{Wilson2010,Wilson2012a,Wahl2013,Gonzalez2014}.  The thermodynamic integration technique
considers the change in Helmholtz free energy for a transformation between two systems
with governing potentials $U_a\left(\mathbf{r_i}\right)$ and
$U_b\left(\mathbf{r_i}\right)$. We define a hybrid potential
$U_{\lambda}=\left(1-\lambda\right)U_a+\lambda U_b$, where $\lambda$ is the fraction of
the potential $U_b\left(\mathbf{r_i}\right)$. The difference in Helmholtz free energy is
then given by
\begin{equation} \label{eqn:td_int}
  \Delta F_{a\to b}\equiv F_b - F_a = \int_{0}^{1}{d\lambda\,\langle U_b\left(\mathbf{r_i}\right) -
  U_a\left(\mathbf{r_i}\right) \rangle_{\lambda}}
\end{equation}
where the bracketed expression represents the ensemble-average over configurations,
$\mathbf{r_i}$, generated in simulations with the hybrid potential at constant volume and
temperature. This technique allows for direct comparisons of the Helmholtz free energy of
DFT phases, $F_{\rm DFT}$, by finding their differences from reference systems with a known
analytic expression, $F_{\rm an}$. 

In practice, it is more computationally efficient to perform the calculation $\Delta
F_{\rm an\to DFT}$ in two steps, each involving an integral of the form of Eqn.
\ref{eqn:td_int}. We introduce an intermediate system governed by classical pair
potentials, $U_{\rm cl}$, found by fitting forces to the DFT trajectories
\citep{Wilson2010,Izvekov2004}. For each pair of elements, find the average force in bins
of radial separation and fit the shape of a potential using a cubic spline function. We
constrain the potential to smoothly approach zero at large radii and use a linear
extrapolation at small radii, where the molecular dynamics simulations provide
insufficient statistics. Examples of these potentials are included in the online
supplementary information. The full energetics of the system is then described as
%$F_{\mathrm{DFT}}=F_{\mathrm{\mathrm{an}}}+\Delta F_{\mathrm{cl}\to \mathrm{DFT}}+\Delta F_{\mathrm{an} \to \mathrm{cl}}$,
\begin{equation} \label{eqn:two_step}
F_{\mathrm{DFT}}=F_{\mathrm{\mathrm{an}}}+\Delta
F_{\mathrm{an} \to \mathrm{cl}}+\Delta F_{\mathrm{cl}\to \mathrm{DFT}}
\end{equation}
where $\Delta F_{\mathrm{cl}\to \mathrm{DFT}}$ requires a small number of DFT-md
simulations, and $\Delta F_{\mathrm{an} \to \mathrm{cl}}$ numerous, but inexpensive
classical Monte Carlo (CMC) simulations. The method depends on a smooth integration of
$\Delta F_{\mathrm{cl}\to \mathrm{DFT}}$ and avoiding any first order phase transitions
with $\lambda$.  We use five $\lambda$ points, for all $\Delta F_{\mathrm{cl}\to
\mathrm{DFT}}$ integrations. For solid MgO we use a combination of classical pair and
one-body harmonic oscillator potentials for $U_{\rm cl}$ \citep{Wilson2012a,Wahl2013}.
Liquids we use only pair potentials. For solids the analytical reference system is an
Einstein solid with atoms in harmonic potentials centered on a perfect lattice, while a
gas of non-interacting particles is used for liquids. We found integrating over 5 lambda
points to be sufficiently accurate, with an increase to 9 lambda points changing our
results by $<0.003$ eV per formula unit. In the online supplementary information we
include two additional tests of the thermodynamic integration method, demonstrating that
our results are not sensitive use of different classical potentials or the integration
path with respect to the interaction between different species in the multicomponent
systems.

All DFT simulations presented here were performed using the Vienna {\it ab initio}
simulation package (VASP) \citep{Kresse1996}. VASP uses projector augmented wave
pseudopotentials \citep{Blochl1994} and the exchange-correlation functional of Perdew,
Burke and Ernzerhof \citep{Perdew1996}. Although the DFT formalism is based on a
zero-temperature theory, DFT-MD simulations at high temperatures have been shown to agree
with theory developed for warm dense matter \citep{Driver2012}. We use an iron

