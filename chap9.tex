\chapter{Conclusions}\label{chap9}

The wide range of topics covered in this dissertation exemplifies the relevance of
first-principles material simulations to a broad range of disciplines within
planetary science. I have presented calculations for materials under pressure ranges
multiple TPa and tens of thousands of Kelvin. I include studies of materials relevant
to the Earth and other terrestrial planets, both today and during their violent
formations, as well as the gas giants. This versatility when it comes exploring
materials at the extreme conditions under which they reside within the planets is a
major beneficial feature of first-principles techniques. Of course the studies
presented here represent just a small selection of the relevant areas of planetary
science to which these simulations are important.

In Chapters 2-4, I presented methods and studies focused on the calculation and
comparison of entropy in materials. This is an area of first-principles simulation
which has seen increasing attention in recent years, due largely to the fact that
these calculations remain computationally expensive. The significant improvements in
the speed and availability of high performance computing over this period of time has
played a huge role in expanding the range of tractable problems for materials
simulation. This is particularly important in planetary science where liquids and
high temperature environments are key to many processes of interest in their deep
interiors. With these newly developed techniques (in particular thermodynamic
integration) we are able to push beyond the more typical equation of state
calculations and begin addressing problems involving simple chemical reactions.

The miscibility of planetary materials is of profound interest across planetary
sciences. The basic structure of the planets reflect the tendency of different
materials to combine or exclude each other. Assessing the extent to which this occurs
for Major elements in deep planetary interiors has been a major focus of my thesis
work. In Chapter 3, I demonstrated that iron metal is soluble in in liquid metallic
hydrogen over the entire range for which these materials might exist in the Jovian
planets.  Combined with similar results on H$_2$O, MgO and SiO$_2$
\citep{Wilson2012a,wilson12b,Gonzalez2013} we now predict that a dense core of Jupiter
would be entirely soluble in the overlying hydrogen-helium envelope. This result has
important consequences for the formation and evolution of the giant planets, although
the specifics of Jupiter's evolution with an eroding core will require additional
modeling efforts to understand fully. 

Similarly, in Chapter 4, I assessed the solubility of analog materials for
terrestrial mantles and cores at temperatures well above those in the present day
Earth, but well within the extreme temperatures present in the aftermath of energetic
giant impacts, such as those hypothesized in the formation of the Earth's moon
\citep{Cuk2012,Canup2012}. In this study I derived solvus closure temperatures for
the Fe-MgO system between $\sim$6000-9000 K over the pressure range within the Earth.
Since heating from giant impacts is naturally heterogeneous, it can therefore be
expected that a portion of the Earth was heated to these temperatures at times during
it's early history. Since the time of publication, experimental results at lower
temperatures \citep{Badro2016} have found solubilities broadly consistent with these
solvus closure temperatures. Thus, a fraction of the planet was likely equilibrated
in an entirely different regime with fully mixed core and mantle material.
Geochemical models for the distribution of elements within the Earth are typically
done based on understanding of material behaviour at lower temperatures, and thus may
require some modification if enough of the planet experienced this very different
material regime.  Assessing the importance of this mantle core miscibility is not
straightforward, as there is presently great uncertainty in processes at work in the
immediate aftermath of these energetic giant impacts, during which the degree of
mechanical mixing may be at least as important as the preferred thermodynamic state
of the material \citep{Nakajima2014,Deguen2014}. Although all of the work presented
in this dissertation is focused on the most abundant elements, it is also possible to
use the same techniques to assess the solubility of trace elements which act as
important tracers for deep mantle processes in terrestrial samples.

Finding the properties of material phases is, in general, only half of the
story when it comes to applying first principles techniques to problems in planetary
science. Correctly using an EOS derived from microscopic principles to model
planetary scale processes is a complex art. Chapter 5 summarizes work towards
building an integrated model for the evolution of the cores of small terrestrial
planets. The interest in this work was focused on peculiarities in the Fe-S-Si phase
diagram, for which I aggregated a combined model from various experimental sources.
However, it is not possible to simply model a planet's core in isolation, since the
heat flow is governed by the mantle and lithosphere above. With sufficient material
information combined with iterative methods it is possible to create a self
consistent model through a planet's interior, even when the precise structure is not
known. The results of such a model will, however, be very sensitive to changes in the
equations of state, for which numerous assumptions must be made to account for
limitations of and inconsistencies between different experimental studies.

With this in mind, Chapters 6-8 dealt with development of new method to relate
existing equations of state to a direct measurable quantity for the gas giant
planets. We demonstrated that the concentric Maclaurin spheroid (CMS) method is
capable of measuring gravitational moments to higher precision in both axisymmetic
cases and cases with a tidal perturbation. Since this method is non-perturbative, it
led to the discovery of some previously undiscovered features of tidal response of a
rotating body: the splitting of Love numbers of the same order, and a dependence on
those harmonics on the orbital distance of the satellite, neither of which occur in
perturbative calculations. Of more immediate, was the result that the primary
tidal Love number, $k_{22}$, is elevated by rotation, bringing predictions in line
with the recent observational result for Saturn \citep{lainey2016}. It remains to be
seen whether Jupiter's tidal response also shows this elevated response.

Finally I detailed a modelling study of the planet Jupiter for interpreting the even
gravitational harmonics from the \textit{Juno} spacecraft's preliminary orbits. The
CMS calculations of the gravitational moments are capable of reproducing the measured
$J_n$, but not without significant modifications to the `preferred' model put forward
before orbital insertion \citep{hubbard2016}. Differences in equations of state have
a very strong influence on results, despite matching the best high pressure
experiments on hydrogen to within experimental uncertainty. The equation of state
using thermodynamic integration to determine entropy, and thus the one favored as
most physically realistic, encounters problems because it is comparatively dense at
low pressures. We demonstrated that a dilute core, expanded to a significant fraction
of the planet's radius is helpful for matching $J_n$ with the preferred equation of
state. However, even with an expanded core it is likely that there is at least one
remaining issue with the current models. We proposed a number of solutions,
including: a lower than expected heavy element content in the outer envelope, a
hotter outer envelope (perhaps due to an extended radiative zone) or a systematic
offset in the DFT densities for hydrogen-helium mixtures. Regardless, it appears that
independent testing of the calculated equations of state may be necessary in order to
make precise statements about the state of Jupiter's deep interior.
