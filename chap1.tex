\chapter{Introduction}\label{chap1}

\begin{quote}
{\small\textit{The world is richer than it is possible to express in any single language. - Ilya Prigogine}}
\end{quote}

%\begin{theorem}
%\tolerance=10000\hbadness=10000
%Koopmanss; Kohn-Sham
%\end{theorem}

Deep inside the Earth and giant planets, materials are under extremly high pressures and temperatures, which are mostly a million atomosphere pressure or higher and thousands degree temperature.
%at the Earth's core-mantle boundary are 1.36 million atomosphere pressures and ~5000 Kelvin temperatures. 
Diverse minerals and phases exist under these conditions and they together govern the specific dynamics of each planet, depending on its particular formation and evolution history. The chemistry and physics of minerals under high-pressure, temperature conditions determine the properties of planet interiores, which is coupled with geological signatures at the surface. Therefore, studying of minerals under extreme conditions has been an important topic in Earth and planetary science, essential for human beings to explore the solar system and beyond, such as exoplanets. 

The physic quantity that governs the stability of phases is the Gibbs free energy $G=U+PV-TS$, where $U$ is the internal energy, $PV$ and $TS$ correspond to the pressure-volume and the temperature-entropy term, respectively. The lower in $G$, the more stable a phase is. Therefore, for phases with similar internal energy, high pressure tends to stabilize those with smaller volumes, while high temperature eases the formation of those with higher entropies.  Competition between the two factors leads to the particular stability region of each phases of a material. The boundary between phases is usually characterized by the Clapeyron slope, which is related to the changes in volume and entropy associated with the phase transition via $dP/dT=\Delta S/\Delta V$. The Clapeyron slope can be positive (for instance the perovskite to post-perovskite transition of MgSiO$_3$) or negative (such as the B1-B2 transition of MgO), leading to variations in the depth of phase transition at regions with temperature anomaly \cite{Karato-book}.

Some important questions for theoretists include: 1) how to accturatly calculate $G$ to determine the most stable phases out of a group of candiate structures? 2) how to precisely obtain the energy-pressure-temperature-volume relations ($E$-$P$-$T$-$V$ relations, or the equation of state) and properties of a system, once the structure is known? and 3) how to guide the design expensive experiments or verify and explain experimental discoveries? First-principles simulations, especially those based on density functional theory (DFT) \cite{ks1965} and quantum Monte Carlo (QMC) \cite{Ceperley2010}, are expected to provide satisfatory solutions to these questions. In this chapter, we introduce the interior of the Earth and the possible structures of Jovian and extrasolar planets, followed by some important questions for mineral physicists and methods in high-pressure studies. Details of first-principles methods will be described in the next chapter.

\section{The interior of the Earth and other planets}
The interior of the Earth is in layers, as has been proven by seismological and geodynamic findings. It mainly consists of the mantle, which is likely dominantly silicates and oxides, and the core, which is presumably Fe-Ni alloys with some light elements \cite{Birch1952,Mcdonough1995}. Based on seismic characteristics, the mantle is divided into the upper mantle (from the bottom of crusts to 410 km depth), the tranzition zone (410-660 km), and the lower mantle (660-2891 km). Typical rocks in the upper mantle (less than $\sim$7 GPa) are peridotite and basalt, where the former consists of olivine, garnet, and pyroxene, and the latter is mainly made of garnet and pyroxene. With increasing pressure at the lower mantle (24-127 GPa), these minerals transform into bridgmanite, ferropericlase, calcium silicate perovskite, stishovite, etc \cite{Trønnes2010}. Down to a few hundred kilometers above the core-mantle boundary ($\sim$136 GPa), there exists the D$^{\prime\prime}$ layer, which is characterized by strong seismic anomalies and lateral heterogeneities. Deeper at the core region, there is a liquid layer at the outside known as the outer core (2891-5150 km, 136-329 GPa), surrounding a solid inner core (5150-6371 km, 329-364 GPa) that has complex anisotropic structure \cite{Mattesini2010}. It is noteworthy that the detailed mineral composition of the lower mantle (and likewise the core) is still in question, which is a topic to be studied in Chapter \ref{chap3} of this dissertation.

%\begin{figure}
%\centering\includegraphics[width=\textwidth]{figs/earth_interior.png}
%\caption{\label{ch1fig1} The interior of the Earth \cite{Jeanloz1993}.}
%\end{figure}

%Therefore, the foundamental chemical systems to study include (Mg$_{1-x^A-y^A}$Fe$_{x^A}$Al$_{y^A}$)(Si$_{1-x^B-y^B}$Fe$_{x^B}$Al$_{y^B}$)O$_3$, Mg$_{1-x}$Fe$_x$O, SiO$_2$, and CaSiO$_3$, the high-pressure phases of each, the phase diagram and behavior of elements in their mixtures, the corresponding equation of states, and physical properties.

%\begin{figure}
%\centering\includegraphics[width=1.0\textwidth]{figs/pyrolite_mantle.png}
%\caption{\label{ch1fig2} Mineral proportion changes in pyrolite, a possible composition model for the Earth's mantle. Adapted from \cite{Lin2013}.}
%\end{figure}

Comparing with the Earth, knowledge of giant-planet interiors is much more limited due to the difficulty in direct detection. Despite of this, data from astronomical observations using telescopes or spacecraft emission have helped constructing models for the composition of Jovian planets \cite{Guillot2004}: Uranus and Neptune may each have a small rocky core surrounded by a thick layer of ice-hydrogen-rock mixture, and another layer near the surface consisting of hydrogen, helium, and ice; Jupiter and Saturn each has a icy rocky core, which is enveloped by a very thick layer of hydrogen-hellium mixtures. Here the ``ices'' refer to hydrides of the most abundant light elements (oxygen, carbon, and nitrogen) that are next to hydrogen, helium and neon, such as water, methane, and ammonia \cite{Stevenson2013}. Novel structures, such as a superionic state, have been predicted for these ices and refreshed our understanding about Uranus and Neptune \cite{Cavazzoni1999,Wilson2013}. Chapter \ref{chap5} and a few more works \cite{Militzer2016a,Militzer2016b} during my doctorate study are toward this end.

%\begin{figure}
%\centering\includegraphics[width=0.6\textwidth]{figs/jovian_planets_interior.png}
%\caption{\label{ch1fig3} The possible interiors Jovian planets. The estimated temperatures and pressures near the surfaces and shell boundaries are labeled. The size of the cores of Jupiter and Saturn are uncertain, even more so for the interiors of Uranus and Neptune. \cite{Guillot2004}}
%\end{figure}

Even less is known about the interiors and compositions of exoplanets, although the confirmed number of which has reached 2951 as of Aug. 5, 2016. Many of the planets recently discovered by the Kepler mission \cite{keplerpj} are bigger than the Earth or similar to Jovian planets. Comparing the planets' mass-radius data with the equation of state (EOS) of typical planet-forming materials enables estimating these planets' composition \cite{Seager2007} and grouping them into ``super Earths'', ``sub-Neptunes'', ``hot Jupiters'', etc. This gives one example that reflects the importance of EOS calculations. In fact, many areas, including shock wave experiments, stellar physics, and high energy density physics, have put requisites on EOS. An example of EOS calculation in wide temperature and pressure ranges on warm dense sodium is given in Chapter \ref{chap6}.

\section{Essential questions for mineral physicists}\label{ch2s2}
In order to understand the formation and evolution of the Earth, Jovians, and other planets beyond the solar system, it is important to have a good knowledge of the foundamental components---the minerals: their various chemistry, phases, and physical properties at high temperatures and pressures corresponding to the interiors of these planets.

In the Earth's lower mantle as well as the rocky parts of Jovian and many extrasolar planets' interiors, the most important systems to study are (Mg$_{1-x-y}$Fe$_{x}$Al$_{y}$)(Si$_{1-x'-y'}$Fe$_{x'}$Al$_{y'}$)O$_3$, 
Mg$_{1-x}$Fe$_x$O, SiO$_2$, and CaSiO$_3$. This includes the structure, texture, mechanic properties (compression behavior, elasticity, etc), transport properties (viscosity, thermal diffusivity and conductivity), as well as the phase diagram of each indivual system individually and their mixtures, at pressures spanning $10^5-10^7$ times atmosphere condition. Accurate data in these aspects are demanding for demystifying global and regional seismic anomalies observed in top-mid lower mantle and core-mantle boundary regions of the Earth. The are also important for screening or justifying geodynamic models for understanding mantle convection and evolution, core-mantle interaction, and plate techtonics.

The Earth's core is of particular importance and interests to the Earth science community. The structure, density-pressure relation, and sound velocities \cite{Badro2014} of Fe and Fe-X (X=Si, O, S, etc.) alloys are useful for estimating the core's composition; the phase diagrams of these alloys and the behavior of the light elements X \cite{Hirose2013,ORourke2016} are important for constraining the inner-outer core boundary and growth of the inner core; the melting temperature \cite{Aquilanti2015} constrains the mantle geotherm; and the thermal conductivity \cite{Ohta2016,Konopkova2016} defines the heat flux that is important for understanding the Earth's magnetic field and the formation of the solid inner core. Tremondous efforts across the world have been invested to working on these topics, but the composition and state of the core are still uncertain.

One common feature of the above questions is the condition of high pressure. At megabar pressurs, materials may exhibit new structures or behaviors that are distinct from those in ambient conditions. For example, at 0 GPa, $d$ electrons of Fe$^{2+}$ in (Mg,Fe)O ferropericlase are in a state with high spin number ($\mathcal{S}=5/2$), as is predicted by the Hund's rule. This ``high-spin'' state is no longer stable at mid-lower mantle pressure ($\sim$40 GPa), but transits into a ``low-spin'' state with $\mathcal{S}=0$, because the changes in crystal field splitting surpasses the spin-paring energy, at such high pressures. The spin transition is often believed to be associated with significant softening in the bulk modulus \cite{Chen2012} and increase in shear anisotropy \cite{Marquardt2009}, potentially result in dramatic changes in the partition of iron between bridgmanite and ferropericlase \cite{Badro2003} that are useful for geodynamic studies, and explain lateral heterogeneities in seismic observations \cite{Wu2014}. Another instance is, meterials under high pressure may have counter-intuitive stoichiometry \cite{WZhang2013}, because of the changes in electronic interaction and lattice dynamics due to compression. These factors should be considered seriously when studying materials at deep-interior conditions of planets.

\section{Methods in high-pressure studies}
Methods for materials studies at high pressure conditions have kept developing in the last few decades. Experimentally there are two main types of techniques---static compression and dynamic compression. The former is often based on large-volume press, such as the multi-anvil apparatus \cite{Ito2015} that can usually attain pressures of $\sim$30 GPa (top of the lower mantle) and sometimes $\sim$100 GPa, or diamond anvil cells \cite{Mao2007}, which enables reaching pressures higher than 360 GPa (center of the Earth); the latter using gas gun\footnote{E.g., Lindhurst Laboratory for Experimental Geophysics at Caltech, and more resources listed on http://mygeologypage.ucdavis.edu/stewart/OLDSITE/ImpactLabs.html.}, intensive laser\footnote{E.g., National Ignition Facility at Lawrence Livermore National Laboratory.}, or high magnetic fields\footnote{E.g., Z Machine at  Sandia National Laboratories.} can achieve pressures and temperatures greater than that at the center of Jupiter, and therefore plays a special role in the studies of Earth and planetary interiors \cite{Ahrens2003}.

Static compression has the advantage of flexibility in reaching the desired temperatures and pressures. Large volume press allows operating with samples in the sizes of centimeters, which is advantageous in studying equilibrium phase relations by analyzing the recovered samples after the pressing, heating, and quenching procedure. Diamond anvil cells are frequently used for studying isolated materials up to very high pressures. The progress in this technique has benefited greatly from the development of new-generation synchrotron radiation facilities, which provide high-energy X-ray beams that are useful to determine the structure of $\mu$m-sized samples confined in chambers with limited opening angles. 

Dynamic compression utilizes shock waves to generates high pressures, which are often accompanied by simultaneous high temperatures. This technique is particularly useful for measuring EOS and detecting phase transitions under the extreme conditions. During progation of the shock wave, the courses of the states of a target mineral is usually along specific paths that are governed by the EOS. Therefore, one can use theoretical models or calculations to provide EOS to guide the design of experiments, which improves the efficiency and could in turn verifies the theoretical data. Techniques such as pre-compression and multi-shock allows reaching states off the principle Hugoniots, and ramp compression enables high pressure measurments at low temperatures, which can be useful for studying pressure-driven phase transformations predicted by ground-state first-principles calculations.

In addition to the collaboration between experiment and simulation, the mineral physics community also has traditions of working closely with other disciplines, such as seismology, geodynamics, geochemistry, and planetary science. Such multi-field synergy is very important for obtaining a complete and convincing picture on the origin and evolution of the Earth and other planets.



