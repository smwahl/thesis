\chapter{Introduction}\label{chap1}

The materials composing Earth's are $\sim$99\% composed of only 8 elements (Fe,
O, Si, Mg, S, Ni, Ca and Al). In Jupiter at least 90\% of the planet is composed
of just hydrogen and helium.  Nonetheless these materials are the subject of
constant study by experiment and simulation. The extreme high pressures and
temperatures under which these materials exist make them difficult to study.
For the Earth these conditions extend to $\sim$300 GPa and $\sim$6000 of Kelvin
\citep{alfe2009} (the temperature value remains contentious); inside Jupiter
the pressures reach in excess of 4 TPa, while the maximum temperatures in the
aftermath of the moon-forming impact would  have been in the tens of thousands
Kelvin. Under such extreme conditions materials can behave in ways that would
not be predicted from their behavior at ambient conditions. Insulating
materials can become metallic at high pressures, and elements that are
nominally excluded can become readily soluble in a material.

An accurate understanding of the properties of planetary materials is essential
in determining their present structure and their evolution over time. Their
properties our inextricably tied to the methods we use to probe the insides of
planets, whether that be through seismology, geodesy, or measurement of their
gravitational or magnetic fields. But understanding their properties alone is
not sufficient. Planetary Science depends on physical models to take these
properties acting on a microscopic material scale and study how they govern
behaviour on planetary length- and time-scales. The results of such models are
often non-unique, due to the relatively small number of direct measurements
that can be made from a planet's surface, and due to the nature of key material
properties such a density  being compatible with a large range of different
compositions. A complete planetary model therefore draws on information from a
variety of sources, including cosmochemical inferences and, on Earth, the
record of rock sample exhumed from the planet's interior.

In this thesis I present a variety of work using computational techniques to
address questions about planetary interiors and evolution, focused on: 1)
determining properties of planetary materials from first-principles physics
simulations, and 2) applying of these derived properties to models of
large-scale planetary structure and processes. In Chapter~\ref{chap2} I discuss
the physics numerical techniques I have used and developed to simulate
planetary materials at high temperatures. Chapters~\ref{chap3} and \ref{chap4}
summarize the results of two studies using these first-principles techniques to
study the dissolution of a rock-ice cores in gas giant planets, and the
high-temperature solubility of terrestrial cores and mantles.
Chapter~\ref{chap5} introduces work to build a thermochemical model of iron 
alloys for understanding the generation of magnetic fields in the cores of
Mercury and other small terrestrial bodies. Chapter~\ref{chap6} shows the 
derivation of a non-perturbative method for calculating self-consistent 
gravitational fields of a fluid planet. Chapter~\ref{chap7} uses this method 
to study the tidal response of Jupiter and Saturn. Finally, Chapter~\ref{chap8}
uses the same technique in combination with first-principles equations of state
to interpret the gravitational field measured by the \textit{Juno} spacecraft 
at Jupiter.

\section{Interior structure and formation of the Earth and other planets}

The interior of the Earth, and presumably most other rocky planets and moons,
are differentiated into into a number of layers, as is demonstrated
seismological and geodynamic studies. It mainly consists of the mantle, which
is likely dominantly of magnesium silicates and oxides, and the core, which is
presumably Fe-Ni alloys with some light elements
\cite{Birch1952,Mcdonough1995}.  The Earth's upper mantle (less than $\sim$7
GPa) and the entire mantles of smaller terrestrial bodies are composed of the
rock peridotite, consisting of the minerals olivine, garnet, and pyroxene. With
increasing pressure at the these minerals undergo a number of phase transitions
through the transition zone (410-660 km), eventually transforming into the
lower mantle (669-2891 km) composition of bridgemanite (formally Mg-perovskite)
and ferropericlase \cite{Tronnes2010}.  Down to a few hundred kilometers above
the core-mantle boundary ($\sim$136 GPa), there exists the D$^{\prime\prime}$
layer, which is characterized by strong seismic anomalies and lateral
heterogeneities. This layer may involve an additional transition from
bridgemanite to Mg-postperovskite and/or a dense partial melt.  The Earth's
core is divided into a liquid outer core (2891-5150 km, 136-329 GPa),
surrounding a solid inner core (5150-6371 km, 329-364 GPa). Smaller rocky
planets are also expected to have at least a thin liquid outer core, due to the
significant freezing point depression of Fe-S alloys.

The structure of giant-planet interiors is much more limited, but data from
telescope observations and spacecraft missions, as well as cosmochemical
studies based on meteorites and the solar photosphere have aided in
constructing models interiors of Jovian planets \cite{Guillot2004}. Jupiter and
Saturn are composed primarily of hydrogen and helium, but must contain tens of
Earth-masses of heavy elements. The hydrogen-helium mixture transitions from an
insulator to a metal at $\sim$100 GPa. It is not known for sure whether these
planets have compositionally distinct layers, or if the ``rocks'' and ``ices''
exist in solution, and are distributed through the planet (Chapters~\ref{chap3}
and \ref{chap7} deal with this question).  Uranus and Neptune may each have a
small rocky core surrounded by a thick layer of ice-hydrogen-rock mixture, and
another layer near the surface consisting of hydrogen, helium, and ice.  Here
the ``ices'' refer to hydrides of the most abundant light elements (oxygen,
carbon, and nitrogen) that are next to hydrogen, helium and neon, such as
water, methane, and ammonia, which condense from the nebula at relatively low
temperatures. 

Even less is known about the interiors and compositions of exoplanets, the
confirmed number of which has reached 3,586 as of March 2017 \cite{Schneider2011}.
The known exoplanets span a large range of masses from smaller than Earth to
greater than Jupiter.  There detection methods do have a selection bias towards
larger planets that are closer in to their stars, thus many fall into to
category of ``hot Jupiters''. Statistical analyses of the Kepler catalogue,
however, suggest that the most numerous category of planets is that in a
``super-Earth'' or ``sub-Neptune''  mass range. Comparing the planets'
mass-radius data with the equation of state (EOS) of typical planet-forming
materials enables estimating these planets' composition \cite{Seager2007} give
some idea of the composition of these planets, although the determination is
non-unique without additional information. The results of the studies in
Chapters~\ref{chap3} and \ref{chap4} have some relevance to exoplanets, as does
Chapter~\ref{chap8} in so far as, many exoplanets are expected to have
analogous interior structures to Jupiter.

The final stages of terrestrial planet formation are expected to be include 
a number of energetic collisions between planetesimals and proto-planets. 
These giant impacts have been invoked to explain a number of features of and
difference between planets in the inner solar system. One of the more extreme
examples is the hypothesized moon-forming impact. Recent studies have suggested
even more energetic, high-angular momentum collisions \citep{Cuk2012,Canup2012} than the 
canonical, Mars-sized impactor scenario \citep{canup2004}. A consequence of 
such a giant impact is that at least a portion of the mantle would have been
molten in the aftermath, and may have been heated to temperatures
much higher than those typically considered in studies of Earth materials. 
We pose this question in Chapter~\ref{chap4}.

For the giant planets there is an ongoing debate regarding the nature of their 
formation: whether a terrestrial-like core must form first for the gas to accrete 
on to, or if they formed directly from collapse of an instability in the gas. 
The concentration of heavy elements towards the center of the planet is 
indicative of the first scenario, although a comprehensive theory for the 
early evolution of the gas giants is not settled. The work presented in 
Chapters~\ref{chap3} and \ref{chap8} are relevant for this debate.

\section{Methods in high-pressure studies}

Although the studies presented here focus on numerical simulations, the
development of numerical techniques has gone hand in hand with experimental
techniques  for studying materials at high pressure.  There are two main
classes of high-pressure experimental techniques: 1) static compression and 2)
dynamic compression. The first involves applying a constant pressure to a
sample. These techniques include ones based on large-volume press, such as the
multi-anvil apparatus \cite{Ito2007} that can attain pressures of $\sim$30 GPa
(top of the lower mantle) and sometimes $\sim$100 GPa, or diamond anvil cells
\cite{Mao2007}, which reach pressures greater than 360 GPa (center of the
Earth). The second class involves generating and monitoring a shock wave in a
sample. The drivers for dynamic compression experiments include gas
gun\footnote{E.g., Lindhurst Laboratory for Experimental Geophysics at Caltech,
    and more resources listed on
http://mygeologypage.ucdavis.edu/stewart/OLDSITE/ImpactLabs.html.}, intensive
laser\footnote{E.g., National Ignition Facility at Lawrence Livermore National
Laboratory.}, or strong magnetic fields\footnote{E.g., Z Machine at  Sandia
National Laboratories.} \cite{Asimow2015}.

Static compression has the advantage of flexibility in reaching a large range
desired temperatures and pressures. Large volume press allow for larger,
centimeter-sized samples, which is helpful for studying equilibrium phase
relations thorough analysis of the recovered samples . Diamond anvil cells are
frequently used for studying isolated materials up to very high pressures,
often with laser-heating to reach the desired temperature. Diamond anvil cell
studies  have benefited greatly from the development of new-generation
synchrotron radiation facilities, which provide high-energy X-ray beams to
determine the structure of $\mu$m-sized samples in situ.

Dynamic compression utilizes shock waves to generate high pressures, which are
necessarily accompanied by simultaneous high temperatures. This technique is
particularly useful for constraining equations of state and detecting phase
transitions under the extreme conditions. During propagation of the shock wave,
the courses of the states of a target mineral is usually along specific paths
that are governed by the EOS. Techniques such as pre-compression and
multi-shock allows reaching states off the principle Hugoniots, and ramp
compression enables high pressure measurements at low temperatures, which can
be useful for studying pressure-driven phase transformations predicted by
ground-state first-principles calculations.

In spite of sweeping advancements in the field, limitations remain for
high-pressure experimental techniques, some of which can be aided by judicious
use of first principles simulations. There are regions of pressure temperature
space that are difficult or impossible to achieve in experiments. Shock
experiments are classically confined to follow the Rankins-Hugoniot curve,
meaning that for a given pressure the temperatures rise much faster than
typical planetary barotropes. Laser heating of diamond anvil cells also has its
limits when it comes to reaching very high temperatures. Thus, there is a range
of conditions in deep planetary interiors, particularly for the giant planets,
that are inaccessible to experiments. There are certain material properties
that very difficult to measure in situ, and only some questions can be
adequately addressed with recovered samples. In some cases, accuracy of
experimentally determined pressures and densities are insufficient to constrain
equations of state satisfactorily (as is the case for experimental
hydrogen-helium equations of state of interest to the work in
Chapter~\ref{chap8}). In spite of these issues, experiments are ultimately the
way to unambiguously verify theoretical predictions. It is clear that both
fields benefit immensely from contributions from the other.

